%
% Positronium-hydrogen Kohn variational method scattering paper
%

%\documentclass[preprint,showpacs,showkeys,preprintnumbers,amsmath,amssymb,longbibliography,pra,aps,superscriptaddress]{revtex4-1}
\documentclass[preprint,showpacs,showkeys,preprintnumbers,amsmath,amssymb,longbibliography,pra,aps]{revtex4-1}
%\documentclass[reprint,showpacs,preprintnumbers,amsmath,amssymb,pra,aps,longbibliography]{revtex4-1}

% Denton's shortcuts
\newcommand{\ee} {\,\text{e}}
\newcommand{\ii}{{\rm{i}}}

% Packages

\usepackage{graphicx}  % Include figure files
%\usepackage[draft]{graphicx}  % Do not include figure files
\usepackage{dcolumn} % Align table columns on decimal point
\usepackage{bm} % bold math
\usepackage{float}  % @TODO: Take this out before submission!
\usepackage{todonotes}
\usepackage[hidelinks]{hyperref}  % For clickable references
\usepackage{amssymb}
\usepackage{xcolor}
\usepackage{wasysym}  % Simply for \CIRCLE and \Circle (could use \circ for the latter)

\usepackage{silence}
\WarningFilter{revtex4-1}{Repair the float}  % Completely innocuous warning - http://tex.stackexchange.com/questions/180762/revtex4-1-warning-repair-the-float-package

% Subfolders containing our images
\graphicspath{{IPython/}{Images/}}

% Remove before submitting
\newcommand{\todoi}{\todo[inline]}

% To center table head
\newcommand*{\thead}[1]{\multicolumn{1}{c}{#1}}

\setlength{\abovecaptionskip}{0pt}  % Reduce space between figure and caption (default of 10pt)
\setlength{\paperheight}{11in}  % To get rid of the annoying hyperref warning


\begin{document}

%\preprint{APS/123-QED}

\title{A Detailed Investigation of Low-Energy Positronium-Hydrogen Scattering}

\author{Denton Woods}
\email{denton.woods@unt.edu}
\homepage{http://www.dentonwoods.com}

\author{S. J. Ward}
\affiliation{Department of Physics, University of North Texas, Denton, Texas 
76203, USA}

\author{P. Van Reeth}
\affiliation{Department of Physics and Astronomy, University College London, 
Gower Street, London WC1E 6BT, UK}

\date{\today}

\begin{abstract}
We investigate the four-body Coulomb process of low-energy elastic positronium
-hydrogen (Ps-H) scattering below the Ps(n=2) excitation threshold with a 
Hylleraas-type basis set. Using the complex Kohn variational method, we 
compute phase shifts through the $^{1,3}$H-wave. The $^{1,3}$S- and
$^{1,3}$P-wave phase shifts can be viewed as benchmark results. The complex
Kohn variational results compare well to a number of other calculations for 
this system. Using the Hylleraas-type basis set, we also compute the binding 
energy of $^1$S positronium hydride. In addition, we present elastic 
differential, elastic integrated, momentum transfer, and ortho-para 
conversion cross sections, and for the singlet, resonances through the
F-wave. Finally, we give a detailed analysis of the scattering lengths and 
effective ranges using multiple effective range theories.
\end{abstract}

   
\pacs{31.15.xt Variational techniques; 34.50.-s Scattering of atoms and
molecules; 36.10.Dr Positronium; 34.80.Bm Elastic scattering}
\keywords{positronium; PsH; positronium hydride; Kohn variational}
\maketitle

\section{\label{sec:Intro}\protect Introduction}


%\preprint{APS/123-QED}

Positronium (Ps) scattering from atoms and molecules is an area of current 
experimental and theoretical interest. The development of energy-tunable
ortho-Ps beams
\cite{Brown1985,Laricchia1987,Zafar1996,Garner1996,Laricchia2008} 
has enabled measurements to be made of Ps scattering from the inert gases He, 
Ne, Ar, Kr, and Xe
\cite{Garner1996,Garner2000,Armitage2002,Laricchia2004,Armitage2006,Laricchia2008,Engbrecht2008,Brawley2010a}
and the molecules H$_2$, N$_2$, O$_2$, CO$_2$, H$_2$O, and SF$_6$
\cite{Garner1996,Garner1998,Garner2000,Laricchia2004,Armitage2006,Beale2006,Brawley2010a}.
The cross sections for Ps scattering from H have not been measured due to the
difficulty of an atomic hydrogen beam, although the binding energy of positronium
hydride (PsH) has been measured in the reaction of a positron with methane,
e$^+$ + CH$_4$ $\to$ CH$_3^+$ + PsH \cite{Schrader1992}. The low-energy region
is of particular interest, because in this energy range, positron and electron 
correlations are dominant. We are presenting
\cite{Conferences1,Conferences2,Conferences3} 
in this paper our work of the application of the complex Kohn variational 
method to elastic Ps(1s)-H(1s) scattering in the energies up to the 
excitation threshold of Ps(n=2) at 5.101 eV.

Ps formation is important in the galactic core \cite{Kinzer1996}, and Ps-atom 
scattering is of interest in the study of solar processes \cite{Crannell1976}
. As well as the basic interest of Ps-atom scattering in atomic physics, Ps 
is also important in material science. As Ps is neutral, it penetrates deeper 
into material than a charged particle, such as a positron. Ps scattering also 
has applications in other areas of physics such as biophysics and 
astrophysics \cite{Laricchia2012}.

Brawley et al. found that for many targets, equal velocity e$^-$ and Ps 
scattering cross sections are similar between 0.5 and 2.0 a.u.
\cite{Brawley2010a,Brawley2010}. This is despite the fact that Ps has twice
the mass as e$^-$ and is neutral. Fabrikant and Gribakin recently found that
for Ps-Kr and Ps-Ar low velocity scattering, the cross sections are comparable
to that of e$^-$-Kr and e$^-$-Ar scattering
\cite{Fabrikant2014,Fabrikant2014a}. The tentative conclusion is that the
positron plays a much smaller role in the scattering process than the electron
in Ps-atom and Ps-molecule scattering.

Ps-H scattering is a fundamental four-body Coulomb process. The Kohn (and 
inverse Kohn) variational method has previously been applied to Ps-H 
collisions by Van Reeth and Humberston \cite{VanReeth2003,VanReeth2004}, who 
computed singlet and triplet $^{1,3}$S and $^{1,3}$P elastic phase shifts. We 
have extended their $^{1,3}$S and $^{1,3}$P Kohn variational calculations in 
multiple ways. First, and foremost, in addition to implementing
the Kohn and inverse Kohn variational methods, we have implemented the 
generalized Kohn and the generalized complex Kohn for the \emph{S} and \emph{T}
matrices. The complex Kohn variational methods are known to suffer
from far fewer anomalous singularities than the Kohn, inverse Kohn and 
generalized Kohn variational methods
\cite{Lucchese1989, Cooper2009, Cooper2010}. The second extension that we
consider is to use the procedure by Todd 
\cite{Todd2007} to systematically remove short-range terms that cause linear 
dependence. This enabled us to compute the phase shifts (and the binding 
energy of PsH) with more short-range Hylleraas terms than the earlier 
Kohn calculations. We added the asymptotic expansion of Drake and Yan
\cite{Drake1995, Yan1997} to improve the short-range integrations. Next, we
have significantly increased the number of integration points for matrix
elements that involve the long-range terms, along with implementing a method to
accelerate the convergence of these integrals. We also extended the 
calculations to the next four partial waves through to the H-wave. Finally, 
we calculate the differential elastic, integrated elastic, momentum transfer, 
and ortho-para conversion cross sections and analyze the resonances through 
the F-wave.

We apply the \emph{S} matrix complex Kohn variational principle to the first 
six partial waves for Ps-H scattering and present results. We confirm the 
previously observed resonances for the first four partial waves and compare 
the positions and widths to that of the earlier Kohn, close coupling (CC) and 
complex rotation calculations (references to be given). In addition, we 
compute the scattering lengths and effective ranges using multiple effective 
range theories. We also use the short-range part of the full scattering 
wavefunction to compute the binding energy of PsH. Comparing the binding 
energy with the most elaborate variational results gives an indication of the 
reliability of describing the Ps-H system at short-distances.

The binding energy of PsH, $E_b$, has been calculated using various methods. 
Ho performed a variational calculation with a Hylleraas basis set
\cite{Ho1986}, and Yan and Ho later did a more extensive calculation
\cite{Yan1999}. Mitroy used the stochastic variational method (SVM) with
1800 explicitly correlated Gaussians (ECGs) \cite{Mitroy2006}, and Bubin
and Adamowicz found the most accurate value using 5000 ECGs in a variational
calculation \cite{Bubin2006}.
 
%This is not meant to be an exhaustive list for this system.

There have been a number of other calculations for Ps-H scattering. A much 
earlier Kohn variational calculation was performed by Page \cite{Page1976} 
for the Ps-H scattering lengths. Drachman and Houston used a stabilization 
method with an effective range theory (ERT) expansion
\cite{Drachman1975,Drachman1976}. At low energies, diffusion Monte Carlo (DMC)
\cite{Chiesa2002}, the SVM \cite{Ivanov2001,Ivanov2002}, CC
\cite{Sinha1997,Campbell1998,Adhikari1999,Sinha2000,Blackwood2002,Blackwood2002b,Walters2004},
static exchange \cite{Hara1975,Ray1997}, and Kohn variational
\cite{Page1976,VanReeth2003,VanReeth2004} methods have been applied. The SVM
with stabilization 
techniques was used to compute accurate low-energy phase shifts and 
scattering lengths for Ps-H collisions \cite{Ivanov2001,Ivanov2002}. A 
disadvantage of the SVM over scattering theory methods such as the Kohn 
variational method is that the phase shifts are determined at energies which 
are not known in advance. Thus, the S- and P-wave effective-range parameters 
have to first be used to compute the phase shifts at the same energy points 
before the cross section could be computed \cite{Ivanov2002}. A disadvantage 
of both the SVM and DMC methods is that they do not give bounds on the 
scattering parameters \cite{VanReeth2003}. This means that it is difficult to 
assess whether the results obtained from these methods are converged with 
respect to improvements in the wavefunction.

Blackwood et al.~\cite{Blackwood2002} performed an elaborate CC calculation 
for Ps scattering from H, which took into account excitation and ionization 
of both the projectile and target. They considered two different coupling 
schemes. The first one, which they refer to as 9Ps9H, included 9 eigen- and 
pseudo-states of Ps and also of H. The second scheme, which they refer to as 
14Ps14H, is for the singlet only and includes 14 eigen- and pseudo-states of 
Ps and also of H. Good agreement is obtained between the CC
\cite{Blackwood2002} and the SVM \cite{Ivanov2002} for the S-wave scattering
lengths and phase shifts. Walters et al.~\cite{Walters2004}
extended the earlier CC calculations \cite{Blackwood2002} to include the
e$^+$-H$^-$ channel
\cite{Blackwood2002b} and compared their results for the S-wave with the
accurate Kohn variational results \cite{VanReeth2003}. They speculated that the
inclusion of the virtual Ps$^-$ formation channel may be needed to obtain 
agreement with the Kohn variational results for Ps(1s)-H(1s) scattering 
\cite{Blackwood2002}.

Unlike the SVM and DMC methods, the Kohn variational method gives rigorous 
bounds on the scattering lengths and, except for Schwartz singularities, 
gives empirical bounds on the elastic phase shifts. This means that the wave 
function can be systematically improved to the converged results. The Kohn 
variational method is known to yield accurate results and has provided 
benchmark results \cite{VanReeth2003,VanReeth2004} to which results from 
other calculations can be compared. Unlike the SVM, the Kohn variational 
method can treat inelastic as well as elastic scattering and thus can be 
extended to higher energies. 

The most recent calculation of Ps-H scattering uses the confined variational 
method (CVM) \cite{Zhang2012}. This provides accurate results but has the 
drawback of being very computationally expensive. In Ref.~\cite{Zhang2012}, 
they calculate phase shifts for 2 momenta for both $^1S$ and $^3S$. Other 
earlier calculations of Ps-H scattering are given in Refs.~\cite{Massey1954} 
and \cite{Hara1975}.

Phase shifts are expressed in radians. Atomic units are used throughout 
unless otherwise stated. For conversions to electron-volts (eV), we use the 
conversion factor 1 au = {27.21138505(60) eV}
\cite{Mohr2012,*NISTConversions}.

%\emph{Add \cite{Shipman2014}, \cite{Barker2012}, \cite{Laricchia2009}?}
%\emph{Add \cite{Mitroy2002a}?}


\vskip 0.5truecm
\noindent
\section{Theory}

\subsection{The Positronium-Hydrogen System}
\begin{figure}[H]
	\centering
	\includegraphics[height=2in]{PsHCoordinates}
	\caption{Positronium-hydrogen coordinate system}
	\label{fig:PsHCoords}
\end{figure}

We are investigating low-energy elastic scattering of ground-state Ps
with ground-state 
hydrogen, Ps(1s)+H(1s), for incident energies up to the excitation threshold 
of Ps(n=2)+H(1s), which is at an energy of $\tfrac{3}{16}$ au ($5.102$ eV). 
Previous work on Ps-H scattering used the Kohn and inverse Kohn approach
\cite{VanReeth2003, VanReeth2004}. The complex Kohn methods are more stable and
suffer less from Schwartz singularities than the Kohn, inverse Kohn and 
generalized Kohn methods \cite{Lucchese1989,Cooper2009,Cooper2010}. Work by 
Van Reeth and Humberston on e$^+$-He scattering also used the complex Kohn 
variational method \cite{VanReeth1999}. The results presented in Sec.
\ref{sec:Results} use the \emph{S} matrix complex Kohn method, but we present a
general wavefunction that can be used in any of the variants of the Kohn 
variational method, as we describe in Sec. \ref{sec:Kohn}.

For S-wave Ps(1s)-H(1s) elastic scattering, the flexible scattering
wavefunction is given by
\begin{equation}
\Psi_0^{\pm,t} = \widetilde{S}_0 + L_0^{\pm,t} \, \widetilde{C}_0
  + \sum_{i=1}^{N(\omega)} c_{i,0} \phi_{i1},
\label{eq:TrialWave}
\end{equation}
where the superscript $t$ indicates that this is a trial wavefunction. The plus
sign indicates the spatially symmetric singlet case, and the minus sign
indicates the spatially antisymmetric triplet. The total orbital angular
momentum of the system is equal to the orbital angular momentum of the incoming
ground-state Ps, $\ell$. For partial waves $\ell > 0$,
\begin{equation}
\Psi_\ell^{\pm,t} = \widetilde{S}_\ell + L^{\pm,t}_\ell \, \widetilde{C}_\ell
 + \sum_{i=1}^{N(\omega)} c_{i,\ell} \phi_{i1}
 + \!\!\!\sum_{i=N(\omega)+1}^{2N(\omega)} \!\! d_{i,\ell} \phi_{i2}.
\label{eq:TrialWaveHigher}
\end{equation}
The scattering wavefunctions contain both the long-range terms $\bar{S}$ and
$\bar{C}$ and the short-range correlation terms $\phi_{i1}$ for small
interparticle distances. We show the coordinate system for Ps-H in 
Fig. \ref{fig:PsHCoords}. The vector
$\vec{\rho} = \frac{1}{2}\left(\vec{r_1} + \vec{r_2}\right)$ is the position
vector of the center of mass of the Ps atom with respect to the proton. The 
long-range terms of Eqs. (\ref{eq:TrialWave}) and (\ref{eq:TrialWaveHigher})
are given by
\begin{equation}
\label{eq:SCPhiDef}
\begin{bmatrix}
\widetilde{S}_\ell \\ \widetilde{C}_\ell
\end{bmatrix} = \textbf{u}  \begin{bmatrix}
\bar{S}_\ell \\ \bar{C}_\ell
\end{bmatrix} = \begin{bmatrix}
u_{00} & u_{01} \\  u_{10} & u_{11}
\end{bmatrix}
\begin{bmatrix}
\bar{S}_\ell \\ \bar{C}_\ell
\end{bmatrix}, 
\end{equation}
with
\begin{subequations}
\label{eq:SCBarPhiDef}
\begin{align}
\bar{S}_\ell &= \frac{1\pm P_{23}}{\sqrt{2}}Y_{\ell 0}(\theta_\rho,
  \varphi_\rho)\Phi_{Ps,1S}\left(r_{12}\right) \Phi_{H,1S}\left(r_3\right)
  \sqrt{2\kappa} \,j_\ell\left(\kappa\rho\right) \text{ and} \label{eq:SBar} \\
\bar{C}_\ell &= -\frac{1\pm P_{23}}{\sqrt{2}}Y_{\ell 0}(\theta_\rho,
  \varphi_\rho)\Phi_{Ps,1S}\left(r_{12}\right) \Phi_{H,1S}\left(r_3\right)
  \sqrt{2\kappa} \,n_\ell\left(\kappa\rho\right) f_\ell(\rho). \label{eq:CBar}
\end{align}
\end{subequations}

$P_{23}$ is the exchange operator for the two indistinguishable electrons.
$\Phi_{Ps,1s}\left(r_{12}\right)$ and $\Phi_{H,1s}\left(r_3\right)$ are the 
ground-state wavefunctions of Ps and H, respectively. The shielding factor,
$f_\ell(\rho)$, ensures that the singularity of the Neumann function is removed
at the origin and is chosen to be
\begin{equation}
f_\ell(\rho) = \left[1 - \ee^{-\mu \rho} \left(1+\frac{\mu}{2}\rho\right)
\right]^{m_\ell}.
\label{eq:PartialWaveShielding}
\end{equation}
The $m_\ell$ is an integer that can be chosen for each $\ell$ so that $\bar{C}$
behaves like $\bar{S}$ as $\rho \rightarrow 0$. We use this criteria as a 
minimum value for $m_\ell$, and Table \ref{tab:Nonlinear} gives the values we 
use for each partial wave.

We consider a number of variants of the Kohn variational method, described more
fully in Sec. \ref{sec:Kohn}, and $\textbf{u}$ and $L^{\pm}_\ell$ take
different forms depending on each. We use different $\textbf{u}$ matrices to
generate multiple Kohn methods, namely the following:
\begin{subequations}
\label{eq:KohnU}
\begin{align}
&\text{generalized Kohn, } L^{\pm}_\ell = \tan(\delta^{\pm}_\ell-\tau),
 \textbf{u} = \left[ \begin{smallmatrix}
\cos \tau & \sin \tau \\  -\sin \tau & \cos \tau
\end{smallmatrix} \right], \label{eq:GenKohn}\\
&\text{generalized \emph{T} matrix Kohn, } L^{\pm,t}_\ell = T_\ell^{\pm\prime},
 \textbf{u} = \left[ \begin{smallmatrix}
\cos\tau & \sin\tau \\ -\sin\tau + \ii \cos\tau & \cos\tau + \ii \sin\tau
\end{smallmatrix} \right], \text{and}\label{eq:GenTKohn} \\
&\text{generalized \emph{S} matrix Kohn, } L^{\pm,t}_\ell = S_\ell^{\pm\prime},
 \textbf{u} = \left[ \begin{smallmatrix}
-\ii \cos\tau - \sin\tau & -\ii\sin\tau + \cos\tau \\ 
 \ii\cos\tau - \sin\tau & \ii\sin\tau + \cos\tau
\end{smallmatrix} \right]. \label{eq:GenSKohn}
\end{align}
\end{subequations}
For the case of $\tau = 0$, these give the Kohn, the \emph{T} matrix and the 
\emph{S} matrix, respectively. $\tau = \frac{\pi}{2}$ in Eq. (\ref{eq:GenKohn})
gives the inverse Kohn. Note that our generalized \emph{T} matrix is 
different than Cooper et al.\ \cite{Cooper2010}, in that our $\bar{S}$ and
$\bar{C}$ are swapped, which gives an outgoing wave. These are similar to the
$\textbf{u}$ matrices given by Lucchese \cite{Lucchese1989} but are more 
generalized versions, and our \emph{T} matrix and \emph{S} matrix choices are 
a slightly different form.

The short-range terms are highly correlated Hylleraas-type functions, including
all interparticle distances, given by
\begin{equation}
\label{eq:PhiDef}
\bar{\phi}_{ij} = \left(1 \pm P_{23}\right) Y_{\ell 0}(\theta_j,\phi_j)
e^{-(\alpha r_1 + \beta r_2 + \gamma r_3)}
r_j^{\ell} r_1^{k_i} r_2^{l_i} r_{12}^{m_i} r_3^{n_i} r_{13}^{p_i} r_{23}^{q_i}.
\end{equation}
The variable $\omega$ is a non-negative integer that determines the maximum
number of terms in the basis set. For a chosen value of $\omega$, the integer
powers of $r_i$ and $r_{ij}$ are constructed in such a way that 
\begin{equation}
k_i + l_i + m_i + n_i + p_i + q_i \leq \omega,
\end{equation}
with all $k_i, l_i, m_i, n_i, q_i$ and $p_i \geq 0$.
The first set of short-range terms in Eq. (\ref{eq:TrialWaveHigher}), referred
to as the first symmetry, has $j=1$ for $i=1$ to $N(\omega)$. The second
symmetry set of terms exists for $\ell > 0$, with $j=2$ and $i = N(\omega)$ to
$2N(\omega)$.

These short-range terms represent the angular momentum as being placed on 
either the positron ($r_1$) or on the electron on the Ps atom ($r_2$, and $r_3$
with exchange). Following up on the Ref. \cite{VanReeth2004} where the slow 
convergence of the $^3$P phase shift was discussed, we also tried a 
wavefunction where the angular momentum was placed on the electron of the H 
atom ($r_3$) and on the Ps ($\rho$). However, this did not improve 
convergence for us and was even marginally worse in some cases. The numerical 
techniques discussed in Sec. \ref{sec:Numerical} improved our convergence.

For partial waves with $\ell>1$, the orbital angular momentum could also be 
shared between both particles in the Ps atom \cite{Schwartz1961a}. An 
investigation of the contribution to the final results from each symmetry for 
e$^+$-He scattering \cite{VanReeth1997} has revealed that the mixed 
symmetries were not important, and because of the complexity of the 
analytical evaluation of the various matrix elements, we have omitted the 
mixed symmetry terms where angular momentum is shared.

% THE HAMILTONIAN
The Hamiltonian for the fundamental Coulombic system is
\begin{align}
H = -&\frac{1}{2} \nabla_{r_1}^2 - \frac{1}{2} \nabla_{r_2}^2 - \frac{1}{2}
  \nabla_{r_3}^2  \nonumber \\
&+ \frac{1}{r_1} - \frac{1}{r_2} - \frac{1}{r_3} - \frac{1}{r_{12}} -
  \frac{1}{r_{13}}+\frac {1}{r_{23}},
\label{eq:Hamiltonian1}
\end{align}
which can also be expressed in Jacobi coordinates for Ps-H scattering as
\begin{align}
H = -&\frac{1}{4} \nabla_{\rho}^2 - \frac{1}{2} \nabla_{r_3}^2 -
  \nabla_{r_{12}}^2  \nonumber \\
&+ \frac{1}{r_1} - \frac{1}{r_2} - \frac{1}{r_3} - \frac{1}{r_{12}} -
  \frac{1}{r_{13}}+\frac{1}{r_{23}}.
\label{eq:Hamiltonian2}
\end{align}


\subsection{Kohn Variational Methods}
\label{sec:Kohn}
This derivation follows a similar procedure as that of
Refs.~\cite{Lucchese1989,Cooper2010,Armour1991,VanReethThesis}.
The functional for the full wavefunction in Eqs. (\ref{eq:TrialWave}) and
(\ref{eq:TrialWaveHigher}) is (dropping the $\ell$ subscript and the $\pm$ 
superscript for clarity),

\begin{equation}
I[\Psi^t] = \left(\Psi^t, \mathcal{L} \Psi^t \right) = \int \Psi^t \mathcal{L}
  \Psi^t \,d\tau,
\label{eq:IlDefPsi}
\end{equation}
with
\begin{equation}
\mathcal{L} = 2(H - E).
\label{eq:LDef}
\end{equation}
The total energy, E, is given by
\begin{equation}
\label{eq:TotalEnergy}
E = E_H + E_{Ps} + \frac{1}{4}\kappa^2 = E_H + E_{Ps} + E_{\bm \kappa},
\end{equation}
where $E_H$ and $E_{Ps}$ are the ionization energies of H and Ps, respectively.
Notice that the wavefunction is not conjugated, as pointed out by Cooper et al.
\cite{Cooper2010}.

We assume the trial wavefunction is a small variation of the exact wavefunction
$\Psi$, or
\begin{equation}
\Psi^t = \Psi + \delta \Psi
\label{eq:PsiTrialRelation}
\end{equation}
It can be shown that solving for $\delta I$, the variation of $I$, gives a
result for the variational method of
\begin{equation}
\delta I = I[\Psi^t] - I[\Psi] = I[\Psi^t] = (L^t - L + I[\delta \Psi]) \det u.
\label{eq:IlPsiVariation}
\end{equation}
Here $L$ and $L^t$ correspond to the exact and trial wavefunctions,
respectively, with the trial wavefunctions given in Eqns. \ref{eq:TrialWave}
and \ref{eq:TrialWaveHigher}. Neglecting the second order term in
$\delta \Psi$ and realizing that $I[\Psi] = 0$, we get a functional for the
variational $L^v$ of
\begin{equation}
L^v = L^t - I[\Psi^t] / \det u.
\label{eq:ComplexKohnVariation}
\end{equation}

Using the stationary property of the complex Kohn functional, we get
\begin{equation}
\frac{\partial L^v}{\partial L^t} = 0  \text{ and }
  \frac{\partial L^v}{\partial c_i} = 0 \text{ where $i = 1,\ldots,N$},
\label{eq:ComplexKohnStationary}
\end{equation}
which can be written as a matrix equation. For the S-wave, this is
\begin{equation}
\label{eq:ComplexKohnMatrix}
\begin{bmatrix} 
 (\widetilde{C},\mathcal{L}\widetilde{C}) & (\widetilde{C},\mathcal{L}\bar{\phi}_{11}) & \cdots & (\widetilde{C},\mathcal{L}\bar{\phi}_{N1})\\
 (\bar{\phi}_{11},\mathcal{L}\widetilde{C}) & (\bar{\phi}_{11},\mathcal{L}\bar{\phi}_{11}) & \cdots & (\bar{\phi}_{11},\mathcal{L}\bar{\phi}_{N1})\\
 \vdots & \vdots & \ddots & \vdots \\
 (\bar{\phi}_{N1},\mathcal{L}\widetilde{C}) & (\bar{\phi}_{N1},\mathcal{L}\bar{\phi}_{11}) & \cdots & (\bar{\phi}_{N1},\mathcal{L}\bar{\phi}_{N1})
\end{bmatrix}
\begin{bmatrix}
L^t\\
c_1\\
\vdots\\
c_N
\end{bmatrix}
= -
\begin{bmatrix}
(\widetilde{C},\mathcal{L}\widetilde{S}) \\
(\bar{\phi}_{11},\mathcal{L}\widetilde{S}) \\
\vdots \\
(\bar{\phi}_{N1},\mathcal{L}\widetilde{S})
\end{bmatrix}.
\end{equation}
This matrix equation can be rewritten as
$\textbf{\emph{AX}}$ = -$\textbf{\emph{B}}$. For higher partial waves,
the matrix equation looks the same but includes the second symmetry
short-range terms. Finally, we solve for $L^v$, from which we obtain our
phase shifts.
\begin{equation}
L^v = -\frac{1}{\det u} \left( \textbf{\emph{B}}^{tr} \textbf{\emph{X}} +
  (\widetilde{S},\mathcal{L} \widetilde{S}) \right)
\end{equation}
To determine the phase shifts, we use the relation given by Ref.~\cite{Lucchese1989} as
\begin{equation}
\label{eq:GenKohnL}
K_\ell = \tan \delta_\ell = (u_{01} + u_{11} L_\ell)(u_{00} + u_{10}
  L_\ell)^{-1}.
\end{equation}

%\begin{center}
%\line(1,0){250}
%\end{center}

\subsection{PsH Bound State and Optimization of Nonlinear Parameters}
As done earlier by Van Reeth and Humberston \cite{VanReeth2003,VanReeth2004},
we use the short-range correlation part of the S-wave scattering wavefunction
to compute the binding energy, $E_b$, of the $^1S$ PsH system. This allows us
to determine the reliability of using our short-range terms for the Ps-H 
scattering problem. The wavefunction we use for the bound state is
\begin{equation}
\label{eq:BoundWavefn}
\Psi^\pm = \sum_{i=1}^{N(\omega)} c_i \phi_{1i}^\pm,
\end{equation}

where $\phi_{1i}^\pm$ is the same as in Eq. (\ref{eq:PhiDef}). The
Rayleigh-Ritz method is used, and the optimization of the nonlinear parameters
is done using both the Newton \cite{Yan1999} and simplex methods \cite{GSL}.
For the $^3$S case and $\ell \neq 0$, even though there is not a bound state,
we use 
the same method to optimize the nonlinear parameters to get the lowest 
energy. For $\ell > 0$, as in the scattering problem, the wavefunction 
includes both sets of short-range terms for this optimization. While not 
necessarily the optimum choice, these optimized nonlinear parameters are used 
in the scattering problem. There is an additional nonlinear parameter $\mu$ 
in the scattering problem that would be impractical to simultaneously 
optimize with these nonlinear parameters due to computational costs. Once 
these nonlinear parameters are chosen, $\mu$ is chosen so that the phase 
shifts are the most positive. The choices of all parameters are given in 
Table \ref{tab:Nonlinear}.

\subsection{Born Approximation}

The Born approximation to the \emph{K} matrix \cite{Bransden2003} is given by
using the first term in Eqs.~(\ref{eq:TrialWave}) and
(\ref{eq:TrialWaveHigher}) in Eq.~\ref{eq:GenKohnL}. For the Kohn variational
method, this leads to the Born approximation given by
\begin{equation}
\label{eq:Born}
\tan\delta_\ell \approx -(\widetilde{S}_\ell,\mathcal{L}\widetilde{S}_\ell )\,.
\end{equation}
We also consider a modified Born approximation that uses both long-range terms
in Eqns. (\ref{eq:TrialWave}) and (\ref{eq:TrialWaveHigher}).

\subsection{Cross Sections}

We calculate the integrated and differential elastic cross sections using
\cite{Bransden2003}, respectively,
\begin{equation}
\label{eq:TotalCross}
\sigma_{el}^\pm = \frac{4}{\kappa^2} \sum_{\ell=0}^\infty (2\ell+1) \sin^2
   \delta_\ell^\pm
\end{equation}
and
\begin{align}
\label{eq:DiffCross}
\nonumber \frac{d\sigma_{el}^\pm}{d\Omega} = \frac{1}{\kappa^2} & \sum_{\ell=0}
  ^\infty \sum_{\ell^\prime=0}^\infty (2\ell+1)(2\ell^\prime+1) \exp
  \left\{\ii \left[\delta_\ell(\kappa) - \delta_{\ell^\prime}(\kappa) \right]
  \right\} \\
& \times \sin\delta_\ell^\pm(\kappa) \sin\delta_{\ell^\prime}^\pm(\kappa) P_\ell
  (\cos\theta) P_{\ell^\prime}(\cos\theta)\,.
\end{align}
The momentum transfer cross sections can be useful in plasma applications
\cite{Wang2014, McEachran2014}. These cross sections have been measured for Ps
with multiple atomic and molecular targets \cite{Nagashima1998,Saito2003}. The
momentum transfer cross sections are given by \cite{Bransden2003}
\begin{equation}
\label{eq:MomentumCross}
\sigma_{m}^\pm = \frac{4}{\kappa^2} \sum_{\ell=0}^\infty (\ell+1) \sin^2
  (\delta_\ell^\pm - \delta_{\ell+1}^\pm) .
\end{equation}
The singlet for each of the partial waves contributes $1/4$ to the integrated,
differential, and momentum transfer cross sections, while the triplet
contributes $3/4$. The ortho-para conversion cross sections give the conversion
of the projectile ortho-Ps to para-Ps by \cite{Hara1975}
\begin{equation}
\label{eq:OrthoParaCross}
\sigma_{c} = \frac{1}{4 \kappa^2} \sum_{\ell=0}^\infty (2 \ell+1) \sin^2
  (\delta_\ell^+ - \delta_\ell^-).
\end{equation}
Equations (\ref{eq:TotalCross}), (\ref{eq:MomentumCross}), and
(\ref{eq:OrthoParaCross}) are in units of $\pi a_0^2$, and
Eq. (\ref{eq:DiffCross}) is in units of $a_0^2 / \rm{sr}$. 


\subsection{Effective Range Theories}

The scattering length is defined as \cite{Bransden2003}
\begin{equation}
\label{eq:ScatLen}
a_\ell^\pm = -\lim_{\kappa \to 0}
  \frac{\tan{\delta_\ell^\pm}}{\kappa^{2\ell+1}}.
\end{equation}
We use the approximation with small $\kappa$ of
\begin{equation}
\label{eq:ScatLenApprox}
a_\ell^\pm \approx
  - \frac{\tan{\delta_\ell^\pm}}{\kappa^{2\ell+1}}.
\end{equation}
To avoid confusion with the Bohr radius, $a_0$, we define $a = a_{\ell=0}$. The
Kohn variational methods give an exact upper bound on the scattering length
\cite{Joachain1979}. The scattering length can give information about whether
there is a bound state in the system, and at zero energy \cite{Buckman1989},
\begin{equation}
\label{eq:ScatLenCross}
\sigma_m^\pm = \sigma_{el}^\pm = 4 (a^\pm)^2 .
\end{equation}

For short-range potentials, the effective range $r_0$ is given by
\cite{Bethe1949,Blatt1949}
\begin{equation}
\label{eq:EffectiveRangeShort}
\kappa \cot\delta_0^\pm = -\frac{1}{a^\pm} + \frac{1}{2} r_0^\pm \kappa^2.
\end{equation}
The effective range is a measure of the interaction region of the scattering
problem. This is the fitting used throughout the literature
\cite{Ivanov2002,VanReeth2003,Blackwood2002,Walters2004}. As pointed out in
Ref.~\cite{Fabrikant2014}, the van der Waals interaction should be taken into
account for low-energy Ps-atom scattering. An effective range theory (ERT)
expansion for the van der Waals interaction is given by \cite{Drake2006}
\begin{equation}
\label{eq:EffectiveRangeLongAu}
\kappa \cot\delta_0^\pm = -\frac{1}{a} + \frac{1}{2} r_0^\pm \kappa^2 - 
  \frac{4 \pi C}{15 a^2} \kappa^3 - 
  \frac{16 C}{15 a} \kappa^4 \ln \left(\kappa \right).
\end{equation}
We use the van der Waals coefficient, C, given by Martin and Fraser as
34.78473 au \cite{Martin1980}.

Gao has developed a quantum defect theory (QDT) treatment, solving the
Schr\"{o}dinger equation for an attractive $r^{-6}$ potential to find
an expression relating the phase shifts to a quantity he refers to as
$K_\ell^0(E_{\bm \kappa})$ \cite{Gao1998}.
\begin{equation}
\label{eq:GaoZEqn}
\tan\delta_\ell = [Z_{ff} - K_\ell^0(E_{\bm \kappa}) Z_{gf}]^{-1}
  [K_\ell^0(E_{\bm \kappa}) Z_{gg} - Z_{fg}]
\end{equation}
The Z functions in this equation are described fully in his paper. The phase
shifts are fitted to Eq. (\ref{eq:GaoZEqn}) to determine $K_\ell^0(E)$ for each
$\kappa$ value. $K_\ell^0(\epsilon)$ must be expanded in a Taylor series as
\begin{equation}
\label{eq:GaoKTaylor}
K_\ell^0(E_{\bm \kappa}) = K_\ell^0(0) + {K_\ell^0}^\prime(0) E_{\bm \kappa}
  + \ldots.
\end{equation}
From this, $K_\ell^0(0)$ and ${K_\ell^0}^\prime(0)$ are determined.
$K_\ell^0(0)$ slowly varies with the energy $E_{\bm \kappa}$, and
${K_\ell^0}^\prime(0)$ gives the quickly varying part. In Ref. \cite{Gao1998a},
Gao performed an expansion of this expression for low $\kappa$ to generate
expressions for $a_\ell$ and $r_0$:

\begin{equation}
\label{eq:GaoScatLenS}
a_{\ell=0} = \frac{2\pi}{[\Gamma(1/4)]^2} \frac{K_{\ell=0}^0(0) - 1}
  {K_{\ell=0}^0(0)} \beta_6
\end{equation}

\begin{equation}
\label{eq:GaoScatLenP}
a_{\ell=1} = -\frac{\pi}{18[\Gamma(1/4)]^2} \frac{K_{\ell=1}^0(0) + 1}
  {K_{\ell=1}^0(0)} \beta_6^3
\end{equation}

\begin{align}
\label{eq:GaoEffRange}
r_{\ell=0} = &\frac{[\Gamma(1/4)]^2}{3\pi} \frac{[K_{\ell=0}^0(0)]^2 + 1}
  {[K_{\ell=0}^0(0) - 1]^2} \beta_6 \nonumber \\
&+ \frac{[\Gamma(1/4)]^2}{\pi} \frac{{K_{\ell=0}^0}^\prime(0)^2(\hbar^2/2\mu)
  (1/\beta_6)^2}{[K_{\ell=0}^0(0) - 1]^2} \beta_6.
\end{align}
The term $\beta_6$ is related to $C$ by $\beta_6 = (2\mu C/\hbar^2)^{1/4}$.

Lastly, we use an expression from Ref. \cite{Blackwood2002} to get an estimate
of the $^1$S effective range, given by
\begin{equation}
\label{eq:BlackwoodERT}
r_0^+ = \frac{a^+ \sqrt{4 E_b} - 1}{2 a^+ E_b}.
\end{equation}
No equivalent expression exists for $r_0^-$, as there is not a $^3$S bound state.


\section{Numerics}
\label{sec:Numerical}

\subsection{Short-Short Integrations}
\label{sec:ShortInt}
For the PsH bound state and $^{1,3}S$ Ps(1s)-H(1s) elastic scattering, we use
the efficient asymptotic expansion method presented by Drake and Yan
\cite{Drake1995} for the evaluation of correlated integrals of the form
\begin{align}
\label{eq:ShortInt}
I(&j_1,j_2,j_3,j_{12},j_{23},j_{31}; \bar{\alpha}, \bar{\beta}, \bar{\gamma}) =
  \nonumber \\
&\int
d \textbf{r}_1 d \textbf{r}_2 d \textbf{r}_3
r_1^{j_1} r_2^{j_2} r_3^{j_3} r_{12}^{j_{12}}
r_{23}^{j_{23}} r_{31}^{j_{31}}
e^{-(\bar{\alpha} r_1 + \bar{\beta} r_2 + \bar{\gamma} r_3)}\, .
\end{align}
These integrals arise from evaluation of the matrix elements
$(\phi_i, L \phi_j)$, $(\phi_i, H \phi_j)$ and $(\phi_i, \phi_j)$, where $H$
is the full Hamiltonian given in Eqs. (\ref{eq:Hamiltonian1}) and
(\ref{eq:Hamiltonian2}). To verify our calculation of these integrals for the
S-wave and P-wave, we also use the recursion relations of Pachucki
\cite{Pachucki2004}.

For $\ell > 0$, we use two different methods to perform these integrations. 
The first is the method used by Van Reeth \cite{VanReethThesis}. In this 
method, we rotate and then integrate over external angles, reducing these 
integrals down to the form of Eq. (\ref{eq:ShortInt}). We then use the 
asymptotic expansion method \cite{Drake1995} to solve. This works through the 
D-wave, as the integrals become too singular for $\ell > 2$. For all partial 
waves, we can use a more general method from Yan and Drake \cite{Yan1997}. 
This C++ code for the second method is slower but has been used mainly for
$\ell > 2$, through the H-wave. These integrals have the form of
\begin{widetext}
\begin{align}
\label{eq:ShortIntGen}
\nonumber I(\ell_1^\prime m_1^\prime, \ell_2^\prime m_2^\prime, &\ell_3^\prime m_3^\prime; j_1,j_2,j_3,j_{12},j_{23},j_{31}; \bar{\alpha}, \bar{\beta}, \bar{\gamma}) = \int d \textit{\textbf{r}}_1 d \textit{\textbf{r}}_2 d \textit{\textbf{r}}_3
r_1^{j_1} r_2^{j_2} r_3^{j_3} r_{12}^{j_{12}}
r_{23}^{j_{23}} r_{31}^{j_{31}}
e^{-(\bar{\alpha} r_1 + \bar{\beta} r_2 + \bar{\gamma} r_3)} \\
& \times Y_{\ell_1^\prime m_1^\prime}^* (\textit{\textbf{r}}_1) Y_{\ell_2^\prime m_2^\prime}^* (\textit{\textbf{r}}_2) Y_{\ell_3^\prime m_3^\prime}^* (\textit{\textbf{r}}_3) Y_{\ell_1 m_1} (\textit{\textbf{r}}_1) Y_{\ell_2 m_2} (\textit{\textbf{r}}_2) Y_{\ell_3 m_3} (\textit{\textbf{r}}_3)\, .
\end{align}
\end{widetext}

\subsection{Long-Range Integrations}
\label{sec:LongInt}
We evaluate the long-range--long-range and short-range--long-range matrix 
elements in Eq. (\ref{eq:ComplexKohnMatrix}) using the standard Gauss-
Laguerre and Gauss-Legendre quadratures. Due to the cusp at $r_1 = r_2$ and
$r_2 = r_3$ in the integrands, we split the $r_2$ and $r_3$ integrations into 
Gauss-Legendre quadratures before the cusp and Gauss-Laguerre after the cusp. 
Previous calculations \cite{VanReeth2003,VanReeth2004} treated these cusps as 
unimportant by 25 au, while we have extended it to 100 au before we consider 
them unimportant. We find that this increases the convergence of the matrix 
elements.

To further improve the convergence of the short-range--long-range matrix 
elements, we note that the biggest source of difficulty in converging these 
comes from the Gaussian-Laguerre quadratures in the $r_1$, $r_2$ and $r_3$ 
integrations -- especially $r_1$. We increase the number of integration 
points to more than seven times as many in previous work
\cite{VanReeth2003,VanReeth2004} to better represent the integrands. This
brute force approach 
can increase the computational time greatly, so we take another approach to 
further increase the accuracy. Specifically, the tails of the integrands are 
negligible, and the integrand closer to the origin is not represented 
adequately. To resolve this, for each of the Gauss-Laguerre quadratures, we 
introduce an extra $e^{-\lambda r_i}$ and remove it with $e^{\lambda r_i}$ 
after the quadrature, bringing the abscissae closer to the origin without 
increasing the number of integration points. For further details on 
integrations, refer to Ref. \cite{Woods2015}.

\subsection{Todd's procedure}
\label{sec:Todd}
We use a method from Todd \cite{Todd2007} to remove short-range terms that 
contribute to linear dependence. This is a variation of the procedure from
L\"uchow and Kleindienst \cite{Luchow1992}. They use multiple blocks, while we 
optimize with a single block. They also use a criteria of $\Delta E$ to 
determine when to discard terms. Instead, we compare the lowest eigenvalues 
from the separate calculations using the upper and lower triangular matrices 
in LAPACK's \texttt{dsygv} routine \cite{LAPACK}, discarding terms when they 
cause the difference to be greater than a predetermined threshold. We refer 
here to the reordering introduced by this method as ``Todd ordering'', as 
compared to the original ordering indicated by increasing $\omega$. The 
summation limits in Eqs. (\ref{eq:TrialWave}) and (\ref{eq:TrialWaveHigher}) 
are then indicated by $N^\prime(\omega)$ instead of $N(\omega)$ to indicate 
that this is not the full set described after Eq. (\ref{eq:PhiDef}). When we 
do convergence checks via extrapolations, we put the terms back in the 
original ordering but with the set of $N^\prime(\omega)$ terms.

\subsection{Selection of Short-Range Terms}
\label{sec:Truncation}
We observe that the short-range terms used in the same order as the output of 
the Todd algorithm will generate better-converged phase shifts, and we can 
include more short-range terms before linear dependence becomes a problem. 
The phase shifts are calculated using this set of short-range terms for the 
generalized Kohn variational method for multiple $\tau$ values in
Eq.~(\ref{eq:GenKohn}). We further truncate this basis set where the phase
shifts for the different Kohn methods begin to diverge, as seen in
Fig. \ref{fig:swave-phase-divergence}. The results in Sec. \ref{sec:Results}
use the \emph{S} matrix complex Kohn with this set of short-range terms.

\begin{figure}[H]
	\centering
	\includegraphics[width=3.3in]{swave-phase-divergence}
	\caption{(Color online) Divergence of the $^1$S phase shifts with respect
to number of short-range terms}
	\label{fig:swave-phase-divergence}
\end{figure}

For $^1$S, we noticed an additional property of the phase shift graphs. When
$\delta_0^+$ is plotted for one Kohn method with respect to $N(\omega)$ for 
multiple $\mu$ values ($\mu$ defined in Eq. (\ref{eq:PartialWaveShielding})), 
the phase shifts agree well for the different $\mu$ values up until linear 
dependence becomes apparent visually. We use this criteria to determine where 
to truncate the basis set, stopping at 1505 terms for $^1S$. We did not see a 
similar effect for $^3$S or other partial waves.

\subsection{Fittings}
As with the previous Kohn calculation of Van Reeth and Humberston
\cite{VanReeth2004}, we fit our computed phase shifts near the resonances for
the S-wave and P-wave to the resonance formula
\begin{align}
\label{eq:ResonanceFit}
\delta(E_{\bm \kappa}) = A &+ B E_{\bm \kappa} + C E_{\bm \kappa}^2 + \arctan
  \left[ \frac{^1\Gamma}{2(^1E_R - E_{\bm \kappa})} \right]  \nonumber \\
& + \arctan \left[ \frac{^2\Gamma}{2(^2E_R - E_{\bm \kappa})} \right]
\end{align}
to extract out the positions ($^1E_R$ and $^2E_R$) and widths ($^1\Gamma$ and 
$^2\Gamma$) of the two resonances. This formula is comprised of the
Breit-Wigner resonance terms \cite{Breit1936,Macek1970} for the two resonances
and allows for a slowly varying polynomial background. The $^1$D and $^1$F 
resonance fits are performed without the second $\arctan$ term, as they have 
a single resonance. The data from each variant of the Kohn variational 
methods is fitted using the MATLAB \cite{MATLAB} nonlinear fitting routine 
\texttt{nlinfit} with all eight possible weightings. For each variant of the 
Kohn variational method, the four parameters are determined for each of the 
eight fits and compared.

We extrapolate the complex Kohn phase shifts in Tables
\ref{tab:SWaveSingletPhase}, \ref{tab:SWaveTripletPhase}, and
\ref{tab:SWaveSingletPhase} according to the empirical formula
\cite{VanReeth2003}
\begin{equation}
\label{eq:Extrap}
\tan\delta_\ell^\pm(\omega) = \tan\delta_\ell^\pm(\omega\to\infty) +
  \frac{c}{\omega^p}\, .
\end{equation}
We use these extrapolated values to estimate the convergence of our phase 
shifts and the error in our final results. A similar method is used to
extrapolate the scattering lengths by fitting to
\begin{equation}
\label{eq:ExtrapA}
a_\ell^\pm(\omega) = a_\ell^\pm(\omega\to\infty) + \frac{d}{\omega^q}\, .
\end{equation}
The difference between the scattering length and the extrapolated scattering 
length is considered the error in tables \ref{tab:SWaveScatLenERT} and
\ref{tab:PWaveScatLen}. No convergence pattern is seen for the effective range.

\subsection{Nonlinear Parameters and Terms Used}
\label{sec:Parameters}

\begin{table}[H]
  \centering
	\begin{ruledtabular}
    \begin{tabular}{cccccccccc}
    Param. & $^1$S & $^3$S & $^1$P & $^3$P & $^1$D & $^3$D & $^{1,3}$F & $^{1,3}$G & $^{1,3}$H \\
    \colrule
	$\omega$           & 7     & 7     & 7     & 7     & 6     & 6     & 5    & 5   & 5 \\
	$N^\prime(\omega)$ & 1505  & 1633  & 1000  & 1000  & 916   & 919   & 462  & 462 & 462 \\
	$\alpha$           & 0.586 & 0.323 & 0.397 & 0.310 & 0.359 & 0.356 & 0.5  & 0.5 & 0.5 \\
	$\beta$            & 0.580 & 0.334 & 0.376 & 0.311 & 0.368 & 0.365 & 0.6  & 0.6 & 0.6 \\
	$\gamma$           & 1.093 & 0.975 & 0.962 & 0.995 & 0.976 & 0.976 & 1.1  & 1.1 & 1.1 \\
	$\mu$              & 0.9   & 0.9   & 0.9   & 0.9   & 0.7   & 0.7   & 0.7  & 0.7 & 0.7 \\
	$m_\ell$           & 1     & 1     & 3     & 3     & 7     & 7     & 7    & 9   & 11 \\
    \end{tabular}
  \end{ruledtabular}
  \caption{Parameters for each partial wave}
  \label{tab:Nonlinear}
\end{table}

Table \ref{tab:Nonlinear} shows the number of terms, $N^\prime(\omega)$, used 
for each partial wave. This table also shows the parameters $\alpha$, $\beta$,
$\gamma$ in Eq.~(\ref{eq:PhiDef}) and the parameters $\mu$ and $m_\ell$ in 
Eq.~(\ref{eq:PartialWaveShielding}) used for each partial wave.


\section{Results}
\label{sec:Results}

\subsection{Bound State Results}

The PsH bound state calculation is used to determine the reliability of the 
short-range part of the wavefunction to describe Ps-H scattering at small 
distances. The Rayleigh-Ritz variational method provides a true bound on the 
total and binding energies, which converge well with respect to $\omega$. To 
be consistent, we report the results obtained with the same set of nonlinear 
parameters that we use in the scattering calculation, as given in
Table\ref{tab:Nonlinear}. We note that we are able to obtain a better value for
the binding energy with higher $\omega$, but we are unable to use this many
terms in the full scattering calculations for Ps-H. Table \ref{tab:BoundEnergy} 
compares our energies for PsH with that obtained by other groups.

Our calculation yields a better value for these quantities than the earlier 
variational calculation \cite{VanReeth2003,VanReeth2004} but not as good as 
the variational calculation of Yan and Ho \cite{Yan1999}, who also used 
Hylleraas wave functions. While we do not obtain the best value of the 
binding energy (this was not the purpose of our calculation), we obtain 
results for this quantity which compare favorably with the most elaborate 
calculation in the literature, which used 5000 ECGs \cite{Bubin2006}. Our 
calculation of the binding energy gives us some confidence in the reliability 
of the short-range part of the scattering wave function to describe the $^1S$ 
PsH system.

\squeezetable  % Makes the table smaller
\begin{table*}
\begin{center}
%\begin{tabular}{|l|l|c|l|l|}
\begin{ruledtabular}  % From http://www.latex-community.org/forum/viewtopic.php?f=45&t=20722
\begin{tabular}{l c l l}
%\toprule
Method & Terms & \thead{$E$ (au)} & \thead{$E_b$ (eV)}\\
%\hline
\colrule
%\midrule
Current work & 1505 & $-$0.789 189 725 & 1.066 406 705 \\
Variational Hylleraas $(\omega = 6)$ \cite{VanReeth2003} & 721 & $-$0.789 156 & 1.065 5$^\star$ \\
Variational Hylleraas \cite{Ho1986} & 396 & $-$0.788 945$^\star$ & 1.059 75 \\
Variational Hylleraas $(\omega \rightarrow \infty)$ \cite{Yan1999} & --- & $-$0.789 196 714 7$^\star$ & 1.066 596 896 \\
CC 14Ps14H \cite{Blackwood2002} & --- & $-$0.786 5 & 0.994$^\star$ \\
CC 14Ps14H + $\text{H}^-$ \cite{Walters2004} & --- & $-$0.787 9 & 1.03$^\star$\\
ECGs with SVM \cite{Mitroy2006} & 1800 & $-$0.789 196 740$^\star$ & 1.066 597 58 \\
ECGs variational \cite{Bubin2006} & 5000 & $-$0.789 196 765 251$^\star$ & 1.066 598 271 959 \\
%\bottomrule
\end{tabular}
\end{ruledtabular}
\caption{PsH energy comparisons. The values marked by an asterisk are the
reported values, and the other values are obtained by using the conversion
factor given in Ref. \cite{Mohr2012,*NISTConversions}.}
\label{tab:BoundEnergy}
\end{center}
\end{table*}

\subsection{Phase Shifts and Cross Sections}

In Tables \ref{tab:SWaveSingletPhase} and \ref{tab:SWaveTripletPhase}, we 
show the $^{1,3}$S phase shifts using the \emph{S} matrix complex Kohn 
variational method. To the accuracy given, the results agree from the various 
methods described in Sec. \ref{sec:Kohn}, after removing any obvious Schwartz 
singularities. We compute extrapolated values from $\omega = 4$ to
$\omega = 7$
to estimate the phase shifts as $\omega \rightarrow \infty$ using Eq.
(\ref{eq:Extrap}). By computing the percentage difference between the extrapolated 
phase shifts and the computed phase shifts at $\omega=7$, we estimate that 
the $^1$S phase shifts have converged to better than about $0.22\%$ for the 
range $\kappa=0.1$ to $0.7$ and that the $^3$S phase shifts have converged to 
better than $0.27\%$ for the same range of $\kappa$.

In these tables, we compare our results with the earlier Kohn variational 
results \cite{VanReeth2003,VanReeth2004} and with the elaborate CC results of 
Refs. \cite{Blackwood2002,Walters2004}. The current $\omega = 7$ results are 
in excellent agreement with the earlier Kohn $\omega = 6$ variational 
results, being either identical or slightly lower, indicating that the 
earlier S-wave calculations were well-converged. The slight difference in 
phase shifts between the previous and current Kohn methods can be attributed 
to two factors. Using Todd's procedure (described in Sec. \ref{sec:Todd}) 
allows us to use more terms (see Table \ref{tab:Nonlinear}) than the earlier 
Kohn calculations \cite{VanReeth2003,VanReeth2004}, which used 721 terms. 
This slightly increases the phase shifts, but we also use more integration 
points in these calculations, which has the effect of slightly decreasing the 
phase shifts.

The complex Kohn results are in good agreement with the CC results of the 
Walters group \cite{Blackwood2002,Walters2004}. For the singlet, the complex 
Kohn phase shifts are slightly larger than the CC results. Because of the 
empirical bounds on the Kohn variational results and, in practice on the CC
(except for the Buttle correction), the complex Kohn results could be slightly 
more accurate than the CC.
The situation is reversed for the triplet. In this case, the Kohn results are 
slightly more negative than the CC results; however, the complex Kohn and CC 
results are in very good agreement.

The recent CVM S-wave results from Zhang and Yan \cite{Zhang2012} agree 
extremely well with the Kohn results, even for the triplet. In Fig.
\ref{fig:swave-comparisons},
we compare the $^{1,3}$S phase shifts obtained from 
the complex Kohn variational method with various other calculations. Figure 
\ref{fig:spd-wave-phases}(a) compares the complex Kohn phase shifts over the 
energy range up to the Ps(n=2) threshold with the CC and CVM results. The 
inset in this figure shows the small discrepancy with the CC phase shifts, 
but excellent agreement between the three sets of results is evident. 

Tables \ref{tab:PWavePhase} and \ref{tab:DWavePhase} give the $^{1,3}$P and
$^{1,3}$D
 phase shifts that we determine using the complex Kohn variational 
method. The small percentage differences with the extrapolated values for the 
P-wave indicates that our phase shifts are well converged. Similar to $^{1,3}$
S, the $^1$P phase shifts are above the CC results, and our $^3$P phase 
shifts are slightly below. Figure \ref{fig:spd-wave-phases}(b) shows that the 
complex Kohn and CC results agree relatively well.

We have had difficulty performing extrapolations on the D-wave phase shifts. 
For $\kappa = 0.2 - 0.7$, the percentage difference between the $^1$D 
extrapolations and $\omega = 6$ results is less than $11\%$. The percentage 
differences for $^3$D for the same range are less than $23\%$, with the 
exception of $\kappa = 0.4$. This has a larger percentage difference of $48.7
\%$, and this is also where the $^3$D phase shifts go from positive to 
negative. It should be noted that the extrapolation is an empirical fit. The
$\omega = 5$ and $\omega = 6$ phase shifts differ by no more than 8.6\% for
$^1$D and 10.3\% for $^3$D. Both the $^1$D and $^3$D phase shifts are below the 
CC, though Fig. \ref{fig:spd-wave-phases}(c) shows that the complex Kohn and 
CC also agree reasonably well. The phase shifts are small for $^3$D, and its 
contribution to the total cross sections given in Fig.
\ref{fig:combined-cross-sections} is small.


\begin{table*}
\centering
\begin{ruledtabular}
\begin{tabular}{c c c c c c c}
$\kappa$ (au) & $\delta_0^+ (\omega = 7)$ & $\delta_0^+ (\omega \rightarrow \infty)$ & \% Diff$^+$ & $\delta_0^+$ (Kohn) \cite{VanReeth2003} & $\delta_0^+$ (CC 14Ps14H+H$^-$) \cite{Walters2004} & $\delta_0^+$ (CVM) \cite{Zhang2012} \\
\colrule
0.1 & $-0.427$ & $-0.426$ & $0.223\%$ & $-0.427$ & $-0.432$ & $-0.42636$ \\
0.2 & $-0.820$ & $-0.819$ & $0.010\%$ & $-0.820$ & $-0.833$ & $-0.81973$ \\
0.3 & $-1.161$ & $-1.161$ & $0.040\%$ & $-1.161$ & $-1.179$ & --- \\
0.4 & $-1.446$ & $-1.446$ & $0.022\%$ & $-1.446$ & $-1.466$ & --- \\
0.5 & $-1.678$ & $-1.677$ & $0.031\%$ & $-1.677$ & $-1.699$ & --- \\
0.6 & $-1.858$ & $-1.857$ & $0.040\%$ & $-1.857$ & $-1.884$ & --- \\
0.7 & $-1.964$ & $-1.963$ & $0.045\%$ & $-1.964$ & $-2.012$ & --- \\
\end{tabular}
\end{ruledtabular}
\caption{$^1$S phase shifts. \% Diff$^+$ is the percent difference between the
 current complex Kohn $\omega = 7$ and $\omega \rightarrow \infty$ results.}
\label{tab:SWaveSingletPhase}
\end{table*}

\begin{table*}
\centering
\begin{ruledtabular}
\begin{tabular}{c c c c c c c}
$\kappa$ (au) & $\delta_0^- (\omega = 7)$ & $\delta_0^- (\omega \rightarrow \infty)$ & \% Diff$^-$ & $\delta_0^-$ (Kohn) \cite{VanReeth2003} & $\delta_0^-$ (CC 14Ps14H) \cite{Blackwood2002} & $\delta_0^-$ (CVM) \cite{Zhang2012} \\
\colrule
0.1 & $-0.215$ & $-0.214$ & $0.120\%$ & $-0.214$ & $-0.206$ & $-0.21464$ \\
0.2 & $-0.431$ & $-0.431$ & $0.063\%$ & $-0.432$ & $-0.414$ & $-0.43159$ \\
0.3 & $-0.645$ & $-0.645$ & $0.094\%$ & $-0.645$ & $-0.624$ & --- \\
0.4 & $-0.850$ & $-0.849$ & $0.130\%$ & $-0.850$ & $-0.838$ & --- \\
0.5 & $-1.041$ & $-1.040$ & $0.166\%$ & $-1.040$ & $-1.037$ & --- \\
0.6 & $-1.217$ & $-1.214$ & $0.273\%$ & $-1.215$ & $-1.213$ & --- \\
0.7 & $-1.375$ & $-1.372$ & $0.250\%$ & $-1.373$ & $-1.367$ & --- \\
\end{tabular}
\end{ruledtabular}
\caption{$^3$S phase shifts. \% Diff$^-$ is the percent difference between the
current complex Kohn $\omega = 7$ and $\omega \rightarrow \infty$ results.}
\label{tab:SWaveTripletPhase}
\end{table*}

\begin{table*}
\begin{center}
\begin{ruledtabular}
\begin{tabular}{c c c c c c c c c c}
$\kappa$ (au) & $\delta_1^+ (\omega = 7)$ & $\delta_1^+ (\omega \rightarrow \infty)$ & \% Diff$^+$ & $\delta_1^+$ (CC) \cite{Walters2004} & $\delta_1^- (\omega = 7)$ & $\delta_1^- (\omega \rightarrow \infty)$ & \% Diff$^-$ & $\delta_1^-$ (CC) \cite{Blackwood2002} \\
\colrule
0.1 & $0.226^{-1}$ & $0.227^{-1}$ & $0.465\%$ & $0.221^{-1}$ & $-0.178^{-2}$ & $-0.172^{-2}$ & $3.176\%$ & $-0.953^{-3}$ \\
0.2 & $0.191$      & $0.192$      & $0.306\%$ & $0.183$      & $-0.167^{-1}$ & $-0.165^{-1}$ & $0.993\%$ & $-0.122^{-1}$ \\
0.3 & $0.609$      & $0.611$      & $0.314\%$ & $0.580$      & $-0.552^{-1}$ & $-0.540^{-1}$ & $0.749\%$ & $-0.456^{-1}$ \\
0.4 & $0.994$      & $0.996$      & $0.205\%$ & $0.956$      & $-0.115$      & $-0.114$      & $0.698\%$ & $-0.104$ \\
0.5 & $1.140$      & $1.142$      & $0.140\%$ & $1.106$      & $-0.183$      & $-0.182$      & $0.749\%$ & $-0.178$ \\
0.6 & $1.162$      & $1.163$      & $0.137\%$ & $1.134$      & $-0.248$      & $-0.246$      & $0.896\%$ & $-0.247$ \\
0.7 & $1.152$      & $1.154$      & $0.181\%$ & $1.133$      & $-0.292$      & $-0.281$      & $1.237\%$ & $-0.295$ \\
\end{tabular}
\end{ruledtabular}
\caption{$^{1,3}$P phase shifts. \% Diff$^\pm$ is the percent difference
between the current complex Kohn $\omega = 7$ and $\omega \rightarrow \infty$
results. Powers of 10 are denoted by exponents.}
\label{tab:PWavePhase}
\end{center}
\end{table*}


\begin{table*}
\begin{center}
\begin{ruledtabular}
\begin{tabular}{c c c c c}
$\kappa$ (au) & $\delta_2^+ (\omega = 6)$ & $\delta_2^+$ (CC 14Ps14H+H$^-$) \cite{Walters2004} & $\delta_2^- (\omega = 6)$ & $\delta_2^-$ (CC 14Ps14H) \cite{Blackwood2002} \\
\colrule
$0.1$ & $1.358^{-4}$ & $1.46^{-4}$ & $5.808^{-5}$ & $8.48^{-5}$ \\
$0.2$ & $2.987^{-3}$ & $3.15^{-3}$ & $7.120^{-4}$ & $1.15^{-3}$ \\
$0.3$ & $1.592^{-2}$ & $1.65^{-2}$ & $1.065^{-3}$ & $2.84^{-3}$ \\
$0.4$ & $4.933^{-2}$ & $4.95^{-2}$ & $-2.002^{-3}$ & $2.37^{-3}$ \\
$0.5$ & $1.113^{-1}$ & $1.08^{-1}$ & $-1.122^{-2}$ & $-4.66^{-3}$ \\
$0.6$ & $2.027^{-1}$ & $1.94^{-1}$ & $-2.647^{-2}$ & $-1.85^{-2}$ \\
$0.7$ & $3.215^{-1}$ & $3.02^{-1}$ & $-4.451^{-2}$ & $-3.27^{-2}$ \\
\end{tabular}
\end{ruledtabular}
\caption{$^{1,3}$D phase shifts. Powers of 10 are denoted by exponents.}
\label{tab:DWavePhase}
\end{center}
\end{table*}



% Caption idea from http://tex.stackexchange.com/questions/102925/how-can-i-insert-the-symbols-into-the-caption-of-a-figure
% Could also do similar caption to Laricchia's paper (doi:10.1088/1742-6596/194/1/012036)
\begin{figure}[H]
	\centering
	\includegraphics[width=3.3in]{swave-comparisons}
	\caption{(Color online) Comparison of $^1$S (a) and $^3$S (b) phase shifts
with results from other groups. Results are ordered according to year of
publication. This work -- solid curves;
\mbox{\textcolor{blue}{$\times$} -- CC \cite{Walters2004};}
\mbox{$\CIRCLE$ -- Kohn \cite{VanReeth2003};}
\mbox{\textcolor{red}{\textbf{+}} -- CC \cite{Blackwood2002};}
\mbox{$\blacktriangle$ -- DMC \cite{Chiesa2002};} 
\mbox{$\triangledown$ -- SVM 2002 \cite{Ivanov2002};} 
\mbox{$\Circle$ -- SVM 2001 \cite{Ivanov2001};} 
\mbox{\textcolor{red}{$\vartriangle$} -- 6-state CC \cite{Sinha2000};} 
\mbox{$\blacksquare$ -- 5-state CC \cite{Adhikari1999};} 
\mbox{$\square$ -- Coupled-pseudostate \cite{Campbell1998};} 
\mbox{$\vartriangle$ -- 3-state CC \cite{Sinha1997};} 
\mbox{\textcolor[RGB]{0,127,0}{$\bigstar$} -- CC \cite{Ray1997};} 
\mbox{$\triangleright$ -- Stabilization \cite{Drachman1976};} 
\mbox{\textcolor{red}{$\blacklozenge$} -- Stabilization \cite{Drachman1975};}
\mbox{\textcolor{blue}{$\lozenge$} -- Static-exchange \cite{Hara1975};}
\mbox{$\blacktriangledown$ -- Static-exchange \cite{Fraser1961}.}}
	\label{fig:swave-comparisons}
\end{figure}


\begin{figure}[H]
	\centering
	\includegraphics[width=3.3in]{spd-wave-phases}
	\caption{(Color online) Phase shifts for elastic Ps-H scattering: (a) S-
wave; (b) P-wave; (c) D-wave. Insets in (a) and (c) show a zoomed in view of 
the low energy regions. Current $^1$S and $^3$S are the solid blue (dark gray)
 and solid red (light gray), respectively. The $^1$S CC phase shifts
\cite{Walters2004} are given by \mbox{\textcolor{blue}{$\times$}}, and the
$^3$S CC phase shifts \cite{Blackwood2002} are given by
\mbox{\textcolor{red}{\textbf{+}}}. The CVM $^1$S and $^3$S phase shifts
\cite{Zhang2012} are blue (dark gray) and red (light gray) circles,
respectively. Vertical dashed lines denote the complex rotation resonance
positions \cite{Yan1999,Yan1998a,Ho1998}}
	\label{fig:spd-wave-phases}
\end{figure}

\begin{figure}[H]
	\centering
	\includegraphics[width=3.3in]{fgh-wave-phases}
	\caption{(Color online) Phase shifts for elastic Ps-H scattering:
(a) F-wave; (b) G-wave; (c) H-wave. Singlet phase shifts are given in blue
(dark gray), and triplet phase shifts are red (light gray).}
	\label{fig:fgh-wave-phases}
\end{figure}

Figure \ref{fig:fgh-wave-phases} shows the F-wave, G-wave and H-wave complex 
Kohn phase shifts compared to their Born approximations. For these three 
partial waves, the modified Born is approximately equivalent to the Born, so 
we show only the Born in these graphs. Even for these higher partial waves, 
the Born approximation is not particularly good, being much lower. The Born 
approximation does not even have the correct sign for the phase shifts for 
the G-wave and H-wave triplet. This comparison is not shown for the first 
three partial waves, as there is very little agreement. The e$^+$-H and
e$^+$-He Born results agree more closely with the Kohn calculations than
for the Ps-H system \cite{VanReeth2014}. The major difference is that for Ps
scattering, the correlations are more important, and the Born approximation
cannot describe them well. The F-wave has a resonance above the Ps(n=2)
threshold, but the beginning of the resonance is evident in Fig.
\ref{fig:fgh-wave-phases}(a).

We perform full complex Kohn calculations on all first six partial waves, but 
we do more extensive calculations for the first three partial waves, as shown 
by the parameters and terms used in Sec. \ref{sec:Parameters}. For the F-wave 
through the H-wave, the phase shifts and partial elastic cross sections 
become very small, so their overall contribution to the total elastic cross 
sections becomes negligible by the H-wave. The H-wave only contributes up to
$0.009\%$ of the integrated elastic cross sections near the threshold and much 
less at lower energies. For Fig. \ref{fig:combined-cross-sections}, we 
extract the CC data of Ref. \cite{Walters2004} using the CurveSnap program 
\cite{CurveSnap}.

\begin{figure}[H]
	\centering
	\includegraphics[width=3.3in]{combined-cross-sections}
	\caption{(Color online) Integrated elastic cross sections. CC data is from
Ref. \cite{Walters2004}.}
	\label{fig:combined-cross-sections}
\end{figure}

The integrated cross sections for the singlet and triplet come from the phase 
shifts using Eq. (\ref{eq:TotalCross}). For the spin-weighted complex Kohn in 
Fig. \ref{fig:combined-cross-sections}, the singlet contributes $1/4$ to the 
total, and the triplet contributes $3/4$. There is good comparison with the 
CC cross sections \cite{Walters2004} for much of the range, but there is a 
clear shift in the positions of the resonances, which can also be seen in 
Tables \ref{tab:SWaveResonances}, \ref{tab:PWaveResonances}, and
\ref{tab:DWaveResonances}.

The differential cross sections shown in Figs.
\ref{fig:diff-cross-section-2D-theta}, \ref{fig:diff-cross-section-2D-kappa},
and \ref{fig:combined-diff-cross-sections} require more partial waves, and
the contributions to these do 
not become negligible until the H-wave. The percent difference when adding 
the H-wave is approximately $4.1\%$ at its largest near the resonances and an 
average of $0.27\%$ throughout the full $E$ and $\theta$ range. As expected, 
Fig. \ref{fig:diff-cross-section-2D-kappa} shows that the cross section is 
isotropic at zero incident energy and becomes slightly more backward peaked 
as the energy is increased up to about \mbox{0.5 eV} ($\kappa = 0.27$). 
However, around this energy, there is an abrupt change in the differential 
cross section, becoming very strongly forward peaked, reaching a maximum 
around \mbox{1 eV} ($\kappa = 0.38)$ and staying nearly constant thereafter. 
Also of interest is the angular dependence of the resonances shown in Fig. 
\ref{fig:diff-cross-section-2D-theta}, for which we find that the main 
contribution is also forward peaked with some presence at large angles and 
little contribution at $\pi/2$.

\begin{figure}[H]
	\centering
	\includegraphics[width=3.3in]{diff-cross-section-2D-kappa}
	\caption{(Color online) Elastic differential cross sections for selected
incident Ps momenta}
	\label{fig:diff-cross-section-2D-kappa}
\end{figure}

\begin{figure}[H]
	\centering
	\includegraphics[width=3.3in]{diff-cross-section-2D-theta}
	\caption{(Color online) Elastic differential cross sections for selected
angles}
	\label{fig:diff-cross-section-2D-theta}
\end{figure}

\begin{figure}[H]
	\centering
	\includegraphics[width=3.3in]{combined-diff-cross-sections}
	\caption{(Color online) Elastic differential cross sections}
	\label{fig:combined-diff-cross-sections}
\end{figure}

\begin{figure}[H]
	\centering
	\includegraphics[width=3.3in]{cross-section-comparisons}
	\caption{(Color online) Comparison of cross sections. CC data is from
Ref.~\cite{Blackwood2002}. Static exchange data is from Ref.~\cite{Hara1975}.}
	\label{fig:cross-section-comparisons}
\end{figure}

From Eq. (\ref{eq:ScatLenCross}), $\sigma_m = \sigma_{el}$ for zero energy, 
and we see this in Fig. \ref{fig:cross-section-comparisons}. For very low 
energies ($E_{\bm \kappa} < 10^{-6}$ eV),
$\sigma_m = \sigma_{el} = 32.45$ $\pi a_0^2$.
The momentum transfer cross section differs from the integrated elastic cross
section past very low energies, and as 
noted in Ref. \cite{Blackwood2002c}, this is due to the differential elastic 
cross section only being isotropic at very low energy
(see Fig. \ref{fig:diff-cross-section-2D-kappa}).

The ortho-para conversion cross section, $\sigma_c$, is much lower than the 
elastic integrated and momentum transfer cross sections. Similar to the 
elastic integrated cross sections, the CC $\sigma_c$ curve \cite{Blackwood2002}
is approximately the same as ours, but the resonances are clearly shifted. 
The static exchange $\sigma_m$ curve \cite{Hara1975} does not agree well with 
the complex Kohn results, but the static exchange is an earlier calculation 
that does not predict resonances.

\subsection{Resonances}
\label{sec:Resonances}
The $^1$S and $^1$P partial waves have two resonances each before the Ps(n=2) 
threshold, and $^1$D has one resonance before. There is a resonance just 
after the threshold for $^1$F, with the onset of the resonance before the 
threshold. We fit the phase shifts in the resonance region to
Eq.~(\ref{eq:ResonanceFit}) for $^1$S and $^1$P. We perform the $^1$D and $^1$F
resonance fits without the second arctan term, as they have a single resonance.

In Tables \ref{tab:SWaveResonances}, \ref{tab:PWaveResonances},
\ref{tab:DWaveResonances}, and \ref{tab:FWaveResonances},
we compare the complex Kohn resonance parameters with results from other
calculations. The first line in each of these tables has the average of the
resonance parameters due to all variants of the Kohn variational method after
Schwartz singularities are removed, and the errors are given by the standard 
deviation. The second line has the $S$ matrix complex Kohn results.

These Rydberg resonances correspond to the quasibound state of e$^+$ with the 
H$^-$ ion \cite{Drachman1979}. Figure \ref{fig:spd-wave-phases}(a) shows the 
two $^1$S Rydberg resonances below the Ps(n=2) channel. The first resonance 
is associated with the $2s$ state \cite{DiRienzi2002b} and was first 
calculated by Hazi and Taylor using a stabilization method \cite{Hazi1970}. 
They have been computed very accurately by Yan and Ho using the complex 
rotation method \cite{Yan1999} and by the Walters' group using the CC 
approach \cite{Walters2004}. The first $^1$P resonance is associated with the 
$3p$ state, not the $2p$ state \cite{DiRienzi2002b}, while the $^1D$ 
resonance corresponds with the $3d$ state \cite{DiRienzi2002a}.

Excellent agreement between the parameters is achieved between the two sets 
of Kohn calculations (present and earlier). Good agreement is achieved 
between the resonance positions obtained with the complex rotation 
calculations of Yan and Ho \cite{Yan1999,Yan1998a,Ho1998,Ho2000} and our 
work. There is less agreement with the resonance parameters for the narrow 
second resonances in the $^1$S and $^1$P calculation.

The CC results, 9HPsPs \cite{Blackwood2002} and 9H9Ps+H$^-$ \cite{Walters2004},
are comparable to the Kohn and complex rotation calculations. This 
comparison confirms the importance of the H$^-$ channel in bringing the 
position of the first resonance, $^1E_R$, into better agreement with the Kohn 
and complex rotation calculations.

We observe no resonances in the triplet for any of these partial waves, which 
is consistent with the discussion by Blackwood et al. \cite{Blackwood2002}, 
who predicted that there should be no resonances for the triplet. Ray
\cite{Ray2006} obtained a triplet resonance in a 3-state CC approximation, but
we see no evidence of this resonance.

\begin{table*}
\begin{center}
\begin{ruledtabular}
\begin{tabular}{l l l l l}
Method & \thead{$^1E_R$ (eV)} & \thead{$^1\Gamma$ (eV)} & \thead{$^2E_R$ (eV)} & \thead{$^2\Gamma$ (eV)} \\
\colrule
Current work & $4.0065 \pm 0.0001$ & $0.0955 \pm 0.0001$ & $5.0272 \pm 0.0029$ & $0.0608 \pm 0.0007$ \\
Current work \emph{S} matrix complex Kohn & $4.0065$ & $0.0955$ & $5.0277$ & $0.0607$ \\
Complex rotation (Yan and Ho 1999) \cite{Yan1999} & $4.0058 \pm 0.0005$ & $0.0952 \pm 0.0011$ & $4.9479 \pm 0.0014$ & $0.0585 \pm 0.0027$ \\
Stabilization (Yan and Ho 2003) \cite{Yan2003} & $4.007$ & $0.0969$ & $4.953$ & $0.0574$ \\
Kohn variational (Van Reeth and Humberston 2004) \cite{VanReeth2004} & $4.0072 \pm 0.0020$ & $0.0956 \pm 0.010$ & $5.0267 \pm 0.0020$ & $0.0597 \pm 0.0010$ \\
CC (Walters et al. 2004) \cite{Walters2004} & $4.149$ & $0.103$ & $4.877$ & $0.0164$ \\
\end{tabular}
\end{ruledtabular}
\caption{$^1$S resonance parameters} % title of Table
\label{tab:SWaveResonances}
\end{center}
\end{table*}


\begin{table*}
\begin{center}
\begin{ruledtabular}
\begin{tabular}{l l l l l}
Method & \thead{$^1E_R$ (eV)} & \thead{$^1\Gamma$ (eV)} & \thead{$^2E_R$ (eV)} & \thead{$^2\Gamma$ (eV)} \\
\colrule
Current work & $4.2856 \pm 0.0001$ & $0.0445 \pm 0.0001$ & $5.0577 \pm 0.0004$ & $0.0459 \pm 0.0005$ \\
Current work \emph{S} matrix complex Kohn & $4.2856$ & $0.0445$ & $5.0579$ & $0.0459$ \\
Complex rotation (Yan and Ho 1998) \cite{Yan1998a} & $4.2850 \pm 0.0014$ & $0.0435 \pm 0.0027$ & $5.0540 \pm 0.0027$ & $0.0925 \pm 0.0054$ \\
Stabilization (Yan and Ho 2003) \cite{Yan2003} & $4.287$ & $0.0446$ & $5.062$ & $0.0563$ \\
CC (Walters et al. 2004 \cite{Walters2004}) & $4.475$ & $0.0827$ & $4.905$ & $0.0043$ \\
Kohn (Van Reeth and Humberston 2004) \cite{VanReeth2004} & $4.29 \pm 0.01$ & $0.042 \pm 0.005$ & --- & --- \\
\end{tabular}
\end{ruledtabular}
\caption{$^1$P resonance parameters} % title of Table
\label{tab:PWaveResonances}
\end{center}
\end{table*}


\begin{table}[H]
\begin{center}
\begin{ruledtabular}
\begin{tabular}{l l l}
Method & \thead{$^1E_R$ (eV)} & \thead{$^1\Gamma$ (eV)} \\
\colrule
Current work & $4.7188 \pm 0.0003$ & $0.0858 \pm 0.0005$ \\
Current work \emph{S} matrix complex Kohn & $4.7186$ & $0.0864$ \\
Complex rotation (Ho and Yan 1998) \cite{Ho1998} & $4.710 \pm 0.0027$ & $0.0925 \pm 0.0054$  \\
Stabilization (Yan and Ho 2003) \cite{Yan2003} & $4.714$ & $0.0969$ \\
CC (Walters et al. 2004 \cite{Walters2004}) & $4.899$ & $0.0872$ \\
\end{tabular}
\end{ruledtabular}
\caption{$^1$D resonance parameters} % title of Table
\label{tab:DWaveResonances}
\end{center}
\end{table}


\begin{table}[H]
\begin{center}
\begin{ruledtabular}
\begin{tabular}{l l l}
Method & $^1E_R \text{ (eV)}$ & $^1\Gamma \text{ (eV)}$ \\
\colrule
Current work & $5.1867 \pm 0.0021$ & $0.0125 \pm 0.0003$ \\
Current work \emph{S} matrix complex Kohn & $5.1863$ & $0.0125$ \\
Complex rotation (Ho and Yan 2000) \cite{Ho2000} & $5.1661 \pm 0.0014$ & $0.0174 \pm 0.0027$  \\
CC (Walters et al. 2004 \cite{Walters2004}) & $5.200$ & $0.0095$ \\
\end{tabular}
\end{ruledtabular}
\caption{$^1$F resonance parameters} % title of Table
\label{tab:FWaveResonances}
\end{center}
\end{table}



\section{Effective Range Theory}

Similar to the work by Van Reeth and Humberston \cite{VanReeth2003}, we use 
Eq. (\ref{eq:ScatLenApprox}) to find the $^{1,3}$S scattering lengths for small 
values of $\kappa$, given in Table \ref{tab:SWaveScatLenERT}. We see that to 
the accuracy reported, the $^{1,3}$S scattering lengths converge well with 
respect to $\kappa$.

Prior work in the literature for Ps-H scattering
\cite{Blackwood2002,Ivanov2002,Walters2004,VanReeth2003}
uses the ERT expansion for short-range 
interactions in Eq. (\ref{eq:EffectiveRangeShort}). Table
\ref{tab:SWaveScatLenERT} shows the S-wave scattering lengths and effective
ranges from prior Kohn variational \cite{VanReeth2003} and CC
\cite{Blackwood2002,Walters2004} calculations. Some other recent elaborate
calculations of the scattering lengths and effective ranges can be found in
Refs.~\cite{Sinha2000,Ivanov2001,Chiesa2002,Ivanov2002}.
These all agree relatively well with each 
other. Additional calculations of S-wave scattering lengths and effective 
ranges can also be found in
Refs.~\cite{Hara1975,Page1976,Drachman1975,
Drachman1976,Campbell1998,Adhikari1999,Adhikari2001b}.

Reference \cite{Blackwood2002} on the CC calculation does not give a specific 
range where they did the fitting for the effective range, but it appears to 
be a similar range as that of the Kohn variational calculations. Van Reeth 
and Humberston \cite{VanReeth2003} do the fitting for the $\kappa = 0-0.5$ 
range for both the singlet and triplet. In the third line of Table
\ref{tab:SWaveScatLenERT}, we use a similar $\kappa$ range and reproduce nearly
their same results but with our $\omega = 7$ values. Figure
\ref{fig:swave-ERT-short} has our updated version 
of the graph presented in Figure 5 of their paper. They note
that the singlet forms a relatively straight line, but the 
triplet curves down at low energies. We see this same behavior and 
investigate it further.

When we perform the fitting only at very small momenta
($\kappa = 0.001 - 0.009$ in
Table \ref{tab:SWaveScatLenERT}) and compare to the higher momentum
($\kappa = 0.1 - 0.5$), we see that the singlet results stay nearly the same. 
However, for the triplet, the effective range changes significantly. This 
corresponds with the triplet curving downward in
Fig. \ref{fig:swave-ERT-short}.
The question as to whether this is a real feature or numerical inaccuracy 
at low momenta can be answered by looking at the convergence of the phase 
shifts at these $\kappa$ values. As Blackwood et al. note \cite{Blackwood2002}
, numerical accuracy is difficult to obtain at very small $\kappa$. We 
estimate that for $\kappa = 0.001 - 0.009$, based on the extrapolations, the
$^1$S and $^3$S complex Kohn variational phase shifts converge to within
$0.28\%$ and $0.37\%$ of their correct values, respectively.

The previous Kohn calculations \cite{VanReeth2003} were performed using
Eq.~(\ref{eq:EffectiveRangeShort}), which works well for short-range
interactions. Equation (\ref{eq:EffectiveRangeLongAu}) includes terms due
to the van der Waals interaction. Van Reeth and Humberston \cite{VanReeth2003}
tried this expansion out to the $\kappa^3$ term but did not notice a
difference from Eq. (\ref{eq:EffectiveRangeShort}) for either $^1$S or $^3$S.
In Table \ref{tab:SWaveScatLenERT}, we see that adding the extra van der
Waals terms does not change the scattering length and only changes the
effective range slightly. We also tried the Hinckelmann and Spruch
expression \cite{Hinckelmann1971}, but we obtained unsatisfactory results,
since $d$ is not known and was chosen as a free parameter in our system.

We also investigate this low $\kappa$ range using the quantum-defect theory 
of Gao \cite{Gao1998} in Eqs. (\ref{eq:GaoScatLenS}) and
(\ref{eq:GaoEffRange}), which give scattering lengths and effective ranges
that agree relatively 
well with the long-range van der Waals expression. The scattering lengths 
obtained with the QDT expansion agree with results using the other ERTs and 
scattering length definition. However, the effective ranges obtained with the 
QDT expansion do not agree with the other ERTs.
%The QDT expansion does not agree well for the effective ranges but gets the 
scattering lengths correct. This uses only the slowly varying first term in 
Eq. (\ref{eq:GaoKTaylor}). 

\begin{figure}[H]
	\centering
	\includegraphics[width=3.3in]{swave-ERT-short}
	\caption{(Color online) $^1$S and $^3$S phase shifts, plotted as
$\kappa \cot \delta_0^\pm$ versus $\kappa^2$. The inset shows a magnified
portion of the same data as denoted by the gray box in the lower left.}
	\label{fig:swave-ERT-short}
\end{figure}

\begin{table*}
\begin{center}
\begin{ruledtabular}
\begin{tabular}{l c c c c c}
Model & $\kappa$ & $a^+$ & $r_0^+$ & $a^-$ & $r_0^-$ \\
\colrule
Approximation to def. - Eq. (\ref{eq:ScatLenApprox}) & $0.001$ & $4.331 \pm 0.012$ & --- & $2.137 \pm 0.008$ & --- \\
ERT Short - Eq. (\ref{eq:EffectiveRangeShort}) & $0.001 - 0.009$ & $4.331 \pm 0.012$ & 2.197 & $2.137 \pm 0.008$ & 2.035 \\
ERT Short - Eq. (\ref{eq:EffectiveRangeShort}) & $0.1 - 0.5$ & $4.308 \pm 0.003$ & 2.275 & $2.162 \pm 0.003$ & 1.343 \\
ERT VDW - Eq. (\ref{eq:EffectiveRangeLongAu}) & $0.001 - 0.009$ & $4.331 \pm 0.012$ & 2.221 & $2.137 \pm 0.008$ & 2.137 \\
QDT - Eqs. (\ref{eq:GaoZEqn}), (\ref{eq:GaoKTaylor}) & $0.002, 0.003$ & $4.331 \pm 0.012$ & 2.210 & $2.137 \pm 0.008$ & 2.151 \\
QDT expansion - Eqs. (\ref{eq:GaoScatLenS}), (\ref{eq:GaoEffRange}) & $0.001$ & $4.331 \pm 0.012$ & 2.535 & $2.137 \pm 0.008$ & 3.085 \\
Eq. (\ref{eq:BlackwoodERT}) & --- & --- & 2.106 & --- & --- \\
\colrule
Kohn 721 terms \cite{VanReeth2003} & $0 - 0.5$ & 4.334 & \,\,--- & 2.143 & \,\,--- \\
Kohn extrapolated \cite{VanReeth2003} & $0 - 0.5$ & 4.311 & 2.27 & 2.126 & 1.39 \\
CC 14Ps14H \cite{Blackwood2002} & --- & 4.41 & 2.19 & 2.06 & 1.47 \\
CC 14Ps14H+H$^-$ \cite{Walters2004} & --- & 4.327 & \,\,--- & \,\,--- & \,\,--- \\
SVM \cite{Ivanov2002} & --- & 4.34 & 2.39 & 2.22 & 1.29 \\
\end{tabular}
\end{ruledtabular}
\caption{$^{1,3}$S scattering lengths and effective ranges}
\label{tab:SWaveScatLenERT}
\end{center}
\end{table*}

From Eq. (\ref{eq:ScatLenCross}), we can compare the scattering lengths to 
the cross sections at zero energy (taken as $E_{\bm \kappa} = 10^{-7}$ eV
 here) with (in units of $\pi a_0^2$)
\begin{subequations}
\label{eq:CrossScatLen}
\begin{align}
\sigma_{el}^+ = \sigma_m^+ = 75.03 & \approx 4 (a^+)^2 = 75.02 \\
\sigma_{el}^- = \sigma_m^- = 18.27 & \approx 4 (a^-)^2 = 18.26 \, .
\end{align}
\end{subequations}
This gives us some confidence in the accuracy of our calculations at low 
energies, particularly for the S-wave, which is the dominant partial wave at 
low energies.

The P-wave does not have effective ranges but does have scattering lengths 
\cite{Levy1963}. We are unable to use the very low $\kappa = 0.0001$ that we 
use for the S-wave due to numerical inaccuracies. Table \ref{tab:PWaveScatLen}
 gives the scattering lengths using the definition and quantum defect theory 
expressions. The $^1$P scattering lengths from the different methods agree 
relatively well, but there is less agreement for the $^3$P scattering 
lengths. The SVM \cite{Ivanov2002} $^1$P scattering length is comparable to 
the complex Kohn values, but the $^3$P differs significantly. This appears 
mainly due to SVM $^3$P phase shifts being much lower than the complex Kohn
(and the CC of Ref. \cite{Blackwood2002}).

\begin{table}[H]
\begin{center}
\begin{ruledtabular}
\begin{tabular}{l l c c}
Model & $\kappa$ & $a_1^+$ & $a_1^-$ \\
\colrule
Approximation to def. - Eq. (\ref{eq:ScatLenApprox}) & 0.01 & $-22.130 \pm 0.173$ & $1.4530 \pm 0.1104$ \\
QDT - Eq. (\ref{eq:GaoZEqn}) & 0.01, 0.02 & $-22.200 \pm 0.173$ & $1.4158 \pm 0.1107$ \\
QDT expansion - Eq. (\ref{eq:GaoScatLenP}) & 0.01 & $-22.198 \pm 0.172$ & $1.4102 \pm 0.1104$ \\
\colrule
SVM \cite{Ivanov2002} & --- & $-20.7$ & $6.80$ 
\end{tabular}
\end{ruledtabular}
\caption{$^{1,3}$P scattering lengths}
\label{tab:PWaveScatLen}
\end{center}
\end{table}


\section{Conclusion}

We present complex Kohn variational results for Ps(1s) scattering from H(1s) 
below the Ps(n=2) threshold. We extend the earlier Kohn variational 
calculations \cite{VanReeth2003,VanReeth2004} with a larger basis set, which 
is made possible by an increase in the number of integration points and, for
$^{1,3}D$, the introduction and subsequent removal of the $e^{-\lambda r_i}$
term to the Gauss-Laguerre quadratures. These give better converged phase 
shifts, and we also use the $S$ matrix complex Kohn variational method to get 
more stable phase shifts.
%We describe a method 
%to identify short-range terms that contribute to linear dependence, allowing 
%us to use a larger basis set \cite{Todd2007}.

The $^{1,3}$S and $^{1,3}$P phase shifts can be considered benchmark results. The 
discrepancy in the $^{1,3}$D phase shifts between the complex Kohn variational 
and CC methods needs further investigation, especially for $^3$D. The partial 
waves with $\ell > 2$ have very small phase shifts and do not contribute 
greatly to the overall cross sections. We present the elastic differential, elastic 
integrated, momentum transfer and ortho-para conversion cross sections using 
the complex Kohn variational phase shifts for
the first six partial waves.

We calculate resonance positions and widths for the $^1$S, $^1$P, $^1$D, and
$^1$F partial waves, which favorably compare to the accurate complex 
rotation results of Refs. \cite{Yan1999,Yan1998a,Ho1998,Ho2000}. We also 
provide a detailed investigation of the effective ranges and scattering 
lengths for $^{1,3}$S, along with the $^{1,3}$P scattering lengths. We 
present results using an approximation of the definition of the scattering
length, a short-range 
effective range expansion, an expansion incorporating the van der Waals 
interaction, and a quantum defect theory model and its expansion.
The $^{1,3}$S scattering 
lengths agree well with previous work and are likely the most accurate to 
date. The $^3$S effective range we calculate is higher than previously 
reported. The $^1$P scattering length agrees with the SVM \cite{Ivanov2002}, 
but the complex Kohn $^3$P scattering length is much smaller.



\begin{acknowledgments}
We wish to thank Drs. Y.~K.~Ho, J.~W.~Humberston, K.~Pachucki and Z.~C.~Yan 
for discussions. S.~J.~W. acknowledges support from NSF under grant no.
PHYS-0968638 and from UNT through the UNT faculty research grant GA9150. 
Computational resources were provided by UNT's High Performance Computing 
Services, a project of Academic Computing and User Services division of the 
University Information Technology with additional support from UNT Office of 
Research and Economic Development.
\end{acknowledgments}


\bibliography{PRA-PsH-Draft}



\end{document}

%
% ****** End of file apssamp.tex ******

