% -*- root: Dissertation.tex -*-
\documentclass[Dissertation.tex]{subfiles} 
\begin{document}

\clearpage
\pagebreak
\newpage

\chapter{General Ps-H Formalism}
\label{chp:General}


%\lettrine{\textcolor{startcolor}{F}}{or} the first two partial waves of the
%Ps-H system, I was able to compare my code and results, including individual 
%matrix elements, with code and results from Van Reeth \cite{VanReethPrivate}.
%As there was no available code to compare against for the D-
%wave, I took steps to write a general code to compare against. The long-range
%code, covered in \cref{sec:GeneralLong}, is a straightforward 
%generalization from the previous codes. The short-range code, discussed in 
%\cref{sec:GeneralShort}, uses a different formalism, enabling us to 
%compare two different methods for the first three partial waves.

\iftoggle{UNT}{To}{\lettrine{\textcolor{startcolor}{T}}{o}}
extend this work to higher partial 
waves, we could have continued as before, performing derivations and writing 
new code for each partial wave. This takes time and is error prone, so during
the course of writing the D-wave code, I investigated creating a general
formalism that works for arbitrary $\ell$. The long-range
code, covered in \cref{sec:GeneralLong}, is a straightforward 
generalization from the previous codes. The short-range code, discussed in 
\cref{sec:GeneralShort}, uses the Laplacian formalism instead of the
gradient-gradient that we used in the S-, P-, and D-wave short-short codes,
which also enabled us to compare the two different methods for the first three
partial waves.



\section{General Long-Range Matrix Elements}
\label{sec:GeneralLong}

If the mixed symmetry terms in \cref{sec:MixedTerms} are not included for the
D-wave, the P-wave and D-wave long-range codebases are very similar. The S-wave 
is not much different as well, but it only has a single symmetry. We also
only treat the first two symmetries for $\ell \geq 2$.

\subsection{Long-Long Matrix Elements}
\label{sec:GeneralLongLong}

One major difference for similar matrix elements is the angular integrations. 
As shown in \cref{chp:AngularInt}, the external angular integrations 
for each partial wave have different results, due to the different spherical 
harmonics. $\widetilde{S}_\ell$ and $\widetilde{C}_\ell$ are also of a similar form between the 
partial waves, as seen in \cref{eq:TildeSCDef,eq:SCBarDef}.
Other than the spherical harmonics, the spherical Bessel functions are 
different, as is the shielding function for $\widetilde{C}_\ell$. The spherical Bessel 
functions are easily called through the GNU Scientific Library \cite{GSL} for 
any $\ell$-value. The power of the shielding function varies as
$m_\ell \geq (2\ell + 1)$. All of this allows us to easily generalize the long-long 
integrations.

%From \cref{eq:PartialWaveShielding}, the general shielding function for $\widetilde{C}$ to keep it regular at the origin is given by
%\begin{equation}
  %%\label{eq:PartialWaveShielding}
  %f_\ell(\rho) = \left[1 - \ee^{-\mu \rho} \left(1+\frac{\mu}{2}\rho\right)
  %\right]^{m_\ell}.
%\end{equation}
%From \cref{}, the derivatives $f_\ell^\prime(\rho)$ and $f_\ell^{\prime\prime}(\rho)$ are needed. In the \emph{Mathematica} notebook ``General Shielding Function.nb'', I found out that the derivatives can be written generally as
%\beq
%\label{eq:Shielding1Der}
%f_\ell^\prime(\rho) = -\frac{\mu m_\ell (\mu  \rho +1) \left[1-\frac{1}{2} e^{-\mu  \rho } (\mu  \rho +2)\right]^{m_\ell}}{\mu  \rho -2 e^{\mu  \rho }+2}
%\eeq
%and
%\beq
%\label{eq:Shielding1Der}
%f_\ell^{\prime\prime}(\rho) = \frac{\mu^2 m_\ell \left[-2 \mu  \rho  e^{\mu  \rho }+ m_\ell (\mu  \rho +1)^2-1\right] \left[1-\frac{1}{2} e^{-\mu  \rho } (\mu  \rho +2)\right]^{m_\ell}} {\left(\mu  \rho -2 e^{\mu  \rho }+2\right)^2}.
%\eeq


\subsection{Short-Long Matrix Elements}
\label{sec:GeneralShortLong}
The short-range functions are also easily generalizable. For the P-wave and
D-wave, these are seen in \cref{eq:PWavePhiBar,eq:DWavePhiBar}. The $\phi_i$
and $\phi_j$ parts for each are the same. This fact can only 
be easily used for the short-long calculations and not the short-short 
calculations, due to the action of the $\mathcal{L}$-operator on these terms. The 
formalism we have used for the short-long terms at every stage only operates
$\mathcal{L}$ on the long-range parts. \Cref{sec:GeneralShort} covers 
the calculations of the short-short terms using a general formalism.

%\subsection{Implementation}
%\label{sec:GeneralLongImp}
%\todoi{Do we need to worry about implementation discussion? Maybe in an appendix if needed.}


\section{General Short-Range--Short-Range}
\label{sec:GeneralShort}

The short-short integrals are more complicated due to the 
gradient-gradient operator acting on the short-range terms (see \cref{eq:BoundGradient}).
As an example of the nature of these, refer to \cref{sec:DWaveShortShort}.
The derivations were previously performed by rotating 
the coordinate system and integrating over external angles first, as 
described in \cref{chp:AngularInt}. This approach works, but it 
cannot be extended to higher partial waves without completing a full 
derivation for each partial wave. %This also introduces terms that may be too 
%singular for higher partial waves.
Drake and Yan \cite{Yan1997} expand upon 
their previous work for the three-electron Hylleraas integrals to include the 
spherical harmonics using the Laplacian formalism instead, which still
has a complicated form but works for arbitrary $\ell$.
It is this approach that we take here. All equations in this
section have been rederived for the Ps-H system in my notes \cite{Wiki,figshare}.
The work of Harris \cite{Harris2005a} may also be directly
applicable to this problem, but this has not yet been explored.


\subsection{Hamiltonian}
\label{sec:GenShortHam}
As we have for the bound state problem, the Hamiltonian is given compactly
as in \cref{eq:BoundHamiltonian} by
\beq
\label{eq:GenHam1}
H = \sum_{i=1}^3 \left(-\frac{1}{2} \nabla_i^2 - \frac{1}{r_i} \right) + \sum_{i>j}^3 \frac{1}{r_{ij}}.
\eeq
As in Ref.~\cite{Yan1997}, when the Laplacians are expanded, this is given by
the form
\begin{align}
\label{eq:GenHam2}
\nonumber H = -\frac{1}{2} & \left[ \sum_{i=1}^3 \left( \frac{\partial^2}{\partial r_i^2} + \frac{2}{r_i} \frac{\partial}{\partial r_i} - \frac{\ell(\ell+1)}{r_i^2} \right) + \sum_{i>j}^3 \left( 2 \frac{\partial^2}{\partial r_i^2} + \frac{4}{r_{ij}} \frac{\partial}{\partial r_i} \right) + \sum_{i \neq j}^3 \left(\frac{r_i^2 - r_j^2 + r_{12}^2}{r_i r_j} \right) \right. \\
\nonumber &  \left. + \frac{r_{12}^2 + r_{13}^2 - r_{23}^2}{r_{12} r_{13}} \frac{\partial^2}{\partial r_{12} \partial r_{13}} + \frac{r_{12}^2 + r_{23}^2 - r_{13}^2}{r_{12} r_{23}} \frac{\partial^2}{\partial r_{12} \partial r_{23}} + \frac{r_{13}^2 + r_{23}^2 - r_{12}^2}{r_{13} r_{23}} \frac{\partial^2}{\partial r_{13} \partial r_{23}} \right. \\
& \left. + \sum_{i>j}^3 \frac{1}{r_{ij}} \frac{r_i}{r_j} \frac{\partial}{\partial r_{ij}} \left( \hat{\boldsymbol{r}_i} \cdot \hat{\nabla}_j^Y \right) + \sum_{i>j}^3 \frac{1}{r_{ji}} \frac{r_j}{r_i} \frac{\partial}{\partial r_{ji}} \left( \hat{\boldsymbol{r}_j} \cdot \hat{\nabla}_i^Y \right) \right].
\end{align}
% & \left. + \sum_{i>j}^3 \frac{1}{r_{ij}} \frac{r_i}{r_j} \frac{\partial}{\partial r_{ij}} \left( r_i^\hat \cdot \nabla_j^Y \right) \right]

\noindent This is the form when the mass polarization terms are unimportant
(with $\mu \rightarrow 0$). As defined in Ref. \cite{Yan1997},
$\hat{\boldsymbol{r}_j} = \boldsymbol{r}_i/r_i$ and $\hat{\nabla}_i^Y = r_i \nabla_i^Y$.
The terms involving $\nabla_i^Y$ only operate on the spherical 
harmonics, and they will be discussed in \cref{sec:GenSphHarm}.

The wavefunction we use is slightly different from the form Yan and Drake \cite{Yan1997}
use. Specifically, we handle the antisymmetrization operator differently in 
our code, and we do not include the $\Omega$ function, given by Equation (10) 
in their paper. Part of the difference here is that we are working with two 
electrons and one positron, whereas they are working with systems that have 
three electrons, such as binding energy calculations of Li.

\subsection{General Integrals}
\label{sec:GenOverlap}
%As mentioned in the previous section, Drake and Yan's wavefunction is similar but not exactly the same. Thus, we cannot use the equations directly from their paper. Using their notation, the overlap integrals are given by
%\begin{align}
%\nonumber I^1(1) = \left< \phi_L^1 \left| \phi_R^1 \right. \right> = \sum_{\text{all } m_i^\prime m_i} & \int \mathrm{d}\boldsymbol{r}_1 \, \mathrm{d}\boldsymbol{r}_2 \, \mathrm{d}\boldsymbol{r}_3 \, r_1^{\tilde{j}_1} r_2^{\tilde{j}_2} r_3^{\tilde{j}_3} r_{12}^{\tilde{j}_{12}} r_{23}^{\tilde{j}_{23}} r_{31}^{\tilde{j}_{31}} \ee^{-(\tilde{\alpha} r_1 + \tilde{\beta} r_2 + \tilde{\gamma} r_3)} \\
%& \times Y_{\ell_1^\prime m_1^\prime}^* (\boldsymbol{r}_1) Y_{\ell_2^\prime m_2^\prime}^* (\boldsymbol{r}_2) Y_{\ell_3^\prime m_3^\prime}^* (\boldsymbol{r}_3) Y_{\ell_1 m_1} (\boldsymbol{r}_1) Y_{\ell_2 m_2} (\boldsymbol{r}_2) Y_{\ell_3 m_3} (\boldsymbol{r}_3).
%\end{align}
%This can be split into angular and radial parts, given by
%\begin{align}
%\nonumber I^1(1) = & \sum_{q_{12}=0}^{M_{12}} \sum_{q_{23}=0}^{M_{23}} \sum_{q_{31}=0}^{M_{31}} \sum_{k_{12}=0}^{L_{12}} \sum_{k_{23}=0}^{L_{23}} \sum_{k_{31}=0}^{L_{31}} C^1(1) \\
%& \times I_R \left(q_{12}, q_{23}, q_{31}, k_{12}, k_{23}, k_{31}; \tilde{j}_1, \tilde{j}_2, \tilde{j}_3, \tilde{j}_{12}, \tilde{j}_{23}, \tilde{j}_{31}; \tilde{\alpha}, \tilde{\beta}, \tilde{\gamma} \right),
%\end{align}
%where $\tilde{j}_1 = j_1^\prime + j_1$, etc. The $I_R$ function is only dependent on the radial coordinates and is given by equation (30) of their paper. The $C^1(1)$ function differs from theirs, as we do not have the 3j symbols in their $\Omega$ function. This is given in two parts as
%\begin{align}
%\nonumber C^1(1) =\, & (-1)^{q_{12} + q_{23} + q_{31}} \left(\ell_1^\prime,\ell_2^\prime,\ell_3^\prime,\ell_1,\ell_2,\ell_3 \right)^{1/2} \sum_{n_1 n_2 n_3} (n_1,n_2,n_3) \ThreeJSymbol{\ell_1^\prime,0}{\ell_1,0}{n_1,0}  \\
%& \times \ThreeJSymbol{\ell_2^\prime,0}{\ell_2,0}{n_2,0} \ThreeJSymbol{\ell_3^\prime,0}{\ell_3,0}{n_3,0} \ThreeJSymbol{q_{31},0}{q_{12},0}{n_1,0} \ThreeJSymbol{q_{12},0}{q_{23},0}{n_2,0} \nonumber \\
%& \times \ThreeJSymbol{q_{23},0}{q_{31},0}{n_3,0} \SixJSymbol{n_1,n_2,n_3}{q_{23},q_{31},q_{12}} \tilde{C}^1(1)
%\end{align}
%\noindent and
%\begin{align}
%\tilde{C}^1(1) = & \sum_{\text{all } m_i^\prime m_i t_i} (-1)^{m_1 + m_2 + m_3} \ThreeJSymbol{n_1,t_1}{n_2,t_2}{n_3,t_3} \nonumber \\
%& \times \ThreeJSymbol{l_1^\prime,-m_1^\prime}{l_1,m_1}{n_1,t_1} \ThreeJSymbol{l_2^\prime,-m_2^\prime}{l_2,m_2}{n_2,t_2} \ThreeJSymbol{l_1^\prime,-m_1^\prime}{l_3,m_3}{n_3,t_3}.
%\end{align}
%From the selection rules for the 3j symbol, $t_i = m_i^\prime - m_i$. The factor $(2\ell+1)$ appears often in these types of derivations, so we adopt the shorthand notation of $(l,m,n,\ldots) = (2l+1)(2m+1)(2n+1) \ldots$.

Using the notation in Ref.~\cite{Yan1997}, the terms in \cref{eq:GenHam2} without the spherical harmonic operator, $\hat{\nabla}_i^Y$, will be of the general form
\begin{align}
\label{eq:ShortIntGen}
\nonumber I(\ell_1^\prime m_1^\prime, \ell_2^\prime m_2^\prime, &\ell_3^\prime m_3^\prime, \ell_1 m_1, \ell_2 m_2, \ell_3 m_3; j_1,j_2,j_3,j_{12},j_{23},j_{31}; \bar{\alpha}, \bar{\beta}, \bar{\gamma}) \\
= & \; \int d \textit{\textbf{r}}_1 d \textit{\textbf{r}}_2 d \textit{\textbf{r}}_3
r_1^{j_1} r_2^{j_2} r_3^{j_3} r_{12}^{j_{12}}
r_{23}^{j_{23}} r_{31}^{j_{31}}
e^{-(\bar{\alpha} r_1 + \bar{\beta} r_2 + \bar{\gamma} r_3)}  \nonumber \\
& \times Y_{\ell_1^\prime m_1^\prime}^* (\textit{\textbf{r}}_1) Y_{\ell_2^\prime m_2^\prime}^* (\textit{\textbf{r}}_2) Y_{\ell_3^\prime m_3^\prime}^* (\textit{\textbf{r}}_3) Y_{\ell_1 m_1} (\textit{\textbf{r}}_1) Y_{\ell_2 m_2} (\textit{\textbf{r}}_2) Y_{\ell_3 m_3} (\textit{\textbf{r}}_3)\, .
\end{align}
As noted on page~\pageref{BraNote} for the Kohn-type variational methods, these
should not be conjugated, but for this work with $m = 0$ (so excluding the mixed
symmetry terms described in \cref{sec:MixedTerms}), the real-valued short-range
terms will be the same with and without the conjugate.
The method described here is an extension of that in \cref{sec:ShortInt}.
Note that $\alpha$, $\beta$, and $\gamma$ are not necessarily the same as those
in \cref{eq:PhiDef}. These will be $2\alpha$, $2\beta$, and $2\gamma$ for the
direct-direct terms and $2\alpha$, $\beta+\gamma$, $\gamma+\beta$ for the
direct-exchange terms. After some manipulation, this integral can be written as
\cite{Yan1997}
\begin{align}
\label{eq:ShortIntGen2}
I(\ell_1^\prime m_1^\prime, &\ell_2^\prime m_2^\prime, \ell_3^\prime m_3^\prime, \ell_1 m_1, \ell_2 m_2, \ell_3 m_3; j_1,j_2,j_3,j_{12},j_{23},j_{31}; \bar{\alpha}, \bar{\beta}, \bar{\gamma})  \nonumber \\
= & \; \sum_{q_{12}=0}^{M_{12}} \sum_{q_{23}=0}^{M_{23}} \sum_{q_{31}=0}^{M_{31}} \sum_{k_{12}=0}^{L_{12}} \sum_{k_{23}=0}^{L_{23}} \sum_{k_{31}=0}^{L_{31}}  \nonumber \\
& \times I_{\rm{ang}}(\ell_1^\prime m_1^\prime, \ell_2^\prime m_2^\prime, \ell_3^\prime m_3^\prime, \ell_1 m_1, \ell_2 m_2, \ell_3 m_3; q_{12}, q_{23}, q_{31})  \nonumber \\
& \times I_{\rm{R}}(q_{12}, q_{23}, q_{31}, k_{12}, k_{23}, k_{31}; j_1, j_2, j_3, j_{12}, j_{23}, j_{31}; \bar{\alpha}, \bar{\beta}, \bar{\gamma}).
\end{align}
For even values of $j_{12}$, $M_{12} = \frac{1}{2}j_{12}$ and $L_{12} = \frac{1}{2}j_{12} - q_{12}$. For odd values of $j_{12}$, $M_{12} = \infty$ and $L_{12} = \frac{1}{2}(j_{12}+1)$. The same type of upper limits apply to the $j_{23}$ and $j_{31}$ terms.

The angular part of $I$ is given by
\begin{align}
\label{eq:DrakeGenAng}
I_{\rm{ang}}(\ell_1^\prime m_1^\prime, & \ell_2^\prime m_2^\prime, \ell_3^\prime m_3^\prime, \ell_1 m_1, \ell_2 m_2, \ell_3 m_3; q_{12}, q_{23}, q_{31})  \nonumber \\
= & \;(-1)^{m_1^\prime + m_2^\prime + m_3^\prime + q_{12} + q_{23} + q_{31}} (\ell_1^\prime, \ell_2^\prime, \ell_3^\prime, \ell_1, \ell_2, \ell_3)^{1/2} \sum_{n_1 n_2 n_3} (n_1,n_2,n_3)  \nonumber \\
& \times \SixJSymbol{n_1,n_2,n_3}{q_{23},q_{31},q_{12}} \ThreeJSymbol{n_1, m_1^\prime - m_1}{n_2, m_2^\prime - m_2}{n_3, m_3^\prime - m_3}  \nonumber \\
& \times \ThreeJSymbol{\ell_1^\prime,-m_1^\prime}{\ell_1,m_1}{n_1, m_1^\prime - m_1} \ThreeJSymbol{\ell_2^\prime,-m_2^\prime}{\ell_2,m_2}{n_2,m_2^\prime - m_2}  \nonumber \\
& \times \ThreeJSymbol{\ell_1^\prime,-m_1^\prime}{\ell_3,m_3}{n_3,m_3^\prime - m_3} \ThreeJSymbol{\ell_1^\prime,0}{\ell_1,0}{n_1,0} \ThreeJSymbol{\ell_2^\prime,0}{\ell_2,0}{n_2,0}  \nonumber \\
& \times \ThreeJSymbol{\ell_3^\prime,0}{\ell_3,0}{n_3,0} \ThreeJSymbol{q_{31},0}{q_{12},0}{n_1,0} \ThreeJSymbol{q_{12},0}{q_{23},0}{n_2,0} \ThreeJSymbol{q_{23},0}{q_{31},0}{n_3,0}.
\end{align}
This expression includes summations over both the Wigner 3-j and 6-j
coefficients \cite{Edmonds1996,Brink1993,Rose1995}, also known as the 3-j and
6-j symbols. The 3-j symbols are related to the Clebsch-Gordan coefficients by
\cite[p.46]{Edmonds1996}
\begin{equation}
\label{eq:Clebsch3J}
\ThreeJSymbol{j_1,m_1}{j_2,m_2}{j_3,m_3} = (-1)^{j_1 - j_2 + m_3} (2 j_3 + 1)^{-1/2} \ClebschGordon{j_1,m_1}{j_2,m_2}{j_3,-m_3}.
\end{equation}
The advantage of the 3-j symbols over the Clebsch-Gordan coefficients is that 
they are a more symmetric representation of angular momentum. The factor
$(2\ell+1)$ appears often in these types of derivations, so we adopt their 
shorthand notation of $(l,m,n,\ldots) = (2l+1)(2m+1)(2n+1) \cdots$.

The radial part of \cref{eq:ShortIntGen2} is 
\begin{align}
\label{eq:ShortGenIR}
I_{\rm{R}}(q_{12}, q_{23}, q_{31}&, k_{12}, k_{23}, k_{31}; j_1, j_2, j_3, j_{12}, j_{23}, j_{31}; \bar{\alpha}, \bar{\beta}, \bar{\gamma})  \nonumber \\
= & \; C_{j_{12} q_{12} k_{12}} C_{j_{23} q_{23} k_{23}} C_{j_{31} q_{31} k_{31}}  \nonumber \\
& \times W_{\rm{R}}(q_{12}, q_{23}, q_{31}, k_{12}, k_{23}, k_{31}; j_1, j_2, j_3, j_{12}, j_{23}, j_{31}; \bar{\alpha}, \bar{\beta}, \bar{\gamma}).
\end{align}
The $C_{jqk}$ coefficients are the same as in \cref{eq:Ccoeff}. The $W_{\rm{R}}$ function is built from the $W$ functions in \cref{eq:Wfunc} as
\begin{align}
\label{eq:ShortGenWR}
W_{\rm{R}}(&q_{12}, q_{23}, q_{31}, k_{12}, k_{23}, k_{31}; j_1, j_2, j_3, j_{12}, j_{23}, j_{31}; \bar{\alpha}, \bar{\beta}, \bar{\gamma})  \nonumber \\
=\; \; & W(j_1 + 2 + s_{12} + s_{31}, j_2 + 2 + j_{12} - s_{12} + s_{23}, j_3 + 2 + j_{23} - s_{23} + j_{31} - s_{31}; \bar{\alpha}, \bar{\beta}, \bar{\gamma})  \nonumber \\
+ & W(j_1 + 2 + s_{12} + s_{31}, j_3 + 2 + s_{23} + j_{31} - s_{31}, j_2 + 2 + j_{12} - s_{12} + j_{23} - s_{23}; \bar{\alpha}, \bar{\gamma}, \bar{\beta})  \nonumber \\
+ & W(j_2 + 2 + s_{12} + s_{23}, j_1 + 2 + j_{12} - s_{12} + s_{31}, j_3 + 2 + j_{23} - s_{23} + j_{31} - s_{31}; \bar{\beta}, \bar{\alpha}, \bar{\gamma})  \nonumber \\
+ & W(j_2 + 2 + s_{12} + s_{23}, j_3 + 2 + j_{23} - s_{23} + s_{31}, j_1 + 2 + j_{12} - s_{12} + j_{31} - s_{31}; \bar{\beta}, \bar{\gamma}, \bar{\alpha})  \nonumber \\
+ & W(j_3 + 2 + s_{23} + s_{31}, j_1 + 2 + s_{12} + j_{31} - s_{31}, j_2 + 2 + j_{12} - s_{12} + j_{23} - s_{23}; \bar{\gamma}, \bar{\alpha}, \bar{\beta})  \nonumber \\
+ & W(j_3 + 2 + s_{23} + s_{31}, j_2 + 2 + s_{12} + j_{23} - s_{23}, j_1 + 2 + j_{12} - s_{12} + j_{31} - s_{31}; \bar{\gamma}, \bar{\beta}, \bar{\alpha}),
\end{align}
with $s_{ij} = q_{ij} + 2 k_{ij}$.
Note that there are similarities with \cref{eq:FourBodyExpansion}.



\subsection{Spherical Harmonic Terms}
\label{sec:GenSphHarm}

The terms in \cref{eq:GenHam2} involving the spherical harmonic operator
$\hat{\nabla}_i^Y$ have to be handled differently than the other terms. In
Ref.~\cite{Yan1997}, they do different permutations for the three-electron
problem, so for the $p$ in their paper, we always use $p = 1$.

\begin{align}
\label{eq:GenY12}
I^1(\hat{\boldsymbol{r}_1} \cdot \hat{\nabla}_2^Y) = &\sum_{q_{12}=0}^{M_{12}} \sum_{q_{23}=0}^{M_{23}} \sum_{q_{31}=0}^{M_{31}} \sum_{k_{12}=0}^{L_{12}} \sum_{k_{23}=0}^{L_{23}} \sum_{k_{31}=0}^{L_{31}} \sum_{T_1 T_2} b(\ell_2; T_2) C^1(\hat{\textit{\textbf{r}}}_1 \bm{\cdot} \hat{\textit{\textbf{r}}}_2)  \nonumber \\
& \times I_{\rm{R}}(q_{12}, q_{23}, q_{31}, k_{12}, k_{23}, k_{31}; j_1, j_2, j_3, j_{12}, j_{23}, j_{31}; \bar{\alpha}, \bar{\beta}, \bar{\gamma})
\end{align}
The $b$ function here is defined by
\begin{align}
\label{eq:bfunc}
b(\ell; \ell - 1) &= \ell + 1  \nonumber \\
b(\ell; \ell + 1) &= -\ell.
\end{align}
For any other values of the arguments, $b$ gives 0. $C^1$ is given by
\begin{align}
\label{eq:CP12}
C^1(\hat{\textit{\textbf{r}}}_1 & \bm{\cdot} \hat{\textit{\textbf{r}}}_2) = (\ell_1^\prime, \ell_2^\prime, \ell_3^\prime, \ell_1, \ell_2, \ell_3)^{1/2} (-1)^{q_{12} + q_{23} + q_{31}} \sum_{n_1 n_2 n_3} (n_1, n_2, n_3, T_1, T_2)  \nonumber \\
& \times \ThreeJSymbol{1,0}{\ell_1,0}{T_1,0} \ThreeJSymbol{1,0}{\ell_2,0}{T_1,0} \ThreeJSymbol{\ell_1^\prime,0}{T_1,0}{n_1,0} \ThreeJSymbol{\ell_2^\prime,0}{T_2,0}{n_2,0} \ThreeJSymbol{\ell_3^\prime,0}{\ell_3,0}{n_3,0}  \nonumber \\
& \times \ThreeJSymbol{q_{31},0}{q_{12},0}{n_1,0} \ThreeJSymbol{q_{12},0}{q_{23},0}{n_2,0} \ThreeJSymbol{q_{23},0}{q_{31},0}{n_3,0} \SixJSymbol{n_1,n_2,n_3}{q_{23},q_{31},q_{12}} \tilde{C}^1(\hat{\textit{\textbf{r}}}_1 \bm{\cdot} \hat{\textit{\textbf{r}}}_2),
\end{align}
with
\begin{align}
\label{eq:CPtilde12}
\tilde{C}^1(\hat{\textit{\textbf{r}}}_1 & \bm{\cdot} \hat{\textit{\textbf{r}}}_2) = \sum_\mu (-1)^{\mu-m_3} \ThreeJSymbol{1,\mu}{\ell_1,m_1}{T_1,-\mu-m_1} \ThreeJSymbol{1,-\mu}{\ell_2,m_2}{T_2,\mu-m_2}  \nonumber \\
& \times \ThreeJSymbol{\ell_3^\prime,-m_3^\prime}{\ell_3,m_3}{n_3,m_3^\prime-m_3} \ThreeJSymbol{\ell_1^\prime,-m_1^\prime}{T_1,\mu+m_1}{n_1,m_1^\prime-\mu-m_1}  \nonumber \\
& \times \ThreeJSymbol{\ell_2^\prime,-m_2^\prime}{T_2,-\mu+m_2}{n_2,m_2^\prime+\mu-m_2} \ThreeJSymbol{n_1,m_1^\prime-\mu-m_1}{n_2,m_2^\prime+\mu-m_2}{n_3,m_3^\prime-m_3}.
\end{align}
The values that $\mu$ can take are $-1$, $0$, and $1$.

Considering the properties of the 3-j symbols in \cref{eq:CPtilde12}, the 
limits for the $T_1$ summation in \cref{eq:GenY12} are $\abs{1 - \ell_1}$ to
$1 + \ell_1$. Similarly, for the $T_2$ summation, the limits are
$\abs{1 - \ell_2}$ to $1 + \ell_2$. However, considering that the $b$
function in \cref{eq:bfunc} can give 0, not all $T_2$ in this range are used.

Unlike the previous section, the expressions here are not exactly the same as 
that in Yan and Drake \cite{Yan1997}, since as noted, their wavefunction has 
an extra set of coefficients. % given by
%\begin{align}
%\Omega(\ell_1, \ell_2, \ell_{12}, &\ell_3, L, M_L, m_1, m_2, m_3) = (-1)^{\ell_1 - \ell_2 + m_{12} + \ell_{12} - \ell_3 + M_L} (\ell_{12}, L)^{1/2}  \nonumber \\
%& \times \ThreeJSymbol{\ell_1, m_1}{\ell_2,m_2}{\ell_{12},-m_{12}} \ThreeJSymbol{\ell_{12},m_{12}}{\ell_3,m_3}{L,-M_L}.
%\end{align}
This addition to their wavefunction is included in the
$\tilde{C}^1(\hat{\textit{\textbf{r}}}_1 \bm{\cdot} \hat{\textit{\textbf{r}}}_2)$
and allows them to reduce these using graphical methods as in
Refs.~\cite{Lindgren2012,Brink1993}. Unfortunately, such simplifications are not possible with the 
form of \cref{eq:GeneralWaveTrial}.

We note that $C^1(\hat{\textit{\textbf{r}}}_1 \bm{\cdot} \hat{\textit{\textbf{r}}}_2) = C^1(\hat{\textit{\textbf{r}}}_2 \bm{\cdot} \hat{\textit{\textbf{r}}}_1)$, but $I^1(\hat{\boldsymbol{r}_1} \cdot \hat{\nabla}_2^Y) \neq I^1(\hat{\boldsymbol{r}_2} \cdot \hat{\nabla}_1^Y)$. Namely,
\begin{align}
\label{eq:GenY21}
I^1(\hat{\boldsymbol{r}_2} \cdot \hat{\nabla}_1^Y) = &\sum_{q_{12}=0}^{M_{12}} \sum_{q_{23}=0}^{M_{23}} \sum_{q_{31}=0}^{M_{31}} \sum_{k_{12}=0}^{L_{12}} \sum_{k_{23}=0}^{L_{23}} \sum_{k_{31}=0}^{L_{31}} \sum_{T_1 T_2} b(\ell_1; T_1) C^1(\hat{\textit{\textbf{r}}}_1 \bm{\cdot} \hat{\textit{\textbf{r}}}_2)  \nonumber \\
& \times I_{\rm{R}}(q_{12}, q_{23}, q_{31}, k_{12}, k_{23}, k_{31}; j_1, j_2, j_3, j_{12}, j_{23}, j_{31}; \bar{\alpha}, \bar{\beta}, \bar{\gamma}).
\end{align}
More generally, these integrals can be written as
\begin{align}
\label{eq:GenYij}
I^1(\hat{\boldsymbol{r}_i} \cdot \hat{\nabla}_j^Y) = &\sum_{q_{12}=0}^{M_{12}} \sum_{q_{23}=0}^{M_{23}} \sum_{q_{31}=0}^{M_{31}} \sum_{k_{12}=0}^{L_{12}} \sum_{k_{23}=0}^{L_{23}} \sum_{k_{31}=0}^{L_{31}} \sum_{T_i T_j} b(\ell_j; T_j) C^1(\hat{\textit{\textbf{r}}}_i \bm{\cdot} \hat{\textit{\textbf{r}}}_j)  \nonumber \\
& \times I_{\rm{R}}(q_{12}, q_{23}, q_{31}, k_{12}, k_{23}, k_{31}; j_1, j_2, j_3, j_{12}, j_{23}, j_{31}; \bar{\alpha}, \bar{\beta}, \bar{\gamma}),
\end{align}
and $C^1(\hat{\textit{\textbf{r}}}_i \bm{\cdot} \hat{\textit{\textbf{r}}}_j) = C^1(\hat{\textit{\textbf{r}}}_j \bm{\cdot} \hat{\textit{\textbf{r}}}_i)$. Using cyclic permutations,
\begin{align}
\label{eq:CP23}
C^1(\hat{\textit{\textbf{r}}}_2 & \bm{\cdot} \hat{\textit{\textbf{r}}}_3) = (\ell_1^\prime, \ell_2^\prime, \ell_3^\prime, \ell_1, \ell_2, \ell_3)^{1/2} (-1)^{q_{12} + q_{23} + q_{31}} \sum_{n_1 n_2 n_3} (n_1, n_2, n_3, T_2, T_3)  \nonumber \\
& \times \ThreeJSymbol{1,0}{\ell_2,0}{T_2,0} \ThreeJSymbol{1,0}{\ell_3,0}{T_3,0} \ThreeJSymbol{\ell_1^\prime,0}{\ell_1,0}{n_1,0} \ThreeJSymbol{\ell_2^\prime,0}{T_2,0}{n_2,0} \ThreeJSymbol{\ell_3^\prime,0}{T_3,0}{n_3,0}  \nonumber \\
& \times \ThreeJSymbol{q_{31},0}{q_{12},0}{n_1,0} \ThreeJSymbol{q_{12},0}{q_{23},0}{n_2,0} \ThreeJSymbol{q_{23},0}{q_{31},0}{n_3,0} \SixJSymbol{n_1,n_2,n_3}{q_{23},q_{31},q_{12}} \tilde{C}^1(\hat{\textit{\textbf{r}}}_2 \bm{\cdot} \hat{\textit{\textbf{r}}}_3)
\end{align}
with
\begin{align}
\label{eq:CPtilde23}
\tilde{C}^1(\hat{\textit{\textbf{r}}}_2 & \bm{\cdot} \hat{\textit{\textbf{r}}}_3) = \sum_\mu (-1)^{\mu-m_1} \ThreeJSymbol{1,\mu}{\ell_2,m_2}{T_2,-\mu-m_2} \ThreeJSymbol{1,-\mu}{\ell_3,m_3}{T_3,\mu-m_3}  \nonumber \\
& \times \ThreeJSymbol{\ell_1^\prime,-m_1^\prime}{\ell_1,m_1}{n_1,m_1^\prime-m_1} \ThreeJSymbol{\ell_2^\prime,-m_2^\prime}{T_2,\mu+m_2}{n_2,m_2^\prime-\mu-m_2}  \nonumber \\
& \times \ThreeJSymbol{\ell_3^\prime,-m_3^\prime}{T_3,-\mu+m_3}{n_3,m_3^\prime+\mu-m_3} \ThreeJSymbol{n_1,m_1^\prime-m_1}{n_2,m_2^\prime-\mu-m_2}{n_3,m_3^\prime+\mu-m_3},
\end{align}
and
\begin{align}
\label{eq:CP31}
C^1(\hat{\textit{\textbf{r}}}_3 & \bm{\cdot} \hat{\textit{\textbf{r}}}_1) = (\ell_1^\prime, \ell_2^\prime, \ell_3^\prime, \ell_1, \ell_2, \ell_3)^{1/2} (-1)^{q_{12} + q_{23} + q_{31}} \sum_{n_1 n_2 n_3} (n_1, n_2, n_3, T_3, T_1)  \nonumber \\
& \times \ThreeJSymbol{1,0}{\ell_3,0}{T_3,0} \ThreeJSymbol{1,0}{\ell_1,0}{T_1,0} \ThreeJSymbol{\ell_2^\prime,0}{\ell_2,0}{n_2,0} \ThreeJSymbol{\ell_3^\prime,0}{T_3,0}{n_3,0} \ThreeJSymbol{\ell_1^\prime,0}{T_1,0}{n_1,0}  \nonumber \\
& \times \ThreeJSymbol{q_{31},0}{q_{12},0}{n_1,0} \ThreeJSymbol{q_{12},0}{q_{23},0}{n_2,0} \ThreeJSymbol{q_{23},0}{q_{31},0}{n_3,0} \SixJSymbol{n_1,n_2,n_3}{q_{23},q_{31},q_{12}} \tilde{C}^1(\hat{\textit{\textbf{r}}}_3 \bm{\cdot} \hat{\textit{\textbf{r}}}_1)
\end{align}
with
\begin{align}
\label{eq:CPtilde31}
\tilde{C}^1(\hat{\textit{\textbf{r}}}_3 & \bm{\cdot} \hat{\textit{\textbf{r}}}_1) = \sum_\mu (-1)^{\mu-m_2} \ThreeJSymbol{1,\mu}{\ell_3,m_3}{T_3,-\mu-m_3} \ThreeJSymbol{1,-\mu}{\ell_1,m_1}{T_1,\mu-m_1}  \nonumber \\
& \times \ThreeJSymbol{\ell_2^\prime,-m_2^\prime}{\ell_2,m_2}{n_2,m_2^\prime-m_2} \ThreeJSymbol{\ell_3^\prime,-m_3^\prime}{T_3,\mu+m_3}{n_3,m_3^\prime-\mu-m_3}  \nonumber \\
& \times \ThreeJSymbol{\ell_1^\prime,-m_1^\prime}{T_1,-\mu+m_1}{n_1,m_1^\prime+\mu-m_1} \ThreeJSymbol{n_1,m_1^\prime+\mu-m_1}{n_2,m_2^\prime-m_2}{n_3,m_3^\prime-\mu-m_3}.
\end{align}


\section{Programs}
\label{sec:GenShortProg}
The general codes described in this chapter are available on GitHub \cite{GitHub}.

The general short-short code implementing the equations described in \cref{sec:GeneralShort}
is at least an order of magnitude slower than the corresponding S-, P-, and D-wave
short-short code developed separately for each of the first three partial waves.
On the Talon 2 \cite{Talon2} cluster, an $\omega = 5$ run of the general short-short code
takes approximately a full day running on a single node with 16 cores.

A rough analysis of the code shows that the bulk of the processing time is 
spent calculating the angular parts, i.e. the 3-j and 6-j symbols. Due to the 
symmetries inherent in the 3-j and 6-j symbols, different inputs can generate 
the same output. Also, many input values to these do not satisfy the 
selection rules for the 3-j and 6-j symbols
\cite[p.1054-1064]{Messiah1999} \cite{Edmonds1996}, giving a result of 0.

Multiple papers \cite{Luscombe1998,Wei1998,Rasch2004,Johansson2015} have
detailed strategies to exploit the symmetries to speed up calculations of the
3-j and 6-j symbols. Storing a full lookup table with the full parameter
space of the 6 input variables would be prohibitively memory-intensive.
The most recent of these by Johansson and Forss{\'{e}}n \cite{Johansson2015}
provides a very promising algorithm with full code that could be used to speed
up this general short-short code.



%\cref{eq:FourBodyExpansion}
%
%\cref{eq:ShortIntGen2}
%
%\cref{eq:GenYij}





\biblio
\end{document}