\documentclass[Dissertation.tex]{subfiles} 
\begin{document}

\clearpage
\pagebreak
\newpage

\chapter{Higher Partial Waves}
\label{chp:HigherWaves}

\todoi{Mention $\alpha$, $\beta$, and $\gamma$ used and any other parameters}
\todoi{Talk about how angular integrations done}

\Cref{tab:SphHarm,tab:SphBess,tab:SphNeum} show the spherical harmonics, spherical Bessel and spherical Neumann functions needed in \cref{eq:GeneralWaveTrial} to use the higher partial waves.

\todoi{Split up tables into each respective section}
{
\renewcommand{\arraystretch}{1.5}
\begin{table}[H]
\centering
\begin{tabular}{l l}
\toprule\\[-1.2cm]
Partial Wave & $\SphericalHarmonicY{\ell}{0}{\theta}{\phi}$ \\
\midrule
F-Wave & $\sqrt{\frac{7}{16\pi}} \left(5 \cos^3\theta - 3 \cos\theta \right)$ \\
G-Wave & $\sqrt{\frac{9}{256\pi}} \left(35 \cos^4\theta - 30 \cos^2\theta + 3 \right)$ \\
H-Wave & $\sqrt{\frac{11}{256\pi}} \left(63 \cos^5\theta - 70 \cos^3\theta + 15 \cos\theta \right)$ \\
\bottomrule
\end{tabular}
\caption{Spherical harmonics for partial waves $\ell = 3$ through 5}
\label{tab:SphHarm}
\end{table}
}

{
\renewcommand{\arraystretch}{1.5}
\begin{table}[H]
\centering
\begin{tabular}{l l}
\toprule\\[-1.2cm]
Partial Wave & $j_\ell(z)$ \\
\midrule
F-Wave & $\frac{\left(z^2-15\right) \cos (z)}{z^3}-\frac{3 \left(2 z^2-5\right) \sin (z)}{z^4}$ \\
G-Wave & $\frac{5 \left(2 z^2-21\right) \cos (z)}{z^4}+\frac{\left(z^4-45 z^2+105\right) \sin (z)}{z^5}$ \\
H-Wave & $\frac{15 \left(z^4-28 z^2+63\right) \sin (z)}{z^6}+\frac{\left(-z^4+105 z^2-945\right) \cos (z)}{z^5}$ \\
\bottomrule
\end{tabular}
\caption{Spherical Bessel functions for partial waves $\ell = 3$ through 5}
\label{tab:SphBess}
\end{table}
}

{
\renewcommand{\arraystretch}{1.5}
\begin{table}[H]
\centering
\begin{tabular}{l l}
\toprule\\[-1.2cm]
Partial Wave & $n_\ell(z)$ \\
\midrule
F-Wave & $\frac{3 \left(2 z^2-5\right) \cos (z)}{z^4}+\frac{\left(z^2-15\right) \sin (z)}{z^3}$ \\
G-Wave & $\frac{5 \left(2 z^2-21\right) \sin (z)}{z^4}+\frac{\left(-z^4+45 z^2-105\right) \cos (z)}{z^5}$ \\
H-Wave & $\frac{\left(-z^4+105 z^2-945\right) \sin (z)}{z^5}-\frac{15 \left(z^4-28 z^2+63\right) \cos (z)}{z^6}$ \\
\bottomrule
\end{tabular}
\caption{Spherical Neumann functions for partial waves $\ell = 3$ through 5}
\label{tab:SphNeum}
\end{table}
}


\section{Born Approximation}
\label{sec:Born}
\todoi{Put something like these in the cross section chapter}
Only a finite number of partial waves can be used to find the cross sections (\cref{chp:CrossSections}), 
The formulas to find the cross sections (\cref{chp:CrossSections}) have infinite summations, and we have to truncate these

In an attempt to approximate the partial waves past the D-wave, we turned to the Born approximation. The Born approximation comes from using only the first term in \cref{eq:TrialWave,eq:TrialWaveHigher}. Specifically, this is done with the Kohn variational method to get an estimate for the $K$ matrix, giving \cite[p.???]{Bransden2003}
\begin{equation}
\label{eq:Born}
\tan\delta_\ell \approx -(\widetilde{S}_\ell,\mathcal{L}\widetilde{S}_\ell )\,.
\end{equation}
We have also done a modified Born approximation that uses the first two terms ($\bar{S}$ and $\bar{C}$) in \cref{eq:TrialWave,eq:TrialWaveHigher}. This modified Born approximation is very similar and lines up nearly exactly with the Born on most partial waves, so we normally just use the Born approximation.

\begin{figure}[H]
	\centering
	\includegraphics[width=5in]{swave-phase-born}
	\caption[$^{1,3}$S complex Kohn and Born comparison]{$^{1,3}$S phase shift comparison between S-matrix complex Kohn and Born approximation}
	\label{fig:SWavePhaseBorn}
\end{figure}

\begin{figure}[H]
	\centering
	\includegraphics[width=5in]{pwave-phase-born}
	\caption[$^{1,3}$P complex Kohn and Born comparison]{$^{1,3}$P phase shift comparison between S-matrix complex Kohn and Born approximation}
	\label{fig:PWavePhaseBorn}
\end{figure}

\begin{figure}[H]
	\centering
	\includegraphics[width=5in]{dwave-phase-born}
	\caption[$^{1,3}$D complex Kohn and Born comparison]{$^{1,3}$D phase shift comparison between S-matrix complex Kohn and Born approximation}
	\label{fig:DWavePhaseBorn}
\end{figure}

\todoi{Show both Born and modified Born}

These Born approximations were calculated for the first three partial waves but showed huge discrepancies, especially for the S-wave, as seen in \cref{fig:SWavePhaseBorn,fig:PWavePhaseBorn,fig:DWavePhaseBorn}. These somewhat get the triplet shapes right, but it would be a poor approximation for these. The $^1$S, $^1$P, and $^1$D partial waves have resonances, which we would not expect to be represented by these approximations.

Due to the obvious discrepancies with even the D-wave, we decided to do a full Kohn calculation for the F-wave to compare, again finding that the Born approximation does not match up as well as we would like for either $^1$F or $^3$F. We tried the same for the G-wave and H-wave, with the same results. The results of using the Born approximation for the higher partial waves are shown later in this chapter in \cref{fig:FWavePhase,fig:GWavePhase,fig:HWavePhase}. The Born approximation unfortunately does not represent any of the partial waves through the H-wave well for this system.

\todoi{Cross sections with Born}



\section{F-Wave}
\label{sec:FWave}

\subsection{Phase Shifts}
\label{sec:FWavePhase}

\begin{figure}[H]
	\centering
	\includegraphics[width=7in]{fwave-phases}
	\caption{$^{1,3}$F phase shifts}
	\label{fig:FWavePhase}
\end{figure}

We have calculated the F-wave phase shifts for $\omega = 5$ with $\alpha = 0.5$, $\beta = 0.6$, $\gamma = 1.1$ and $\mu = 0.7$ for the first two short-range symmetries using the general codes described in \cref{sec:GeneralLong,sec:GeneralShort}. \Cref{fig:FWavePhase} and \cref{tab:FWavePhase} show the phase shifts for the F-wave.

\setlength{\abovecaptionskip}{6pt}
\setlength{\belowcaptionskip}{6pt}
\begin{table}[H]
\centering
\begin{tabular}{c | c c | c c}
\toprule
$\kappa$ & Kohn $\delta_3^+$ & Born $\delta_3^+$ & Kohn $\delta_3^-$ & Born $\delta_3^-$ \\
\midrule
0.1 &	$1.183^{-6}$ & $1.237^{-7}$ & $1.023^{-6}$ & $-1.232^{-7}$ \\
0.2 &	$1.046^{-4}$ & $2.003^{-6}$ & $8.282^{-5}$ & $-1.387^{-5}$ \\
0.3 &	$1.048^{-3}$ & $6.051^{-5}$ & $6.824^{-4}$ & $-1.926^{-4}$ \\
0.4 &	$4.375^{-3}$ & $5.009^{-4}$ & $1.963^{-3}$ & $-1.092^{-3}$ \\
0.5 & $1.186^{-2}$ & $3.768^{-3}$ & $2.941^{-3}$ & $-3.706^{-3}$ \\
0.6 &	$2.610^{-2}$ & $2.143^{-3}$ & $2.516^{-3}$ & $-8.995^{-3}$ \\
0.7 &	$4.880^{-2}$ & $1.312^{-2}$ & $4.236^{-4}$ & $-1.725^{-2}$ \\
\bottomrule
\end{tabular}
\caption{F-Wave Phase Shifts}
\label{tab:FWavePhase}
\end{table}

We can see that the phase shifts for the (modified) Born approximation do not agree very well with the full Kohn calculation, though they follow roughly the same shape. The triplet Born is fully negative, while the Kohn only goes negative past about $\kappa = 0.7$. The problem with the triplet gets even worse for the G-wave (section \ref{sec:GWave}) and H-Wave (section \ref{sec:HWave}).

\begin{figure}[H]
	\centering
	\includegraphics[width=6in]{fwave-phases-full}
	\caption[Full $^1$F phase shifts]{$^1$F phase shifts showing full resonance past the inelastic threshold}
	\label{fig:FWavePhaseFull}
\end{figure}
\todoi{Remove Born and triplet}

\subsection{Resonance}
\label{sec:FWaveResonance}

There is the start of a resonance shortly before the threshold cutoff in figure \ref{fig:FWavePhase}. Figure \ref{fig:FWavePhaseFull} gives the rough plotting past this resonance. As our code does not contain the open channels required to extend into the region that contains the rest of the resonance, we likely cannot measure the resonance parameters as accurately. Tables \ref{tab:FWaveResonanceBefore} and \ref{tab:FWaveResonanceFull} give the resonance parameter fittings using our MATLAB script (\ref{?}).

Table \ref{tab:FWaveResonanceBefore} shows the calculated resonance parameters using only data before the Ps(n=2) threshold. This is the data that is plotted in figure \ref{fig:FWavePhase}. Table \ref{tab:FWaveResonanceFull} has the calculated values when we consider data through the resonance, in the range of $\kappa = 0.74 - 0.88$. This data is plotted in figure \ref{fig:FWavePhaseFull}. This calculated position of 5.182 eV is very close to the 5.166 eV that Ho and Yan compute using the complex rotation method \cite{Ho2000}. As with the other partial waves, our width of 0.0107 eV is close to their value of 0.0174 eV but does not compare exactly.


\setlength{\abovecaptionskip}{6pt}
\setlength{\belowcaptionskip}{6pt}
\begin{table}[H]
\centering
\begin{tabular}{l c c c c c c}
\toprule
Fitting & a & b & c & $E_R$ (eV) & $\Gamma$ (eV) \\
\midrule
Bisquare	& 0.033382 & -0.017062 & 0.0062664 & 5.1913 & 0.013232 \\
Andrews		& 0.033390 & -0.017066 & 0.0062668 & 5.1913 & 0.013232 \\
Cauchy		& 0.037941 & -0.019297 & 0.0065459 & 5.1899 & 0.012976 \\
Fair		& 0.038145 & -0.019401 & 0.0065597 & 5.1898 & 0.012956 \\
Welsch		& 0.038211 & -0.019432 & 0.0065632 & 5.1898 & 0.012964 \\
Huber		& 0.038173 & -0.019410 & 0.0065601 & 5.1898 & 0.012962 \\
Logistic	& 0.037609 & -0.019191 & 0.0065409 & 5.1895 & 0.012921 \\
Talwar		& 0.033688 & -0.017208 & 0.0062841 & 5.1913 & 0.013220 \\
\bottomrule
\end{tabular}
\caption{Resonance Parameters Before Threshold}
\label{tab:FWaveResonanceBefore}
\end{table}


\setlength{\abovecaptionskip}{6pt}
\setlength{\belowcaptionskip}{6pt}
\begin{table}[H]
\centering
\begin{tabular}{l c c c c c c}
\toprule
Resonance & a & b & c & $E_R$ (eV) & $\Gamma$ (eV) \\
\midrule
Bisquare	& 0.15449 & -0.075289 & 0.013296 & 5.1822 & 0.010698 \\
Andrews		& 0.15449 & -0.075290 & 0.013296 & 5.1822 & 0.010698 \\
Cauchy		& 0.15494 & -0.075481 & 0.013317 & 5.1822 & 0.010694 \\
Fair			& 0.14021 & -0.068646 & 0.012532 & 5.1821 & 0.010672 \\ 
Welsch		& 0.14470 & -0.070811 & 0.012789 & 5.1822 & 0.010681 \\
Huber			& 0.14556 & -0.071145 & 0.012820 & 5.1822 & 0.010680 \\
Logistic	& 0.15389 & -0.075046 & 0.013271 & 5.1822 & 0.010702 \\ 
Talwar		& 0.15466 & -0.075358 & 0.013303 & 5.1822 & 0.010697 \\
\bottomrule
\end{tabular}
\caption{Resonance Parameters}
\label{tab:FWaveResonanceFull}
\end{table}


\setlength{\abovecaptionskip}{6pt}   % 0.5cm as an example
\setlength{\belowcaptionskip}{6pt}   % 0.5cm as an example
\begin{table}[H]
\centering
\begin{tabular}{l l l}
\toprule
Method & $^1E_R \text{ (eV)}$ & $^1\Gamma \text{ (eV)}$ \\
\midrule
This work & $5.1867 \pm 0.0021$ & $0.0125 \pm 0.0003$ \\
CC (9Ps9H + H$^-$) \cite{Walters2004} & $5.200$ & $0.0095$ \\
CC (22Ps1H + H$^-$) \cite{Blackwood2002b} & $5.151$ & $0.010$ \\
Complex rotation \cite{Ho2000} & $5.1661 \pm 0.0014$ & $0.0174 \pm 0.0027$  \\
\bottomrule
\end{tabular}
\caption{F-Wave Resonance Parameters}
\label{tab:FWaveResonanceComparisons}
\end{table}




\section{G-Wave}
\label{sec:GWave}

\begin{figure}[H]
	\centering
	\includegraphics[width=7in]{gwave-phases}
	\caption{$^{1,3}$G phase shifts}
	\label{fig:GWavePhase}
\end{figure}

\setlength{\abovecaptionskip}{6pt}
\setlength{\belowcaptionskip}{6pt}
\begin{table}[H]
\centering
\begin{tabular}{c | c c | c c}
\toprule
$\kappa$ & Kohn $\delta_3^+$ & Born $\delta_3^+$ & Kohn $\delta_3^-$ & Born $\delta_3^-$ \\
\midrule
0.2 &	$6.422^{-5}$ & $1.886^{-7}$ & $6.708^{-6}$ & $-1.886^{-7}$ \\
0.3 &	$1.167^{-4}$ & $5.736^{-6}$ & $1.180^{-4}$ & $-5.736^{-6}$ \\
0.4 &	$6.297^{-4}$ & $5.579^{-5}$ & $5.792^{-4}$ & $-5.578^{-5}$ \\
0.5 &	$1.893^{-3}$ & $2.834^{-4}$ & $1.404^{-3}$ & $-2.833^{-4}$ \\
0.6 &	$4.357^{-3}$ & $9.416^{-4}$ & $2.262^{-3}$ & $-9.405^{-4}$ \\
0.7 &	$8.757^{-3}$ & $9.416^{-4}$ & $2.855^{-3}$ & $-2.313^{-3}$ \\
\bottomrule
\end{tabular}
\caption{G-Wave Phase Shifts}
\label{tab:GWavePhase}
\end{table}

Similarly to the F-wave (section \ref{sec:FWave}), the Born approximation 
does not work well for this partial wave. In fact, the G-wave triplet Kohn 
calculation is fully positive, yet the Born approximation is fully negative. 
This gets the physics wrong, as indicated by Bransden and Joachain
\citep[p.589]{Bransden2003}. The Born approximation gives a repulsive potential
($\delta_4^- < 0$), while the Kohn calculation gives an attractive potential
($\delta_4^- > 0$).

For the S-wave through the F-wave, each has at least one resonance before or 
shortly after the Ps(n=2) threshold. For the higher partial waves, we want to 
use the Born approximation (see \ref{sec:Born}). The Born approximation does 
not capture resonance behavior, so we can only do this for partial waves that 
do not contain resonances in this region. As $\ell$ increases, the resonance positions
(in \cref{tab:SWaveResonancesOther,tab:PWaveResonancesOther,tab:DWaveResonancesOther,tab:FWaveResonanceComparisons})
increase, until they are past the threshold fully for the G-wave. Ho and Yan
\cite{Ho2000} calculate the G-wave resonance at \SI{5.486}{eV} with a width of
\SI{0.0109}{eV}.



\section{H-Wave}
\label{sec:HWave}

\begin{figure}[H]
	\centering
	\includegraphics[width=6in]{hwave-phases}
	\caption{H-Wave Phase Shifts}
	\label{fig:HWavePhase}
\end{figure}

\setlength{\abovecaptionskip}{6pt}
\setlength{\belowcaptionskip}{6pt}
\begin{table}[H]
\centering
\begin{tabular}{c | c c | c c}
\toprule
$\kappa$ & Kohn $\delta_3^+$ & Born $\delta_3^+$ & Kohn $\delta_3^-$ & Born $\delta_3^-$ \\
\midrule
0.2 &	$4.522^{-7}$ & $1.863^{-9}$ & $4.703^{-7}$ & $-2.328^{-9}$ \\
0.3 &	$1.627^{-5}$ & $1.518^{-7}$ & $1.728^{-5}$ & $-1.518^{-7}$ \\
0.4 &	$1.256^{-4}$ & $2.523^{-6}$ & $1.328^{-4}$ & $-2.523^{-6}$ \\
0.5 &	$4.280^{-4}$ & $1.906^{-5}$ & $4.332^{-4}$ & $-1.906^{-5}$ \\
0.6 &	$9.902^{-4}$ & $8.596^{-5}$ & $8.769^{-4}$ & $-8.596^{-5}$ \\
0.7 &	$1.963^{-3}$ & $2.693^{-4}$ & $1.372^{-3}$ & $-2.693^{-4}$ \\
\bottomrule
\end{tabular}
\caption{H-Wave Phase Shifts}
\label{tab:HWavePhase}
\end{table}

The Born approximation is also not sufficient for describing the H-wave. Like the F-wave (section \ref{sec:FWave}) and the G-wave (section \ref{sec:HWave}), the triplet is particularly bad, giving the wrong type of potential. The phase shifts are small for this partial wave, so its contribution to the total cross section is minimal.

\textbf{@TODO:} Include cross section graphs.




\biblio
\end{document}