% -*- root: Dissertation.tex -*-
\documentclass[Dissertation.tex]{subfiles} 
\begin{document}

\clearpage
\pagebreak
\newpage

\chapter{Higher Partial Waves}
\label{chp:HigherWaves}



%\todoi{Split up tables into each respective section}
%{
%\renewcommand{\arraystretch}{1.5}
%\begin{table}
%\centering
%\begin{tabular}{l l}
%\toprule\\[-1.2cm]
%Partial Wave & $\SphericalHarmonicY{\ell}{0}{\theta}{\phi}$ \\
%\midrule
%F-Wave & $\sqrt{\frac{7}{16\pi}} \left(5 \cos^3\theta - 3 \cos\theta \right)$ \\
%G-Wave & $\sqrt{\frac{9}{256\pi}} \left(35 \cos^4\theta - 30 \cos^2\theta + 3 \right)$ \\
%H-Wave & $\sqrt{\frac{11}{256\pi}} \left(63 \cos^5\theta - 70 \cos^3\theta + 15 \cos\theta \right)$ \\
%\bottomrule
%\end{tabular}
%\caption{Spherical harmonics for partial waves $\ell = 3$ through 5}
%\label{tab:SphHarm}
%\end{table}
%}
%
%{
%\renewcommand{\arraystretch}{1.5}
%\begin{table}
%\centering
%\begin{tabular}{l l}
%\toprule\\[-1.2cm]
%Partial Wave & $j_\ell(z)$ \\
%\midrule
%F-Wave & $\frac{\left(z^2-15\right) \cos (z)}{z^3}-\frac{3 \left(2 z^2-5\right) \sin (z)}{z^4}$ \\
%G-Wave & $\frac{5 \left(2 z^2-21\right) \cos (z)}{z^4}+\frac{\left(z^4-45 z^2+105\right) \sin (z)}{z^5}$ \\
%H-Wave & $\frac{15 \left(z^4-28 z^2+63\right) \sin (z)}{z^6}+\frac{\left(-z^4+105 z^2-945\right) \cos (z)}{z^5}$ \\
%\bottomrule
%\end{tabular}
%\caption{Spherical Bessel functions for partial waves $\ell = 3$ through 5}
%\label{tab:SphBess}
%\end{table}
%}
%
%{
%\renewcommand{\arraystretch}{1.5}
%\begin{table}
%\centering
%\begin{tabular}{l l}
%\toprule\\[-1.2cm]
%Partial Wave & $n_\ell(z)$ \\
%\midrule
%F-Wave & $\frac{3 \left(2 z^2-5\right) \cos (z)}{z^4}+\frac{\left(z^2-15\right) \sin (z)}{z^3}$ \\
%G-Wave & $\frac{5 \left(2 z^2-21\right) \sin (z)}{z^4}+\frac{\left(-z^4+45 z^2-105\right) \cos (z)}{z^5}$ \\
%H-Wave & $\frac{\left(-z^4+105 z^2-945\right) \sin (z)}{z^5}-\frac{15 \left(z^4-28 z^2+63\right) \cos (z)}{z^6}$ \\
%\bottomrule
%\end{tabular}
%\caption{Spherical Neumann functions for partial waves $\ell = 3$ through 5}
%\label{tab:SphNeum}
%\end{table}
%}

\iftoggle{UNT}{We}{\lettrine{\textcolor{startcolor}{W}}{e}}
necessarily have to truncate the expressions for the cross
sections to use a finite number of partial waves (\cref{chp:CrossSections}).
By the D-wave, the various cross sections have not
fully converged (see \cref{tab:PercentToCross,tab:PercentDiffCrossFull}), so 
we looked at the contributions from higher partial waves.

There is nothing preventing us from extending this to even higher partial 
waves than the H-wave now that we have determined general expressions for the 
external angular integrations in \cref{sec:AngularInt}. However, phase shifts 
for partial waves past the P-wave are not expected to be fully converged due 
to the lack of inclusion of mixed terms (see \cref{sec:MixedTerms}), and this 
effect likely becomes worse as $\ell$ increases until the Born-Oppenheimer 
approximation matches the phase shifts well.


\section{Born-Oppenheimer Approximation}
\label{sec:Born}
In an attempt to approximate the partial waves past the D-wave, we turned to 
the Born-Oppenheimer (BO) approximation \cite{Oppenheimer1928}.
The BO approximation comes from using only the 
first term in \cref{eq:TrialWave,eq:TrialWaveHigher}. Specifically, this is 
done with the Kohn variational method to get an estimate for the $K$-matrix, 
giving \cite[p.590]{Bransden2003}
\begin{equation}
\label{eq:Born}
\tan\delta_\ell \approx -(\widetilde{S}_\ell,\mathcal{L}\widetilde{S}_\ell ) = -(\bar{S}_\ell,\mathcal{L}\bar{S}_\ell ) = \mp(S_\ell,\mathcal{L}S'_\ell ) \,.
\end{equation}
We have also performed a modified BO approximation that uses the first two 
terms ($\widetilde{S}_\ell$ and $\widetilde{C}_\ell$) in \cref{eq:TrialWave,eq:TrialWaveHigher}.
This modified BO approximation is very similar and lines up nearly exactly 
with the BO on most partial waves, so we normally just use the BO
approximation. We also note that the first Born approximation,
$\tan\delta_\ell \approx -(S_\ell,\mathcal{L}S_\ell )$, cannot be used
here due to \cref{eq:SLS0Test}, which would just give 0.

\begin{figure}
	\centering
	\includegraphics[width=5in]{swave-phase-born}
	\caption[$^{1,3}$S complex Kohn and Born comparison]{$^{1,3}$S phase shift comparison between S-matrix complex Kohn and BO approximation}
	\label{fig:SWavePhaseBorn}
\end{figure}

\begin{figure}
	\centering
	\includegraphics[width=5in]{pwave-phase-born}
	\caption[$^{1,3}$P complex Kohn and Born comparison]{$^{1,3}$P phase shift comparison between S-matrix complex Kohn and BO approximation}
	\label{fig:PWavePhaseBorn}
\end{figure}

\begin{figure}
	\centering
	\includegraphics[width=5in]{dwave-phase-born}
	\caption[$^{1,3}$D complex Kohn and Born comparison]{$^{1,3}$D phase shift comparison between $S$-matrix complex Kohn and several approximations}
	\label{fig:DWavePhaseBorn}
\end{figure}

%\todoi{Do we really need to show the S-wave and P-wave?}

These BO approximations were calculated for the first three partial waves 
but showed huge discrepancies, especially for the S-wave, as seen in
\cref{fig:SWavePhaseBorn,fig:PWavePhaseBorn,fig:DWavePhaseBorn}. These somewhat get 
the triplet shapes right, but it would be a poor approximation for these. The 
$^1$S, $^1$P, and $^1$D partial waves have resonances before the Ps(n=2)
threshold, which we would not expect to be represented by these approximations.
The Ganas approximation described in \cref{sec:GanasPhase} is also included in
\cref{fig:DWavePhaseBorn}. This approximation agrees with the $^1$D phase
shifts much better than the BO.

For the $^1$S, $^1$P, $^1$D, and $^1$F partial waves, each has at least one 
resonance before or shortly after the Ps(n=2) threshold. The BO 
approximation does not capture resonance behavior, so we can really only look 
at this for partial waves that do not contain resonances in this region. As
$\ell$ increases, the resonance positions
(in \cref{tab:SWaveResonancesOther,tab:PWaveResonancesOther,tab:DWaveResonancesOther,tab:FWaveResonanceComparisons})
increase, until they are past the threshold fully for the G-wave.
Ho and Yan \cite{Ho2000} calculate the G-wave resonance
at \SI{5.486}{eV} with a width of \SI{0.0109}{eV}.

Due to the obvious discrepancies with even the D-wave, we decided to do a 
full Kohn calculation for the F-wave to compare, again finding that the BO 
approximation does not match up as well as we would like for either
$^1$F or $^3$F. We tried the same for the G-wave and H-wave, with the same results.
The results of using the BO approximation for the higher partial waves are 
shown later in this chapter in
\cref{fig:FWavePhase,fig:GWavePhase,fig:HWavePhase}. The BO approximation 
unfortunately does not represent any of the partial waves through the H-wave 
well for this system. Preliminary investigation shows that this pattern still holds
for the I-wave ($\ell = 6$), indicating that the cross sections will not match
up well for the BO until a much higher value of $\ell$. Bransden and
Jochain \cite[p.590]{Bransden2003} note that if $\ell \gg \kappa a$, where $a$ is
the range of the potential, the BO approximation will match the phase shifts
relatively well, indicating that the range of the Coulomb potentials is
indeed large.


%\todoi{Cross sections with BO}


\section{Ganas Approximation}
\label{sec:GanasPhase}

Ganas \cite{Ganas1972} gives an expression to estimate phase shifts for
$\ell \geq 2$ using a van der Waals ERT (see \cref{sec:vanderWaalsERT}),
which is given in a more convenient form by
Refs.~\cite{Fabrikant2014a,Mitroy2003a,Swann2015} as
\begin{equation}
\label{eq:vdWPhase}
\delta_\ell(\kappa) \simeq \frac{6 \uppi m C_6 \kappa^4}{(2\ell+5)(2\ell+3)(2\ell+1)(2\ell-1)(2\ell-3)}.
\end{equation}
For Ps scattering, with $m = 2$, this is
\begin{equation}
\label{eq:vdWPhase}
\delta_\ell(\kappa) \simeq \frac{12 \uppi C_6 \kappa^4}{(2\ell+5)(2\ell+3)(2\ell+1)(2\ell-1)(2\ell-3)}.
\end{equation}
The van der Waals coefficient for Ps-H scattering is given in
\cref{sec:vanderWaalsERT} as $C_6 = 34.78473$.

This approximation matches surprisingly well to the $^1$D phase shifts in
\cref{fig:DWavePhaseBorn}. \Cref{fig:FWavePhase,fig:GWavePhase,fig:HWavePhase}
show this approximation for the $^1$F-, $^1$G-, and $^1$H-waves, and it normally
gives a better approximation to the phase shifts than the BO. For the H-wave,
it matches relatively well but overestimates the phase shifts.


\section{Gao Approximation}
\label{sec:GaoPhase}

Gao \cite{Gao1998a} provides a QDT expansion for the van der Waals interaction,
somewhat similar to those attempted for the S-wave and P-wave in
\cref{sec:GaoModel}. For $\ell \geq 2$, the $K_\ell^0$ and its derivatives
do not come into play though. This expression is given in terms of $\tan\delta_\ell$:
\begin{equation}
\label{eq:GaoPhase}
\tan\delta_{\ell \geq 2} = (3 \uppi / 32) \{(\ell+1/2) [(\ell+1/2)^2 - 4][(\ell+1/2)^2 - 1]\}^{-1} (\kappa \beta_6)^4,
\end{equation}
where $\beta_6$ is given in \cref{eq:beta6}.

This approximation matches even better to the $^1$D phase shifts than the Ganas
approximation and matches the same as the Ganas approximation for $\ell \geq 3$.
The behavior of all three of these approximations is shown in
\cref{fig:DWavePhaseBorn,fig:FWavePhase,fig:GWavePhase,fig:HWavePhase}.


\section{F-Wave}
\label{sec:FWave}

Similar to the D-wave (see \cref{sec:DWaveNonlinear}), we investigated the
dependence of the F-wave phase shifts on the nonlinear parameters.
After multiple variations of $\alpha$ and
$\beta$ with a fixed $\gamma$, we settled on using the same set of nonlinear
parameters that we used for the D-wave, as seen in \cref{tab:Nonlinear}, but
with the switchover between the two sets at $\kappa = 0.4$.
We found that to keep $R'(5) < 1$ for $^{1,3}$F, we had to use the 
restricted set described in \cref{sec:Restricted} for $\kappa < 0.4$.
We have calculated the F-wave phase shifts for $\omega = 5$ with these nonlinear
parameters for the first two short-range symmetries using the general codes
described in \cref{sec:GeneralLong,sec:GeneralShort}.

\label{sec:FNonlinear}

\subsection{Phase Shifts}
\label{sec:FWavePhase}

\Cref{fig:FWavePhase} shows the phase shifts for the F-wave.
Similar to the D-wave (page \pageref{DWaveSwitch}), the triplet F-wave starts
positive and becomes negative, though at a higher $\kappa$ of about 0.7
(\SI{3.3}{eV}).

We can see that the phase shifts for the (modified) BO approximation do not 
agree very well with the full Kohn calculation, though they follow roughly 
the same shape. The triplet BO is fully negative, while the Kohn only goes 
negative past about $\kappa = 0.7$. The problem with the triplet gets even 
worse for the G-wave (\cref{sec:GWave}) and H-Wave (\cref{sec:HWave}).
The Ganas approximation \cref{sec:GanasPhase}, which does not distinguish
between singlet and triplet states, matches better with the singlet than the
BO approximation but still disagrees near the resonance.

%\todoi{Add Ganas to table?}
%
%\setlength{\abovecaptionskip}{6pt}
%\setlength{\belowcaptionskip}{6pt}
%\begin{table}
%\centering
%\begin{tabular}{c | c c | c c}
%\toprule
%$\kappa$ & Kohn $\delta_3^+$ & Born $\delta_3^+$ & Kohn $\delta_3^-$ & Born $\delta_3^-$ \\
%\midrule
%0.1 &	$1.183^{-6}$ & $1.237^{-7}$ & $1.023^{-6}$ & $-1.232^{-7}$ \\
%0.2 &	$1.046^{-4}$ & $2.003^{-6}$ & $8.282^{-5}$ & $-1.387^{-5}$ \\
%0.3 &	$1.048^{-3}$ & $6.051^{-5}$ & $6.824^{-4}$ & $-1.926^{-4}$ \\
%0.4 &	$4.375^{-3}$ & $5.009^{-4}$ & $1.963^{-3}$ & $-1.092^{-3}$ \\
%0.5 &	$1.186^{-2}$ & $3.768^{-3}$ & $2.941^{-3}$ & $-3.706^{-3}$ \\
%0.6 &	$2.610^{-2}$ & $2.143^{-3}$ & $2.516^{-3}$ & $-8.995^{-3}$ \\
%0.7 &	$4.880^{-2}$ & $1.312^{-2}$ & $4.236^{-4}$ & $-1.725^{-2}$ \\
%\bottomrule
%\end{tabular}
%\caption{F-Wave Phase Shifts}
%\label{tab:FWavePhase}
%\end{table}

\begin{figure}
	\centering
	\includegraphics[width=\textwidth]{fwave-phases}
	\caption{$^{1,3}$F phase shifts}
	\label{fig:FWavePhase}
\end{figure}


\subsection{Resonance}
\label{sec:FWaveResonance}

\begin{figure}
	\centering
	\includegraphics[width=6in]{fwave-phases-full}
	\caption[Full $^1$F phase shifts]{$^1$F phase shifts showing full resonance past the inelastic threshold}
	\label{fig:FWavePhaseFull}
\end{figure}

There is the start of a resonance shortly before the threshold cutoff in 
\cref{fig:FWavePhase}. \Cref{fig:FWavePhaseFull} gives the rough 
plotting past this resonance. As our code does not contain the open channels 
required to extend into the region that contains the full resonance, 
we likely cannot determine the resonance parameters as accurately.
\Cref{tab:FWaveResonanceComparisons} give the resonance 
parameter fittings using our MATLAB script (\cref{sec:ResonanceFit}).
The first two lines only use data before the Ps(n=2) threshold. The next
two lines has the calculated values when we consider data on both sides of the
resonance, in the range of $\kappa = 0.74 - 0.88$. The two sets of $^a$ and
$^b$ agree well and would likely agree better if we considered the multichannel
problem above the Ps(n=2) threshold.
As with the other partial
waves, these resonance parameters compare reasonably well with the complex
rotation \cite{Ho2000}, though there is more discrepancy with this partial
wave, presumably because the resonance is past the inelastic threshold.
The CC results of Ref.~\cite{Walters2004} agree relatively well, but their
resonance position is higher than both the CC and complex Kohn results.
Interestingly, their less accurate 22Ps1H + H$^-$ calculation
\cite{Blackwood2002b} has a resonance position closer to the complex rotation
result.

\setlength{\abovecaptionskip}{6pt}   % 0.5cm as an example
\setlength{\belowcaptionskip}{6pt}   % 0.5cm as an example
\begin{table}
\centering
\begin{tabular}{l l l}
\toprule
Method & $^1E_R \text{ (eV)}$ & $^1\Gamma \text{ (eV)}$ \\
\midrule
Current work$^a$: Average $\pm$ standard deviation & $5.1867 \pm 0.0021$ & $0.0125 \pm 0.0003$ \\
Current work$^a$: $S$-matrix complex Kohn & $5.1863$ & $0.0125$ \\
Current work$^b$: Average $\pm$ standard deviation & $5.1838 \pm 0.0031$ & $0.0114 \pm 0.0015$ \\
Current work$^b$: $S$-matrix complex Kohn & $5.1857$ & $0.0145$ \\
CC (9Ps9H + H$^-$) \cite{Walters2004} & $5.200$ & $0.0095$ \\
CC (22Ps1H + H$^-$) \cite{Blackwood2002b} & $5.151$ & $0.010$ \\
Complex rotation \cite{Ho2000} & $5.1661 \pm 0.0014$ & $0.0174 \pm 0.0027$  \\
\bottomrule
\end{tabular}
\caption[F-wave resonance parameters]{F-wave resonance parameters. $^a$ denotes that the data is only taken before the Ps(n=2) threshold.
$^b$ indicates that data is used from both before and after the threshold, as described in the text.}
\label{tab:FWaveResonanceComparisons}
\end{table}




\section{G-Wave}
\label{sec:GWave}

In an effort to try to improve the convergence ratio, $R'(5)$,
of the low energy phase shifts, we
looked at the $\mu$ nonlinear parameter in the shielding function, given
in \cref{eq:PartialWaveShielding}, along with the $m_\ell$ power in the same
equation. Interestingly, the $\omega = 5$ phase shifts were very stable with
the variation of $\mu$ from 0.5 to 0.8 (with a constant $m_\ell$), agreeing to
five significant figures. Keeping $\mu$ constant and increasing $m_\ell$ from
9 to 13 yielded the same phase shifts, again agreeing to five significant
figures. The convergence ratios are greater than 1 for $^{1,3}$G when
$\kappa < 0.3$, but the phase shifts are very small in this range
($\lesssim 10^{-5}$). As for the F-wave, we also use the D-wave nonlinear
parameters, but the switchover is at $\kappa = 0.45$.


\begin{figure}
	\centering
	\includegraphics[width=\textwidth]{gwave-phases}
	\caption{$^{1,3}$G phase shifts}
	\label{fig:GWavePhase}
\end{figure}

%\setlength{\abovecaptionskip}{6pt}
%\setlength{\belowcaptionskip}{6pt}
%\begin{table}
%\centering
%\begin{tabular}{c | c c | c c}
%\toprule
%$\kappa$ & Kohn $\delta_3^+$ & Born $\delta_3^+$ & Kohn $\delta_3^-$ & Born $\delta_3^-$ \\
%\midrule
%0.2 &	$6.422^{-5}$ & $1.886^{-7}$ & $6.708^{-6}$ & $-1.886^{-7}$ \\
%0.3 &	$1.167^{-4}$ & $5.736^{-6}$ & $1.180^{-4}$ & $-5.736^{-6}$ \\
%0.4 &	$6.297^{-4}$ & $5.579^{-5}$ & $5.792^{-4}$ & $-5.578^{-5}$ \\
%0.5 &	$1.893^{-3}$ & $2.834^{-4}$ & $1.404^{-3}$ & $-2.833^{-4}$ \\
%0.6 &	$4.357^{-3}$ & $9.416^{-4}$ & $2.262^{-3}$ & $-9.405^{-4}$ \\
%0.7 &	$8.757^{-3}$ & $9.416^{-4}$ & $2.855^{-3}$ & $-2.313^{-3}$ \\
%\bottomrule
%\end{tabular}
%\caption{G-Wave Phase Shifts}
%\label{tab:GWavePhase}
%\end{table}

Similar to the F-wave (section \ref{sec:FWave}), the BO approximation 
does not work well for this partial wave. In fact, the G-wave triplet Kohn 
calculation is fully positive, yet the BO approximation is fully negative. 
This gets the physics wrong, as indicated by Bransden and Joachain
\citep[p.589]{Bransden2003}. The BO approximation gives a repulsive potential
($\delta_4^- < 0$), while the Kohn calculation gives an attractive potential
($\delta_4^- > 0$).


\section{H-Wave}
\label{sec:HWave}

As for the F-wave and G-wave, we also use the D-wave nonlinear
parameters, but the switchover is at a higher $\kappa$ of $0.45$.
The convergence ratios are greater than 1 for $^{1,3}$H when
$\kappa < 0.4$, but the phase shifts are very small in this range
($\lesssim 10^{-5}$). 

\begin{figure}
	\centering
	\includegraphics[width=\textwidth]{hwave-phases}
	\caption{H-wave phase shifts}
	\label{fig:HWavePhase}
\end{figure}

%\setlength{\abovecaptionskip}{6pt}
%\setlength{\belowcaptionskip}{6pt}
%\begin{table}
%\centering
%\begin{tabular}{c | c c | c c}
%\toprule
%$\kappa$ & Kohn $\delta_3^+$ & Born $\delta_3^+$ & Kohn $\delta_3^-$ & Born $\delta_3^-$ \\
%\midrule
%0.2 &	$4.522^{-7}$ & $1.863^{-9}$ & $4.703^{-7}$ & $-2.328^{-9}$ \\
%0.3 &	$1.627^{-5}$ & $1.518^{-7}$ & $1.728^{-5}$ & $-1.518^{-7}$ \\
%0.4 &	$1.256^{-4}$ & $2.523^{-6}$ & $1.328^{-4}$ & $-2.523^{-6}$ \\
%0.5 &	$4.280^{-4}$ & $1.906^{-5}$ & $4.332^{-4}$ & $-1.906^{-5}$ \\
%0.6 &	$9.902^{-4}$ & $8.596^{-5}$ & $8.769^{-4}$ & $-8.596^{-5}$ \\
%0.7 &	$1.963^{-3}$ & $2.693^{-4}$ & $1.372^{-3}$ & $-2.693^{-4}$ \\
%\bottomrule
%\end{tabular}
%\caption{H-Wave Phase Shifts}
%\label{tab:HWavePhase}
%\end{table}

\Cref{fig:HWavePhase} shows the $^{1,3}$H phase shifts.
The BO approximation is also not sufficient for describing the H-wave. Like 
the F-wave (section \ref{sec:FWave}) and the G-wave (section \ref{sec:HWave}),
the triplet is particularly bad, giving the wrong type of potential.
The Ganas approximation agrees relatively well with the $^1$H curve. The 
phase shifts are small for this partial wave, so its contribution to the 
integrated cross section is essentially negligible, and its contribution
to the differential cross section is small (see \cref{chp:CrossSections}).


\section{Singlet/Triplet Comparisons}
\label{sec:SingTripCompare}

Interestingly, there appears to be a pattern concerning the difference between
the singlet and triplet phase shifts for Ps-H scattering as $\ell$ increases.
From \cref{fig:FWavePhase,fig:GWavePhase,fig:HWavePhase}, we see that at low
energies, the singlet and triplet phase shifts are nearly the same. The energy
at which the triplet curve diverges from the singlet becomes higher as $\ell$
increases. To this end, I calculated the percent difference between the singlet
and triplet phase shifts for these three partial waves in
\cref{fig:singlet-triplet-compare}. The sharp features at low energy are likely
due to the linear interpolation used and switching between sets of nonlinear 
parameters.

By \SI{0.5}{eV}, the $^1$F and $^3$F phase shifts differ by more than 50\%. The
$^1$G and $^3$G differ by more than 50\% at around \SI{1.75}{eV}, and the $^1$H
and $^3$H phase shifts diverge to this degree at over \SI{3}{eV}. In fact, the
$^1$H and $^3$H phase shifts are approximately the same up to about \SI{1.5}{eV}.
This suggests that at a high enough value of $\ell$ and above, the singlet and
triplet phase shifts are the same in the full energy range below the Ps(n=2)
threshold.

It should be noted that for each of these, the singlet and triplet
BO phase shifts are the opposite sign of each other, but the cross sections
determined from the BO (\cref{sec:Born}) will be the same due to the
$\sin^2 \! \delta_\ell^\pm$ contribution. The Ganas approximation also does
not differentiate between the singlet and triplet.

\begin{figure}
	\centering
	\includegraphics[width=5in]{singlet-triplet-compare}
	\caption[Singlet and triplet higher partial wave comparisons]{Percentage difference comparison between singlet and triplet higher partial waves}
	\label{fig:singlet-triplet-compare}
\end{figure}

This result is also not without precedent. An analysis of two papers on
e$^+$-H scattering \cite{Shertzer1994,Chen1997} shows that the S-wave and
P-wave singlet and triplet phase shifts do not agree, but the D-wave and
F-wave singlet and triplet phase shifts agree well. It is also interesting
that this occurs as low as $\ell = 2$, while for Ps-H scattering, these do
not agree fully for even $\ell = 5$.

%\todoi{Did Peter find this for his work? What about e$^-$-H?}


\section{Summary}
\label{sec:SummaryHigh}

I was able to generalize the evaluation of the matrix elements for arbitrary
$\ell$ (see \cref{chp:General}), which enabled us to calculate phase shifts
for the F-, G-, and H-waves. The phase shifts are not fully converged, but
they are small and generally get smaller as $\ell$ increases. Including the
mixed symmetry terms may allow us to get better converged phase shifts for
these, but the very small phase shifts ($\lesssim 10^{-5}$) at low $\kappa$ 
are not likely to improve without improved numerics as well. We are able to 
calculate the F-wave resonance parameters well, even though the resonance 
lies just past the Ps(n=2) threshold.



\biblio
\end{document}
