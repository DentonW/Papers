% -*- root: Dissertation.tex -*-
\documentclass[Dissertation.tex]{subfiles} 
\begin{document}

\clearpage
\pagebreak
\newpage

\chapter{Higher Partial Waves}
\label{chp:HigherWaves}


\todoi{May want to look at approximation in Eq. 2 of \cite{Mitroy2003a}}
\todoi{Mention $\alpha$, $\beta$, and $\gamma$ used and any other parameters}
\todoi{Talk about how angular integrations done}

\Cref{tab:SphHarm,tab:SphBess,tab:SphNeum} show the spherical harmonics, spherical Bessel and spherical Neumann functions needed in \cref{eq:GeneralWaveTrial} to use the higher partial waves.

\todoi{Split up tables into each respective section}
{
\renewcommand{\arraystretch}{1.5}
\begin{table}[H]
\centering
\begin{tabular}{l l}
\toprule\\[-1.2cm]
Partial Wave & $\SphericalHarmonicY{\ell}{0}{\theta}{\phi}$ \\
\midrule
F-Wave & $\sqrt{\frac{7}{16\pi}} \left(5 \cos^3\theta - 3 \cos\theta \right)$ \\
G-Wave & $\sqrt{\frac{9}{256\pi}} \left(35 \cos^4\theta - 30 \cos^2\theta + 3 \right)$ \\
H-Wave & $\sqrt{\frac{11}{256\pi}} \left(63 \cos^5\theta - 70 \cos^3\theta + 15 \cos\theta \right)$ \\
\bottomrule
\end{tabular}
\caption{Spherical harmonics for partial waves $\ell = 3$ through 5}
\label{tab:SphHarm}
\end{table}
}

{
\renewcommand{\arraystretch}{1.5}
\begin{table}[H]
\centering
\begin{tabular}{l l}
\toprule\\[-1.2cm]
Partial Wave & $j_\ell(z)$ \\
\midrule
F-Wave & $\frac{\left(z^2-15\right) \cos (z)}{z^3}-\frac{3 \left(2 z^2-5\right) \sin (z)}{z^4}$ \\
G-Wave & $\frac{5 \left(2 z^2-21\right) \cos (z)}{z^4}+\frac{\left(z^4-45 z^2+105\right) \sin (z)}{z^5}$ \\
H-Wave & $\frac{15 \left(z^4-28 z^2+63\right) \sin (z)}{z^6}+\frac{\left(-z^4+105 z^2-945\right) \cos (z)}{z^5}$ \\
\bottomrule
\end{tabular}
\caption{Spherical Bessel functions for partial waves $\ell = 3$ through 5}
\label{tab:SphBess}
\end{table}
}

{
\renewcommand{\arraystretch}{1.5}
\begin{table}[H]
\centering
\begin{tabular}{l l}
\toprule\\[-1.2cm]
Partial Wave & $n_\ell(z)$ \\
\midrule
F-Wave & $\frac{3 \left(2 z^2-5\right) \cos (z)}{z^4}+\frac{\left(z^2-15\right) \sin (z)}{z^3}$ \\
G-Wave & $\frac{5 \left(2 z^2-21\right) \sin (z)}{z^4}+\frac{\left(-z^4+45 z^2-105\right) \cos (z)}{z^5}$ \\
H-Wave & $\frac{\left(-z^4+105 z^2-945\right) \sin (z)}{z^5}-\frac{15 \left(z^4-28 z^2+63\right) \cos (z)}{z^6}$ \\
\bottomrule
\end{tabular}
\caption{Spherical Neumann functions for partial waves $\ell = 3$ through 5}
\label{tab:SphNeum}
\end{table}
}

There is nothing preventing us from extending this to even higher partial 
waves than the H-wave now that we have determined general expressions for the 
external angular integrations in \cref{sec:AngularInt}. However, phase shifts 
for partial waves past the P-wave are not expected to be fully converged due 
to the lack of inclusion of mixed terms (see \cref{sec:MixedTerms}), and this 
effect likely becomes worse as $\ell$ increases.

We necessarily have to truncate the expressions for the cross
sections to use a finite number of partial waves (\cref{chp:CrossSections}).
By the D-wave, the various cross sections have not
fully converged (see \cref{tab:PercentToCross,tab:PercentDiffCrossFull}), so 
we looked at the contributions from higher partial waves.


\section{Born Approximation}
\label{sec:Born}
In an attempt to approximate the partial waves past the D-wave, we turned to 
the first Born approximation. The Born approximation comes from using only the 
first term in \cref{eq:TrialWave,eq:TrialWaveHigher}. Specifically, this is 
done with the Kohn variational method to get an estimate for the $K$-matrix, 
giving \cite[p.590]{Bransden2003}
\begin{equation}
\label{eq:Born}
\tan\delta_\ell \approx -(\widetilde{S}_\ell,\mathcal{L}\widetilde{S}_\ell )\,.
\end{equation}
We have performed a modified Born approximation that uses the first two 
terms ($\widetilde{S}$ and $\widetilde{C}$) in \cref{eq:TrialWave,eq:TrialWaveHigher}.
This modified Born approximation is very similar and lines up nearly exactly 
with the Born on most partial waves, so we normally just use the Born 
approximation.

\begin{figure}[H]
	\centering
	\includegraphics[width=5in]{swave-phase-born}
	\caption[$^{1,3}$S complex Kohn and Born comparison]{$^{1,3}$S phase shift comparison between S-matrix complex Kohn and Born approximation}
	\label{fig:SWavePhaseBorn}
\end{figure}

\begin{figure}[H]
	\centering
	\includegraphics[width=5in]{pwave-phase-born}
	\caption[$^{1,3}$P complex Kohn and Born comparison]{$^{1,3}$P phase shift comparison between S-matrix complex Kohn and Born approximation}
	\label{fig:PWavePhaseBorn}
\end{figure}

\begin{figure}[H]
	\centering
	\includegraphics[width=5in]{dwave-phase-born}
	\caption[$^{1,3}$D complex Kohn and Born comparison]{$^{1,3}$D phase shift comparison between S-matrix complex Kohn and Born approximation}
	\label{fig:DWavePhaseBorn}
\end{figure}

\todoi{Show both Born and modified Born}

These Born approximations were calculated for the first three partial waves 
but showed huge discrepancies, especially for the S-wave, as seen in
\cref{fig:SWavePhaseBorn,fig:PWavePhaseBorn,fig:DWavePhaseBorn}. These somewhat get 
the triplet shapes right, but it would be a poor approximation for these. The 
$^1$S, $^1$P, and $^1$D partial waves have resonances before the Ps(n=2)
threshold, which we would not expect to be represented by these approximations.
The Ganas approximation described in \cref{sec:GanasPhase} is also included in
\cref{fig:DWavePhaseBorn}. This approximation agrees with the $^1$D phase
shifts much better than the Born.

For the $^1$S, $^1$P, $^1$D, and $^1$F partial waves, each has at least one 
resonance before or shortly after the Ps(n=2) threshold. The Born 
approximation does not capture resonance behavior, so we can really only look 
at this for partial waves that do not contain resonances in this region. As
$\ell$ increases, the resonance positions
(in \cref{ tab:SWaveResonancesOther,tab:PWaveResonancesOther,tab:DWaveResonancesOther,tab:FWaveResonanceComparisons})
increase, until they are past the threshold fully for the G-wave.
Ho and Yan \cite{Ho2000} calculate the G-wave resonance
at \SI{5.486}{eV} with a width of \SI{0.0109}{eV}.

Due to the obvious discrepancies with even the D-wave, we decided to do a 
full Kohn calculation for the F-wave to compare, again finding that the Born 
approximation does not match up as well as we would like for either
$^1$F or $^3$F. We tried the same for the G-wave and H-wave, with the same results.
The results of using the Born approximation for the higher partial waves are 
shown later in this chapter in
\cref{fig:FWavePhase,fig:GWavePhase,fig:HWavePhase}. The Born approximation 
unfortunately does not represent any of the partial waves through the H-wave 
well for this system. Preliminary investigation shows that this pattern still holds
for the I-wave ($\ell = 6$), indicating that the cross sections will not match
up well for the Born until a much higher value of $\ell$. Bransden and
Jochain \cite[p.590]{Bransden2003} note that if $\ell \gg \kappa a$, where $a$ is
the range of the potential, the Born approximation will match the phase shifts
relatively well, indicating that the range of this potential is
indeed large.


\todoi{Cross sections with Born}


\section{Ganas Approximation}
\label{sec:GanasPhase}

Ganas \cite{Ganas1972} gives an expression to estimate phase shifts for
$\ell \geq 2$ using a van der Waals ERT (see \cref{sec:vanderWaalsERT}),
which is given in a more convenient form by
Refs.~\cite{Fabrikant2014a,Swann2015} as
\begin{equation}
\label{eq:vdWPhase}
\delta_\ell(\kappa) \simeq \frac{6 \uppi m C_6 \kappa^4}{(2\ell+5)(2\ell+3)(2\ell+1)(2\ell-1)(2\ell-3)}.
\end{equation}
For Ps scattering, with $m = 2$, this is
\begin{equation}
\label{eq:vdWPhase}
\delta_\ell(\kappa) \simeq \frac{12 \uppi C_6 \kappa^4}{(2\ell+5)(2\ell+3)(2\ell+1)(2\ell-1)(2\ell-3)}.
\end{equation}
The van der Waals coefficient for Ps-H scattering is given in
\cref{sec:vanderWaalsERT} as $C_6 = 34.78473$.

This approximation matches surprisingly well to the $^1$D phase shifts in
\cref{fig:DWavePhaseBorn}. \Cref{fig:FWavePhase,fig:GWavePhase,fig:HWavePhase}
show this approximation for the $^1$F-, $^1$G-, and $^1$H-waves, and it normally
gives a better approximation to the phase shifts than the Born. For the H-wave,
it matches relatively well but overestimates the phase shifts.


\section{F-Wave}
\label{sec:FWave}

Similar to the D-wave (see \cref{sec:DWaveNonlinear}), we investigated the
nonlinear parameters for the F-wave. After multiple variations of $\alpha$ and
$\beta$ with a fixed $\gamma$, we settled on using the same set of nonlinear
parameters that we used for the D-wave, as seen in \cref{tab:Nonlinear}.
We have calculated the F-wave phase shifts for $\omega = 5$ with these nonlinear
parameters for the first two short-range symmetries using the general codes
described in \cref{sec:GeneralLong,sec:GeneralShort}. 

\label{sec:FNonlinear}

\subsection{Phase Shifts}
\label{sec:FWavePhase}

\todoi{Note how goes negative for triplet like $^3$D}

\begin{figure}[H]
	\centering
	\includegraphics[width=\textwidth]{fwave-phases}
	\caption{$^{1,3}$F phase shifts}
	\label{fig:FWavePhase}
\end{figure}

\Cref{fig:FWavePhase} and \cref{tab:FWavePhase} show the phase shifts for the F-wave.

\setlength{\abovecaptionskip}{6pt}
\setlength{\belowcaptionskip}{6pt}
\begin{table}[H]
\centering
\begin{tabular}{c | c c | c c}
\toprule
$\kappa$ & Kohn $\delta_3^+$ & Born $\delta_3^+$ & Kohn $\delta_3^-$ & Born $\delta_3^-$ \\
\midrule
0.1 &	$1.183^{-6}$ & $1.237^{-7}$ & $1.023^{-6}$ & $-1.232^{-7}$ \\
0.2 &	$1.046^{-4}$ & $2.003^{-6}$ & $8.282^{-5}$ & $-1.387^{-5}$ \\
0.3 &	$1.048^{-3}$ & $6.051^{-5}$ & $6.824^{-4}$ & $-1.926^{-4}$ \\
0.4 &	$4.375^{-3}$ & $5.009^{-4}$ & $1.963^{-3}$ & $-1.092^{-3}$ \\
0.5 & $1.186^{-2}$ & $3.768^{-3}$ & $2.941^{-3}$ & $-3.706^{-3}$ \\
0.6 &	$2.610^{-2}$ & $2.143^{-3}$ & $2.516^{-3}$ & $-8.995^{-3}$ \\
0.7 &	$4.880^{-2}$ & $1.312^{-2}$ & $4.236^{-4}$ & $-1.725^{-2}$ \\
\bottomrule
\end{tabular}
\caption{F-Wave Phase Shifts}
\label{tab:FWavePhase}
\end{table}

We can see that the phase shifts for the (modified) Born approximation do not 
agree very well with the full Kohn calculation, though they follow roughly 
the same shape. The triplet Born is fully negative, while the Kohn only goes 
negative past about $\kappa = 0.7$ (\SI{3.3}{eV}). The problem with the triplet gets even 
worse for the G-wave (\cref{sec:GWave}) and H-Wave (\cref{sec:HWave}).

\begin{figure}[H]
	\centering
	\includegraphics[width=6in]{fwave-phases-full}
	\caption[Full $^1$F phase shifts]{$^1$F phase shifts showing full resonance past the inelastic threshold}
	\label{fig:FWavePhaseFull}
\end{figure}

\subsection{Resonance}
\label{sec:FWaveResonance}

There is the start of a resonance shortly before the threshold cutoff in 
\cref{fig:FWavePhase}. \Cref{fig:FWavePhaseFull} gives the rough 
plotting past this resonance. As our code does not contain the open channels 
required to extend into the region that contains the rest of the resonance, 
we likely cannot measure the resonance parameters as accurately.
\Cref{tab:FWaveResonanceBefore,tab:FWaveResonanceFull} give the resonance 
parameter fittings using our MATLAB script (\cref{sec:ResonanceFit}).

\Cref{tab:FWaveResonanceBefore} shows the calculated resonance 
parameters using only data before the Ps(n=2) threshold. This is the data 
that is plotted in \cref{fig:FWavePhase}. \Cref{tab:FWaveResonanceFull} has
the calculated values when we consider data through the resonance,
in the range of $\kappa = 0.74 - 0.88$. This data is plotted in
\cref{fig:FWavePhaseFull}. This calculated position of \SI{5.182}{eV} is
close to the \SI{5.166}{eV} that Ho and Yan compute using the complex rotation
method \cite{Ho2000}. As with the other partial waves, our width of
\SI{0.0107}{eV} is close to their value of \SI{0.0174}{eV} but does not 
compare exactly.


\setlength{\abovecaptionskip}{6pt}
\setlength{\belowcaptionskip}{6pt}
\begin{table}[H]
\centering
\begin{tabular}{l c c c c c c}
\toprule
Fitting & a & b & c & $E_R$ (eV) & $\Gamma$ (eV) \\
\midrule
Bisquare	& 0.033382 & -0.017062 & 0.0062664 & 5.1913 & 0.013232 \\
Andrews		& 0.033390 & -0.017066 & 0.0062668 & 5.1913 & 0.013232 \\
Cauchy		& 0.037941 & -0.019297 & 0.0065459 & 5.1899 & 0.012976 \\
Fair		& 0.038145 & -0.019401 & 0.0065597 & 5.1898 & 0.012956 \\
Welsch		& 0.038211 & -0.019432 & 0.0065632 & 5.1898 & 0.012964 \\
Huber		& 0.038173 & -0.019410 & 0.0065601 & 5.1898 & 0.012962 \\
Logistic	& 0.037609 & -0.019191 & 0.0065409 & 5.1895 & 0.012921 \\
Talwar		& 0.033688 & -0.017208 & 0.0062841 & 5.1913 & 0.013220 \\
\bottomrule
\end{tabular}
\caption{Resonance Parameters Before Threshold}
\label{tab:FWaveResonanceBefore}
\end{table}


\setlength{\abovecaptionskip}{6pt}
\setlength{\belowcaptionskip}{6pt}
\begin{table}[H]
\centering
\begin{tabular}{l c c c c c c}
\toprule
Resonance & a & b & c & $E_R$ (eV) & $\Gamma$ (eV) \\
\midrule
Bisquare	& 0.15449 & -0.075289 & 0.013296 & 5.1822 & 0.010698 \\
Andrews		& 0.15449 & -0.075290 & 0.013296 & 5.1822 & 0.010698 \\
Cauchy		& 0.15494 & -0.075481 & 0.013317 & 5.1822 & 0.010694 \\
Fair			& 0.14021 & -0.068646 & 0.012532 & 5.1821 & 0.010672 \\ 
Welsch		& 0.14470 & -0.070811 & 0.012789 & 5.1822 & 0.010681 \\
Huber			& 0.14556 & -0.071145 & 0.012820 & 5.1822 & 0.010680 \\
Logistic	& 0.15389 & -0.075046 & 0.013271 & 5.1822 & 0.010702 \\ 
Talwar		& 0.15466 & -0.075358 & 0.013303 & 5.1822 & 0.010697 \\
\bottomrule
\end{tabular}
\caption{Resonance Parameters}
\label{tab:FWaveResonanceFull}
\end{table}


\setlength{\abovecaptionskip}{6pt}   % 0.5cm as an example
\setlength{\belowcaptionskip}{6pt}   % 0.5cm as an example
\begin{table}[H]
\centering
\begin{tabular}{l l l}
\toprule
Method & $^1E_R \text{ (eV)}$ & $^1\Gamma \text{ (eV)}$ \\
\midrule
This work & $5.1867 \pm 0.0021$ & $0.0125 \pm 0.0003$ \\
CC (9Ps9H + H$^-$) \cite{Walters2004} & $5.200$ & $0.0095$ \\
CC (22Ps1H + H$^-$) \cite{Blackwood2002b} & $5.151$ & $0.010$ \\
Complex rotation \cite{Ho2000} & $5.1661 \pm 0.0014$ & $0.0174 \pm 0.0027$  \\
\bottomrule
\end{tabular}
\caption{F-Wave Resonance Parameters}
\label{tab:FWaveResonanceComparisons}
\end{table}




\section{G-Wave}
\label{sec:GWave}

In an effort to try to improve the convergence ratio, $\mathcal{R}(5)$,
of the low energy phase shifts, we
looked at the $\mu$ nonlinear parameter in the shielding function, given
in \cref{eq:PartialWaveShielding}, along with the $m_\ell$ power in the same
equation. Interestingly, the $\omega = 5$ phase shifts were very stable with
the variation of $\mu$ from 0.5 to 0.8 (with a constant $m_\ell$), agreeing to
five significant figures. Keeping $\mu$ constant and increasing $m_\ell$ from
9 to 13 yielded the same phase shifts, again agreeing to five significant
figures. 


\begin{figure}[H]
	\centering
	\includegraphics[width=\textwidth]{gwave-phases}
	\caption{$^{1,3}$G phase shifts}
	\label{fig:GWavePhase}
\end{figure}

\setlength{\abovecaptionskip}{6pt}
\setlength{\belowcaptionskip}{6pt}
\begin{table}[H]
\centering
\begin{tabular}{c | c c | c c}
\toprule
$\kappa$ & Kohn $\delta_3^+$ & Born $\delta_3^+$ & Kohn $\delta_3^-$ & Born $\delta_3^-$ \\
\midrule
0.2 &	$6.422^{-5}$ & $1.886^{-7}$ & $6.708^{-6}$ & $-1.886^{-7}$ \\
0.3 &	$1.167^{-4}$ & $5.736^{-6}$ & $1.180^{-4}$ & $-5.736^{-6}$ \\
0.4 &	$6.297^{-4}$ & $5.579^{-5}$ & $5.792^{-4}$ & $-5.578^{-5}$ \\
0.5 &	$1.893^{-3}$ & $2.834^{-4}$ & $1.404^{-3}$ & $-2.833^{-4}$ \\
0.6 &	$4.357^{-3}$ & $9.416^{-4}$ & $2.262^{-3}$ & $-9.405^{-4}$ \\
0.7 &	$8.757^{-3}$ & $9.416^{-4}$ & $2.855^{-3}$ & $-2.313^{-3}$ \\
\bottomrule
\end{tabular}
\caption{G-Wave Phase Shifts}
\label{tab:GWavePhase}
\end{table}

Similar to the F-wave (section \ref{sec:FWave}), the Born approximation 
does not work well for this partial wave. In fact, the G-wave triplet Kohn 
calculation is fully positive, yet the Born approximation is fully negative. 
This gets the physics wrong, as indicated by Bransden and Joachain
\citep[p.589]{Bransden2003}. The Born approximation gives a repulsive potential
($\delta_4^- < 0$), while the Kohn calculation gives an attractive potential
($\delta_4^- > 0$).





\section{H-Wave}
\label{sec:HWave}

\begin{figure}[H]
	\centering
	\includegraphics[width=\textwidth]{hwave-phases}
	\caption{H-Wave Phase Shifts}
	\label{fig:HWavePhase}
\end{figure}

\setlength{\abovecaptionskip}{6pt}
\setlength{\belowcaptionskip}{6pt}
\begin{table}[H]
\centering
\begin{tabular}{c | c c | c c}
\toprule
$\kappa$ & Kohn $\delta_3^+$ & Born $\delta_3^+$ & Kohn $\delta_3^-$ & Born $\delta_3^-$ \\
\midrule
0.2 &	$4.522^{-7}$ & $1.863^{-9}$ & $4.703^{-7}$ & $-2.328^{-9}$ \\
0.3 &	$1.627^{-5}$ & $1.518^{-7}$ & $1.728^{-5}$ & $-1.518^{-7}$ \\
0.4 &	$1.256^{-4}$ & $2.523^{-6}$ & $1.328^{-4}$ & $-2.523^{-6}$ \\
0.5 &	$4.280^{-4}$ & $1.906^{-5}$ & $4.332^{-4}$ & $-1.906^{-5}$ \\
0.6 &	$9.902^{-4}$ & $8.596^{-5}$ & $8.769^{-4}$ & $-8.596^{-5}$ \\
0.7 &	$1.963^{-3}$ & $2.693^{-4}$ & $1.372^{-3}$ & $-2.693^{-4}$ \\
\bottomrule
\end{tabular}
\caption{H-Wave Phase Shifts}
\label{tab:HWavePhase}
\end{table}

The Born approximation is also not sufficient for describing the H-wave. Like the F-wave (section \ref{sec:FWave}) and the G-wave (section \ref{sec:HWave}), the triplet is particularly bad, giving the wrong type of potential. The phase shifts are small for this partial wave, so its contribution to the total cross section is minimal.

\todoi{Include cross section graphs?}


\section{Singlet/Triplet Comparisons}
\label{sec:SingTripCompare}
\todoi{Peter's email on 04-24-2015} % (\url{https://mail.google.com/mail/u/0/#inbox/14ceaf6c764393d8})}
comparison of e$^+$-H scattering
\cite{Chen1997} \cite{Shertzer1994}

\begin{figure}[H]
	\centering
	\includegraphics[width=5in]{singlet-triplet-compare}
	\caption[Singlet and triplet higher partial wave comparisons]{Percentage difference comparison between singlet and triplet higher partial waves}
	\label{fig:singlet-triplet-compare}
\end{figure}


\biblio
\end{document}