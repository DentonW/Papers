\documentclass[Dissertation.tex]{subfiles} 
\begin{document}

\clearpage
\pagebreak
\newpage

\chapter{Higher Partial Waves}
\label{chp:HigherWaves}


\section{Resonances}
For the S-wave through the D-wave, each has at least one resonance before the Ps(n=2) threshold. For the higher partial waves, we want to use the Born approximation (see \ref{sec:Born}). The Born approximation does not capture resonance behavior, so we can only do this for partial waves that do not contain resonances in this region. So far, the only publication that has calculated resonance parameters for the F-wave and G-wave is from Ho and Yan \cite{Ho2000}. Their calculated resonance parameters follow in table \ref{tab:HigherResonancesOther} (converted to eV). The Ps(n=2) threshold is at 5.10184 eV. Note that as L increases, the resonance positions (in tables \ref{tab:SWaveResonances}, \ref{tab:PWaveResonances} and \ref{tab:DWaveResonances}) increase, until they are past the threshold for the F-wave and G-wave. 

\setlength{\abovecaptionskip}{6pt}
\setlength{\belowcaptionskip}{6pt}
\begin{table}[H]
\centering
\begin{tabular}{c c c}
\toprule
Partial Wave & $E_R \text{ (eV)}$ & $\Gamma \text{ (eV)}$ \\
\midrule
F-Wave & $5.166$ & $0.0174$ \\
G-Wave & $5.486$ & $0.0109$ \\
\bottomrule
\end{tabular}
\caption{Higher Partial Wave Resonance Parameters} % title of Table
\label{tab:HigherResonancesOther}
\end{table}


\section{F-Wave}
\label{sec:FWave}

\begin{figure}[H]
	\centering
	\includegraphics[width=7in]{fwave-phases}
	\caption{$^{1,3}$F phase shifts}
	\label{fig:FWavePhase}
\end{figure}

We have calculated the F-wave phase shifts for $\omega = 5$ with $\alpha = 0.5$, $\beta = 0.6$, $\gamma = 1.1$ and $\mu = 0.7$ for the first two short-range symmetries using the general codes described in sections \ref{sec:GeneralLong} and \ref{sec:GeneralShort}. Figure \ref{fig:FWavePhase} and table \ref{tab:FWavePhase} show the phase shifts for the F-wave.

\setlength{\abovecaptionskip}{6pt}
\setlength{\belowcaptionskip}{6pt}
\begin{table}[H]
\centering
\begin{tabular}{c | c c | c c}
\toprule
$\kappa$ & Kohn $\delta_3^+$ & Born $\delta_3^+$ & Kohn $\delta_3^-$ & Born $\delta_3^-$ \\
\midrule
0.1 &	$1.183^{-6}$ & $1.237^{-7}$ & $1.023^{-6}$ & $-1.232^{-7}$ \\
0.2 &	$1.046^{-4}$ & $2.003^{-6}$ & $8.282^{-5}$ & $-1.387^{-5}$ \\
0.3 &	$1.048^{-3}$ & $6.051^{-5}$ & $6.824^{-4}$ & $-1.926^{-4}$ \\
0.4 &	$4.375^{-3}$ & $5.009^{-4}$ & $1.963^{-3}$ & $-1.092^{-3}$ \\
0.5 & $1.186^{-2}$ & $3.768^{-3}$ & $2.941^{-3}$ & $-3.706^{-3}$ \\
0.6 &	$2.610^{-2}$ & $2.143^{-3}$ & $2.516^{-3}$ & $-8.995^{-3}$ \\
0.7 &	$4.880^{-2}$ & $1.312^{-2}$ & $4.236^{-4}$ & $-1.725^{-2}$ \\
\bottomrule
\end{tabular}
\caption{F-Wave Phase Shifts}
\label{tab:FWavePhase}
\end{table}

We can see that the phase shifts for the (modified) Born approximation do not agree very well with the full Kohn calculation, though they follow roughly the same shape. The triplet Born is fully negative, while the Kohn only goes negative past about $\kappa = 0.7$. The problem with the triplet gets even worse for the G-wave (section \ref{sec:GWave}) and H-Wave (section \ref{sec:HWave}).

\begin{figure}[H]
	\centering
	\includegraphics[width=6in]{fwave-phases-full}
	\caption{F-Wave Phase Shifts}
	\label{fig:FWavePhaseFull}
\end{figure}

There is the start of a resonance shortly before the threshold cutoff in figure \ref{fig:FWavePhase}. Figure \ref{fig:FWavePhaseFull} gives the rough plotting past this resonance. As our code does not contain the open channels required to extend into the region that contains the rest of the resonance, we likely cannot measure the resonance parameters as accurately. Tables \ref{tab:FWaveResonanceBefore} and \ref{tab:FWaveResonanceFull} give the resonance parameter fittings using our MATLAB script (\ref{?}).

Table \ref{tab:FWaveResonanceBefore} shows the calculated resonance parameters using only data before the Ps(n=2) threshold. This is the data that is plotted in figure \ref{fig:FWavePhase}. Table \ref{tab:FWaveResonanceFull} has the calculated values when we consider data through the resonance, in the range of $\kappa = 0.74 - 0.88$. This data is plotted in figure \ref{fig:FWavePhaseFull}. This calculated position of 5.182 eV is very close to the 5.166 eV that Ho and Yan compute using the complex rotation method \cite{Ho2000}. As with the other partial waves, our width of 0.0107 eV is close to their value of 0.0174 eV but does not compare exactly.


\setlength{\abovecaptionskip}{6pt}
\setlength{\belowcaptionskip}{6pt}
\begin{table}[H]
\centering
\begin{tabular}{l c c c c c c}
\toprule
Fitting & a & b & c & $E_R$ (eV) & $\Gamma$ (eV) \\
\midrule
Bisquare	& 0.033382 & -0.017062 & 0.0062664 & 5.1913 & 0.013232 \\
Andrews		& 0.033390 & -0.017066 & 0.0062668 & 5.1913 & 0.013232 \\
Cauchy		& 0.037941 & -0.019297 & 0.0065459 & 5.1899 & 0.012976 \\
Fair			& 0.038145 & -0.019401 & 0.0065597 & 5.1898 & 0.012956 \\
Welsch		& 0.038211 & -0.019432 & 0.0065632 & 5.1898 & 0.012964 \\
Huber			& 0.038173 & -0.019410 & 0.0065601 & 5.1898 & 0.012962 \\
Logistic	& 0.037609 & -0.019191 & 0.0065409 & 5.1895 & 0.012921 \\
Talwar		& 0.033688 & -0.017208 & 0.0062841 & 5.1913 & 0.013220 \\
\bottomrule
\end{tabular}
\caption{Resonance Parameters Before Threshold} % title of Table
\label{tab:FWaveResonanceBefore}
\end{table}


\setlength{\abovecaptionskip}{6pt}
\setlength{\belowcaptionskip}{6pt}
\begin{table}[H]
\centering
\begin{tabular}{l c c c c c c}
\toprule
Resonance & a & b & c & $E_R$ (eV) & $\Gamma$ (eV) \\
\midrule
Bisquare	& 0.15449 & -0.075289 & 0.013296 & 5.1822 & 0.010698 \\
Andrews		& 0.15449 & -0.075290 & 0.013296 & 5.1822 & 0.010698 \\
Cauchy		& 0.15494 & -0.075481 & 0.013317 & 5.1822 & 0.010694 \\
Fair			& 0.14021 & -0.068646 & 0.012532 & 5.1821 & 0.010672 \\ 
Welsch		& 0.14470 & -0.070811 & 0.012789 & 5.1822 & 0.010681 \\
Huber			& 0.14556 & -0.071145 & 0.012820 & 5.1822 & 0.010680 \\
Logistic	& 0.15389 & -0.075046 & 0.013271 & 5.1822 & 0.010702 \\ 
Talwar		& 0.15466 & -0.075358 & 0.013303 & 5.1822 & 0.010697 \\
\bottomrule
\end{tabular}
\caption{Resonance Parameters} % title of Table
\label{tab:FWaveResonanceFull}
\end{table}


\setlength{\abovecaptionskip}{6pt}   % 0.5cm as an example
\setlength{\belowcaptionskip}{6pt}   % 0.5cm as an example
\begin{table}[H]
\centering
\begin{tabular}{l c c}
\toprule
Method & $^1E_R \text{ (eV)}$ & $^1\Gamma \text{ (eV)}$ \\
\midrule
Kohn $(\omega = 5)$ & $5.1822$ & $0.0107$ \\
Complex rotation (Ho and Yan 2000) \cite{Ho2000} & $5.1661 \pm 0.0014$ & $0.0174 \pm 0.0027$  \\
Close coupling (22Ps1H + H$^-$ \emph{et al} 2002) \cite{Blackwood2002b} & $5.151$ & $0.010$ \\
Close coupling (Walters \emph{et al} 2004 \cite{Walters2004}) & $5.200$ & $0.0095$ \\
\bottomrule
\end{tabular}
\caption{F-Wave Resonance Parameters} % title of Table
\label{tab:FWaveResonanceComparisons}
\end{table}




\section{G-Wave}
\label{sec:GWave}

\begin{figure}[H]
	\centering
	\includegraphics[width=7in]{gwave-phases}
	\caption{$^{1,3}$G phase shifts}
	\label{fig:GWavePhase}
\end{figure}

\setlength{\abovecaptionskip}{6pt}
\setlength{\belowcaptionskip}{6pt}
\begin{table}[H]
\centering
\begin{tabular}{c | c c | c c}
\toprule
$\kappa$ & Kohn $\delta_3^+$ & Born $\delta_3^+$ & Kohn $\delta_3^-$ & Born $\delta_3^-$ \\
\midrule
0.2 &	$6.422^{-5}$ & $1.886^{-7}$ & $6.708^{-6}$ & $-1.886^{-7}$ \\
0.3 &	$1.167^{-4}$ & $5.736^{-6}$ & $1.180^{-4}$ & $-5.736^{-6}$ \\
0.4 &	$6.297^{-4}$ & $5.579^{-5}$ & $5.792^{-4}$ & $-5.578^{-5}$ \\
0.5 &	$1.893^{-3}$ & $2.834^{-4}$ & $1.404^{-3}$ & $-2.833^{-4}$ \\
0.6 &	$4.357^{-3}$ & $9.416^{-4}$ & $2.262^{-3}$ & $-9.405^{-4}$ \\
0.7 &	$8.757^{-3}$ & $9.416^{-4}$ & $2.855^{-3}$ & $-2.313^{-3}$ \\
\bottomrule
\end{tabular}
\caption{G-Wave Phase Shifts}
\label{tab:GWavePhase}
\end{table}

Similarly to the F-wave (section \ref{sec:FWave}), the Born approximation does not work well for this partial wave. In fact, the G-wave triplet Kohn calculation is fully positive, yet the Born approximation is fully negative. This gets the physics wrong, as indicated by Bransden and Joachain \citep[pg. 589]{Bransden2003}. The Born approximation gives a repulsive potential ($\delta_3^- < 0$), while the Kohn calculation gives an attractive potential ($\delta_3^- > 0$).


\section{H-Wave}
\label{sec:HWave}

\begin{figure}[H]
	\centering
	\includegraphics[width=6in]{hwave-phases}
	\caption{H-Wave Phase Shifts}
	\label{fig:HWavePhase}
\end{figure}

\setlength{\abovecaptionskip}{6pt}
\setlength{\belowcaptionskip}{6pt}
\begin{table}[H]
\centering
\begin{tabular}{c | c c | c c}
\toprule
$\kappa$ & Kohn $\delta_3^+$ & Born $\delta_3^+$ & Kohn $\delta_3^-$ & Born $\delta_3^-$ \\
\midrule
0.2 &	$4.522^{-7}$ & $1.863^{-9}$ & $4.703^{-7}$ & $-2.328^{-9}$ \\
0.3 &	$1.627^{-5}$ & $1.518^{-7}$ & $1.728^{-5}$ & $-1.518^{-7}$ \\
0.4 &	$1.256^{-4}$ & $2.523^{-6}$ & $1.328^{-4}$ & $-2.523^{-6}$ \\
0.5 &	$4.280^{-4}$ & $1.906^{-5}$ & $4.332^{-4}$ & $-1.906^{-5}$ \\
0.6 &	$9.902^{-4}$ & $8.596^{-5}$ & $8.769^{-4}$ & $-8.596^{-5}$ \\
0.7 &	$1.963^{-3}$ & $2.693^{-4}$ & $1.372^{-3}$ & $-2.693^{-4}$ \\
\bottomrule
\end{tabular}
\caption{H-Wave Phase Shifts}
\label{tab:HWavePhase}
\end{table}

The Born approximation is also not sufficient for describing the H-wave. Like the F-wave (section \ref{sec:FWave}) and the G-wave (section \ref{sec:HWave}), the triplet is particularly bad, giving the wrong type of potential. The phase shifts are small for this partial wave, so its contribution to the total cross section is minimal.

\textbf{@TODO:} Include cross section graphs.


\section{Resonances}
\label{sec:AllResonances}

\setlength{\abovecaptionskip}{6pt}   % 0.5cm as an example
\setlength{\belowcaptionskip}{6pt}   % 0.5cm as an example
% Similar to the table in Walters 2004
\begin{table}[H]
\centering
\begin{tabular}{l l l l}
\toprule
 & & & Yan and Ho \cite{Yan1999, Yan1998a} \\
Resonance & Current Kohn & Walters \emph{et al} \cite{Walters2004} & Ho and Yan \cite{Ho1998, Ho2000} \\
\midrule
S(1) & $4.0061$ & $4.149$ & $4.0058 \pm 0.0005$ \\
     & $(0.0955)$ & $(0.103)$ & $(0.0952 \pm 0.0011)$ \\
\\
S(2) & $5.0277$ & $4.877$ & $4.9479 \pm 0.0014$ \\
     & $(0.0608)$ & $(0.0164)$ & $(0.0585 \pm 0.0027)$ \\
\\
P(1) & $4.2856$ & $4.475$ & $4.2850 \pm 0.0014$ \\
     & $(0.0445)$ & $(0.0827)$ & $(0.0435 \pm 0.0027)$ \\
\\
P(2) & $5.0581$ & $4.905$ & $5.0540 \pm 0.0027$ \\
     & $(0.0456)$ & $(0.0043)$ & $(0.0585 \pm 0.0054)$ \\
\\
D(1) & $4.724$ & $4.899$ & $4.710 \pm 0.0027$ \\
     & $(0.0819)$ & $(0.0872)$ & $(0.0925 \pm 0.0054)$ \\
\\
F(1) & $5.1822$ & $5.200$ & $5.1661 \pm 0.0014$ \\
     & $(0.0107)$ & $(0.0095)$ & $(0.00174 \pm 0.0027)$ \\
\bottomrule
\end{tabular}
\caption{All resonance parameters (in eV)} % title of Table
\label{tab:AllResonanceComparisons}
\end{table}


\section{Total Cross Sections}
\label{sec:totalcross}

\begin{equation}
\label{eq:TotalCross}
\sigma_{el}^\pm = \frac{4}{\kappa^2} \sum_{\ell=0}^\infty (2\ell+1) \sin^2 \delta_\ell^\pm
\end{equation}

\begin{figure}[H]
	\centering
	\includegraphics[width=6in]{singlet-cross-sections}
	\caption{Combined complex Kohn singlet cross sections}
	\label{fig:singlet-cross-sections}
\end{figure}

\begin{figure}[H]
	\centering
	\includegraphics[width=6in]{triplet-cross-sections}
	\caption{Combined complex Kohn triplet cross sections}
	\label{fig:triplet-cross-sections}
\end{figure}

\begin{figure}[H]
	\centering
	\includegraphics[width=7in]{percentage-cross-sections}
	\caption[Percentage contribution to total cross section]{Percentage contribution to total cross section from each partial wave for singlet (a) and triplet (b).}
	\label{fig:percentage-cross-sections}
\end{figure}

\begin{figure}[H]
	\centering
	\includegraphics[width=7in]{percentage-cross-sections-full}
	\caption[Percentage contribution to total cross section]{Percentage contribution to total cross section from each partial wave for singlet (a) and triplet (b).}
	\label{fig:percentage-cross-sections-full}
\end{figure}

\begin{figure}[H]
	\centering
	\includegraphics[width=6in]{combined-cross-sections}
	\caption{Cross sections. CC results are from Ref. \cite{Walters2004}, and SE results are from Ref. \cite{Hara1975}.}
	\label{fig:combined-cross-sections}
\end{figure}


For the total, differential and momentum transfer cross sections, the singlet contributes $\frac{1}{4}$, and the triplet contributes $\frac{3}{4}$, giving spin-weighted cross sections, i.e. \cite{Ward1987}
\beq
\label{eq:SpinWeightCS}
\sigma = \tfrac{1}{4} \sigma^+ + \tfrac{3}{4} \sigma^-.
\eeq


\label{sec:crosscompare}
\begin{figure}[H]
	\centering
	\includegraphics[width=7in]{spd-singlet-cross-sections}
	\caption{Comparisons of $^1$S, $^1$P, and $^1$D cross sections. CC results are from Ref. \cite{Walters2004}.}
	\label{fig:spd-singlet-cross-sections}
\end{figure}

\begin{figure}[H]
	\centering
	\includegraphics[width=7in]{sp-triplet-cross-sections}
	\caption{Comparisons of $^1$S cross sections. CC results are from Ref. \cite{Walters2004}.}
	\label{fig:sp-triplet-cross-sections}
\end{figure}


\subsection{Differential Cross Sections}
\label{sec:totalcross}

\begin{align}
\label{eq:DiffCross}
\nonumber \frac{d\sigma_{el}^\pm}{d\Omega} = \frac{1}{\kappa^2} & \sum_{\ell=0}^\infty \sum_{\ell^\prime=0}^\infty (2\ell+1)(2\ell^\prime+1) \exp\left\{\ii \left[\delta_\ell(\kappa) - \delta_{\ell^\prime}(\kappa) \right] \right\} \\
& \times \sin\delta_\ell^\pm(\kappa) \sin\delta_{\ell^\prime}^\pm(\kappa) P_\ell(\cos\theta) P_{\ell^\prime}(\cos\theta)\,.
\end{align}

\begin{figure}[H]
	\centering
	\includegraphics[width=5.25in]{diff-cross-sections-singlet}
	\caption{Singlet differential cross sections}
	\label{fig:diff-cross-sections-singlet}
\end{figure}

\begin{figure}[H]
	\centering
	\includegraphics[width=5.25in]{diff-cross-sections-triplet}
	\caption{Triplet differential cross sections}
	\label{fig:diff-cross-sections-triplet}
\end{figure}

\begin{figure}[H]
	\centering
	\includegraphics[width=7in]{diff-cross-sections-combined}
	\caption{Combined differential cross sections}
	\label{fig:diff-cross-sections-combined}
\end{figure}

\begin{figure}[H]
	\centering
	\includegraphics[width=7in]{percent-diff-cross-sections}
	\caption[Percent difference of differential cross sections at selected angles]{Percent difference of $\frac{d\sigma_{el}}{d\Omega}$ for upper limit of summations in \ref{eq:DiffCross} as $\ell = 4$ versus $\ell = 5$ for selected angles}
	\label{fig:percent-diff-cross-sections}
\end{figure}

\begin{figure}[H]
	\centering
	\includegraphics[width=5.5in]{percent-diff-cross-sections-full}
	\caption[Percent difference of differential cross sections at all angles]{Percent difference of $\frac{d\sigma_{el}}{d\Omega}$ for upper limit of summations in \ref{eq:DiffCross} as $\ell = 4$ versus $\ell = 5$ for all angles and energies}
	\label{fig:percent-diff-cross-sections}
\end{figure}

\begin{figure}[H]
	\centering
	\includegraphics[width=5.25in]{diff-cross-section-2D-theta}
	\caption{Differential cross sections for selected $\theta$}
	\label{fig:diff-cross-section-2D-theta}
\end{figure}

\begin{figure}[H]
	\centering
	\includegraphics[width=5.25in]{diff-cross-section-2D-kappa}
	\caption{Differential cross sections for selected $\kappa$}
	\label{fig:diff-cross-section-2D-kappa}
\end{figure}


\subsection{Other Cross Sections and Comparisons}
\label{sec:OtherCross}

\begin{figure}[H]
	\centering
	\includegraphics[width=5.25in]{momentum-cross-sections}
	\caption{Momentum transfer cross sections}
	\label{fig:momentum-cross-sections}
\end{figure}

\begin{figure}[H]
	\centering
	\includegraphics[width=5.25in]{orthopara-cross-sections}
	\caption{Ortho-para conversion cross sections}
	\label{fig:orthopara-cross-sections}
\end{figure}

\begin{figure}[H]
	\centering
	\includegraphics[width=5.25in]{cross-section-comparisons}
	\caption{Comparison of cross sections}
	\label{fig:cross-section-comparisons}
\end{figure}

\end{document}
