% -*- root: Dissertation.tex -*-
\documentclass[Dissertation.tex]{subfiles} 
\begin{document}


\chapter{Introduction}
\label{sec:Introduction}

\todoi{LOOK AT BISHOP THESIS}
\todoi{Switch to BibLaTeX}
\todoi{Lithium?, e- H, e+ He, e- He, McCurdy and Rescigno -> KVM mentioned how used for many systems, established method used by many groups}
\todoi{Mention UCL and Australia groups}
$\LaTeXe$

\section{Positrons and Positronium}
Positrons (denoted by $e^+$) are the antiparticle equivalent to electrons.  The mass and spin are same as that of electrons, but the charge is $+1e$, compared to the charge of $-1e$ on an electron. Paul Dirac first theorized the positron (antielectron) in 1931 \cite{Dirac1931}, and it was discovered shortly thereafter by Carl D. Anderson in 1932 \cite{Anderson1933}.

Stjepan Mohorovi\u{c}i\'{c} theorized the existence of positronium (Ps) in 
1934 \cite{Mohorovicic1934}, but it was not created until 1951 by Martin 
Deutsch \cite{Deutsch1951}. Ps is an exotic atom consisting of a positron and 
an electron.  This atom is similar in some ways to hydrogen but also differs 
in some key aspects. Namely that Ps annihilates, emitting two or three $\gamma$
rays, depending on the spin \cite{Charlton2001}. In the singlet state, also 
known as parapositronium (p-Ps), the lifetime is $\SI{125}{\ps}$ \cite{Czarnecki1999}.
The triplet, or orthopositronium (o-Ps), lasts approximately 1000 times longer 
with a lifetime of $\SI{142}{\ns}$ \cite{Vallery2003}. Due to the the short 
lifetime of p-Ps, the majority of experimental data of Ps-atom and 
Ps-molecule scattering comes from o-Ps.

\subsection{Positronium and Hydrogen Wavefunctions}
The reduced mass for hydrogen is
\begin{equation}
\mu_{\rm{H}} = \frac{m_p m_e}{m_p + m_e}.
\end{equation}

In the limit where the proton has infinite mass, the reduced mass becomes
\begin{equation}
%\lim_{m_p \to \infty} \frac{m_p + m_e}{m_p m_e} = \frac{1}{m_e} = \mu_H^{-1}
\lim_{m_p \to \infty} \frac{m_p m_e}{m_p + m_e} = m_e \approx \mu_{\rm{H}}.
\end{equation}

The positronium system is similar to the hydrogen atom with the proton replaced by a positron, meaning we can no longer use the infinite mass approximation, as $m_{e^+} = m_e$.
\begin{equation}
\mu_{\rm{Ps}} = \frac{m_e^2}{2 m_e} = \frac{m_e}{2}.
\end{equation}

To obtain the Ps wavefunction, we can start with the wavefunction for H and
replace $m_e \rightarrow \frac{m_e}{2}$. The ground state wavefunction for H is
\begin{equation}
\Phi_{100,\rm{H}} = \left(\frac{1}{\pi a_0^3}\right)^{\frac{1}{2}} \!\! \ee^{-\frac{r}{a_0}},
\text{where } a_0 = \frac{\hbar^2}{m_e \,e^2} \;.
\end{equation}
The $100$ represents the $n \ell m$ quantum numbers, and we are only
considering ground state H and Ps. If we denote $a_0 \rightarrow a_0'$, where
we replace $m_e$ by $m_e/2$, the positronium ground state
wavefunction can be written easily as
\begin{equation}
\Phi_{100,\rm{Ps}} = \left(\frac{1}{\pi {a_0'}^3}\right)^{\frac{1}{2}} \!\! \ee^{-\frac{r}{a_0'}},
\text{where } a_0' = \frac{2\hbar^2}{m_e \,e^2} \;.
\end{equation}
Working in atomic units (see \cref{sec:Units}), $a_0 = 1$ (i.e. $\hbar = m_e = e = 1$), giving
\begin{equation}
\Phi_{\rm{H}}\left(r_3\right) \equiv \Phi_{100,\rm{H}}\left(r_3\right) = \frac{1}{\sqrt{\pi}} \ee^{-r_3}
\label{eq:HWave}
\end{equation}
and
\begin{equation}
\Phi_{\rm{Ps}}\left(r_{12}\right) \equiv \Phi_{100,\rm{Ps}}\left(r_{12}\right) = \frac{1}{\sqrt{8 \pi}} \ee^{-r_{12}/2} \;.
\label{eq:PsWave}
\end{equation}

\subsection{Positronium and Hydrogen Energies}

When the Schr\"{o}dinger equation is solved for the hydrogen atom, the energy
is seen to be (neglecting higher-order effects)
\beq
\label{eq:HEnergy}
E_{n,\rm{H}} = -\frac{R_\infty}{n^2}.
\eeq
In Hartree atomic units, $R_\infty = \frac{1}{2}$, giving
\beq
\label{eq:HEnergyAU}
E_{n,\rm{H}} = -\frac{1}{2 n^2}.
\eeq
Due to the reduced mass of half that of hydrogen, the energy of positronium is
\beq
\label{eq:PsEnergyAU}
E_{n,\rm{Ps}} = -\frac{1}{4 n^2}.
\eeq

For a given level $n$, the energy of positronium is half that of hydrogen. The
ionization energy is approximately $\SI{-6.8}{\eV}$, versus that of
$\SI{-13.6}{\eV}$ for hydrogen. \Cref{fig:HPsLevels} compares the energy levels
of positronium and hydrogen.
\begin{figure}[H]
	\centering
	\includegraphics[width=5.5in]{HPsLevels}
	\caption{Hydrogen and positronium energy levels}
	\label{fig:HPsLevels}
\end{figure}

For a detailed analysis of H and Ps theoretical and experimental research,
refer to Ref.~\cite{Karshenboim2005}.

\section{Positronium Hydride}
\label{sec:PsH}
Positronium hydride (PsH) is a molecule \todo{Not exactly true} comprised of 
a hydrogen atom and a positronium atom. After Wheeler \cite{Wheeler1946} 
showed that positrons could be part of what he called a polyelectronic 
compound, Ore shortly thereafter predicted PsH in 1951 \cite{Ore1951}. PsH 
was not experimentally verified until 1992 by Schrader \cite{Schrader1992}. 

\todoi{Use \cite{Bressanini2003}}
\todoi{Use \cite{Heyrovska2011}}



\section{Introduction from PRA Paper}
Positronium (Ps) scattering from atoms and molecules is an area of current 
experimental and theoretical interest. The development of energy-tunable
ortho-Ps beams
\cite{Brown1985,Laricchia1987,Zafar1996,Garner1996,Laricchia2008} 
has enabled measurements to be made of Ps scattering from the inert gases He,
Ne, Ar, Kr, and Xe
\cite{Garner1996,Garner2000,Armitage2002,Laricchia2004,Armitage2006,Laricchia2008,Engbrecht2008,Brawley2010a}
and the molecules H$_2$, N$_2$, O$_2$, CO$_2$, H$_2$O, and SF$_6$
\cite{Garner1996,Garner1998,Garner2000,Laricchia2004,Armitage2006,Beale2006,Brawley2010a}.
Cross sections for Ps scattering from H have not been measured due to the
difficulty of an atomic hydrogen beam, although the binding energy of positronium
hydride (PsH) has been measured in the reaction of a positron with methane,
e$^+$ + CH$_4$ $\to$ CH$_3^+$ + PsH \cite{Schrader1992}. The low-energy region
is of particular interest, because in this energy range, positron and electron 
correlations are dominant. We present
\cite{Conferences1,Conferences2,Conferences3,WoodsDiss2015} 
our work of the application of the complex Kohn variational 
method to elastic Ps(1s)-H(1s) scattering in the energies up to the 
excitation threshold of Ps(n=2) at 5.101 eV.

Ps-H scattering is a fundamental four-body Coulomb process. The Kohn (and 
inverse Kohn) variational method has previously been applied to Ps-H 
collisions by Van Reeth and Humberston \cite{VanReeth2003,VanReeth2004}, who 
computed $^{1,3}S$ and $^{1,3}P$ elastic phase shifts. We 
have extended their $^{1,3}S$ and $^{1,3}P$ Kohn variational calculations in 
multiple ways. First, and foremost, in addition to implementing
the Kohn and inverse Kohn variational methods, we have implemented the 
generalized Kohn, the complex Kohn for the $S$ and $T$ matrices, and the
generalized complex Kohn for the $S$ and $T$
matrices. The complex Kohn variational methods are known to suffer
from far fewer anomalous singularities than the Kohn, inverse Kohn and 
generalized Kohn variational methods
\cite{Lucchese1989, Cooper2009, Cooper2010}. The second extension that we
consider is to use the procedure by Todd 
\cite{Todd2007} to systematically remove short-range terms that cause linear 
dependence. This enables us to compute the phase shifts (and the binding 
energy of PsH) with more short-range Hylleraas terms than the earlier 
Kohn calculations. We add the asymptotic expansion of Drake and Yan
\cite{Drake1995, Yan1997} to improve the accuracy of the short-range
integrations. We have also significantly increased the number of integration
points for matrix
elements that involve the long-range terms, along with implementing a method to
accelerate the convergence of these integrals. We also extend the 
calculations to the next two partial waves through to the $F$-wave, which
enables us to calculate the differential elastic, integrated elastic, momentum
transfer, and ortho-para conversion cross sections.

We present results in this paper obtained using the $S$ matrix complex Kohn 
variational principle for Ps-H scattering. We confirm the previously observed 
resonances for the first four partial waves and compare the positions and 
widths to that of the earlier Kohn \cite{VanReeth2003,VanReeth2004}, close 
coupling (CC) \cite{Walters2004} and complex rotation calculations
\cite{Yan1999,Yan1998a,Ho1998,Ho2000}. In addition, we compute the scattering
lengths and effective ranges using multiple effective range theories.

We also use the short-range part of the full scattering wavefunction to 
compute the binding energy of PsH. Comparing the binding energy with the most 
elaborate variational results gives an indication of the reliability of 
describing the Ps-H system at short distances. The binding energy of PsH,
$E_b$, has been calculated using various methods. Ho \cite{Ho1986} performed a 
variational calculation with a Hylleraas basis set, and Yan and Ho
\cite{Yan1999} later did a more extensive calculation. Mitroy \cite{Mitroy2006}
used the stochastic variational method (SVM) with 1800 explicitly correlated 
Gaussians (ECGs), and Bubin and Adamowicz \cite{Bubin2006} found the most 
accurate value using 5000 ECGs in a variational calculation.

There have been a number of other calculations for Ps-H scattering. A much 
earlier Kohn variational calculation was performed by Page \cite{Page1976} 
for the Ps-H scattering lengths. Drachman and Houston used a stabilization 
method with an effective range theory (ERT) expansion
\cite{Drachman1975,Drachman1976}. At low energies, diffusion Monte Carlo (DMC)
\cite{Chiesa2002}, the SVM \cite{Ivanov2001,Ivanov2002}, CC
\cite{Sinha1997,Campbell1998,Adhikari1999,Sinha2000,Blackwood2002,Blackwood2002b,Walters2004},
static exchange \cite{Hara1975,Ray1997}, and Kohn variational
\cite{Page1976,VanReeth2003,VanReeth2004} methods have been applied. The SVM
with stabilization 
techniques was used to compute low-energy phase shifts and 
scattering lengths for Ps-H collisions \cite{Ivanov2001,Ivanov2002}.

Blackwood et al.~\cite{Blackwood2002} performed an elaborate CC calculation 
for Ps scattering from H, which took into account excitation and ionization 
of both the projectile and target. They considered two different coupling 
schemes. The first one, which they refer to as 9Ps9H, included 9 eigen- and 
pseudo-states of Ps and also of H. The second scheme, which they refer to as 
14Ps14H, is for the singlet only and includes 14 eigen- and pseudo-states of 
Ps and also of H. Good agreement is obtained between the CC
\cite{Blackwood2002} and the SVM \cite{Ivanov2002} for the $S$-wave scattering
lengths and phase shifts. In another paper, Blackwood et
al.~\cite{Blackwood2002b} considered the importance of including the H$^-$
channel in a 22Ps1H coupling scheme, comparing to the previous 22Ps1H
calculations of Campbell et al. \cite{Campbell1998}. Walters et
al.~\cite{Walters2004} extended the earlier CC calculations
\cite{Blackwood2002} to include the e$^+$-H$^-$ channel
\cite{Blackwood2002b} and compared their results for the $S$-wave with the
accurate Kohn variational results \cite{VanReeth2003}. They speculated that the
inclusion of the virtual Ps$^-$ formation channel may be needed to obtain 
agreement with the Kohn variational results for Ps(1s)-H(1s) scattering 
\cite{Blackwood2002}. A recent calculation of Ps-H scattering uses the
confined variational method (CVM) \cite{Zhang2012}. This provides accurate
results but has the drawback of being very computationally expensive. In
Ref.~\cite{Zhang2012}, Zhang and Yan calculate phase shifts for 2 momenta for
both $^1S$ and $^3S$.

Unlike the SVM and DMC methods, the Kohn variational method gives rigorous 
bounds on the scattering lengths and, except for Schwartz singularities, 
gives empirical bounds on the elastic phase shifts. This means that the wave 
function can be systematically improved to the converged results. The Kohn 
variational method is known to yield accurate results and has provided 
benchmark results \cite{VanReeth2003,VanReeth2004} to which results from 
other calculations can be compared. %\sout{Unlike the SVM, the Kohn variational
%method can treat inelastic as well as elastic scattering and thus can be 
%extended to higher energies.}

Phase shifts are expressed in radians. Atomic units are used throughout 
unless otherwise stated. For conversions to electron-volts (eV), we use the 
conversion factor 1 a.u. = {27.21138505(60) eV}
\cite{Mohr2012,*NISTConversions}.


\todoi{ECGs do not match the physical shape of the wavefunction and require large basis sets \cite{Mitroy2002a}}


\section{From proposal}

\begin{itemize}
	\item At low energies, multiple methods for scattering calculations can be used
	\begin{itemize}
		\item Diffusion Monte Carlo (DMC)
		\item Stochastic variational (SVM)
		\item Close-coupling (CC)
		\item Kohn variational
	\end{itemize}

	\item Diffusion Monte Carlo does not give bounds on scattering parameters 13

	\item Stochastic variational (Mitroy et al. (2001), Ivanov et al. (2001, 2002)) 8-10
	\begin{itemize}
		\item Uses explicitly correlated Gaussians
		\item ECGs do not match the physical shape of the wavefunction 8
		\item Large basis sets required 8
		\item Phase shifts are determined at energies which are not known in advance 10
		\item Does not give bounds on scattering parameters 13
	\end{itemize}

	\item Close-coupling (Blackwood et al. (2002), Walters et al. (2004)) 11, 12
	\begin{itemize}
		\item Two possible coupling schemes
		\item 9Ps9H – has 9 eigen- and pseudo-states of Ps and H
		\item 14Ps14H – has 14 eigen- and pseudo-states of Ps and H
		\item Only singlet performed for 14Ps14H 11
	\end{itemize}
\end{itemize}

\section{Motivation}
Positron Research Group at St. Olaf College in Northfield, MN.

Jason Engbrecht

planned to measure low-energy Ps scattering off alkali atoms


\section{Ideas for Introduction}

\begin{figure}[H]
	\centering
	\resizebox{1.0\textwidth}{!}{\includegraphics{ScienceBrawley}}
	\caption{From Ref.~\cite{Brawley2010a}. Reprinted with permission from AAAS.}
	\label{fig:ScienceBrawley}
\end{figure}

also mentioned in the popular science literature in New Scientist \cite{NewScientist2015}

\begin{itemize}
	\item Bound State Versus Scattering Problems
	\item Comparison with Other Scattering Methods
	\item Partial waves
	\item Types of possible targets
\end{itemize}

\todoi{\cite{Biswas2002} mentions motivations, such as \cite{Weber1988,Moxom1998,Tang1993} and \url{http://www.amazon.com/Positron-Positronium-Chemistry-Physical-Theoretical/dp/0444430091}}

%\nicebox{Title}{This is the content of my box}
\todoi{Other groups using Kohn: \cite{McCurdy1987,McCurdy1989,Houston1971,Armour1987,Schwartz1961,Watts1992,DiRienzi2004,Armour2014,Ward1987,Todd2007,Armour2010c}}
\todoi{Look at quotchap (chapter quotes) and thumbs (chapter thumbs) packages}
\todoi{Go through tips in \url{https://www.cs.purdue.edu/homes/dec/essay.dissertation.html}}
\todoi{Nomenclature with \url{http://www.howtotex.com/packages/create-a-simple-nomenclature-with-the-longtable-package/} or the nomencl package}
\todoi{Discussion of methods in \cite{Ivanov2002}}
\todoi{Bold volume numbers: \url{http://tex.stackexchange.com/questions/82499/make-volume-bold-in-custom-bibliography-style-bst}}
\todoi{Possible recreation of figures 2-4 in \cite{Adhikari1999}?}
\todoi{Better WebCite citations: \url{http://www.volkerschatz.com/tex/advbib.html}}
\todoi{Might be nice to do something like this for Hartree atomic units: \url{http://ilan.schnell-web.net/physics/rydberg.pdf}}
\todoi{Short-range term animations: \url{http://youtu.be/uVxK9iGZGm0}}
\todoi{Plotly plots: \url{https://plot.ly/~Denton}}
\todoi{Cite Hylleraas paper on He: \cite{Hylleraas1929}}
\todoi{Consistent H and Ps abbreviations}
\todoi{Mention that S-matrix results throughout unless otherwise specified}
\todoi{Cusp condition mentioned in \cite{Saito1995a}. \url{http://onlinelibrary.wiley.com/doi/10.1002/cpa.3160100201/abstract}}

\section{A Note on Units}
\label{sec:Units}

\begin{verbatim}
http://en.wikipedia.org/wiki/Natural_units
http://www.phys.ksu.edu/personal/cdlin/class/class11a-amo2/atomic_units.pdf
http://ilan.schnell-web.net/physics/rydberg.pdf
http://exciting-code.org/ref:input
http://physics.nist.gov/cuu/Constants/energy.html
\end{verbatim}

Unless otherwise stated, values throughout are given in atomic units, i.e. $\hbar = m_e = e = 4\pi\epsilon_0 = 1$ \cite{Hartree1928}. Energies are given in hartrees, with $\SI{1}{\hartree} = \SI{27.211 385 05(60)}{\electronvolt}$ \cite{Mohr2012,NISTConversions}. Momentum is given as units of $a_0^{-1}$. Cross sections are given in units of $\pi a_0^2$, and differential cross sections are given in units of $a_0^2 / \rm{sr}$, unless otherwise noted.



\biblio
\end{document}