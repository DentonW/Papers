% -*- root: Dissertation.tex -*-
\documentclass[Dissertation.tex]{subfiles} 
\begin{document}


\chapter{Introduction}
\label{sec:Introduction}


\section{Positronium}
\label{sec:Positronium}

Positronium (Ps) is an exotic atom formed from the bound state of a positron
(antielectron, e$^+$) and an electron (e$^-$). \todo{onium?} Stjepan Mohorovi\u{c}i\'{c} 
theorized the existence of Ps in 1934 \cite{Mohorovicic1934}, 
but it was not created until 1951 by Martin Deutsch \cite{Deutsch1951}.
This atom is similar in some ways to hydrogen (H) but also differs 
in some key aspects. Namely that Ps annihilates, emitting two or three $\gamma$
rays, depending on the spin \cite{Charlton2001}. In the singlet state, also 
known as parapositronium (p-Ps), the lifetime is $\SI{125}{\ps}$ \cite{Czarnecki1999}.
The triplet state, or orthopositronium (o-Ps), lasts approximately 1000 times
longer with a lifetime of $\SI{142}{\ns}$ \cite{Vallery2003}. Due to the short 
lifetime of p-Ps, the majority of experimental data of Ps-atom and
Ps-molecule scattering comes from o-Ps.


\subsection{Positronium and Hydrogen Wavefunctions}
\todoi{p and e instead of $p$ and $e$}
The reduced mass for H is
\begin{equation}
\mu_{\rm{H}} = \frac{m_p m_e}{m_p + m_e}.
\end{equation}

In the limit where the proton has infinite mass, the reduced mass becomes
\begin{equation}
%\lim_{m_p \to \infty} \frac{m_p + m_e}{m_p m_e} = \frac{1}{m_e} = \mu_H^{-1}
\lim_{m_p \to \infty} \frac{m_p m_e}{m_p + m_e} = m_e \approx \mu_{\rm{H}}.
\end{equation}

The Ps exotic atom is similar to the H atom with the proton replaced 
e$^+$, meaning we can no longer use the infinite mass approximation, 
as $m_{e^+} = m_{e^-} = m_e$.
\begin{equation}
\mu_{\rm{Ps}} = \frac{m_e^2}{2 m_e} = \frac{m_e}{2}.
\end{equation}

To obtain the Ps wavefunction, we can start with the wavefunction for H and
replace $m_e \rightarrow \frac{m_e}{2}$. The ground state wavefunction for H is
\begin{equation}
\Phi_{100,\rm{H}} = \left(\frac{1}{\pi a_0^3}\right)^{\frac{1}{2}} \!\! \ee^{-\frac{r}{a_0}},
\text{where } a_0 = \frac{\hbar^2}{m_e \,e^2} \;.
\end{equation}
The $100$ represents the $n \ell m$ quantum numbers, and we are only
considering ground state H and Ps. If we denote $a_0 \to a_\mu$, where
we replace $m_e$ by $m_e/2$, the positronium ground state
wavefunction can be written easily as
\begin{equation}
\Phi_{100,\rm{Ps}} = \left(\frac{1}{\pi {a_\mu}^3}\right)^{\frac{1}{2}} \!\! \ee^{-\frac{r}{a_\mu}},
\text{where } a_\mu = \frac{2\hbar^2}{m_e \,e^2} \;.
\end{equation}
Working in atomic units (see \cref{sec:Units}), $a_0 = 1$ (i.e. $\hbar = m_e = e = 1$), giving
\begin{equation}
\Phi_{\rm{H}}\left(r_3\right) \equiv \Phi_{100,\rm{H}}\left(r_3\right) = \frac{1}{\sqrt{\pi}} \ee^{-r_3}
\label{eq:HWave}
\end{equation}
and
\begin{equation}
\Phi_{\rm{Ps}}\left(r_{12}\right) \equiv \Phi_{100,\rm{Ps}}\left(r_{12}\right) = \frac{1}{\sqrt{8 \pi}} \ee^{-r_{12}/2} \;.
\label{eq:PsWave}
\end{equation}

\subsection{Positronium and Hydrogen Energies}

When the Schr\"{o}dinger equation is solved for the hydrogen atom, the energy
is seen to be (neglecting higher-order effects)
\beq
\label{eq:HEnergy}
E_{n,\rm{H}} = -\frac{R_\infty}{n^2}.
\eeq
In Hartree atomic units, $R_\infty = \frac{1}{2}$, giving
\beq
\label{eq:HEnergyAU}
E_{n,\rm{H}} = -\frac{1}{2 n^2}.
\eeq
Due to the reduced mass of half that of hydrogen, the energy of positronium is
\beq
\label{eq:PsEnergyAU}
E_{n,\rm{Ps}} = -\frac{1}{4 n^2}.
\eeq

For a given level $n$, the energy of positronium is half that of hydrogen. The
ionization energy is approximately $\SI{-6.8}{\eV}$, versus that of
$\SI{-13.6}{\eV}$ for hydrogen. \Cref{fig:HPsLevels} compares the energy levels
of positronium and hydrogen.
\begin{figure}[H]
	\centering
	\includegraphics[width=5.5in]{HPsLevels}
	\caption{Hydrogen and positronium energy levels}
	\label{fig:HPsLevels}
\end{figure}

For a detailed analysis of H and Ps theoretical and experimental research,
refer to Ref.~\cite{Karshenboim2005}.




\section{Motivation}
\label{sec:Motivation}

Ps formation is important in the galactic core \cite{Kinzer1996}, and Ps-atom 
scattering is of interest in the study of solar processes \cite{Crannell1976}.
As well as the basic interest of Ps-atom scattering in atomic physics, Ps 
is also important in material science. As Ps is neutral, it penetrates deeper 
into material than a charged particle, such as a positron. Ps scattering also 
has applications in other areas of physics such as biophysics and 
astrophysics \cite{Laricchia2012}. A brief overview of the state of the art
in antimatter atomic physics is Ref.~\cite{Walters2010}, and a more in-depth
review of Ps collisions is Ref.~\cite{Laricchia2012}.

With the increased interest in antihydrogen ($\bar{\rm{H}}$) production at CERN
\cite{ALPHACollaboration2011}, there have been investigations by groups
exploring alternate mechanisms other than dumping antiprotons ($\bar{\rm{p}}$)
into a cloud of e$^+$ and relying on the reaction of
$\bar{\rm{p}} + \rm{e}^+ + \rm{e}^+ \to \bar{\rm{H}} + \rm{e}^+$.
Refs.~\cite{Kadyrov2013,Elkilany2014} explore the inelastic low-energy
reaction of $\bar{\rm{p}} + \rm{Ps} \to \bar{\rm{H}} + \rm{e}^-$, where the Ps
is in its ground state. A very recent paper \cite{Kadyrov2015} (also mentioned
in the popular science press \cite{Kadyrov2015b}) found that if the target Ps
is in an excited state ($1 < n \leq 3$), the $\bar{\rm{H}}$ production rate is
improved by several orders of magnitude.

The original %impetus
motivation for this research was a proposed experiment to measure 
low-energy Ps scattering from alkali metals by Jason Engbrecht of the Positron 
Research Group at St. Olaf College. There has not been much theoretical work
on these systems so far. Unfortunately, it appears that this project
is put on hold indefinitely, and the group's website \cite{Engbrecht2013} no 
longer exists. We started investigating Ps-H scattering and plan to extend
this work into Ps scattering from the alkali metals.

However, there is still interesting ongoing experimental work on Ps scattering,
though with different targets. The University College London (UCL) Positron
Group \cite{UCL2015} has developed energy-tunable o-Ps beams
\cite{Brown1985,Laricchia1987,Zafar1996,Garner1996,Laricchia2008} over the
course of many years. This group has been able to study Ps scattering from
He, Ne, Ar, Kr, and Xe
\cite{Garner1996,Garner2000,Armitage2002,Laricchia2004,Armitage2006,Laricchia2008,Engbrecht2008,Brawley2010a}
and the H$_2$, N$_2$, O$_2$, CO$_2$, H$_2$O, and SF$_6$ molecules
\cite{Garner1996,Garner1998,Garner2000,Laricchia2004,Armitage2006,Beale2006,Brawley2010a}.

A recent development in the field is the surprising discovery that Ps 
scattering is electron-like \cite{Brawley2010,Brawley2010a}, which was also reported in the
popular science literature \cite{NewScientist2015}. If the cross sections are 
plotted with respect to the velocity of the incoming projectile, not momentum 
like typical, e$^-$ and Ps scattering look similar when the target is the same.
This is despite the Ps projectile having twice the mass of e$^-$ and being 
electrically neutral versus negatively charged. It can be seen in
\cref{fig:ScienceBrawley} that e$^+$ scattering looks different when compared
to these two. Fabrikant and Gribakin
\cite{Fabrikant2014,Fabrikant2014a} compare low-energy e$^-$ and Ps scattering
from Kr and Ar targets, also finding that the cross sections are similar for
e$^-$ and Ps projectiles. The tentative conclusion is that the
e$^+$ plays a much smaller role in the scattering process than the e$^-$
in Ps-atom and Ps-molecule scattering. This shows that there is still
plenty of work to do to understand Ps scattering more fully, but it
does suggest that for certain cases, a decent first approximation to Ps
scattering can be made by using e$^-$ scattering data.

\begin{figure}[H]
	\centering
	\resizebox{1.0\textwidth}{!}{\includegraphics{ScienceBrawley}}
	\caption[Comparisons of e$^-$, e$^+$, and Ps scattering]{Comparisons of
e$^-$, e$^+$, and Ps scattering from different atomic and molecular targets 
from Ref.~\cite{Brawley2010a}. Reprinted with permission from AAAS.}
	\label{fig:ScienceBrawley}
\end{figure}


\section{Partial Waves and Kohn Variational Methods}
\label{sec:KohnIntro}

The most common way of approaching scattering problems is to use the complete
set of Legendre polynomials to expand the scattering wavefunction. For a
central potential, this can be written as \cite{Bransden2003}
\beq
\label{eq:PartialWave}
\Psi(k,r,\theta) = \sum_{\ell=0}^\infty R_\ell(k,r) P_\ell(\cos\theta).
\eeq
The method of partial waves evaluates each term in this summation separately,
with each referred to as a partial wave, and each has a different angular 
momentum. The typical naming of each partial wave starting from $\ell = 0$
is the S-, P-, D-, F-, G-, H-wave, etc., which is similar to the usual 
spectroscopic notation. For low energies, usually only a few terms in this
expansion are required, and for very low energies, the S-wave ($\ell = 0$) is 
typically the only significant contribution.

The Kohn variational method \cite{Kohn1948} and its variants, derived and 
described in \cref{chp:WaveKohn}, have been used successfully in many 
scattering problems, such as e$^-$-H \cite{Schwartz1961}, e$^-$-methane
\cite{McCurdy1989}, H-$\bar{\rm{H}}$ \cite{Armour2002}, e$^+$-H$_2$
\cite{Cooper2008}, e$^-$-Ps \cite{Ward1987}, e$^+$-He \cite{VanReeth1997},
and nucleon-nucleon scattering \cite{Tomio1995,Kievsky1997}.
The Kohn variational method and its variants suffer from well-known spurious
singularities (see \cref{eq:SchwartzSing}), so they are often used in conjunction
with each other to identify these. To avoid cumbersome wording in this document,
the Kohn variational method and variants of the method are simply referred to as
``Kohn methods''. Complex Kohn methods that use spherical Hankel functions 
instead of the spherical Bessel and Neumann functions are often used due to a 
smaller, but nonzero, chance of the Schwartz singularities
\cite{McCurdy1989,Lucchese1989,Cooper2010}. This work uses the Kohn,
inverse Kohn, generalized Kohn \cite{Armour1991}, $S$-matrix complex Kohn,
and $T$-matrix complex Kohn variational methods.


\section{Ps-H Scattering}
\label{sec:ScatIntro}

For this work \cite{Woods2015,Conferences1,Conferences2,Conferences3}, we
have computed phase shifts for the first six partial waves of Ps-H scattering
and resonance parameters for $^1$S through $^1$F
(\cref{chp:SWave,chp:PWave,chp:DWave,chp:HigherWaves}).
We also calculate scattering lengths and effective ranges for $^{1,3}$S
(\cref{sec:ScatteringLength}), scattering lengths for $^{1,3}$P
(\cref{sec:PWaveScatLen}), and multiple cross sections
(\cref{chp:CrossSections}). Each of these is compared to previously published
research where possible.

This work is an extension of the earlier work on Ps-H collisions using the 
Kohn and inverse Kohn variational methods by Van Reeth and Humberston
\cite{VanReeth2003,VanReeth2004}. In these papers, they calculated $^{1,3}$S
and $^{1,3}$P phase shifts and obtained resonance parameters, scattering lengths
and effective ranges from these. The most important difference from this earlier 
work is that we have increased the number of partial waves examined from two 
to six, requiring developing a general formalism and code that works for 
arbitrary partial waves. This allows us to calculate elastic integrated, elastic
differential, and momentum transfer cross sections (\cref{chp:CrossSections}),
which would not be possible with only the
$^{1,3}$S- and $^{1,3}$P-waves. We also develop a very general form
(\cref{chp:WaveKohn}) of the scattering wavefunction and codes that allows us to
calculate phase shifts with not only the Kohn and inverse Kohn variational methods
but also with the generalized Kohn, $S$-matrix complex Kohn, $T$-matrix complex
Kohn, generalized $S$-matrix complex Kohn, and generalized $T$-matrix complex
Kohn variational methods. Over the previous work, we also perform a thorough
analysis of the van der Waals contribution to the $^{1,3}$S scattering lengths
and effective ranges and investigate the $^{1,3}$P scattering lengths
(\cref{sec:ScatteringLength}).

We have increased the number of short-range terms (\cref{eq:PhiDef}) used 
over prior work \cite{VanReeth2003,VanReeth2004}. This is enabled by several
changes. The largest improvement has been the introduction of the Todd method
(\cref{sec:ToddBound}), which selects the ``best'' set of short-range terms
from the full set determined by $\omega$ in \cref{eq:GeneralWaveTrial}. Some
short-range terms contribute more to linear dependence than others, and this
method removes those in a systematic manner. This is often an improvement 
over the restriction in powers that Van Reeth and Humberston
\cite{VanReeth2003} did, though we still use that restricted basis set for
the $^{1,3}$F-wave (\cref{sec:FWave}). We also implement the asymptotic
expansion \cite{Drake1995,Yan1997} for the short-range--short-range integrals
instead of only doing a direction summation (\cref{sec:AsymptoticExpansion}).
This gives much more accurate short-range--short-range integrals, allowing us
to use more short-range terms and solve larger matrices in
\cref{eq:GeneralKohnMatrix}. We specifically noted a threefold increase in the
number of terms we could use for the $^3$S state when the asymptotic expansion
was included. We use approximately seven times as many
integration points as this previous work (\cref{sec:SelQuadPoints}).
For $\ell \geq 1$ especially, an increase in the number of
integration points for the long-range--long-range and long-range--short-range
integrals \cref{sec:CompLong} lead to more stable results and the ability to
use more short-range terms. For $\ell \geq 2$ (not investigated by the prior
work), we also introduced extra exponentials in several coordinates to the
Gauss-Laguerre quadratures (\cref{sec:ExtraExp}) that are subsequently removed,
increasing the convergence rate of the long-range--short-range integrals.

The Ps-H collision problem has been treated by multiple groups with different
methods over the years, each with associated strengths and weaknesses.
\footnote{The following three paragraphs are used from Ref.~\cite{Woods2015}.}A much 
earlier Kohn variational calculation was performed by Page \cite{Page1976} 
for the Ps-H scattering lengths. Drachman and Houston
\cite{Drachman1975,Drachman1976} used
a stabilization method with an effective range theory (ERT) expansion.
At low energies, diffusion Monte Carlo (DMC)
\cite{Chiesa2002}, the SVM \cite{Ivanov2001,Ivanov2002}, CC
\cite{Sinha1997,Campbell1998,Adhikari1999,Sinha2000,Blackwood2002,Blackwood2002b,Walters2004},
static exchange \cite{Hara1975,Ray1997,*Ray1996}, Kohn variational
\cite{Page1976,VanReeth2003,VanReeth2004}, and inverse Kohn
variational \cite{VanReeth2003,VanReeth2004} methods have been applied. The SVM
with stabilization 
techniques was used to compute low-energy phase shifts and 
scattering lengths for Ps-H collisions \cite{Ivanov2001,Ivanov2002}.

The first Born approximation has been used by Massey and Mohr \cite{Massey1954}
and later by McAlinden et al.~\cite{McAlinden1996}.
Blackwood et al.~\cite{Blackwood2002} performed an elaborate CC calculation 
for Ps scattering from H, which took into account excitation and ionization 
of both the projectile and target. They considered two different coupling 
schemes. The first one, which they refer to as 9Ps9H, included 9 eigen- and 
pseudo-states of Ps and also of H. The second scheme, which they refer to as 
14Ps14H, was used for $S$-wave singlet scattering only and included
14 eigen- and pseudo-states of 
Ps and also of H. Good agreement was obtained between the CC
\cite{Blackwood2002} and the SVM \cite{Ivanov2002} for the $^{1,3}S$-wave scattering
lengths and phase shifts. In another paper, Blackwood et
al.~\cite{Blackwood2002b} considered the importance of including the H$^-$
channel in a 22Ps1H coupling scheme, comparing with the previous 22Ps1H
calculations of Campbell et al. \cite{Campbell1998}. Walters et
al.~\cite{Walters2004} extended the earlier CC calculations
\cite{Blackwood2002} to include the e$^+$-H$^-$ channel
\cite{Blackwood2002b} and compared their results for the $S$-wave with the
Kohn variational results \cite{VanReeth2003}.
A recent calculation of Ps-H scattering by Zhang and Yan
\cite{Zhang2012} used the
confined variational method (CVM) to calculate phase shifts for two momenta
for both $^1S$ and $^3S$. This method provides accurate
results but has the drawback of being very computationally expensive.

The Kohn variational method gives rigorous upper bounds on the scattering 
lengths and, except for Schwartz singularities, empirical lower bounds on the 
phase shifts. This means that the wave function can be systematically 
improved to the converged results. The Kohn and inverse Kohn variational 
methods are known to yield accurate results and have provided benchmark 
results \cite{VanReeth2003,VanReeth2004} with which results from other 
calculations can be compared. The Kohn methods can generate very accurate 
phase shifts, but the choice of trial wavefunction can make computation very 
difficult. Then there are the spurious Schwartz singularities, but these can 
often be mitigated by using complex Kohn methods.


\section{Positronium Hydride}
\label{sec:PsH}
Positronium hydride (PsH) is a bound state comprised of 
a hydrogen atom and a positronium atom. After Wheeler \cite{Wheeler1946} 
showed that positrons could be part of what he called a polyelectronic 
compound, Ore shortly thereafter predicted PsH in 1951 \cite{Ore1951}. PsH 
was not experimentally verified until 1992 by Schrader \cite{Schrader1992}
using the reaction e$^+$ + CH$_4$ $\to$ CH$_3^+$ + PsH.

We first investigated the bound state of PsH instead of Ps-H scattering, as 
it is a simpler problem and has been studied extensively in the literature
(see \cref{tab:BoundEnergyOther} on page \pageref{tab:BoundEnergyOther}).
The purpose of first studying PsH was not to try to contribute more accurate
results but to develop the experience with the short-range Hylleraas-type 
correlation terms that we used in both the $^1$S PsH and $^1$S Ps-H
scattering problems. 
The full discussion of our work on PsH is found in \cref{chp:PsHBound}.
There are dozens of calculations of the ground state
or binding energy of PsH given in \cref{tab:BoundEnergyOther}.
Our binding energy of \SI{1.066 406}{eV} compares well with the most accurate
value from Ref.~\cite{Bubin2006} of \SI{1.066 598}{eV}, which gives confidence
in the short-range part of our scattering wavefunction in
\cref{sec:GeneralWave}.



\section{Positronium Hydride Structure}
\label{sec:PsHStructure}
There has been some discussion in the literature about whether PsH is more 
like an atomic structure or more like a molecule with Ps and H. We did not 
attempt an analysis of this problem, since the PsH system is not the goal of 
this work. Bressanini and Morosi \cite{Bressanini2003} give a good overview 
of the lack of consensus on this problem. 

Frolov and Smith \cite{Frolov1997c} note that they expect PsH to be a cluster 
consisting of a Ps atom and an H atom. Then from their calculations, they 
conclude that it acts as some kind of sum of H$^-$ with Ps$^-$.

Saito \cite{Saito2000} attempted to answer this question using Ho's
\cite{Ho1986} Hylleraas basis set by plotting e$^+$ and e$^-$ densities. Saito's 
conclusion was that PsH has an atomic structure but also has a diatomic 
molecular structure, or Ps with H. Bromley and Mitroy \cite{Bromley2001} also 
state that PsH has a molecular structure, comparing it ``to a light isotope 
of the H$_2$ molecule.''

Biswas and Darewych \cite{Biswas2002} find that the difference between the
S(1) resonance and the binding energy (\cref{sec:BoundSinglet}) for various 
calculations of differing accuracy is roughly constant. They suggest that 
this means that PsH is less like e$^+$ orbiting H$^-$ and more like a 
diatomic molecule.

Bressanini and Morosi \cite{Bressanini2003} perform calculations on PsH with 
a highly optimized one term wavefunction to determine the structure and 
conclude that PsH cannot be viewed as Ps+H or e$^-$ orbiting around H$^-$. 
They state, ``Keeping in mind the quantum nature of the leptons and so the 
impossibility of defining a structure, we suggest to look at PsH as a 
hydrogen negative ion with the positron that, staying more distant from the 
nucleus than the electrons, correlates its motion with those of both the 
electrons. Its attraction on the electrons squeezes them nearer to each other 
and nearer to the nucleus.''

Heyrovska \cite{Heyrovska2011} treats PsH as a molecule and calculates bond
lengths to try to gain a better understanding of its structure. However, this
preprint does not settle the debate.


\section{Final Notes}
\label{sec:Units}

%\begin{verbatim}
%http://en.wikipedia.org/wiki/Natural_units
%http://www.phys.ksu.edu/personal/cdlin/class/class11a-amo2/atomic_units.pdf
%http://ilan.schnell-web.net/physics/rydberg.pdf
%http://exciting-code.org/ref:input
%http://physics.nist.gov/cuu/Constants/energy.html
%\end{verbatim}

Unless otherwise stated, values throughout are given in atomic units, i.e.
$\hbar = m_e = e = 4\pi\epsilon_0 = 1$ \cite{Hartree1928}. Energies are given
in hartrees, with $\SI{1}{\hartree} = \SI{27.211 385 05(60)}{\electronvolt}$ 
\cite{Mohr2012,NISTConversions}. Momentum is given as units of $a_0^{-1}$,
where $a_0$ is the Bohr radius. Cross sections are given in units of
$\pi a_0^2$, and differential cross sections are given in units of
$a_0^2 / \rm{sr}$, unless otherwise noted.

Many of the figures in this document are adapted from our paper submitted to
Physical Review A \cite{Woods2015}.






%\section{From proposal}
%
%\begin{itemize}
	%\item At low energies, multiple methods for scattering calculations can be used
	%\begin{itemize}
		%\item Diffusion Monte Carlo (DMC)
		%\item Stochastic variational (SVM)
		%\item Close-coupling (CC)
		%\item Kohn variational
	%\end{itemize}
%
	%\item Diffusion Monte Carlo does not give bounds on scattering parameters 13
%
	%\item Stochastic variational (Mitroy et al. (2001), Ivanov et al. (2001, 2002)) 8-10
	%\begin{itemize}
		%\item Uses explicitly correlated Gaussians
		%\item ECGs do not match the physical shape of the wavefunction 8
		%\item Large basis sets required 8
		%\item Phase shifts are determined at energies which are not known in advance 10
		%\item Does not give bounds on scattering parameters 13
	%\end{itemize}
%
	%\item Close-coupling (Blackwood et al. (2002), Walters et al. (2004)) 11, 12
	%\begin{itemize}
		%\item Two possible coupling schemes
		%\item 9Ps9H – has 9 eigen- and pseudo-states of Ps and H
		%\item 14Ps14H – has 14 eigen- and pseudo-states of Ps and H
		%\item Only singlet performed for 14Ps14H 11
	%\end{itemize}
%\end{itemize}


%\section{Ideas for Introduction}
%
%
%\todoi{LOOK AT BISHOP THESIS}
%\todoi{Switch to BibLaTeX}
%
%\todoi{\cite{Biswas2002} mentions motivations, such as \cite{Weber1988,Moxom1998,Tang1993} and \url{http://www.amazon.com/Positron-Positronium-Chemistry-Physical-Theoretical/dp/0444430091}}
%
%\nicebox{Title}{This is the content of my box}
%\todoi{ECGs do not match the physical shape of the wavefunction and require large basis sets \cite{Mitroy2002a}}
%\todoi{Other groups using Kohn: \cite{McCurdy1987,McCurdy1989,Houston1971,Armour1987,Schwartz1961,Watts1992,DiRienzi2004,Armour2014,Ward1987,Todd2007,Armour2010c}}
%\todoi{Look at quotchap (chapter quotes) and thumbs (chapter thumbs) packages}
%\todoi{Go through tips in \url{https://www.cs.purdue.edu/homes/dec/essay.dissertation.html}}
%\todoi{Nomenclature with \url{http://www.howtotex.com/packages/create-a-simple-nomenclature-with-the-longtable-package/} or the nomencl package}
%\todoi{Discussion of methods in \cite{Ivanov2002}}
%\todoi{Bold volume numbers: \url{http://tex.stackexchange.com/questions/82499/make-volume-bold-in-custom-bibliography-style-bst}}
%%\todoi{Possible recreation of figures 2-4 in \cite{Adhikari1999}?}
%\todoi{Better WebCite citations: \url{http://www.volkerschatz.com/tex/advbib.html}}
%\todoi{Might be nice to do something like this for Hartree atomic units: \url{http://ilan.schnell-web.net/physics/rydberg.pdf}}
%\todoi{Short-range term animations: \url{http://youtu.be/uVxK9iGZGm0}}
%\todoi{Plotly plots: \url{https://plot.ly/~Denton}}
%\todoi{Cite Hylleraas paper on He: \cite{Hylleraas1929}}
%%\todoi{Consistent H and Ps abbreviations}
%\todoi{Mention that S-matrix results throughout unless otherwise specified}
%\todoi{Cusp condition mentioned in \cite{Saito1995a}. \url{http://onlinelibrary.wiley.com/doi/10.1002/cpa.3160100201/abstract}}
%\todoi{\cite{Shantarovich2003} for applications of Ps scattering}
%\todoi{Add \cite{Messiah1999} for several things in scattering theory chapter}
%\todoi{Can we get the Schraeder referenced as [4] in \cite{Heyrovska2011}?}



%\section{Positrons and Positronium}
%Positrons (denoted by $e^+$) are the antiparticle equivalent to electrons.  The mass and spin are same as that of electrons, but the charge is $+1e$, compared to the charge of $-1e$ on an electron. Paul Dirac first theorized the positron (antielectron) in 1931 \cite{Dirac1931}, and it was discovered shortly thereafter by Carl D. Anderson in 1932 \cite{Anderson1933}.


\biblio
\end{document}
