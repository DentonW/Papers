\documentclass[Introduction.tex]{subfiles} 
\begin{document}


\chapter{Introduction}
\label{sec:Introduction}

\todoi{LOOK AT BISHOP THESIS}
\todoi{Switch to BibLaTeX}

\section{Positrons and Positronium}
Positrons (denoted by $e^+$) are the antiparticle equivalent to electrons.  The mass and spin are same as that of electrons, but the charge is $+1e$, compared to the charge of $-1e$ on an electron.  Paul Dirac first theorized the positron (antielectron) in 1928, and it was discovered shortly thereafter by Carl D. Anderson in 1932. \textbf{@TODO: Cite this.}

Stjepan Mohorovi\u{c}i\'{c} theorized the existence of positronium in 1934, but it was not discovered until 1951 by Martin Deutsch.  Positronium is an exotic atom consisting of a positron and an electron.  This atom is similar in some ways to hydrogen but also differs in some key aspects.  

\subsection{Positronium and Hydrogen Wavefunctions}
The reduced mass for hydrogen is
\begin{equation}
\mu_H = \frac{m_p m_e}{m_p + m_e}.
\end{equation}

In the limit where the proton has infinite mass, the reduced mass becomes
\begin{equation}
\lim_{m_p \to \infty} \frac{m_p + m_e}{m_p m_e} = \frac{1}{m_e} = \mu_H^{-1}
\end{equation}
\begin{equation}
\mu_H \approx m_e
\end{equation}

The positronium system is similar to the hydrogen atom with the proton replaced by a positron, meaning we can no longer use the infinite mass approximation, as $m_{e^+} = m_e$.
\begin{equation}
\mu_{Ps} = \frac{m_e^2}{2 m_e} = \frac{m_e}{2}.
\end{equation}

To obtain the Ps wavefunction, we can start with the wavefunction for H and replace $m_e \rightarrow \frac{m_e}{2}$.  The ground state wavefunction for H is
\begin{equation}
\Psi_{100,H} = \left(\frac{1}{\pi a_0^3}\right)^{\frac{1}{2}} \!\! \ee^{-\frac{r}{a_0}},
\text{where } a_0 = \frac{\hbar^2}{m e^2}.
\end{equation}

If we denote $a_0 \rightarrow a_0'$, where we replace $m_e$ by $\frac{m_e}{2}$, the positronium ground state wavefunction can be writen easily as
\begin{equation}
\Psi_{100,Ps} = \left(\frac{1}{\pi {a_0'}^3}\right)^{\frac{1}{2}} \!\! \ee^{-\frac{r}{a_0'}},
\text{where } a_0' = \frac{2\hbar^2}{m e^2}.
\end{equation}

Working in atomic units, $a_0 = 1$ (i.e. $\hbar = m_e = e = 1$), giving
\begin{equation}
\Psi_{100,H}\left(r_3\right) = \frac{1}{\sqrt{\pi}} \ee^{-r_3}
\label{eq:HWave}
\end{equation}
and
\begin{equation}
\Psi_{100,Ps}\left(r_{12}\right) = \frac{1}{\sqrt{8 \pi}} \ee^{-r_{12}/2}.
\label{eq:PsWave}
\end{equation}

\subsection{Positronium and Hydrogen Energies}

When the Schr\"{o}dinger equation (equation \ref{}) is solved for the hydrogen atom, the energy is seen to be (neglecting higher-order effects)
\beq
\label{eq:HEnergy}
E_{n,H} = -\frac{R_\infty}{n^2}.
\eeq

\noindent In Hartree atomic units, $R_\infty = \frac{1}{2}$, giving
\beq
\label{eq:HEnergyAU}
E_{n,H} = -\frac{1}{2 n^2}.
\eeq

\noindent Due to the reduced mass of half that of hydrogen, the energy of positronium is
\beq
\label{eq:PsEnergyAU}
E_{n,Ps} = -\frac{1}{4 n^2}.
\eeq

\noindent For a given level \emph{n}, the energy of positronium is half that of hydrogen.  The ionization energy is approximately -6.8 eV, versus that of -13.6 eV for hydrogen.  Figure \ref{fig:HPsLevels} compares the energy levels of positronium and hydrogen.

\begin{figure}[H]
	\centering
	\includegraphics[width=5.5in]{HPsLevels}
	\caption{Hydrogen and positronium energy levels}
	\label{fig:HPsLevels}
\end{figure}

\section{Positronium Hydride}
\label{sec:PsH}
Positronium hydride (PsH) is a molecule comprised of a hydrogen atom and a positronium atom.  Predicted by A. Ore in 1951 \cite{Ore1951}, positronium hydride was not experimentally verified until 1992 by Schrader \cite{Schrader1992}. There was the question of whether PsH was atomic or molecular, but Saito \cite{Saito2000}

Use \cite{Heyrovska2011} and \cite{Saito2000}


\nicebox{Title}{This is the content of my box}
\todoi{Look at quotchap (chapter quotes) and thumbs (chapter thumbs) packages}
\todoi{Go through tips in \url{https://www.cs.purdue.edu/homes/dec/essay.dissertation.html}}
\todoi{Nomenclature with \url{http://www.howtotex.com/packages/create-a-simple-nomenclature-with-the-longtable-package/} or the nomencl package}



\section{Ideas for Introduction}

\begin{itemize}
	\item Bound State Versus Scattering Problems
	\item Comparison with Other Scattering Methods
	\item Partial waves
	\item Types of possible targets
\end{itemize}



\section{A Note on Units}
\label{sec:Units}

\begin{verbatim}
http://en.wikipedia.org/wiki/Natural_units
http://www.phys.ksu.edu/personal/cdlin/class/class11a-amo2/atomic_units.pdf
http://ilan.schnell-web.net/physics/rydberg.pdf
http://exciting-code.org/ref:input
http://physics.nist.gov/cuu/Constants/energy.html
\end{verbatim}


Unless otherwise stated, values throughout are given in atomic units, i.e. $\hbar = m_e = e = 4\pi\epsilon_0 = 1$. Energies are given in hartrees, with $\SI{1}{\hartree} = \SI{27.211 385 05(60)}{\electronvolt}$ \cite{Mohr2012,NISTConversions}. Momentum is given as units of $a_0^{-1}$. Cross sections are given in units of $\pi a_0^2$, unless otherwise noted.



\end{document}