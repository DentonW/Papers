\pdfbookmark[1]{Abstract}{abstract}

\thispagestyle{plain}
\begin{center}
    \large
    \textbf{Abstract}
\end{center}

Positronium-hydrogen (Ps-H) scattering is of interest, as it is a fundamental 
four-body Coulomb problem. We have investigated low-energy Ps-H scattering 
below the Ps(n=2) excitation threshold using the Kohn variational method and 
variants of the method with a trial wavefunction that includes highly 
correlated Hylleraas-type short-range terms. With this, we have computed 
phase shifts for the first six partial waves for both the singlet and triplet 
states. The $^{1,3}$S and $^{1,3}$P phase shifts are highly accurate results
and could potentially be viewed as benchmark results.
Resonance positions and widths for the $^1$S-, $^1$P-, $^1$D-, and
$^1$F-waves have been calculated.

We present elastic integrated, elastic differential, and momentum transfer 
cross sections using all six partial waves and note interesting features of 
each. We use multiple effective range theories, including several that 
explicitly take into account the long-range van der Waals interaction, to 
investigate scattering lengths and effective ranges.

We give an elegant formalism that combines all variants of the Kohn variational
method into a single form. Along with this, we have also developed a general
formalism for Kohn-type matrix elements that allows us to evaluate arbitrary 
partial waves with a single codebase. Computational strategies we have 
developed and use in this work are also discussed. Finally, possible future 
extensions of this work are given.
