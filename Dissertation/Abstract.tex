\pdfbookmark[1]{Abstract}{abstract}

\thispagestyle{plain}
\begin{center}
    \large
    \textbf{Abstract}
\end{center}

Positronium-hydrogen (Ps-H) scattering is of interest, as it is a fundamental 
four-body Coulomb problem. We have investigated low-energy Ps-H scattering 
below the Ps(n=2) excitation threshold using the Kohn variational method and 
variants of the method with a trial wavefunction that includes highly 
correlated Hylleraas-type short-range terms. With this, we have computed 
phase shifts for the first six partial waves for both the singlet and triplet 
states. The $^{1,3}$S and $^{1,3}$P phase shifts are highly accurate results
and could potentially be viewed as benchmark results.
Resonance positions and widths for the $^1$S-, $^1$P-, $^1$D-, and
$^1$F-waves have been calculated.

We give an elegant formalism that combines all variants of the Kohn variational
method into a single form. Along with this, we have also developed a general
formalism that allows us to evaluate arbitrary partial waves with a single
codebase.

Computational strategies we have developed and use in this work are presented.

We present elastic integrated, elastic differential, and momentum transfer 
cross sections using all six partial waves. 

\todoi{Taken from PRA paper}
The differential cross section exhibits interesting features, including a 
change from slightly backward peaked to forward peaked scattering as the 
energy of the incident positronium increases and rich structure due to 
multiple resonances near the Ps(n=2) threshold.

We use multiple effective range theories, including several that explicitly
take into account the long-range van der Waals interaction, to investigate
scattering lengths and effective ranges.

Possible future extensions of this work are also given.

including