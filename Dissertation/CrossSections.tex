\documentclass[Dissertation.tex]{subfiles} 
\begin{document}

\clearpage
\pagebreak
\newpage

\chapter{Cross Sections}
\label{chp:CrossSections}

\todoi{Discussion like in PRA paper about how related to scattering length}

\subsection{Integrated Cross Sections}
\label{sec:totalcross}

\begin{equation}
\label{eq:PartialCross}
\sigma_{el,\ell}^\pm = \frac{4}{\kappa^2} (2\ell+1) \sin^2 \delta_\ell^\pm
\end{equation}

\begin{equation}
\label{eq:TotalCross}
\sigma_{el}^\pm = \frac{4}{\kappa^2} \sum_{\ell=0}^\infty (2\ell+1) \sin^2 \delta_\ell^\pm
\end{equation}

\begin{figure}[H]
	\centering
	\includegraphics[width=5.25in]{singlet-cross-sections}
	\caption{Combined complex Kohn singlet cross sections}
	\label{fig:singlet-cross-sections}
\end{figure}

\begin{figure}[H]
	\centering
	\includegraphics[width=5.25in]{triplet-cross-sections}
	\caption{Combined complex Kohn triplet cross sections}
	\label{fig:triplet-cross-sections}
\end{figure}

\begin{figure}[H]
	\centering
	\includegraphics[width=\textwidth]{percentage-cross-sections}
	\caption[Percentage contribution to total cross section]{Percentage contribution to total cross section from each partial wave for the singlet (a) and triplet (b).}
	\label{fig:percentage-cross-sections}
\end{figure}

\begin{figure}[H]
	\centering
	\includegraphics[width=\textwidth]{percentage-cross-sections-full}
	\caption[Percentage contribution to total cross section]{Percentage contribution to total cross section from each partial wave for the singlet (a) and triplet (b).}
	\label{fig:percentage-cross-sections-full}
\end{figure}

\begin{figure}[H]
	\centering
	\includegraphics[width=6in]{combined-cross-sections}
	\caption[Cross sections.]{Cross sections. CC results are from Ref. \cite{Walters2004}, and SE results are from Ref. \cite{Hara1975}.}
	\label{fig:combined-cross-sections}
\end{figure}


For the total, differential and momentum transfer cross sections, the singlet contributes $\frac{1}{4}$, and the triplet contributes $\frac{3}{4}$, giving spin-weighted cross sections, i.e. \cite{Ward1987}
\beq
\label{eq:SpinWeightCS}
\sigma = \tfrac{1}{4} \sigma^+ + \tfrac{3}{4} \sigma^-.
\eeq

\todoi{Put partial wave cross sections in each chapter/section?}

\label{sec:crosscompare}
\begin{figure}[H]
	\centering
	\includegraphics[width=\textwidth]{spd-singlet-cross-sections}
	\caption[Comparisons of $^1$S, $^1$P, and $^1$D cross sections.]{Comparisons of $^1$S, $^1$P, and $^1$D cross sections. CC results are from Ref. \cite{Walters2004}.}
	\label{fig:spd-singlet-cross-sections}
\end{figure}

\begin{figure}[H]
	\centering
	\includegraphics[width=\textwidth]{spd-triplet-cross-sections}
	\caption[Comparisons of $^3$S, $^3$P, and $^3$D cross sections.]{Comparisons of $^3$S, $^3$P, and $^3$D cross sections. CC results for $^3$S and $^3$P are from Ref. \cite{Walters2004}. CC results for $^3$D are from Ref. \cite{Blackwood2002}.}
	\label{fig:spd-triplet-cross-sections}
\end{figure}


\begin{figure}[H]
	\centering
	\includegraphics[width=\textwidth]{fgh-born-cross-sections}
	\caption[Comparisons of F-, G-, and H-wave cross sections]{Comparisons of F-, G-, and H-wave cross sections between complex Kohn and Born approximation}
	\label{fig:fgh-born-cross-sections}
\end{figure}


\section{Differential Cross Sections}
\label{sec:diffcross}

\todoi{Make sure final graphs have the full 50 $\theta$ terms}

\begin{align}
\label{eq:DiffCross}
\nonumber \frac{d\sigma_{el}^\pm}{d\Omega} = \frac{1}{\kappa^2} & \sum_{\ell=0}^\infty \sum_{\ell^\prime=0}^\infty (2\ell+1)(2\ell^\prime+1) \exp\left\{\ii \left[\delta_\ell(\kappa) - \delta_{\ell^\prime}(\kappa) \right] \right\} \\
& \times \sin\delta_\ell^\pm(\kappa) \sin\delta_{\ell^\prime}^\pm(\kappa) P_\ell(\cos\theta) P_{\ell^\prime}(\cos\theta)\,.
\end{align}

\begin{figure}[H]
	\centering
	\includegraphics[width=5.25in]{diff-cross-sections-singlet}
	\caption{Singlet differential cross sections}
	\label{fig:diff-cross-sections-singlet}
\end{figure}

\begin{figure}[H]
	\centering
	\includegraphics[width=5.25in]{diff-cross-sections-triplet}
	\caption{Triplet differential cross sections}
	\label{fig:diff-cross-sections-triplet}
\end{figure}

\begin{figure}[H]
	\centering
	\includegraphics[width=\textwidth]{diff-cross-sections-combined}
	\caption{Combined differential cross sections}
	\label{fig:diff-cross-sections-combined}
\end{figure}

\begin{figure}[H]
	\centering
	\includegraphics[width=\textwidth]{percent-diff-cross-sections}
	\caption[Percent difference of differential cross sections at selected angles]{Percent difference of $\frac{d\sigma_{el}}{d\Omega}$ for upper limit of summations in \ref{eq:DiffCross} as $\ell = 4$ versus $\ell = 5$ for selected angles}
	\label{fig:percent-diff-cross-sections}
\end{figure}

\begin{figure}[H]
	\centering
	\includegraphics[width=5.5in]{percent-diff-cross-sections-full}
	\caption[Percent difference of differential cross sections at all angles]{Percent difference of $\frac{d\sigma_{el}}{d\Omega}$ for upper limit of summations in \ref{eq:DiffCross} as $\ell = 4$ versus $\ell = 5$ for all angles and energies}
	\label{fig:percent-diff-cross-sections-full}
\end{figure}

\begin{figure}[H]
	\centering
	\includegraphics[width=5.25in]{diff-cross-section-2D-theta}
	\caption{Differential cross sections for selected $\theta$}
	\label{fig:diff-cross-section-2D-theta}
\end{figure}

\begin{figure}[H]
	\centering
	\includegraphics[width=5.25in]{diff-cross-section-2D-kappa}
	\caption{Differential cross sections for selected $\kappa$}
	\label{fig:diff-cross-section-2D-kappa}
\end{figure}


\section{Other Cross Sections and Comparisons}
\label{sec:OtherCross}

\todoi{Reword this paragraph. It's currently copied from the PRA article.}
The momentum transfer cross sections can be useful in plasma applications \cite{Wang2014, McEachran2014}. These cross sections have been measured for Ps with multiple atomic and molecular molecular targets \cite{Nagashima1998,Saito2003}. The momentum transfer cross sections are given by \cite{Bransden2003}
\begin{equation}
\label{eq:MomentumCross}
\sigma_{m}^\pm = \frac{4}{\kappa^2} \sum_{\ell=0}^\infty (\ell+1) \sin^2 (\delta_\ell^\pm - \delta_{\ell+1}^\pm) .
\end{equation}
The ortho-para conversion cross sections give the conversion of the projectile ortho-Ps to para-Ps by \cite{Hara1975}
\begin{equation}
\label{eq:OrthoParaCross}
\sigma_{c} = \frac{1}{4 \kappa^2} \sum_{\ell=0}^\infty (2 \ell+1) \sin^2 (\delta_\ell^+ - \delta_\ell^-).
\end{equation}

\begin{figure}[H]
	\centering
	\includegraphics[width=5.25in]{momentum-cross-sections}
	\caption{Momentum transfer cross sections}
	\label{fig:momentum-cross-sections}
\end{figure}

\begin{figure}[H]
	\centering
	\includegraphics[width=5.25in]{orthopara-cross-sections}
	\caption{Ortho-para conversion cross sections}
	\label{fig:orthopara-cross-sections}
\end{figure}
\todoi{OP in \cite{Biswas1998}}

\begin{figure}[H]
	\centering
	\includegraphics[width=5.25in]{cross-section-comparisons}
	\caption{Comparison of cross sections}
	\label{fig:cross-section-comparisons}
\end{figure}





\biblio
\end{document}