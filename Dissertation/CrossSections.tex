% -*- root: Dissertation.tex -*-
\documentclass[Dissertation.tex]{subfiles} 
\begin{document}

\clearpage
\pagebreak
\newpage

\chapter{Cross Sections}
\label{chp:CrossSections}

\todoi{Discussion like in PRA paper about how related to scattering length}
\todoi{integrated elastic vs. elastic integrated}

We get the phase shifts directly from the Kohn variational methods, but a more relevant quantity for experiments is the cross section. Cross sections essentially give the strength of the interaction and are a quantity that can be measured experimentally.

\begin{figure}[H]
	\centering
	\includegraphics[width=4.5in]{ScatteringDiagram}
	\caption[General scattering diagram]{General scattering diagram. Modified from \url{http://commons.wikimedia.org/wiki/File:ScatteringDiagram.svg}. Permission granted under the Creative Commons Attribution-Share Alike 3.0 Unported license.}
	\label{fig:ScatteringDiagram}
\end{figure}

\Cref{fig:ScatteringDiagram} shows a general scattering diagram. If we have
azimuthal symmetry, as we assume for Ps-H scattering, $\varphi$ is unimportant.
The quantity that we are most interested in from this is the ratio
$\frac{d\sigma}{d\Omega}$, which is is the differential cross section.
The integrated cross section (also known simply as the cross section) is
related to the differential cross section by integrating over $d\Omega$:
\begin{equation}
\label{eq:TotalDiffCross}
\sigma = \int \frac{d\sigma}{d\Omega} \, d\Omega
 = \int_0^{2\uppi} \int_0^{\uppi} \frac{d\sigma}{d\Omega} \sin\theta \, d\theta \, d\varphi.
\end{equation}


\section{Integrated Cross Sections}
\label{sec:totalcross}

The partial wave cross sections can be 
related to the phase shifts by
\citep[p.584]{Bransden2003}
\begin{equation}
\label{eq:PartialCross}
\sigma_{el,\ell}^\pm = \frac{4}{\kappa^2} (2\ell+1) \sin^2 \delta_\ell^\pm.
\end{equation}
In addition to the relation of the integrated cross sections to the
differential cross sections in \cref{eq:TotalDiffCross},
using the partial wave expansion, the integrated cross sections can also be 
expressed as \citep[p.]{Bransden2003}
\begin{equation}
\label{eq:TotalCross}
\sigma_{el}^\pm = \sum_{\ell=0}^\infty \sigma_{el,\ell}^\pm = \frac{4}{\kappa^2} \sum_{\ell=0}^\infty (2\ell+1) \sin^2 \delta_\ell^\pm.
\end{equation}
We consider the Ps and H to be unpolarized, so
for each of the cross sections other than the ortho-para conversion, the 
singlet contributes $\frac{1}{4}$, and the triplet contributes $\frac{3}{4}$, 
giving spin-weighted cross sections, i.e. \cite{Ward1987,Blackwood2002}
\beq
\label{eq:SpinWeightCS}
\sigma = \tfrac{1}{4} \sigma^+ + \tfrac{3}{4} \sigma^-.
\eeq

\Cref{fig:singlet-cross-sections,fig:triplet-cross-sections} show the partial
wave cross sections for the singlet and triplet, respectively. The ``Total''
in each is the sum of each of the singlet or triplet partial waves through the
H-wave.

The triplet cross section in \cref{fig:triplet-cross-sections} is dominated 
almost completely by the $^3$S partial wave. The $^3$P partial wave 
contributes less, and the $^3$D-wave barely registers on the graph. The 
higher partial waves contribute nearly negligible amounts. The total follows
closely with $^3$S, but the contribution from $^3$P is evident.

The singlet cross sections in \cref{fig:singlet-cross-sections} are more
interesting due to their larger partial wave cross sections and the resonances
from the first four partial waves. The ``Total'' has peaks from each of the
first three partial waves, giving it a more complicated structure than the
total from the triplet in \cref{fig:triplet-cross-sections}. As mentioned
in \cref{sec:Resonances}, each of these resonances goes through a phase shift
change of $\uppi$. With the $(2\ell+1)$ factor in \cref{eq:TotalCross},
higher partial wave resonances have a larger contribution to the total
cross sections, as long as the background does not change significantly.
In \cref{fig:singlet-cross-sections}, the $^1$D resonance clearly has the
most significant contribution to the integrated elastic cross section out of
the resonances for the first three partial waves. The $^1$F resonance barely
contributes, but this is because the resonance lies past the inelastic 
threshold.

Also interesting is the minimum in the total singlet cross section at 
approximately $\SI{0.25}{eV}$ and then the maximum at $\SI{0.74}{eV}$. The dip
here at low energy is due to the mixing of the $^1$S and $^1$P cross sections.
The $^1$S-wave cross section is decreasing rapidly while the $^1$P-wave cross
section is increasing, giving this feature. The maximum is due primarily to the
$^1$P-wave.

\begin{figure}[H]
	\centering
	\includegraphics[width=5.25in]{singlet-cross-sections}
	\caption{Complex Kohn singlet cross sections}
	\label{fig:singlet-cross-sections}
\end{figure}

\begin{figure}[H]
	\centering
	\includegraphics[width=5.25in]{triplet-cross-sections}
	\caption{Complex Kohn triplet cross sections}
	\label{fig:triplet-cross-sections}
\end{figure}

From \cref{eq:TotalCross}, for exact cross sections, we have to do an infinite
summation. In practice, we add partial waves until the cross section no longer
changes a significant amount. In \cref{fig:percentage-cross-sections}(a), we
consider what the percentage contribution to the total cross section for
the singlet partial waves is at the 7 ``standard'' $\kappa$ values. From this,
we can see the trend that the $^1$S-wave is by far the greatest contribution at
small $\kappa$, but the $^1$P-wave becomes the dominant contribution through
most of the rest of the energy range. When $\kappa \geq 0.5$, the $^1$D-wave is
no longer a negligible contribution. The $^1$F-wave barely contributes, even
for $\kappa = 0.7$, and the $^1$G- and $^1$H-wave are not shown due to their
insignificant contributions.

The corresponding bar chart for the triplet is in
\cref{fig:percentage-cross-sections}(b). As qualitatively described earlier,
the contribution to the total elastic integrated cross section for
the triplet is mainly due to the $^3$S-wave. The $^3$P-wave contributes about
$20\%$ at $\kappa = 0.7$, and the $^3$D-wave contribution is nearly negligible.

\begin{figure}[H]
	\centering
	\includegraphics[width=\textwidth]{percentage-cross-sections}
	\caption[Percentage contribution to total cross section]{Percentage contribution to total cross section from each partial wave for the singlet (a) and triplet (b).}
	\label{fig:percentage-cross-sections}
\end{figure}

\Cref{fig:percentage-cross-sections-full} also shows the percent contributions to 
the total integrated elastic cross sections from each partial wave but shows
this data for the entire energy range. The triplet graph in
\cref{fig:percentage-cross-sections-full}(b) again shows that only the $^3$S- and
$^3$P-waves give very significant contributions. The $^3$D-wave gives a
non-negligible but small contribution near the inelastic threshold

The corresponding singlet graph in \cref{fig:percentage-cross-sections-full}(a)
is more difficult to interpret due to the resonances. The $^1$S- and
$^1$P-wave dominate until about $\SI{3}{eV}$, and the $^1$P-wave is the largest 
contribution from approximately $\SI{0.5}{eV}$ to slightly over $\SI{4}{eV}$.
The $^1$D-wave starts being a more dominant contribution past $\SI{4}{eV}$, and
it has spikes in the percentage when the $^1$S and $^1$P resonances go to their
minima.

\begin{figure}[H]
	\centering
	\includegraphics[width=\textwidth]{percentage-cross-sections-full}
	\caption[Percentage contribution to total cross section]{Percentage contribution to total cross section from each partial wave for the singlet (a) and triplet (b).}
	\label{fig:percentage-cross-sections-full}
\end{figure}

For a more quantitative approach, we calculate the percent contributions of the
different partial waves to the total spin-weighted cross section as presented
in \cref{tab:PercentToCross}. The second and third columns give the
contribution from the singlet and triplet combined for each partial wave. From
this, it is noticeable that the F-wave should be included, though the average
is less than $0.6\%$. The G-wave and H-wave barely contribute, with the H-wave
contribution from both the singlet and triplet less than $0.002\%$ on 
average.

For the last 4 columns of \cref{tab:PercentToCross}, we also compare each
partial wave's contribution to the total spin-weighted cross section, but we
separate out the singlet and triplet. From columns 4 and 5 for the singlet,
we see that the $^1$F-wave is important, and it gives most of the combined
$^{1,3}$F-wave contribution to the elastic integrated cross section. The 
$^3$F-wave contribution is $0.0011\%$ on average, and the $^3$D contribution 
is less than $0.41\%$ through the entire energy range. We include all partial
waves through the H-wave in our final results, but these would be relatively
well converged if we stopped adding partial waves to \cref{eq:TotalCross} at
the F-wave. For comparison, the CC \cite{Walters2004} cross section shown in
\cref{fig:combined-cross-sections} uses partial waves through the G-wave, but
their graph would likely not change if they included the H-wave.

\begin{table}[H]
\centering
\begin{tabular}{cllllll}
\toprule
$\ell$ & Max. $^{1,3}\ell$ \% & Avg. $^{1,3}\ell$ \% & Max. $^1\ell$ \% & Avg. $^1\ell$ \% & Max. $^3\ell$ \% & Avg. $^3\ell$ \% \\
\midrule
S-wave & 100.0\%  & 60.61\%   & 57.80\%  & 20.80\%  & 20.86\%   & 13.27\% \\
P-wave & 45.97\%  & 30.21\%   & 42.89\%  & 24.36\%  & 4.323\%   & 1.947\%  \\
D-wave & 42.07\%  & 8.565\%   & 41.61\%  & 8.178\%  & 0.401\%   & 0.129\%  \\
F-wave & 3.782\%  & 0.596\%   & 3.765\%  & 0.593\%  & 0.00606\% & 0.0011\%  \\
G-wave & 0.103\%  & 0.022\%   & 0.0994\% & 0.0206\% & 0.00106\% & 0.00046\%  \\
H-wave & 0.0083\% & 0.0019\%  & 0.0062\% & 0.0013\% & 0.00070\% & 0.00021\% \\ 
\bottomrule
\end{tabular}
\caption{Percent contribution to the elastic integrated cross section from each partial wave for both the maximum and average for the entire energy range}
\label{tab:PercentToCross}
\end{table}

Finally, combining the singlet and triplet total integrated elastic cross 
sections using the spin-weighting in \cref{eq:SpinWeightCS} gives the result
in \cref{fig:combined-cross-sections}. We include the spin-weighted singlet
and triplet cross sections for comparison with the combined total cross section.
The comparison to the CC \cite{Walters2004} and SE \cite{Hara1975} results is
made possible by using the CurveSnap \cite{CurveSnap} program to extract the
curves from the respective papers. The SE curve gives a decent approximation to
the background without any resonances. We would not expect the SE to give
resonance information, but it is a good first approximation, except at low
energy, where it overestimates the cross section greatly. There is
good agreement between the complex Kohn and CC cross sections, but the CC
results have resonances shifted to higher energy. The complex Kohn resonances
correspond better than the CC to the resonance positions that are given in the
complex rotation results of Yan and Ho \cite{Yan1999,Yan1998a,Ho1998,Ho2000}, as
seen in \cref{tab:SWaveResonancesOther,tab:PWaveResonancesOther,tab:DWaveResonancesOther,tab:FWaveResonanceComparisons}.

\begin{figure}[H]
	\centering
	\includegraphics[width=6in]{combined-cross-sections}
	\caption[Cross sections.]{Cross sections. CC results are from Ref. \cite{Walters2004}, and SE results are from Ref. \cite{Hara1975}.}
	\label{fig:combined-cross-sections}
\end{figure}
\todoi{Just put 'spin-weighted' in caption}

For partial wave cross section data available from the CC papers
\cite{Walters2004,Blackwood2002,Blackwood2002b}, we compare the complex Kohn
partial wave cross sections to these in
\cref{fig:spd-singlet-cross-sections,fig:spd-triplet-cross-sections}.
The CC 9Ps9H + H$^-$ cross sections are more accurate than the CC 22Ps1H + H$^-$,
as pointed out in those papers. They also have CC data with the 9Ps9H
approximation \cite{Blackwood2002}, but that gives less accurate results than
the CC 22Ps1H + H$^-$.

Other than the shifted resonance positions, the CC results tend to match well
with the complex Kohn. However, there are some features that are worth
noticing. For $^1$P in \cref{fig:spd-singlet-cross-sections}, the CC
9Ps9H + H$^-$ maximum is lower than the complex Kohn maximum. The CC
22Ps1H + H$^-$ maximum is even lower, so we would expect that if more
eigen- and pseudo-states are included, the CC maximum would likely match up 
with the complex Kohn. Also, the $^3$D CC cross section is much smaller than
the complex Kohn. This is a reflection of the fact that in
\cref{fig:DWavePhase,tab:DWaveComparisons}, the CC results are higher than the
complex Kohn, making them less negative and the corresponding cross sections
smaller. Noting the magnitude of the $^3$D cross section for both methods, the
contribution to the full summed elastic integrated cross section is small.
As a simple test, we tried replacing the complex Kohn $\kappa = 0.1 - 0.7$
phase shifts with those of the CC, and the change to the cross section is
less than 0.084\% for this range. Consequently, for the full summed
cross section, the discrepancy between the complex Kohn and the CC results
does not change things significantly.

\begin{figure}[H]
	\centering
	\includegraphics[width=\textwidth]{spd-singlet-cross-sections}
	\caption[Comparisons of $^1$S, $^1$P, and $^1$D cross sections.]{Comparisons of $^1$S, $^1$P, and $^1$D cross sections. CC 9Ps9H + H$^-$ results are from Ref.~\cite{Walters2004}, and CC 22Ps1H + H$^-$ results are from Ref.~\cite{Blackwood2002b}.}
	\label{fig:spd-singlet-cross-sections}
\end{figure}

\begin{figure}[H]
	\centering
	\includegraphics[width=\textwidth]{spd-triplet-cross-sections}
	\caption[Comparisons of $^3$S, $^3$P, and $^3$D cross sections.]{Comparisons of $^3$S, $^3$P, and $^3$D cross sections. CC results for $^3$S and $^3$P are from Ref. \cite{Walters2004}. CC results for $^3$D are from Ref. \cite{Blackwood2002}.}
	\label{fig:spd-triplet-cross-sections}
\end{figure}

Despite the disappointing performance of the Born and modified Born 
approximations for the phase shifts of the higher partial waves in
\cref{chp:HigherWaves}, we also compare the Born approximation cross sections
to see if the discrepancy is as large. In \cref{fig:fgh-born-cross-sections},
we see that the Born approximation does not match up with either the singlet
or triplet for the F-, G-, or H-waves. Interestingly, we see that the plots
for the singlet and the Born look approximately the same in (a), (b), and (c)
but with a different vertical scale. This seems to indicate that the Born and complex
Kohn results will only be approximately equal at a much higher value of $\ell$
than $\ell = 5$. The phase shifts become very small for partial waves higher
than the H-wave, so we would not be able to investigate these much further 
without improving the accuracy of our long-range code. Similar to what we
noticed in \cref{sec:SingTripCompare}, we also note that the singlet and
triplet phase shifts match for higher $\kappa$ as $\ell$ increases.

\begin{figure}[H]
	\centering
	\includegraphics[width=\textwidth]{fgh-born-cross-sections}
	\caption[Comparisons of F-, G-, and H-wave cross sections]{Comparisons of F-, G-, and H-wave cross sections between complex Kohn and Born approximation}
	\label{fig:fgh-born-cross-sections}
\end{figure}


\section{Differential Cross Sections}
\label{sec:diffcross}

\todoi{Make sure final graphs have the full 50 $\theta$ terms}

The differential cross section is important through \cref{eq:TotalDiffCross},
but this also gives information about the angular dependence, $\theta$, of the
system. The differential cross sections can be calculated from the phase shifts by
\begin{align}
\label{eq:DiffCross}
\nonumber \frac{d\sigma_{el}^\pm}{d\Omega} = \frac{1}{\kappa^2} & \sum_{\ell=0}^\infty \sum_{\ell^\prime=0}^\infty (2\ell+1)(2\ell^\prime+1) \exp\left\{\ii \left[\delta_\ell(\kappa) - \delta_{\ell^\prime}(\kappa) \right] \right\} \\
& \times \sin\delta_\ell^\pm(\kappa) \sin\delta_{\ell^\prime}^\pm(\kappa) P_\ell(\cos\theta) P_{\ell^\prime}(\cos\theta)\,.
\end{align}
This expression has the complex-valued exponential, for which we can use the
well-known Euler formula of
\begin{equation}
\label{eq:ComplexExp}
\ee^{\ii x} = \cos x + \ii \sin x
\end{equation}
to split this into real-valued and imaginary-valued parts. As long as the
finite truncation of the upper limits of the summations are the same, the
imaginary part becomes 0 to within numerical accuracy. So we can use the
approximation of
\begin{align}
\label{eq:DiffCross1}
\nonumber \frac{d\sigma_{el}^\pm}{d\Omega} \approx \frac{1}{\kappa^2} & \sum_{\ell=0}^{\ell_{max}} \sum_{\ell^\prime=0}^{\ell_{max}} (2\ell+1)(2\ell^\prime+1) \cos \left[\delta_\ell(\kappa) - \delta_{\ell^\prime}(\kappa) \right] \\
& \times \sin\delta_\ell^\pm(\kappa) \sin\delta_{\ell^\prime}^\pm(\kappa) P_\ell(\cos\theta) P_{\ell^\prime}(\cos\theta)\,.
\end{align}

Graphs of the differential cross sections for the singlet and triplet are found
in \cref{fig:diff-cross-sections-singlet,fig:diff-cross-sections-triplet},
respectively. Note that the $\theta$ axis is plotting backwards so that the
features are visible instead of being obscured by higher differential cross
sections in the front of the graph.
The triplet $\frac{d\sigma_{el}^-}{d\Omega}$ is not particularly
remarkable, being smooth and having a maximum at intermediate energies for the
forward direction $(\theta = 0)$. The values are all less than 7 $a_0^2/\textrm{sr}$.
It is also of note that in the backward scattering direction $(\theta = \uppi)$,
the triplet differential cross section quickly becomes very small.

In contrast, the singlet differential cross section, $\frac{d\sigma_{el}^+}{d\Omega}$,
shown in \cref{fig:diff-cross-sections-singlet} has a much more complicated
structure. The resonances are clearly visible in the plot, and the highest peak
extending to 110.5 $a_0^2/\textrm{sr}$ is due to the D-wave resonance, which is
also the dominant resonance in the integrated cross section shown in
\cref{fig:singlet-cross-sections,fig:combined-cross-sections}. The features in
the integrated cross section graphs can easily be matched up to features in the
differential cross section, including the rest of the resonances. The maximum
and dip described in \cref{sec:totalcross} for the integrated cross section can
also be seen in \cref{fig:diff-cross-sections-singlet}, especially in the
forward direction.

\begin{figure}[H]
	\centering
	\includegraphics[width=5.25in]{diff-cross-sections-singlet}
	\caption{Singlet differential cross section}
	\label{fig:diff-cross-sections-singlet}
\end{figure}

\begin{figure}[H]
	\centering
	\includegraphics[width=5.25in]{diff-cross-sections-triplet}
	\caption{Triplet differential cross section}
	\label{fig:diff-cross-sections-triplet}
\end{figure}

Similar to the integrated cross section, we combine these using the same type
of spin-weighting given in \cref{eq:SpinWeightCS}. Due to the nearly featureless
nature of the triplet in \cref{fig:diff-cross-sections-triplet},
the combined differential cross section in 
\cref{fig:diff-cross-sections-combined} looks very similar to 
\cref{fig:diff-cross-sections-singlet}, but the 1/4 weighting of the singlet
brings the vertical scale down. The forward direction is enhanced slightly by
the maximum in the triplet in \cref{fig:diff-cross-sections-triplet}.

\begin{figure}[H]%
    \centering
    \subfloat{{\includegraphics[width=6in]{diff-cross-sections-combined} }} \\%
    %\qquad
    \subfloat{{\includegraphics[width=6in]{diff-cross-sections-combined-2} }}%
    \caption{(Color online) The combined spin-weighted elastic differential cross section for Ps-H scattering at two different viewing angles}%
    \label{fig:combined-diff-cross-sections}%
\end{figure}

With the data in \cref{fig:diff-cross-sections-combined}, it is illustrative 
to plot 2-dimensional versions to see trends more clearly. In
\cref{fig:diff-cross-section-2D-theta}, we restrict $\theta$ and vary
$E_{\bm \kappa}$ to see the the energy-dependence for several $\theta$ values.
We are particularly interested in the forward ($\theta = 0\degree$) and backward
($\theta = 180\degree$) directions but also include $\theta = 90\degree$.
From this figure, it is clear that scattering in the forward direction is
dominant past approximately $\SI{0.46}{eV}$. The maximum for forward scattering
is due to the $^1$D resonance. Particularly interesting is the contribution
from backward scattering for low $E_{\bm \kappa}$ and the dip in the forward
scattering direction, which corresponds to the dip in
\cref{fig:combined-cross-sections}. All angles give essentially the same value
at very low energy, as we would expect.


\begin{figure}[H]
	\centering
	\includegraphics[width=5.25in]{diff-cross-section-2D-theta}
	\caption{Differential cross sections for selected $\theta$}
	\label{fig:diff-cross-section-2D-theta}
\end{figure}

In \cref{fig:diff-cross-section-2D-kappa}, we instead fix values of
$E_{\bm \kappa}$ and plot with respect to $\theta$. The legend gives the $\kappa$ 
value instead, so it is clear what specific values we are plotting. At low $
\kappa$ (0.05 in the plot), the differential cross section is nearly 
isotropic, with a slight bias toward backward scattering. As the momentum is 
increased to 0.2, backward scattering is more prominent. Between $\kappa = 0.2$
and 0.3, there is an abrupt change in the differential cross section, where 
it becomes much more forward peaked, with a decreasing contribution to the 
backward direction, and a minimum at approximately $100\degree$. As the 
momentum is increased further, the differential cross section becomes very 
strongly forward peaked, with even further decreases at larger angles and a 
nearly constant value from about $100\degree$.  We see from
\cref{fig:combined-diff-cross-sections} that the majority of the scattering
takes place between 0 and 1 radians.

\begin{figure}[H]
	\centering
	\includegraphics[width=5.25in]{diff-cross-section-2D-kappa}
	\caption{Differential cross sections for selected $\kappa$}
	\label{fig:diff-cross-section-2D-kappa}
\end{figure}



We find that the elastic differential cross section converges slower with 
respect to $\ell$ than the elastic integrated cross section in
\cref{sec:totalcross}. The differential cross section is more difficult to 
quantitatively evaluate the convergence than the integrated cross section, 
since it has mixing between all of the included partial waves. Due to the
double summation, there are $\ell_{max}^2$ terms in the differential cross
section. We calculate $\frac{d\sigma_{el}}{d\Omega}$ for two subsequent
values of $\ell_{max}$, then find the percent difference between these.

\Cref{fig:percent-diff-cross-sections-full-g,fig:percent-diff-cross-sections-full-h}
shows this percentage difference for all angles and energies. These two figures
look similar but have different vertical scales, as we would hope for if
there was convergence. There are two important trends here. One is that the 
differential cross section is well converged at low energies, and the other 
is that it is less converged in the backward scattering direction than in the 
forward direction. In \cref{fig:percent-diff-cross-sections-full-h}, which
includes all partial waves through the H-wave, the percentage differences are
all below 4\%, and most of the $E_{\bm \kappa}$ and $\theta$ range is 
much less than this. This indicates that the differential cross section is
relatively well converged.

\Cref{fig:percent-diff-cross-sections-g,fig:percent-diff-cross-sections-h}
show the same data but for selected angles. We see here that
$\theta = 90\degree$ is the best converged angle out of the three, and
the backward scattering direction of $\theta = 180\degree$ is the worst
converged. Again, as seen in \cref{fig:percent-diff-cross-sections-h}, the
differential cross section is relatively well converged if we include the
H-wave.

We would expect that as $\ell$ increases, angles near $\theta = 180\degree$ 
will be the most sensitive to adding terms to the differential cross section. 
The partial wave expansion \cite[p.583]{Bransden2003} has a $\LegendreP{\ell,\cos\theta}$ for 
each term. If we set $\theta = 0\degree$, each of the Legendre polynomials
equals 1, meaning that each term added is positive. If $\theta = 180\degree$,
the Legendre polynomial alternates between 1 and -1, i.e. $(-1)^\ell$. This is
then an alternating series, which we would expect to converge more slowly.
The minimum for $\theta = 90\degree$ can be explained by every $\ell$ odd term
equaling 0 from the Legendre polynomial.

\todoi{I should really define a symbol (maybe $\Delta$?) for the percentage difference.}


\begin{figure}[H]
	\centering
	\includegraphics[width=5.5in]{percent-diff-cross-sections-full-g}
	\caption[Percent difference of differential cross sections at all angles]{Percent difference of $\frac{d\sigma_{el}}{d\Omega}$ for upper limit of summations in \ref{eq:DiffCross} as $\ell_{max} = 3$ versus $\ell_{max} = 4$ for all angles and energies}
	\label{fig:percent-diff-cross-sections-full-g}
\end{figure}

\begin{figure}[H]
	\centering
	\includegraphics[width=5.5in]{percent-diff-cross-sections-full-h}
	\caption[Percent difference of differential cross sections at all angles]{Percent difference of $\frac{d\sigma_{el}}{d\Omega}$ for upper limit of summations in \ref{eq:DiffCross} as $\ell_{max} = 4$ versus $\ell_{max} = 5$ for all angles and energies}
	\label{fig:percent-diff-cross-sections-full-h}
\end{figure}

\begin{figure}[H]
	\centering
	\includegraphics[width=\textwidth]{percent-diff-cross-sections-g}
	\caption[Percent difference of differential cross sections at selected angles]{Percent difference of $\frac{d\sigma_{el}}{d\Omega}$ for upper limit of summations in \ref{eq:DiffCross} as $\ell_{max} = 3$ versus $\ell_{max} = 4$ for selected angles}
	\label{fig:percent-diff-cross-sections-g}
\end{figure}

\todoi{Do a graph with selected energy - one at the threshold?}

\begin{figure}[H]
	\centering
	\includegraphics[width=\textwidth]{percent-diff-cross-sections-h}
	\caption[Percent difference of differential cross sections at selected angles]{Percent difference of $\frac{d\sigma_{el}}{d\Omega}$ for upper limit of summations in \ref{eq:DiffCross} as $\ell_{max} = 4$ versus $\ell_{max} = 5$ for selected angles}
	\label{fig:percent-diff-cross-sections-h}
\end{figure}

Similar to \cref{tab:PercentToCross}, \cref{tab:PercentDiffCrossFull} gives 
the average and maximum percent differences for adding partial waves to the
differential cross section (both the singlet and triplet). The average 
percent difference for adding the G-wave is less than 1\% but has a fairly
large maximum percent difference. Adding the H-wave is a much less
significant contribution, and our differential cross section looks to be
relatively well converged by the H-wave.

\begin{table}[H]
\centering
\begin{tabular}{c...c}
\toprule
$\ell$  &  \multicolumn{1}{c}{Avg. \% Diff.}  &  \multicolumn{1}{c}{Max. \% Diff.} \\
\midrule
1& 49.11\% & 140.8\%  \\
2& 18.54\% & 122.6\%  \\
3& 4.50\%  & 58.1\%   \\
4& 0.86\%  & 12.1\%   \\
5& 0.26\%  & 3.8\%    \\
\bottomrule
\end{tabular}
\caption[Convergence of the full differential cross section]{Percent difference of the elastic differential cross section for each partial wave $\ell$ with respect to $\ell - 1$ for both the maximum and average for the entire $E_{\bm \kappa}$ and $\theta$ range}
\label{tab:PercentDiffCrossFull}
\end{table}

\todoi{Double check that the singlet and triplet aren't considered completely separate}
\Cref{tab:PercentDiffCrossSingTrip} separates the singlet and triplet
contributions to the differential cross section. For this table, when the
singlet contributions are analyzed, the triplet summations have
$\ell_{max} = 5$. Likewise, when the singlet contributions are analyzed,
$\ell_{max} = 5$ for the triplet. We see that with respect to $\ell$, the
triplet differential cross section converges much quicker for both the average
and maximum. Columns 5 and 6 give the values of $E_{\bm \kappa}$ and $\theta$
where the maximum percent difference is located for the singlet, and columns
8 and 9 give the maximum percent difference location for the triplet.
Unsurprisingly, other than the case of $\ell = 1$, the angles that are the most
sensitive are 0 and $\pi$ radians. The most sensitive energies are in the
resonance region, and most of these are near the inelastic threshold of
$\SI{5.102}{eV}$.


\begin{table}[H]
\centering
\begin{tabular}{c....c..c}
\toprule
%$\ell$ & \multicolumn{1}{c}{Avg. $\%^+$ Diff.}  & \multicolumn{1}{c}{Avg. $\%^-$ Diff.}  & \multicolumn{1}{c}{Max. $\%^+$ Diff.}  & \multicolumn{1}{c}{Max. $E_{\bm \kappa}^+$} & \multicolumn{1}{c}{Max. $\theta^+$} & \multicolumn{1}{c}{Avg. $\%^-$ Diff.}  & \multicolumn{1}{c}{Max. $E_{\bm \kappa}^-$} & \multicolumn{1}{c}{Max. $\theta^-$} \\
 & \multicolumn{1}{c}{Avg.}  & \multicolumn{1}{c}{Avg.}  & \multicolumn{1}{c}{Max.}  & \multicolumn{1}{c}{Max.} & Max. & \multicolumn{1}{c}{Max.}  & \multicolumn{1}{c}{Max.} & Max. \\
$\ell$ & \multicolumn{1}{c}{$\%^+$ Diff.}  & \multicolumn{1}{c}{$\%^-$ Diff.}  & \multicolumn{1}{c}{$\%^+$ Diff.}  & \multicolumn{1}{c}{$E_{\bm \kappa}^+$} (eV) & $\theta^+$ (rad) & \multicolumn{1}{c}{$\%^-$ Diff.}  & \multicolumn{1}{c}{$E_{\bm \kappa}^-$} (eV) & $\theta^-$ (rad) \\
\midrule
1 & 40.34\% & 9.84\% & 162.28\% & 4.686 & 0       & 50.11\% & 4.289 & 0.705   \\
2 & 17.55\% & 3.31\% & 121.89\% & 4.686 & 0       & 24.72\% & 4.354 & $\uppi$ \\
3 & 4.47\%  & 0.39\% & 51.89\%  & 5.072 & $\uppi$ & 3.31\%  & 5.102 & $\uppi$ \\
4 & 0.80\%  & 0.20\% & 9.22\%   & 5.067 & 0       & 1.48\%  & 5.055 & 0       \\
5 & 0.21\%  & 0.15\% & 3.15\%   & 5.061 & $\uppi$ & 1.64\%  & 4.354 & $\uppi$ \\
\bottomrule
\end{tabular}
\caption[Convergence of the singlet and triplet differential cross sections]{Percent difference of the elastic differential cross section for each partial wave $\ell$ with respect to $\ell - 1$ for both the maximum and average for the entire $E_{\bm \kappa}$ and $\theta$ range. The values of $E_{\bm \kappa}^\pm$ and $\theta^\pm$ given are where $\frac{d\sigma_{el}^\pm}{d\Omega}$ is at its maximum value given in columns 4 and 7.}
\label{tab:PercentDiffCrossSingTrip}
\end{table}




\section{Other Cross Sections and Comparisons}
\label{sec:OtherCross}

\todoi{Reword this paragraph. It's mainly copied from the PRA article. Also mention \cite{Engbrecht2008} so that it's clear there is experimental interest. Good reason for the $1-\cos\theta$? Make sure to mention any groups who do these calculations, even if I don't use them in my figures.}

The momentum transfer cross sections can be useful in plasma applications 
\cite{Wang2014, McEachran2014}. These cross sections have been measured for 
Ps with multiple atomic and molecular molecular targets
\cite{Nagashima1998,Saito2003}. The momentum transfer cross section is
similar to \cref{eq:TotalDiffCross} but with a weighting factor of
$(1 - \cos\theta)$ \cite{Walters2004}:
\begin{equation}
\label{eq:MomentumCrossInt}
\sigma_m = \int (1 - \cos\theta) \frac{d\sigma_{el}}{d\Omega} \, d\Omega.
\end{equation}
The momentum transfer cross sections can also be written in terms of the
phase shifts as \citep[p. 589]{Bransden2003}
\begin{equation}
\label{eq:MomentumCross}
\sigma_{m}^\pm = \frac{4}{\kappa^2} \sum_{\ell=0}^\infty (\ell+1) \sin^2 (\delta_\ell^\pm - \delta_{\ell+1}^\pm) .
\end{equation}
This is the expression used in this work.

\Cref{fig:momentum-cross-sections} shows the momentum transfer cross section 
from the complex Kohn and the static exchange \cite{Hara1975}. The static 
exchange curve is extracted from Figure 1 of their paper using CurveSnap
\cite{CurveSnap}. The spin-weighting is from \cref{eq:SpinWeightCS}. The static 
exchange \todo{check } gives a decent approximation but overestimates the 
momentum transfer cross section through much of the energy range and greatly 
overestimates for less than $\SI{0.25}{eV}$, going to a maximum of
$\SI{57.7}{$\pi a_0^2$}$ at zero energy. The SE approximation also does not
reproduce any of the resonance features from the singlet.

\begin{figure}[H]
	\centering
	\includegraphics[width=5.25in]{momentum-cross-sections}
	\caption{Momentum transfer cross sections. Static exchange data is from Ref.~\cite{Hara1975}.}
	\label{fig:momentum-cross-sections}
\end{figure}

The ortho-para conversion cross sections give the conversion of the projectile ortho-Ps to para-Ps by \cite{Hara1975}
\begin{equation}
\label{eq:OrthoParaCross}
\sigma_{c} = \frac{1}{4 \kappa^2} \sum_{\ell=0}^\infty (2 \ell+1) \sin^2 (\delta_\ell^+ - \delta_\ell^-).
\end{equation}

\todoi{Ortho-para in \cite{Baltenko1983}}

\begin{figure}[H]
	\centering
	\includegraphics[width=5.25in]{orthopara-cross-sections}
	\caption{Ortho-para conversion cross sections}
	\label{fig:orthopara-cross-sections}
\end{figure}

\begin{figure}[H]
	\centering
	\includegraphics[width=5.25in]{cross-section-comparisons}
	\caption{Comparison of cross sections}
	\label{fig:cross-section-comparisons}
\end{figure}

As expected, $\sigma_m \approx \sigma_{el}$ at very low energy. 
At zero energy, $\sigma_m = \sigma_{el}$ should hold, and for Ps-H, we find that
$E_{\bm \kappa} < 10^{-6}$ eV, $\sigma_m = \sigma_{el} = 32.45$ $\pi a_0^2$. 
As Blackwood et al.\ \cite{Blackwood2002c} note, this is due to the differential
cross section being essentially isotropic at low energy (and exactly isotropic at
zero energy), as seen in
\cref{fig:diff-cross-sections-combined,fig:diff-cross-section-2D-kappa,fig:diff-cross-sections-combined-1}.
If $\sigma_m < \sigma_{el}$, the scattering is mainly forward peaked, and if
$\sigma_m > \sigma_{el}$, the scattering is mainly backward peaked \cite{Thumm1993}.
Past very low energy, where $\sigma_m = \sigma_{el}$, we see that $\sigma_m > \sigma_{el}$
until approximately $\SI{0.46}{eV}$ or $\kappa = 0.26$. This is backward peaked
(concentrated in the $\theta = \pi$ direction), which corresponds to the
findings from the differential cross section in \cref{sec:diffcross}. Beyond
$\SI{0.46}{eV}$, $\sigma_m < \sigma_{el}$ for the rest of the energy range,
indicating that the scattering is primarily forward peaked. The $\sigma_m$ curve,
as seen in \cref{fig:cross-section-comparisons}, gets close to the $\sigma_{el}$
curve for the dips in some of the resonances at approximately $\SI{4.3}{eV}$
and $\SI{5.1}{eV}$, showing that the scattering is concentrated nearly equally
in the forward and backward directions at these energy values.
This effect can slightly be seen in \cref{fig:combined-diff-cross-sections},
but it is most clearly shown in \cref{fig:diff-cross-section-2D-theta}.





\todoi{Other papers describing relationship between elastic and momentum-transfer?}

\todoi{Ortho-para conversion ratio in \cite{Massey1954,Biswas1998,Raya,Rayb}}

\todoi{Revisit ortho-para conversion cross section. Does this really show the ortho-Ps
converting to para-Ps? Because we are not considering the spin of Ps but rather the
spins of the electrons. We removed this from the paper.}


\biblio
\end{document}