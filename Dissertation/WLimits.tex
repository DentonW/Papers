\documentclass[main.tex]{subfiles} 
\begin{document}

\chapter{W Function Limits}
\label{chp:WLimits}
\textbf{@TODO: Write this all in terms of $j_i$, $k_i$, $l_i$, etc. instead of $k_1$, $k_2$, $k_{12}$, etc.}

\lettrine{P}{recalculating} the results of the W function yields significant improvements to the speed of this program.  The number of terms in the trial function increases greatly with increasing $\omega$.  The number of elements of the matrices Phi2HPhi and PhiPhi is $N(\omega)^2$.  Only half of each of these matrices is calculated, since the upper triangular portion mirrors the bottom triangle.  The W function is called repeatedly for each element, but the parameters are not unique between elements of the matrices.

A four-dimensional matrix is constructed with ranges of all possible inputs to the W function, and the output of the W function is stored as the elements.  The range of the parameters to the W function needs to be determined so that no extra computation takes place for impossible inputs and that values are not read outside of the matrix.

\section{\texorpdfstring{$\alpha$, $\beta$ and $\gamma$} {alpha, beta and gamma} Definitions}

The last three parameters of the W function are $\alpha$, $\beta$ and $\gamma$.  There are exactly six permutations of these parameters.  All six of these occur in the T function for a set of l, m and n parameters.  Since there are only six possibilities from equation (6) of Drake and Yan \cite{Drake1995}, the last dimension of the WMatrix is 6.  If this fact was not used, a six-dimensional matrix would have to be used.

\section{Limits of convergence on W}

The first three parameters of the W function are called $l$, $m$ and $n$ (in order).  From \cite{}, there are three conditions for the first three parameters that determine whether W converges.
\[l \geq 0, ~~~~ l+m+1 \geq 0, ~~~~ l+m+n+2 \geq 0\]
The other limits that must be taken into account are that $j_{\mu \nu} \geq -1$ and $j_{\mu} \geq -2$.  However, for this program, none of $j_{\mu \nu}$ or $j_{\mu}$ are more singular than -1.  Also, the lower limits of $\Omega$ and $q$ are 0.

Note that since we are calculating the inner products of $\phi_i$ and $\phi_j$, the upper limit on any of $j_{\mu \nu}$ and $j_{\mu}$ is $2\Omega$, not $\Omega$.

\section{\emph{l} Parameter}
The first parameter, \emph{l}, is
\begin{equation}
	\label{l_Param}j_1 + 2 + 2q + 2 k_{12} + 2 k_{31}.
\end{equation}
\underline{Min}: Let $j_{12} = -1$, $j_{23} = j_{31} = 0$.  From this, we have the upper limits of the summations of equation () of \cite{Drake1995} as $k_{12} = \tfrac{1}{2} (j_{12} + 1) = 0$, $k_{23} = 0$ and $k_{31} = 0$.  With $j_1$ as its minimum of -1 and the above conditions, this becomes 1.

\noindent\underline{Max}: Choosing $j_1 = 2 \Omega$ and $j_{12} = j_{13} = -1$, (\ref{l_Param}) becomes $2 \Omega + 2q + 4$.  This is the upper limit of $l$.

\section{\emph{m} Parameter}
The \emph{m} parameter is
\begin{equation}
	\label{m_Param}j_2 + 2 + j_{23} - 2 k_{23} + 2 k_{12}.
\end{equation}
\underline{Min}: 
We have to look at three conditions for this, which yield all the same result.  This is because the upper limits of the summations change if one of the $j_{\mu \nu}$ or $j_{\mu}$ is even.  This can be seen from Ref. \cite{Perkins1969}.
\begin{itemize}
	\item $j_2$, $j_{23}$ and $j_{12}$ all odd.  In this case, (\ref{m_Param}) becomes $j_2 + 2 + j_{12}.$ \\ If $j_2 = j_{12} = -1$, this equation becomes 0.
	\item $j_{23}$ is even.  Then the upper limit of the summation becomes $\frac{j_{23}}{2}-q$ by \cite{Perkins1969}.  If $j_{23} = 2 \Omega$, then $\frac{j_{23}}{2}-q = \Omega - q$.  Minimizing this by $\Omega = 0$ and $q = 0$ gives
	\[j_2 + 2 + 2 k_{12} = j_2 + 2 + j_{12} + 1.\]
	With $j_2 = j_{12} = -1$, this equation reduces to 1, giving a lower limit higher than the all odd case.
	\item $j_{12}$ is even.  Then the summation upper limit is $\frac{j_{12}}{2}-q$ by \cite{Perkins1969}.  (\ref{m_Param}) becomes $j_2 + 2 + j_{23} - (j_{23}+1)$ if $q = j_{12} = 0$.  Setting $j_2$ as its minimum of -1 gives a lower limit of 0.
\end{itemize}
\underline{Max}: We have to examine the case of all $j_{\mu \nu}$ and $j_{\mu}$ being odd and the case of at least one being even.
\begin{itemize}
  \item If $j_{23}$ and $j_{31}$ are odd, (\ref{m_Param}) becomes
\[j_3 + 2 + j_{23} - 2 k_{23} + 2 k_{31} = j_3 + 2 + j_{23} - 2\cdot\tfrac{1}{2} (j_{23} + 1) + 2\cdot\tfrac{1}{2} (j_{31} + 1) = j_{3} + 2 + j_{31}.\]
If $j_3 = 2\Omega - 1$ and $j_{31} = 1$, then this equation becomes $2\Omega + 2$.  This is not the largest this parameter can be, which is given for the next case.
  \item If $j_{23} = 0$, (\ref{m_Param}) becomes
  \[j_3 + 2 + j_{23} + 2\cdot\tfrac{1}{2}(j_{31}+1) = j_3 + 2 + j_{23} + j{31} + 1 = j_3 + j_{23} + j_{31} + 3.\]  Note that this has no q-dependence, and the q summation is finite.  If one of $j_3$, $j_{23}$ or $j_{31}$ is $2\Omega$, and the others are 0, then this becomes $2\Omega + 3$.
\end{itemize}

\section{\emph{n} Parameter}
The \emph{n} parameter is
\begin{equation}
	\label{n_Param}j_1 + 2 + j_{12} - 2 q - 2 k_{12} + j_{31} - 2 k_{23}.
\end{equation}
\underline{Min}: If $j_{12}$ and $j_{31}$ are odd, then (\ref{n_Param}) becomes
	\[j_1 + 2 + j_{12} - 2q - 2\cdot\tfrac{1}{2}(j_{12}+1) + j_{31} - 2\cdot\tfrac{1}{2}(j_{31}+1) = j_1 - 2q.\]
\underline{Max}: When $j_{12} = 0$, $k_{12} = 0$, (\ref{n_Param}) becomes $j_2 + 2 + j_{12} + 2k_{23}$.  Then if $j_2 = 2\Omega$ or $j_{12} = 2\Omega$, this becomes $2\Omega + 2$.  With the minimum of $j_1$ as -1, the lower limit of the $n$ parameter is $-1 - 2q$.

\section{Summary}
Table \ref{tab:WLimits} summarizes these results.

\begin{table}[H]
\begin{center}
\begin{tabular}{| c | c | c |}
  \hline
  Parameter & Minimum & Maximum \\
  \hline
  l & 1 & $2\Omega + 2q + 4$ \\
  m & 0 & $2\Omega + 3$ \\
  n	& $-1 - 2q$ & $2\Omega + 2$ \\
  \hline
\end{tabular}
\caption{W Function Limits Summary}
\label{tab:WLimits}
\end{center}
\end{table}

\end{document}