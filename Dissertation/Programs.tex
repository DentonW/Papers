% -*- root: Dissertation.tex -*-
\documentclass[Dissertation.tex]{subfiles} 
\begin{document}


\chapter{Program Descriptions}
\label{chp:Programs}


\todoi{Should mention the Mathematica notebooks to generate Fortran lists for short-range matrix elements and possibly other programs}
\todoi{\url{https://github.com/DentonW/Ps-H-Scattering}}



\begin{figure}
	\centering
	\includegraphics[width=4.5in]{ProgramFlowchart}
	\caption{Flowchart for programs and scripts}
	\label{fig:ProgramFlowchart}
\end{figure}


%\begin{lstlisting}[label=some-code,caption=Some Code]
{
\singlespacing

\lstset{captionpos=b,frame=lines,numbers=left,basicstyle=\tiny,numberstyle=\tiny,numbersep=5pt,breaklines=true,showstringspaces=false,basicstyle=\footnotesize,emph={label}}
\begin{lstlisting}[caption=Set of integration points in m07n7.txt]
Number of quadrature points for the various matrix elements

Long-long: r1 Lag, r2 Leg, r2 Lag, r3 Leg, r3 Lag, r12 Leg, r13 Leg, phi23
75 40 40 40 40 25 25 25
Long-long 2/r23 term: r1 Lag, r2 Leg, r2 Lag, r3 Leg, r3 Lag, phi12, r13 Leg, r23 Leg
75 40 40 40 40 25 25 25

Short-long with qi = 0: r1 Lag, r2 Leg, r2 Lag, r3 Leg, r3 Lag, r12 Leg, r13 Leg
100 65 45 65 45 45 45 45
Short-long 2/r23 term with qi = 0: r1 Lag, r2 Leg, r2 Lag, r3 Leg, r3 Lag, r12 Leg, phi13, r23 Leg
115 65 45 65 60 45 45 45
Short-long full with qi > 0: r1 Lag, r2 Leg, r2 Lag, r3 Leg, r3 Lag, r12 Leg, r13 Leg, phi23
100 65 45 65 45 45 45 45

r2 and r3 integration cusp unimportant
100.0
100.0
Nonlinear parameter mu
0.7
Power of shielding function
7
Lambda
1.0 1.0 1.0
\end{lstlisting}

}



\section{XML, MySQL Database, and IPython}
\label{sec:XMLSQLIPy}

\todoi{Make sure to mention the form of the database described in the book I have been reading over MySQL and web programming}

Normal forms:
\cite{Phlonx} and \cite{Kent1983}


\url{http://plot.ly/~Denton}


%\section{}
%\label{sec:}

Hylleraas Three-Electron Integral Recursion Relations at \url{http://dx.doi.org/10.6084/m9.figshare.1419455}

\url{http://www.labarchives.com/}





\section{Extrapolation Program}
I wrote a Python \cite{Python} program to extrapolate the phase shifts for a 
run for all Kohn method variants used, including the 35 values of $\tau$ used 
in each of the generalized Kohn, generalized $S$-matrix Kohn, and generalized
$T$-matrix Kohn. The extrapolation is performed with a least-squares fitting 
using the \texttt{polyfit} function of the SciPy package \cite{SciPy} to the 
form of \cref{eq:PhaseExtrap}. This program can do the extrapolation over any 
interval of $\omega$ values requested. The extrapolations from the 109 total 
Kohn variants are compared to see if there is any large discrepancy between 
them, indicating numerical instability. The phase shift can have a 
singularity in the generalized Kohn method as seen in \cref{fig:schwartz-singularity}, so 
care must be taken to ensure extrapolations are not taken around this interval.


\biblio
\end{document}
