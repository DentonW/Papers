\documentclass[Dissertation.tex]{subfiles} 
\begin{document}


\chapter{S-Wave}
\label{chp:SWave}

\lettrine{\textcolor{startcolor}{T}}{he} S-wave is the simplest of the partial waves to consider, since it is the lowest angular momentum state, and the spherical harmonics are constant, so there is no angular dependence.

As a condensed notation, the S-wave singlet and triplet are referred to as $^1$S and $^3$S, respectively (as shown in the nomenclature on \pageref{chp:nomenclature}).

%\section{Wavefunction}
%\label{sec:SWaveFn}
%Our trial wavefunction for the s-wave partial wave of positronium scattering from hydrogen is
%\begin{equation}
%\Psi_t^\pm = \bar{S} + \tan \eta_t^\pm \, \bar{C} + \sum_{i=1}^N c_i \bar{\phi}_i^t ,
%\label{eq:SWaveTrialSimple}
%\end{equation}
%
%\noindent where
%
%\begin{subequations}\label{SCphiBarDef}
%\begin{align}
%\bar{S} &= \frac{\left( 1 \pm P_{23} \right) S}{\sqrt{2}} \label{SBarDef} \\
%\bar{C} &= \frac{\left( 1 \pm P_{23} \right) C}{\sqrt{2}} \label{CBarDef} \\
%\bar{\phi}_i^t &= \left( 1 \pm P_{23} \right) \phi_i \label{PhiBarDef}
%\end{align}
%\end{subequations}
%
%\noindent and
%
%\begin{subequations}\label{eq:SCPhiDef}
%\begin{align}
%S &= Y_{0,0}\!\left( \theta_\rho, \phi_\rho \right) \Phi_{Ps}\left(r_{12}\right) \Phi_H\left(r_3\right) \sqrt{2\kappa} \,j_0\!\left(\kappa\rho\right) \label{eq:SDef} \\
%C &= -Y_{0,0}\!\left( \theta_\rho, \phi_\rho \right) \Phi_{Ps}\left(r_{12}\right) \Phi_H\left(r_3\right) \sqrt{2\kappa} \,n_0\!\left(\kappa\rho\right) \left[1 - \exp(-\mu \rho) (1+\frac{\mu}{2}\rho)\right] \label{eq:CDef} \\
%\phi_i &= e^{-\left(\alpha r_1 + \beta r_2 + \gamma r_3 \right)} r_1^{k_i} r_2^{l_i} r_{12}^{m_i} r_3^{n_i} r_{13}^{p_i} r_{23}^{q_i}. \label{eq:PhiDef}
%\end{align}
%\end{subequations}
%
%\noindent The $\frac{1}{\sqrt{2}}$ and $Y_{0,0}\left( \theta_\rho, \phi_\rho \right) = \frac{1}{\sqrt{4\pi}}$ for $\bar{\phi}_i^t$ are included in the $c_i$ coefficients in (\ref{eq:SWaveTrialSimple}).
%
%$\rho$ and $\rho'$ are defined as (refer to figure \ref{fig:PsHCoords})
%\begin{subequations}
%\begin{align}
%\vec{\rho} &= \frac{1}{2}\left(\vec{r_1} + \vec{r_2}\right) \label{eq:RhoDef}\\
%\vec{\rho}^\prime &= \frac{1}{2}\left(\vec{r_1} + \vec{r_3}\right) \label{eq:RhopDef}.
%\end{align}
%\end{subequations}
%As a shortcut, we can also write
%\begin{equation}
%S^\prime = P_{23} S \text{ and } C^\prime = P_{23} C.
%\label{eq:SCprime}
%\end{equation}
%The short-range Hylleraas-type terms are given by the $\phi_i$ in equation \ref{eq:SWaveTrialSimple}. 

%The Hamiltonian for the fundamental Coulombic system is
%\begin{align}
%\label{eq:Hamiltonian1}
%H = -\frac{1}{2} \nabla_{r_1}^2 - \frac{1}{2} \nabla_{r_2}^2 - \frac{1}{2} \nabla_{r_3}^2 + \frac{1}{r_1} - \frac{1}{r_2} - \frac{1}{r_3} - \frac{1}{r_{12}} -\frac {1}{r_{13}} + \frac{1}{r_{23}}.
%\end{align}
%
%\noindent The Hamiltonian can also be expressed in terms of other variables as
%\begin{align}
%H = -\frac{1}{4} \nabla_{\rho}^2 - \frac{1}{2} \nabla_{r_3}^2 - \nabla_{r_{12}}^2 + \frac{1}{r_1} - \frac{1}{r_2} - \frac{1}{r_3} - \frac{1}{r_{12}} - \frac{1}{r_{13}} + \frac{1}{r_{23}}
%\label{eq:Hamiltonian2Dup}
%\end{align}
%and
%\begin{align}
%H = -\frac{1}{4} \nabla_{\rho^\prime}^2 - \frac{1}{2} \nabla_{r_2}^2 - \nabla_{r_{13}}^2 + \frac{1}{r_1} - \frac{1}{r_2} - \frac{1}{r_3} - \frac{1}{r_{12}} - \frac{1}{r_{13}} + \frac{1}{r_{23}}.
%\label{eq:Hamiltonian3}
%\end{align}





\section{Matrix Elements}

%\subsection{Total Energy and H/Ps Wavefunctions}
%
%
%From equation \ref{eq:Wavenumber}, since positronium has twice the mass of a positron, $m \to 2m$, the kinetic energy of the positronium atom is $T = \frac{1}{4} \kappa^2$, where $\kappa$ is the wavenumber of Ps.  Including the dissociation energy of H and Ps to get the total energy gives
%\beq
%E_T = E_H + E_{Ps} + \frac{1}{4} \kappa^2.
%\eeq
%For the ground states of H and Ps,
%\beq
%E_T = -\frac{1}{2} - \frac{1}{4} + \frac{1}{4} \kappa^2 = -\frac{3}{4} + \frac{1}{4} \kappa^2 \:\: \text{(in a.u.)}.
%\label{eq:EnergyTotal}
%\eeq
%
%Some important cancellations are made in the matrix element equations by using the H and Ps ground state wavefunctions. Separately, the H and Ps equations are respectively (for large values of $\rho$)
%\beq
%\left(-\frac{1}{2}\nabla_{r_3}^2 - \frac{1}{r_3}\right) \Phi_H(r_3) = E_H \Phi_H(r_3)
%\label{eq:HEqn}
%\eeq
%\beq
%\left(-\nabla_{r_{12}}^2 - \frac{1}{r_{12}}\right) \Phi_{Ps}(r_{12}) = E_{Ps} \Phi_{Ps}(r_{12})
%\label{eq:PsEqn}
%\eeq

\subsection{Matrix Elements Involving Long-Range Terms}
\label{sec:MatrixLong}
%The short-range--long-range and long-range--long-range matrix elements have a similar analysis. For all of these, the effect of the $\mathcal{L} = 2(H-E)$ operator on the long-range terms must be considered, and then integrations over the external angles (see \Cref{chp:AngularInt}) are performed. The remaining 6-dimensional integral is then numerically integrated as described in sections \ref{sec:LongLongInt} and \ref{sec:ShortLongInt}.

\subsubsection{Elements with \texorpdfstring{$\mathcal{L}S$}{LS}}
\label{sec:LSElements}
The $(\tilde{C},\mathcal{L}\tilde{S})$ and $(\tilde{\phi}_i,\mathcal{L}\tilde{S})$ matrix elements have the $\mathcal{L}$-operator acting on $S$. While not explicitly in the matrix given by \cref{eq:GeneralKohnMatrix}, $(\tilde{S},\mathcal{L}\tilde{S})$ is required in the final \cref{eq:GenDFDT2}. Also from the general $\tilde{S}$ and $\tilde{C}$, \cref{eq:GenSCMatrix} gives that the evaluation of $(\tilde{S},\mathcal{L}\tilde{S})$ is needed. We first look at $L$ operating on $S$ and then determine the form of these matrix elements.

%\begin{align}
%\nonumber \mathcal{L}S &= \left(-\frac{1}{2}\nabla_\rho^2 - \nabla_{r_3}^2 - 2\nabla_{r_{12}}^2 + \frac{2}{r_1} - \frac{2}{r_2} - \frac{2}{r_3} - \frac{2}{r_{12}} - \frac{2}{r_{13}} + \frac{2}{r_{23}} - 2 E_T\right) \Phi_{Ps}(r_{12}) \Phi_H(r_3) \frac{\sin(\kappa\rho)}{\kappa\rho} \sqrt{\frac{2\kappa}{4\pi}} \\
%&= \left(-\frac{1}{2}\nabla_\rho^2 - \nabla_{r_3}^2 - 2\nabla_{r_{12}}^2 + \frac{2}{r_1} - \frac{2}{r_2} - \frac{2}{r_3} - \frac{2}{r_{12}} - \frac{2}{r_{13}} + \frac{2}{r_{23}} - 2 E_H - 2 E_{Ps} - \frac{1}{2}\kappa^2 \right) \Phi_{Ps}(r_{12}) \Phi_H(r_3) \frac{\sin(\kappa\rho)}{\kappa\rho} \sqrt{\frac{2\kappa}{4\pi}}.
%\end{align}
%
%Since $\Phi_H$ and $\frac{\sin(\kappa\rho)}{\kappa\rho}$ are independent of $r_3$, we can use (\ref{eq:HEqn}) to replace the $\left(-\frac{1}{2}\nabla_{r_3}^2 - \frac{1}{r_3}\right)$ with $E_H$, which gives a cancellation with the $-2 E_H$:
%\beq
%\mathcal{L}S = \left(-\frac{1}{2}\nabla_\rho^2 - 2\nabla_{r_{12}}^2 + \frac{2}{r_1} - \frac{2}{r_2} - \frac{2}{r_{12}} - \frac{2}{r_{13}} + \frac{2}{r_{23}} - 2 E_{Ps} - \frac{1}{2}\kappa^2 \right) \Phi_{Ps}(r_{12}) \Phi_H(r_3) \frac{\sin(\kappa\rho)}{\kappa\rho} \sqrt{\frac{2\kappa}{4\pi}}
%\eeq
%
%\noindent We can do the same with (\ref{eq:PsEqn}):
%\beq
%\mathcal{L}S = \left(-\frac{1}{2}\nabla_\rho^2 + \frac{2}{r_1} - \frac{2}{r_2} - \frac{2}{r_{13}} + \frac{2}{r_{23}} - \frac{1}{2}\kappa^2 \right) \Phi_{Ps}(r_{12}) \Phi_H(r_3) \frac{\sin(\kappa\rho)}{\kappa\rho} \sqrt{\frac{2\kappa}{4\pi}}
%\label{LS1}
%\eeq
%
%The only part that explicitly depends on $\rho$ is $j_0(\kappa\rho)$.  We can see that $j_0(\kappa\rho)$ is an eigenfunction of $\nabla_\rho^2$:
%From the file 'PVR Thesis.nb',
%\beq
%\nabla_\rho^2 \: j_0(\kappa\rho) = \frac{1}{\rho^2 } \frac{\partial}{\partial\rho} \left[ \rho^2 \frac{\partial}{\partial\rho} \left( \frac{\sin(\kappa\rho)}{\kappa\rho} \right)\right] = -\kappa^2 \, \frac{\sin(\kappa\rho)}{\kappa\rho} = -\kappa^2 j_0(\kappa\rho)
%\label{eigenj0}
%\eeq
%
%\noindent Similarly,
%\beq
%\nabla_\rho^2 \: n_0(\kappa\rho) = -\frac{1}{\rho^2 } \frac{\partial}{\partial\rho} \left[ \rho^2 \frac{\partial}{\partial\rho} \left( \frac{\cos(\kappa\rho)}{\kappa\rho} \right)\right] = \kappa^2 \, \frac{\cos(\kappa\rho)}{\kappa\rho} = -\kappa^2 n_0(\kappa\rho)
%\label{eigenn0}
%\eeq
%
%\noindent So $j_0(\kr)$ is an eigenfunction of $-\nabla_\rho^2$ with eigenvalue $-\kappa^2$.  Now using (\ref{eigenj0}) in (\ref{LS1}) gives
%\begin{align}
%\label{eq:LS2}
%\nonumber \mathcal{L}S &= \left( \frac{2}{r_1} - \frac{2}{r_2} - \frac{2}{r_{13}} + \frac{2}{r_{23}} \right) \Phi_{Ps}(r_{12}) \Phi_H(r_3) \frac{\sin(\kr)}{\kr} \sqrt{\frac{2\kappa}{4\pi}} \\
%&= \left( \frac{2}{r_1} - \frac{2}{r_2} - \frac{2}{r_{13}} + \frac{2}{r_{23}} \right) S
%\end{align}
%
%\beq
%(S,\mathcal{L}S) = \left( \left[ \frac{2}{r_1} - \frac{2}{r_2} - \frac{2}{r_{13}} + \frac{2}{r_{23}} \right] S^2 \right)
%\label{eq:SLS}
%\eeq
%
%The potential energy terms in brackets are antisymmetric, i.e. replacing $1 \leftrightarrow 2$.  Also notice from (\ref{eq:SCPhiDef}) that S and C are symmetric.  So under the $1 \leftrightarrow 2$ permutation, $(S,\mathcal{L}S)$ and $(C,\mathcal{L}S)$ are antisymmetric.  Thus, we have
%\beq
%(S,\mathcal{L}S) = 0 \text{ and } (C,\mathcal{L}S) = 0.
%\eeq
%
%From (\ref{SCphiBarDef}),
%\beq
%(\bar{S},\mathcal{L}\bar{S}) = \frac{1}{2} \left((S \pm S^\prime),\mathcal{L}(S \pm S^\prime)\right) = \frac{1}{2} \left[(S,\mathcal{L}S) \pm (S,\mathcal{L}S^\prime) \pm (S^\prime,\mathcal{L}S) + (S^\prime,\mathcal{L} S^\prime) \right]
%\eeq
%
%\noindent From the symmetry of these elements (as in (\ref{})),
%\beq
%(S,\mathcal{L}S) = (S^\prime,\mathcal{L}S^\prime) = 0.
%\eeq
%
%\noindent Also,
%\beq
%(S,\mathcal{L}S^\prime) = (S^\prime,\mathcal{L}S).
%\eeq
%
%%\beq
%%(\bar{S},\mathcal{L}\bar{S}) = \pm \left(S^\prime,\mathcal{L}S\right) = \pm \left(S^\prime, \left[ \frac{2}{r_1} - \frac{2}{r_2} - \frac{2}{r_{13}} + \frac{2}{r_{23}} \right] S\right)
%%\label{eq:SbarLSbar}
%%\eeq
%
%%The same computation applies for $\bar{C}$:
%%\beq
%%(\bar{C},\mathcal{L}\bar{S}) = \pm \left(C^\prime,\mathcal{L}S\right) = \pm \left(C^\prime, \left[ \frac{2}{r_1} - \frac{2}{r_2} - \frac{2}{r_{13}} + \frac{2}{r_{23}} \right] S\right)
%%\label{eq:CbarLSbar}
%%\eeq
%
%We can use (\ref{eq:SLCandCLS}) to find $(\bar{S},\mathcal{L}\bar{C})$ from this:
%\beq
%\left(\bar{S},\mathcal{L}\bar{C}\right) = \left(\bar{C},\mathcal{L}\bar{S}\right) + 1
%\label{eq:SLCandCLSBar}
%\eeq
%
%$\mathcal{L}S'$ is constructed by performing the $2 \leftrightarrow 3$ permutation of (\ref{eq:LS2}):
%\beq
%\label{eq:LSP2}
%\mathcal{L}S' = \left( \frac{2}{r_1} - \frac{2}{r_3} - \frac{2}{r_{12}} + \frac{2}{r_{23}} \right) S'.
%\eeq

So $(\bar{\phi}_i, \mathcal{L}\bar{S})$ is a straightforward calculation.
\beq
\label{eq:PhiBarLSBar1}
(\bar{\phi}_i, \mathcal{L}\bar{S}) = \left( (\phi_i \pm \phi_i') L \frac{(S \pm S')}{\sqrt{2}}\right)
= \frac{1}{\sqrt{2}} \left(\phi_i L S \pm \phi_i L S' \pm \phi_i' L S + \phi_i' L S'\right)
\eeq

Again, from the properties of the $P_{23}$ permutation operator,
\beq
(\phi_i,\mathcal{L}S) = (\phi_i',\mathcal{L}S') \text{ and } (\phi_i,\mathcal{L}S') = (\phi_i',\mathcal{L}S).
\eeq

Equation (\ref{eq:PhiBarLSBar1}) becomes
\begin{subequations}
\label{eq:PhiBarLSBar2}
\begin{align}
(\bar{\phi}_i, \mathcal{L}\bar{S}) = \frac{1}{\sqrt{2}} \left[2(\phi_i,\mathcal{L}S) \pm 2(\phi_i',\mathcal{L}S)\right] &= \frac{2}{\sqrt{2}} \left[(\phi_i,\mathcal{L}S) \pm (\phi_i',\mathcal{L}S)\right] \label{eq:PhiBarLSBar2a} \\
 &= \frac{2}{\sqrt{2}} \left[(\phi_i,\mathcal{L}S) \pm (\phi_i,\mathcal{L}S')\right]  \label{eq:PhiBarLSBar2b}
\end{align}
\end{subequations}

Notice that (\ref{eq:PhiBarLSBar2a}) and (\ref{eq:PhiBarLSBar2b}) are equivalent ways of writing this expression.  Either could be used, depending on the form desired for the computation.
From (\ref{eq:LS2}) and (\ref{eq:LSP2}),
\begin{align}
(\bar{\phi}_i, \mathcal{L}\bar{S}) &= \frac{2}{\sqrt{2}} \left[\left( \phi_i \left( \frac{2}{r_1} - \frac{2}{r_2} - \frac{2}{r_{13}} + \frac{2}{r_{23}} \right) S\right) \pm \left( \phi_i' \left( \frac{2}{r_1} - \frac{2}{r_2} - \frac{2}{r_{13}} + \frac{2}{r_{23}} \right) S\right)\right] \\
 &= \frac{2}{\sqrt{2}} \left[\left( \phi_i \left( \frac{2}{r_1} - \frac{2}{r_2} - \frac{2}{r_{13}} + \frac{2}{r_{23}} \right) S\right) \pm \left( \phi_i \left( \frac{2}{r_1} - \frac{2}{r_3} - \frac{2}{r_{12}} + \frac{2}{r_{23}} \right) S'\right)\right]
\end{align}

\noindent We also have that
\beq
(\bar{\phi}_i, \mathcal{L}\bar{S}) = (\bar{S}, \mathcal{L}\bar{\phi}_i).
\label{eq:PhiLSPerm}
\eeq


\subsubsection{Elements with \texorpdfstring{$\mathcal{L}C$}{LC}}
\label{sec:LCElements}
%Now we need to do the same analysis for $L\bar{C}$ as we just did for $L\bar{S}$. The shielding factor, $f_{sh}(\rho)$, complicates the derivatives slightly.
%\begin{align}
%\mathcal{L}C = & \left(-\frac{1}{2}\nabla_\rho^2 - \nabla_{r_3}^2 - 2\nabla_{r_{12}}^2 + \frac{2}{r_1} - \frac{2}{r_2} - \frac{2}{r_3} - \frac{2}{r_{12}} - \frac{2}{r_{13}} + \frac{2}{r_{23}} - 2 E_H - 2 E_{Ps} - \frac{1}{2}\kappa^2 \right) \nonumber \\
 %& \times \Phi_{Ps}(r_{12}) \Phi_H(r_3) \frac{\sin(\kappa\rho)}{\kappa\rho} \sqrt{\frac{2\kappa}{4\pi}} \left[1 + e^{-\mu\rho} \left(1 + \frac{\mu}{2} \rho \right) \right]
%\label{LC1}
%\end{align}
%
%\noindent Again, we use (\ref{eq:HEqn}) and (\ref{eq:PsEqn}) to simplify this expression.
%\begin{align}
%\mathcal{L}C = & \left(-\frac{1}{2}\nabla_\rho^2 + \frac{2}{r_1} - \frac{2}{r_2} - \frac{2}{r_3} - \frac{2}{r_{12}} - \frac{2}{r_{13}} + \frac{2}{r_{23}}  - \frac{1}{2}\kappa^2\right) \nonumber \\
 %& \times \Phi_{Ps}(r_{12}) \Phi_H(r_3) \frac{\sin(\kappa\rho)}{\kappa\rho} \sqrt{\frac{2\kappa}{4\pi}} \left[1 + e^{-\mu\rho} \left(1 + \frac{\mu}{2} \rho \right) \right]
%\label{LC2}
%\end{align}
%
%From (\ref{eigenn0}), $n_0(\kr)$ is an eigenfunction of $\nabla_\rho^2$ with eigenvalue $-\kappa^2$.  Then $\displaystyle -\frac{1}{2} \nabla_\rho^2 \frac{\cos(\kr)}{\kr}$ could be replaced by $\displaystyle \frac{1}{2}\kappa^2\frac{\cos(\kr)}{\kr}$ if the shielding term could be ignored.  However, it also depends on $\rho$, so we have to calculate this separately.
%
%\begin{align}
%\nonumber -\frac{1}{2} \nabla_\rho^2 & \left\lbrace \frac{\cos(\kr)}{\kr} \left[1 - e^{-\mr} \left(1 + \frac{\mu}{2} \rho \right)\right]\right\rbrace \\
%\nonumber &= \frac{e^{-\mr}}{2\kappa\rho} \left[ \frac{\mu^3 \rho}{2} \cos(\kr) + \kappa^2 \left(-1 + e^{\mr} -\frac{\mr}{2} \right) \cos(\kr) + \kappa\mu (1+\mr) \sin(\kr) \right] \\
%\nonumber &= \frac{e^{-\mr} \mu^3\rho}{4\kr} \cos(\kr) + \frac{\kappa^2}{2} \left[-e^{-\mr} + 1 - \frac{\mr}{2} e^{-\mr}\right] \frac{\cos(\kr)}{\kr} + \frac{e^{-\mr}}{2\kr} \kappa\mu (1 + \mr)\sin(\kr) \\
 %&= \frac{e^{-\mr} \mu^3\rho}{4\kr} \cos(\kr) + \frac{\kappa^2}{2} \left[1 - e^{-\mr}\left(1 + \frac{\mu}{2} \rho \right) \right] \frac{\cos(\kr)}{\kr} + \frac{e^{-\mr}}{2\kr} \kappa\mu (1 + \mr)\sin(\kr)
%\label{laplacianshield}
%\end{align}
%
%\noindent The second term here cancels the $\displaystyle \frac{1}{2}\kappa^2$ in parentheses in (\ref{LC2}).  Now we have
%\begin{align}
%\mathcal{L}C = \: & \Phi_{Ps}(r_{12}) \Phi_H(r_3) \sqrt{\frac{2\kappa}{4\pi}} \left\{ \left(\frac{2}{r_1} - \frac{2}{r_2} - \frac{2}{r_{13}} + \frac{2}{r_{23}} \right) \frac{\cos(\kr)}{\kr} \left[1 - e^{-\mr} \left(1 + \frac{\mu}{2}\rho \right) \right] \right. \nonumber\\
%& + \left. \frac{e^{-\mr} \mu^3 \rho}{4} \frac{\cos(\kr)}{\kr} + \frac{e^{-\mr}}{2} \kappa \mu (1+\mu\rho) \frac{\sin(\kr)}{\kr} \right\}
%\label{eq:LC}
%\end{align}
%
%To find $\mathcal{L}\bar{C}$, we also have to analyze $\mathcal{L}C^\prime$.  From (\ref{eq:SCprime}) and (\ref{eq:SCPhiDef}),
%\beq
%C^\prime = P_{23} C = -Y_{0,0}(\theta_{\rho^\prime}, \varphi_{\rho^\prime}) \Phi_{Ps}(r_{13}) \Phi_H(r_2) \sqrt{2\kappa} n_0(\kr) \left[1 - e^{-\mu\rho^\prime}\left(1 + \frac{\mu}{2} \rho^\prime \right) \right].
%\eeq
%
%\noindent The calculation is exactly the same as that for $LC$, except $\rho\to\rho^\prime$ and $2\leftrightarrow3$.
%
%\begin{align}
%\mathcal{L}C^\prime = & \left(-\frac{1}{2}\nabla_{\rhop}^2 - \nabla_{r_2}^2 - 2\nabla_{r_{13}}^2 + \frac{2}{r_1} - \frac{2}{r_2} - \frac{2}{r_3} - \frac{2}{r_{12}} - \frac{2}{r_{13}} + \frac{2}{r_{23}} - 2 E_H - 2 E_{Ps} - \frac{1}{2}\kappa^2 \right) \nonumber \\
 %& \times \Phi_{Ps}(r_{13}) \Phi_H(r_2) \frac{\sin(\kappa\rhop)}{\kappa\rhop} \sqrt{\frac{2\kappa}{4\pi}} \left[1 + e^{-\mu\rhop} \left(1 + \frac{\mu}{2} \rhop \right) \right]
%\label{LCP1}
%\end{align}
%
%We can modify (\ref{eq:HEqn}) and (\ref{eq:PsEqn}) to simplify this:
%\begin{align}
%&\left(-\frac{1}{2}\nabla_{r_2}^2 - \frac{1}{r_2}\right) \Phi_H(r_2) = E_H \Phi_H(r_2) \text{ and} \\
%&\left(-\nabla_{r_{13}}^2 - \frac{1}{r_{13}}\right) \Phi_{Ps}(r_{13}) = E_{Ps} \Phi_{Ps}(r_{13}).
%\end{align}
%
%\beq
%\mathcal{L}C^\prime = \left(-\frac{1}{2}\nabla_{\rhop}^2 + \frac{2}{r_1} - \frac{2}{r_3} - \frac{2}{r_{12}}  + \frac{2}{r_{23}} - \frac{1}{2}\kappa^2 \right) \Phi_{Ps}(r_{13}) \Phi_H(r_2) \frac{\sin(\kappa\rhop)}{\kappa\rhop} \sqrt{\frac{2\kappa}{4\pi}} \left[1 + e^{-\mu\rhop} \left(1 + \frac{\mu}{2} \rhop \right) \right]
%\label{eq:LCP2}
%\eeq
%
%Modifying the result from (\ref{laplacianshield}) to use $\rhop$ instead of $\rho$ yields
%\begin{align}
%\nonumber -\frac{1}{2} \nabla_\rhop^2 & \left\lbrace \frac{\cos(\krp)}{\krp} \left[1 - e^{-\mrp} \left(1 + \frac{\mu}{2} \rhop \right)\right]\right\rbrace \\
 %&= \frac{e^{-\mrp} \mu^3\rhop}{4\kr} \cos(\krp) + \frac{\kappa^2}{2} \left[1 - e^{-\mrp}\left(1 + \frac{\mu}{2} \rhop \right) \right] \frac{\cos(\krp)}{\krp} + \frac{e^{-\mrp}}{2\krp} \kappa\mu (1 + \mrp)\sin(\krp)
%\end{align}
%
%\noindent Similar to (\ref{eq:LC}), (\ref{eq:LCP2}) becomes
%\begin{align}
%\mathcal{L}C^\prime = \: & \Phi_{Ps}(r_{13}) \Phi_H(r_2) \sqrt{\frac{2\kappa}{4\pi}} \left\{ \left(\frac{2}{r_1} - \frac{2}{r_3} - \frac{2}{r_{12}} + \frac{2}{r_{23}} \right) \frac{\cos(\krp)}{\krp} \left[1 - e^{-\mrp} \left(1 + \frac{\mu}{2}\rhop \right) \right] \right. \nonumber\\
%& + \left. \frac{e^{-\mrp} \mu^3 \rhop}{4} \frac{\cos(\krp)}{\krp} + \frac{e^{-\mrp}}{2} \kappa \mu (1+\mu\rhop) \frac{\sin(\krp)}{\krp} \right\}
%\label{eq:LCP}
%\end{align}
%
%\noindent Combining (\ref{eq:LC}) and (\ref{eq:LCP}) gives
%\begin{align}
%\mathcal{L}\bar{C} = \:\: &\frac{1}{\sqrt{2}} \mathcal{L}(C \pm C^\prime) = \frac{1}{\sqrt{2}} (\mathcal{L}C \pm \mathcal{L}C^\prime) \nonumber \\
%= \:\: &\frac{1}{\sqrt{8\pi}} \Phi_{Ps}(r_{12}) \Phi_H(r_3) \sqrt{2\kappa} \nonumber  \\
%&\times \left\{ \frac{\kappa\mu}{2} e^{-\mr} (1+\mr) \frac{\sin(\kr)}{\kr} + \frac{\mu^3 \rho}{4} e^{-\mr} \frac{\cos(\kr)}{\kr} \right. \nonumber \\
%&+ \left. \left(\frac{2}{r_1} - \frac{2}{r_2} - \frac{2}{r_{13}} + \frac{2}{r_{23}}\right) \frac{\cos(\kr)}{\kr} \left[1 - e^{-\mr} \left(1 + \frac{\mu}{2}\rho\right)\right] \right\} \nonumber \\
%\pm &\frac{1}{\sqrt{8\pi}} \Phi_{Ps}(r_{13}) \Phi_H(r_2) \sqrt{2\kappa} \nonumber  \\
%&\times \left\{ \frac{\kappa\mu}{2} e^{-\mrp} (1+\mrp) \frac{\sin(\krp)}{\krp} + \frac{\mu^3 \rhop}{4} e^{-\mrp} \frac{\cos(\krp)}{\krp} \right. \nonumber \\
%&+ \left. \left(\frac{2}{r_1} - \frac{2}{r_3} - \frac{2}{r_{12}} + \frac{2}{r_{23}}\right) \frac{\cos(\krp)}{\krp} \left[1 - e^{-\mrp} \left(1 + \frac{\mu}{2}\rhop\right)\right] \right\}
%\label{LCBar}
%\end{align}

The properties of the permutation operator give that
\beq
(C,\mathcal{L}C) = (C',\mathcal{L}C') \text{ and } (C,\mathcal{L}C') = (C',\mathcal{L}C)
\eeq

Using these two relations gives that
\begin{align}
\left(\bar{C},\mathcal{L}\bar{C}\right) &= \frac{1}{2}\left(\left(C \pm C'\right),\mathcal{L}(C \pm C')\right) \nonumber\\
 &= \frac{1}{2}\left[(C,\mathcal{L}C) \pm (C,\mathcal{L}C') \pm (C',\mathcal{L}C) + (C',\mathcal{L}C')\right] = \frac{1}{2}\left[2(C,\mathcal{L}C) \pm 2(C',\mathcal{L}C)\right] \nonumber\\
 &= (C,\mathcal{L}C) \pm (C',\mathcal{L}C).
 \label{CLC1}
\end{align}

Substitute (\ref{eq:CDef}) and (\ref{eq:LC}) in (\ref{CLC1}) to get
\begin{align}
\left(\bar{C},\mathcal{L}\bar{C}\right) = &(C,\mathcal{L}C) \pm (C',\mathcal{L}C) = \left((C \pm C'),\mathcal{L}C\right) \nonumber\\
 = &\left((C \pm C') \Phi_{Ps}(r_{12}) \Phi_H(r_3) \sqrt{\frac{2\kappa}{4\pi}} \left\{ \left(\frac{2}{r_1} - \frac{2}{r_2} - \frac{2}{r_{13}} + \frac{2}{r_{23}} \right) \frac{\cos(\kr)}{\kr} \left[1 - e^{-\mr}\left(1 + \frac{\mu}{2}\rho \right) \right] \right.\right. \\
   &+ \left.\left. \frac{e^{-\mr} \mu^3 \rho}{4} \frac{\cos(\kr)}{\kr} + \frac{e^{-\mr}}{2} \kappa\mu (1+\mr) \frac{\sin(\kr)}{\kr} \right\} \right).
 \label{CBarLCBar1}
\end{align}

These can be split up to simplify:
\begin{align}
\left(\bar{C},\mathcal{L}\bar{C}\right) = &\sqrt{\frac{2\kappa}{4\pi}} \frac{\kappa\mu}{2} \left((C \pm C') e^{-\mr}(1+\mr) \frac{\sin(\kr)}{\kr} \Phi_{Ps}(r_{12}) \Phi_H(r_3) \right)  \nonumber\\
 &+ \sqrt{\frac{2\kappa}{4\pi}} \frac{\mu^3}{4} \left((C \pm C') e^{-\mr} \frac{\cos(\kr)}{\kr} \Phi_{Ps}(r_{12}) \Phi_H(r_3) \right) \nonumber\\
 &+ \sqrt{\frac{2\kappa}{4\pi}} \left((C \pm C') \left(\frac{2}{r_1} - \frac{2}{r_3} - \frac{2}{r_{12}} + \frac{2}{r_{23}}\right) \frac{\cos(\kr)}{\kr} \left[1 - e^{-\mr}\left(1 + \frac{\mu}{2}\rho \right)\right] \Phi_{Ps}(r_{12}) \Phi_H(r_3) \right) \nonumber\\
= &\frac{\kappa\mu}{2} \left((C \pm C') e^{-\mr}(1+\mr) S\right) \nonumber\\
 &+ \sqrt{\frac{2\kappa}{4\pi}} \frac{\mu^3}{4} \left((C \pm C') e^{-\mr} \frac{\cos(\kr)}{\kr} \Phi_{Ps}(r_{12}) \Phi_H(r_3) \right) \nonumber\\
 &+ \left((C \pm C') \left(\frac{2}{r_1} - \frac{2}{r_3} - \frac{2}{r_{12}} + \frac{2}{r_{23}}\right) C \right).
 \label{CBarLCBar2}
\end{align}

Looking at just the last term:
\beq
\left((C \pm C') \left(\frac{2}{r_1} - \frac{2}{r_3} - \frac{2}{r_{12}} + \frac{2}{r_{23}}\right) C \right) = \left( \left(\frac{2}{r_1} - \frac{2}{r_3} - \frac{2}{r_{12}} + \frac{2}{r_{23}}\right) C^2 \right) \pm \left( \left(\frac{2}{r_1} - \frac{2}{r_3} - \frac{2}{r_{12}} + \frac{2}{r_{23}}\right) C'C \right)
\eeq

\noindent The first set of parentheses has the same form as (\ref{eq:SLS}).  These terms in parentheses are antisymmetric with respect to the $1 \leftrightarrow 2$ permutation.  Also, $C$ is symmetric with respect to this permutation.  So the first set of parentheses is 0.  Thus,
\beq
\left((C \pm C') \left(\frac{2}{r_1} - \frac{2}{r_3} - \frac{2}{r_{12}} + \frac{2}{r_{23}}\right) C \right) = \pm \left(C' \left(\frac{2}{r_1} - \frac{2}{r_3} - \frac{2}{r_{12}} + \frac{2}{r_{23}}\right) C \right)
\eeq

\noindent We finally have that
\begin{align}
\left(\bar{C},\mathcal{L}\bar{C}\right) = \: &\frac{\kappa\mu}{2} \left((C \pm C') e^{-\mr}(1+\mr) S\right) \nonumber\\
 &+ \sqrt{\frac{2\kappa}{4\pi}} \frac{\mu^3}{4} \left((C \pm C') e^{-\mr} \frac{\cos(\kr)}{\kr} \Phi_{Ps}(r_{12}) \Phi_H(r_3) \right) \nonumber\\
 &\pm \left(C' \left(\frac{2}{r_1} - \frac{2}{r_3} - \frac{2}{r_{12}} + \frac{2}{r_{23}}\right) C \right).
 \label{CBarLCBar}
\end{align}

\textbf{@TODO:} Put this elsewhere.
We chose to absorb both the $\frac{1}{\sqrt{2}}$ and $Y_{0,0}(\theta_\rho,\varphi_\rho) = \frac{1}{\sqrt{4\pi}}$ into the $c_i$ constants of the short-range terms (\ref{eq:PhiDef}).


The analysis for $(\bar{\phi}_i, \mathcal{L}\bar{C})$ and $(\bar{C}, L\bar{\phi}_i)$ is similar to the analysis in section \ref{sec:LSElements}.
\begin{subequations}
\begin{align}
(\bar{\phi}_i, \mathcal{L}\bar{C}) &= \frac{2}{\sqrt{2}} \left[(\phi_i,\mathcal{L}C) \pm (\phi_i',\mathcal{L}C)\right] \label{PhiBarLCBar2a} \\
 &= \frac{2}{\sqrt{2}} \left[(\phi_i,\mathcal{L}C) \pm (\phi_i,\mathcal{L}C')\right]  \label{PhiBarLCBar2b}
\end{align}
\end{subequations}

Using (\ref{eq:LC}) and (\ref{eq:LCP}) in the above gives our final result for $(\bar{\phi}_i, \mathcal{L}\bar{C})$.
\begin{subequations}
\begin{align}
(\bar{\phi}_i, \mathcal{L}\bar{C}) = \sqrt{\frac{\kappa}{\pi}} &\left( (\phi_i \pm \phi_i') \Phi_{Ps}(r_{12}) \Phi_H(r_3) \left\{ \left(\frac{2}{r_1} - \frac{2}{r_2} - \frac{2}{r_{13}} + \frac{2}{r_{23}}\right) \frac{\cos(\kr)}{\kr} \left[1 - e^{-\mr}\left(1 + \frac{\mu}{2}\rho\right)\right] \right.\right. \nonumber\\
&+ \left.\left.\frac{e^{-\mr}\mu^3\rho}{4} \frac{\cos(\kr)}{\kr} + \frac{e^{-\mr}}{2} \kappa\mu (1+\mr) \frac{\sin(\kr)}{\kr}  \right\}\right) \\
= \sqrt{\frac{\kappa}{\pi}} &\left( \phi_i \Phi_{Ps}(r_{12}) \Phi_H(r_3) \left\{ \left(\frac{2}{r_1} - \frac{2}{r_2} - \frac{2}{r_{13}} + \frac{2}{r_{23}}\right) \frac{\cos(\kr)}{\kr} \left[1 - e^{-\mr}\left(1 + \frac{\mu}{2}\rho\right)\right] \right.\right. \nonumber\\
&+ \left.\left.\frac{e^{-\mr}\mu^3\rho}{4} \frac{\cos(\kr)}{\kr} + \frac{e^{-\mr}}{2} \kappa\mu (1+\mr) \frac{\sin(\kr)}{\kr}  \right\}\right) \nonumber\\
\pm \sqrt{\frac{\kappa}{\pi}} &\left( \phi_i \Phi_{Ps}(r_{12}) \Phi_H(r_3) \left\{ \left(\frac{2}{r_1} - \frac{2}{r_3} - \frac{2}{r_{12}} + \frac{2}{r_{23}}\right) \frac{\cos(\krp)}{\krp} \left[1 - e^{-\mrp}\left(1 + \frac{\mu}{2}\rhop\right)\right] \right.\right. \nonumber\\
&+ \left.\left.\frac{e^{-\mr}\mu^3\rhop}{4} \frac{\cos(\krp)}{\krp} + \frac{e^{-\mrp}}{2} \kappa\mu (1+\mrp) \frac{\sin(\krp)}{\krp}  \right\}\right)
\end{align}
\end{subequations}

Similar to (\ref{eq:PhiLSPerm}), we have
\beq
(\bar{\phi}_i, \mathcal{L}\bar{C}) = (\bar{C}, L\bar{\phi}_i).
\label{eq:PhiLCPerm}
\eeq

\subsection{Matrix Elements Involving Only Short-Range Terms}
\label{sec:MatrixShort}
The short-range--short-range interactions are calculated in a separate document ``3.24 Using Spherical Coordinates''.

\beq
\label{eq:SWaveShortShort}
\left(\bar{\phi}_i, L \bar{\phi}_j\right) = \int \left[ \sum_{l=1}^3 \boldsymbol{\nabla}_{\!\mathbf{r}_l} \bar{\phi}_i \boldsymbol{\cdot} \boldsymbol{\nabla}_{\!\mathbf{r}_l} \bar{\phi}_j + \left( \frac{2}{r_1} - \frac{2}{r_2} - \frac{2}{r_3} - \frac{2}{r_{12}} - \frac{2}{r_{13}} + \frac{2}{r_{23}} - 2 E_T \right) \bar{\phi}_i \bar{\phi}_j \right] d\tau.
\eeq

Lastly, we need to prove that $\left(\bar{\phi}_i, \mathcal{L} \bar{\phi}_j\right) = \left(\bar{\phi}_j, \mathcal{L} \bar{\phi}_i\right)$. Consider the functional
\beq
F = (\phi_i, \mathcal{L} \phi_j) - (\phi_j, \mathcal{L} \phi_i).
\eeq

Using \cref{eq:Hamiltonian1},
\begin{align}
	F=&\left({-\phi_i,{\frac {1}{2}{\nabla }_{{r}_{1}}^{2}\phi_j}}\right)+\left({\phi_j,{\frac {1}{2}{\nabla }_{{r}_{1}}^{2}\phi_i}}\right)+
	\left({-\phi_i,{\frac {1}{2}{\nabla }_{{r}_{2}}^{2}\phi_j}}\right)+\left({\phi_j,{\frac {1}{2}{\nabla }_{{r}_{2}}^{2}\phi_i}}\right) \nonumber \\
	&+\left({-\phi_i,{\frac {1}{2}{\nabla }_{{r}_{3}}^{2}\phi_j}}\right)+\left({\phi_j,{\frac {1}{2}{\nabla }_{{r}_{3}}^{2}\phi_i}}\right).
\end{align}

From (\ref{eq:PhiDef}), the $\bar{\phi}_i$ (and $\bar{\phi}_j$) have a decaying exponential dependence in $r_1$, $r_2$ and $r_3$, so the surface terms in the integrations will vanish.  Combining this fact with (\ref{PhiBarDef}) gives that $F = 0$. Then
\beq
\left(\bar{\phi}_i, \mathcal{L} \bar{\phi}_j\right) = \left(\bar{\phi}_j, \mathcal{L} \bar{\phi}_i\right).
\label{PhiLPhiPerm}
\eeq

The claims in (\ref{eq:PhiLSPerm}) and (\ref{eq:PhiLCPerm}) are proven using the same method. Both (\ref{eq:PhiLSPerm}) and (\ref{eq:PhiLCPerm}) include $\bar{\phi}_i$, so these terms do have a decaying exponential dependence, making the claims true.

%(\bar{\phi}_i, \mathcal{L}\bar{C}) = \frac{\sqrt{2\kappa}}{4\pi} &\left( (\phi_i \pm \phi_i') \Phi_{Ps}(r_{12}) \Phi_H(r_3) \left\{ \left(\frac{2}{r_1} - \frac{2}{r_2} - \frac{2}{r_{13}} + \frac{2}{r_{23}}\right) \frac{\cos(\kr)}{\kr} \left[1 - e^{-\mr}\left(1 + \frac{\mu}{2}\rho\right)\right] \right.\right. \nonumber\\
%&+ \frac{e^{-\mr}\mu^3\rho}{4} \frac{\cos(\kr)}{\kr} + \frac{e^{-\mr}}{2} \kappa\mu (1+\mr) \frac{\sin(\kr)} \left.\left. \right\}\right)






\subsection{Summary of Matrix Elements}
\label{SWaveMatrixSummary}
This section 



\section{Results}

\subsection{Singlet}
The extrapolation becomes less stable for higher $\kappa$, as is discussed in section (\ref{sec:Extrapolation}).


\subsection{Phase Shifts}

All runs here were performed using the set of integration points described in \cref{sec:QuadraturePoints}.

\begin{table}[H]
\centering
\begin{tabular}{c c c c c c c c}
\toprule
$\kappa$ & $\delta^+ (\omega = 7)$ & $\delta^- (\omega = 7)$ & $\delta^+ (\omega \rightarrow \infty)$ & $\delta^- (\omega \rightarrow \infty)$ & \% Diff$^+$ & \% Diff$^-$ \\
\midrule
0.1 & -0.427 & -0.215 & -0.426 & -0.214 & 0.223\% & 0.120\% \\
0.2 & -0.820 & -0.431 & -0.819 & -0.432 & 0.019\% & 0.063\% \\
0.3 & -1.161 & -0.645 & -1.161 & -0.645 & 0.040\% & 0.094\% \\
0.4 & -1.446 & -0.850 & -1.446 & -0.850 & 0.022\% & 0.130\% \\
0.5 & -1.678 & -1.041 & -1.677 & -1.040 & 0.031\% & 0.166\% \\
0.6 & -1.858 & -1.217 & -1.857 & -1.215 & 0.040\% & 0.273\% \\
0.7 & -1.964 & -1.375 & -1.963 & -1.373 & 0.045\% & 0.250\% \\
\bottomrule
\end{tabular}
\caption{S-Wave complex Kohn phase shifts}
\label{tab:SWavePhase}
\end{table}

Table \ref{tab:SWavePhase} contains the phase shifts for regular intervals of $\kappa$, which we can compare to the results from other groups in Tables \ref{tab:SWaveSingletOther} and \ref{tab:SWaveTripletOther} starting on page \pageref{tab:SWaveSingletOther}.  Our phase shifts at $\omega = 7$ from this table are exactly the same as Van Reeth and Humberston's results for $\omega = 6$ \cite{VanReeth2003}, with some exceptions in the last digit.  Figure \ref{fig:SWavePhaseOmega=7} has the fuller set of phase shifts plotted with respect to the positronium momentum, $\kappa$.

\textbf{@TODO:} What are our results for $\omega = 6$?

As mentioned in Section \ref{sec:Extrapolation}, extrapolation to $\omega = \infty$ has been problematic with larger numbers of terms.

\begin{figure}[H]
	\centering
	\includegraphics[width=7in]{swave-phases}
	\caption{$^{1,3}$S phase shifts}
	\label{fig:SWavePhase}
\end{figure}


\begin{table}[H]
\centering
\begin{tabular}{c c c c c c c}
\toprule
 & $(\omega = 6)$ & $(\omega \rightarrow \infty)$ &  &  &  & \\
$\kappa$ & $\delta^+$ \cite{VanReeth2003} & $\delta^+$ \cite{VanReeth2003} & $\delta^+$ \cite{Blackwood2002} & $\delta^+$ \cite{Walters2004} & $\delta^+$ \cite{Ray1997} & $\delta^+$ \cite{Adhikari1999} \\
\midrule
$0.05$ & --- & --- & $-0.219$ & --- & --- & --- \\
$0.1$ & $-0.427$ & $-0.425$ & $-0.434$ & $-0.428$ & $-0.692$ & $-0.362$ \\
$0.2$ & $-0.820$ & $-0.817$ & $-0.834$ & $-0.825$ & $-1.212$ & $-0.702$ \\
$0.3$ & $-1.161$ & $-1.158$ & $-1.178$ & $-1.167$ & $-1.592$ & $-1.002$ \\
$0.4$ & $-1.446$ & $-1.443$ & $-1.467$ & $-1.453$ & $-1.902$ & $-1.252$ \\
$0.5$ & $-1.677$ & $-1.674$ & $-1.704$ & $-1.685$ & $-2.142$ & $-1.462$ \\
$0.6$ & $-1.857$ & $-1.852$ & $-1.890$ & $-1.867$ & $-2.362$ & $-1.622$ \\
$0.7$ & $-1.964$ & $-1.959$ & $-2.018$ & $-1.992$ & $-2.512$ & $-1.712$ \\
$0.8$ &    --- &    --- &    --- &    --- & $-2.652$ & $-2.163$ \\
\bottomrule
\end{tabular}
\caption{S-Wave Singlet Results from Other Groups}
\label{tab:SWaveSingletOther}
\end{table}

\todoi{Put labels on these}
\todoi{Add \cite{Biswas2001} results}

\begin{table}[H]
\centering
\begin{tabular}{c c c c c c}
\toprule
 & $(\omega = 6)$ & $(\omega \rightarrow \infty)$ &  &  &   \\
$\kappa$ & $\delta^-$ \cite{VanReeth2003} & $\delta^-$ \cite{VanReeth2003} & $\delta^-$ \cite{Blackwood2002} & $\delta^-$ \cite{Ray1997} & $\delta^-$ \cite{Adhikari1999} \\
\midrule
$0.05$ & --- & --- & $-0.103$ & --- & --- \\
$0.1$ & $-0.215$ & $-0.214$ & $-0.206$ & $-0.252$ & $-0.167$ \\
$0.2$ & $-0.432$ & $-0.431$ & $-0.414$ & $-0.502$ & $-0.327$ \\
$0.3$ & $-0.645$ & $-0.645$ & $-0.624$ & $-0.722$ & $-0.474$ \\
$0.4$ & $-0.850$ & $-0.849$ & $-0.838$ & $-0.942$ & $-0.602$ \\
$0.5$ & $-1.040$ & $-1.038$ & $-1.037$ & $-1.142$ & $-0.706$ \\
$0.6$ & $-1.215$ & $-1.211$ & $-1.213$ & $-1.332$ & $-0.784$ \\
$0.7$ & $-1.373$ & $-1.366$ & $-1.367$ & $-1.502$ & $-0.833$ \\
$0.8$ &    --- &    --- &    --- & $-1.652$ & $-0.851$ \\
\bottomrule
\end{tabular}
\caption{S-Wave Triplet Results from Other Groups}
\label{tab:SWaveTripletOther}
\end{table}


\begin{figure}[H]
	\centering
	\includegraphics[width=5.25in]{swave-comparisons}
	\caption[Comparison of S-wave phase shifts]{Comparison of $^1$S (a) and $^3$S (b) phase shifts with results from other groups. Results are ordered according to year of publication. This work -- solid curves; \mbox{\textcolor{blue}{$\times$} -- CC \cite{Walters2004};} \mbox{$\CIRCLE$ -- Kohn \cite{VanReeth2003};} \mbox{\textcolor{red}{\textbf{+}} -- CC \cite{Blackwood2002};} \mbox{$\blacktriangle$ -- DMC \cite{Chiesa2002};} \mbox{$\triangledown$ -- SVM 2002 \cite{Ivanov2002};} \mbox{$\Circle$ -- SVM 2001 \cite{Ivanov2001};} \mbox{\textcolor{red}{$\vartriangle$} -- 6-state CC \cite{Sinha2000};} \mbox{$\blacksquare$ -- 5-state CC \cite{Adhikari1999};} \mbox{$\square$ -- Coupled-pseudostate \cite{Campbell1998};} \mbox{$\vartriangle$ -- 3-state CC \cite{Sinha1997};} \mbox{\textcolor[RGB]{0,127,0}{$\bigstar$} -- CC \cite{Ray1997};} \mbox{$\triangleright$ -- Stabilization \cite{Drachman1976};} \mbox{\textcolor{red}{$\blacklozenge$} -- Stabilization \cite{Drachman1975};} \mbox{\textcolor{blue}{$\lozenge$} -- Static-exchange \cite{Hara1975};} \mbox{$\blacktriangledown$ -- Static-exchange \cite{Fraser1961}.}}
	\label{fig:SWaveComparisons}
\end{figure}

Figure \ref{fig:SWaveComparisons} shows the results from various groups for calculations of the singlet S-wave.  Our results are extremely close to Van Reeth's \cite{VanReeth2003}, so they would follow along the solid line as well.  Several groups have results that cluster very closely to his and ours, namely Blackwood \cite{Blackwood2002}, Walters \cite{Walters2004}, Chiesa \cite{Chiesa2002} and Ivanov \cite{Ivanov2002}.






\subsection{Resonances}
\label{sec:SWaveResonances}

There are two very obvious resonances before the Ps(n=2) channel threshold for $^1$S scattering, which can be seen in figure \ref{fig:SWavePhase}.  This threshold is at an incoming Ps energy of approximately 5.102 eV.  

\subsubsection{Resonance Model}
\label{sec:ResonanceModel}
The resonance positions and widths can be calculated to high accuracy by fitting the data to the following standard curve:

\beq
\label{eq:ResonanceCurve}
\delta(E) = A + BE + CE^2 + \arctan\left[ \frac{^1\Gamma}{2\left(^1E_R-E\right)} \right] + \arctan\left[ \frac{^2\Gamma}{2\left(^2E_R-E\right)} \right]
\eeq
The polynomial part of the above equation corresponds to hard-sphere scattering. The arctangent parts correspond to the first and second Fano resonances \cite{Fano1961,Macek1970,Hazi1979}, with $^1E_R$ and $^2E_R$ as the positions of the resonances and $^1\Gamma$ and $^2\Gamma$ as the widths of the respective resonances.

\todoi{Refer back to the Bransden and Joachain book and/or the Ray paper about decomposing this equation into the different parts. Refer to the discussions in Peter's paper and Yan/Ho's paper about the resonances corresponding to 2S and 3S.  Discussion on energy levels elsewhere for H and Ps, along with what the threshold is. Ray has an analysis about the resonance width versus the lifetime.}

\begin{figure}[H]
	\centering
	\includegraphics[width=7in]{triplet-false-resonance}
	\caption{$^3$S plot showing false resonance (Schwartz singularity) at low $N$}
	\label{fig:triplet-false-resonance}
\end{figure}

\subsubsection{Resonance Graphs}
\label{sec:ResonanceGraphs}

\begin{figure}[H]
	\centering
	\includegraphics[width=7in]{ResOmega=6,pg1}
	\caption{First set of resonance fitting graphs for $\omega = 6$}
	\label{fig:ResOmega=6,pg1}
\end{figure}

\begin{figure}[H]
	\centering
	\includegraphics[width=7in]{ResOmega=6,pg2}
	\caption{Second set of resonance fitting graphs for $\omega = 6$}
	\label{fig:ResOmega=6,pg2}
\end{figure}

\begin{figure}[H]
	\centering
	\includegraphics[width=7in]{ResOmega=6,Residuals}
	\caption{Graphs of residuals for the resonance fittings for $\omega = 6$}
	\label{fig:ResOmega=6,Residuals}
\end{figure}

\begin{figure}[H]
	\centering
	\includegraphics[width=7in]{ResOmega=7,pg1}
	\caption{First set of resonance fitting graphs for $\omega = 7$}
	\label{fig:ResOmega=7,pg1}
\end{figure}

\begin{figure}[H]
	\centering
	\includegraphics[width=7in]{ResOmega=7,pg2}
	\caption{Second set of resonance fitting graphs for $\omega = 7$}
	\label{fig:ResOmega=7,pg2}
\end{figure}

\begin{figure}[H]
	\centering
	\includegraphics[width=7in]{ResOmega=7,Residuals}
	\caption{Graphs of residuals for the resonance fittings for $\omega = 7$}
	\label{fig:ResOmega=7,Residuals}
\end{figure}


\subsubsection{Resonance Values and Errors}
\label{sec:ResonanceErrors}

\begin{figure}[H]
	\centering
	\includegraphics[width=5in]{KohnFitting}
	\caption{Kohn fitting ($\tau = 0.0$) with different weighting functions for $\omega = 7$}
	\label{fig:KohnFitting}
\end{figure}

\begin{figure}[H]
	\centering
	\includegraphics[width=5in]{07Fitting}
	\caption{Fitting for $\tau = 0.7$ with different weighting functions for $\omega = 7$}
	\label{fig:07Fitting}
\end{figure}

\begin{figure}[H]
	\centering
	\includegraphics[width=5in]{pi4Fitting}
	\caption{Fitting for $\frac{\pi}{4}$ with different weighting functions for $\omega = 7$}
	\label{fig:pi4Fitting}
\end{figure}

The results for the resonance parameters normally differ slightly for each of the Kohn methods.  To determine these, we gathered the data for all $\kappa$ values below the Ps(2s)+H(1s) channel threshold for a given Kohn method.  An example of the resultant graph for $\tau = 0.0$ (Kohn) can be seen in figure \ref{fig:KohnFitting}.  As described in section \ref{sec:ResonanceFit}, MATLAB's nonlinear fitting routine, nlinfit, is used with 8 different weighting functions to try various fits.  All 8 weighting functions give at least respectable fits to the data, but some fits are obviously worse than others.  For our data, the Fair, Logistic and Huber do not fit well for the second resonance.  The Cauchy will fit it close but not as close as the Andrews, Bisquare, Talwar and Welsch.  Out of the closest fittings, the Talwar is usually the worst of the four.

The closeness of the fits can be determined somewhat by plotting the residuals or looking at the values of the residual sum of squares, but these can be misleading near the resonances.  The MATLAB script that does the fittings also generates graphs for each fit with the data points superimposed.  The quality of the fitting is easily evaluated visually.  Due to the narrowness of the second resonance, the poor fits will usually overestimate its width and/or give the wrong position, causing the curve to not intersect the data points.

An example of a good set of fits is given in figure \ref{fig:KohnFitting}.  The graphs in this section are plotted with respect to the positronium energy, unlike the graphs in section \ref{sec:ResonanceGraphs}, which are plotted with respect to the momentum, $\kappa$.  All three of the plotted fitting curves follow the data nicely, with the exception of the extreme points near $\delta^+ = 0$.  This is not a failure of the fitting but instead a side effect of using the model given by \ref{eq:ResonanceCurve}.  The range of $\arctan$ is $(-\frac{\pi}{2},\frac{\pi}{2})$, so the $\arctan$ parts of this model cannot bring the phase shift from the background (near 2.0) all the way to 0.0, as it can only add up to $\frac{\pi}{2}$.  Regardless of this, the fits are nearly perfect in this example.

For some values of $\tau$, we have what appear to be Schwartz singularities.  Section \ref{sec:SchwartzSing} has more discussion on how these arise.  Figure \ref{fig:07Fitting} has a series of data points to the right of the second singularity that do not agree with runs for other values of $\tau$.  MATLAB attempts to fit with these, leading to incorrect parameters for the second resonance.  This type of behavior is observed for $\kappa = 0.7$ and $0.8$, so these runs are not included in the final results.  A similar problem is seen in figure \ref{fig:pi4Fitting}.  One data point at 5 eV is obviously shifted from where it ``should'' be.  The different weighting functions assign different importance to this point, so all three weighting functions gives different positions and widths of the second resonance.  This run for $\kappa = \frac{\pi}{4}$ was also discarded. 

The resonance parameters are calculated separately for each Kohn method (Kohn, inverse Kohn, etc.).  The ``good'' fits for all Kohn methods are all compiled in an Excel spreadsheet, where the mean, mode and errors are determined for each of the four resonance parameters.

The error for each parameter could be determined one of three ways.  The errors given in table \ref{tab:SWaveResonances} are simply the standard deviation.  We could have also used the standard error, which is given as $SE = \sigma / \sqrt{n}$.  The standard error is typically too small, since we are using approximately 100 tests.  For instance, using $\omega = 7$, for $^1E_R$, we get a standard error of $0.00001$, which implies a greater precision and agreement than our results have.  Another possible method for determining the error that we have looked at is to find the maximum deviation from the mean.  However, this overemphasizes the importance of outliers.  Using $\omega = 6$, the maximum deviation from the mean is $0.00586$ for $^2E_R$, which is $4.0$ times as large as the standard deviation. 



\subsubsection{Resonance Parameters}

\setlength{\abovecaptionskip}{6pt}   % 0.5cm as an example
\setlength{\belowcaptionskip}{6pt}   % 0.5cm as an example
\begin{table}[H]
\footnotesize
\centering

\begin{tabular}{l l l l l}
\toprule
Method & \thead{$^1E_R \text{ (eV)}$} & \thead{$^1\Gamma \text{ (eV)}$} & \thead{$^2E_R \text{ (eV)}$} & \thead{$^2\Gamma \text{ (eV)}$} \\
\midrule
This work & $4.0065 \pm 0.0001$ & $0.0955 \pm 0.0001$ & $5.0277 \pm 0.0018$ & $0.0608 \pm 0.0005$ \\
CC (9Ps9H + H$^-$) \cite{Walters2004} & $4.149$ & $0.103$ & $4.877$ & $0.0164$ \\
Kohn variational \cite{VanReeth2004} & $4.0072 \pm 0.0020$ & $0.0956 \pm 0.010$ & $5.0267 \pm 0.0020$ & $0.0597 \pm 0.0010$ \\
Stabilization \cite{Yan2003} & $4.007$ & $0.0969$ & $4.953$ & $0.0574$ \\
CC (22Ps1H + H$^-$) \cite{Blackwood2002b} & $4.141$ & $0.071$ & $4.963$ & $0.033$ \\
CC (9Ps9H) \cite{Blackwood2002} & $4.37$ & $0.10$ & --- & --- \\
Optical potential \cite{DiRienzi2002b} & $4.021$ & $0.0259$ & --- & --- \\
Five-state CC \cite{Adhikari2001e} & $4.01$ & $0.15$ & --- & --- \\
Complex rotation \cite{Yan1999} & $4.0058 \pm 0.0005$ & $0.0952 \pm 0.0011$ & $4.9479 \pm 0.0014$ & $0.0585 \pm 0.0027$ \\
Coupled-pseudostate \cite{Campbell1998} & $4.55$ & $0.084$ & --- & --- \\
Complex rotation \cite{Ho1978} & $4.013 \pm 0.014$ & $0.075 \pm 0.027$ & --- & --- \\
Complex rotation \cite{Drachman1975} & $4.455 \pm 0.010$ & $0.062 \pm 0.015$ & --- & --- \\
Stabilization \cite{Hazi1970} & $5.8366$ & $0.2693$ & --- & --- \\
\bottomrule
\end{tabular}
\caption{S-wave resonance parameters}
\label{tab:SWaveResonancesOther}
\end{table}

\textbf{@TODO:} Create comparison graph like in \cite{Ho1998}.
\textbf{@TODO:} Discussion of Peter's previous calculations and omitted terms.








\biblio
\end{document}