% -*- root: Dissertation.tex -*-
\documentclass[Dissertation.tex]{subfiles} 
\begin{document}


\chapter{S-Wave}
\label{chp:SWave}

%\lettrine{\textcolor{startcolor}{T}}{he} S-wave is the simplest of the 
%partial waves to consider, since it is the lowest angular momentum state, and 
%the spherical harmonics are constant, so giving no angular dependence. As a 
%condensed notation, the S-wave singlet and triplet are referred to as $^1S$ 
%and $^3S$, respectively (as shown in the nomenclature on page
%\pageref{chp:nomenclature}).

\iftoggle{UNT}{The}{\lettrine{\textcolor{startcolor}{T}}{he}}
S-wave is the simplest of the 
partial waves to consider, since the spherical harmonic
$\SphericalHarmonicY{0}{0}{\theta_\rho}{\varphi_\rho}$ is constant. The form
of the Kohn matrix elements was derived in \cref{chp:WaveKohn} for general
$\ell$, but when we started this project, we went through the S-wave derivation
first and wrote code to compute this partial wave only. This chapter shows the
matrix elements specifically for the S-wave, which can be shown from the
general results in \cref{chp:WaveKohn}. The code described in
\cref{chp:General} works for any partial wave but is slower than the S-wave
specific code (\cref{chp:Programs}).

Van Reeth and Humberston \cite{VanReeth2003} previously performed $^{1,3}$S-wave
Ps-H calculations using the Kohn and inverse Kohn. We extend their work here
and used Van Reeth's notes and code \cite{VanReethPrivate} for guidance, 
though we rederived everything, and I wrote my own codes.

\section{Wavefunction}
\label{sec:SWaveFn}

The general wavefunction for any $\ell$ is given in \cref{eq:GeneralWaveTrial}, and the S-wave version is
\beq
\Psi_0^{\pm,t} = \widetilde{S}_0 + L_0^{\pm,t} \, \widetilde{C}_0 + \sum_{i=1}^{N'(\omega)} c_i \bar{\phi}_i.
\label{eq:SWaveTrial}
\eeq
The S-wave has only one set of short-range terms. $\widetilde{S}_0$,
$\widetilde{C}_0$, and $\bar{\phi}_i$ are given by
\cref{eq:TildeSCDef,eq:PhiDef}. The forms of the short-range terms in
\cref{eq:PhiDef} are equivalent for the S-wave ($\ell = 0$).
There is one notable difference about what we did here for the S-wave versus the
general code, namely that we absorb the spherical harmonic on the short-range
terms into the $c_i$ coefficients, similar to what we did with the $\frac{1}{\sqrt{2}}$.


\section{Matrix Elements with \texorpdfstring{$\mathcal{L}\bar{C}_0$}{LC}}
\label{sec:LCElements}

%\subsection{Elements with \texorpdfstring{$\mathcal{L}C$}{LC}}
These can be calculated using the results in \cref{sec:LCTerms}, but the S-wave
code was written using these derivations.%, before we came up with a general formalism. 
The analysis for $\mathcal{L}C_0$ is more difficult than that of
$\mathcal{L}S_0$ in \cref{eq:LSFinal} and \cref{eq:LSPrimeFinal}.
The shielding factor, $f_0(\rho)$, complicates the derivatives slightly.
\begin{align}
\mathcal{L}C_0 = & \left(-\frac{1}{2}\nabla_\rho^2 - \nabla_{r_3}^2 - 2\nabla_{r_{12}}^2 + \frac{2}{r_1} - \frac{2}{r_2} - \frac{2}{r_3} - \frac{2}{r_{12}} - \frac{2}{r_{13}} + \frac{2}{r_{23}} - 2 E_H - 2 E_{Ps} - \frac{1}{2}\kappa^2 \right) \nonumber \\
 & \times \Phi_{Ps}(r_{12}) \Phi_H(r_3) \frac{\sin(\kappa\rho)}{\kappa\rho} \sqrt{\frac{2\kappa}{4\pi}} \left[1 + e^{-\mu\rho} \left(1 + \frac{\mu}{2} \rho \right) \right]
\label{eq:LC1}
\end{align}

\noindent Again, we use \cref{eq:HEqn,eq:PsEqn}) to simplify this expression.
\begin{align}
\mathcal{L}C_0 = & \left(-\frac{1}{2}\nabla_\rho^2 + \frac{2}{r_1} - \frac{2}{r_2} - \frac{2}{r_3} - \frac{2}{r_{12}} - \frac{2}{r_{13}} + \frac{2}{r_{23}}  - \frac{1}{2}\kappa^2\right) \nonumber \\
 & \times \Phi_{Ps}(r_{12}) \Phi_H(r_3) \frac{\sin(\kappa\rho)}{\kappa\rho} \sqrt{\frac{2\kappa}{4\pi}} \left[1 + e^{-\mu\rho} \left(1 + \frac{\mu}{2} \rho \right) \right]
\label{eq:LC2}
\end{align}

Similar to \cref{sec:SphBess2}, $n_0(\kr)$ is an eigenfunction of $\nabla_\rho^2$ with eigenvalue $-\kappa^2$.
%Then $\displaystyle -\frac{1}{2} \nabla_\rho^2 \frac{\cos(\kr)}{\kr}$ could be replaced by $\displaystyle \frac{1}{2}\kappa^2\frac{\cos(\kr)}{\kr}$ if the shielding term could be ignored.  However, it also depends on $\rho$, so we have to calculate this separately.
To properly take into account the shielding function, we use the code in
\cref{fig:LCMath} for the S-wave, yielding
\beq
\label{eq:LCMathS}
\frac{1}{2} \left(\Laplacian_\rho + \kappa^2\right) \SphericalHarmonicY{0}{0}{\theta_\rho}{\varphi_\rho} n_0(\kappa\rho) f_0(\rho) = \frac{2 \kappa  f^\prime(\rho ) \sin (\kappa \rho )-f^{\prime\prime}(\rho) \cos (\kappa \rho )}{2 \kappa \rho}.
\eeq
The $f^\prime(\rho)$ and $f^{\prime\prime}(\rho)$ are given in \cref{eq:Shielding1Der}.
%\begin{align}
%\nonumber -\frac{1}{2} \nabla_\rho^2 & \left\lbrace \frac{\cos(\kr)}{\kr} \left[1 - e^{-\mr} \left(1 + \frac{\mu}{2} \rho \right)\right]\right\rbrace \\
%\nonumber &= \frac{e^{-\mr}}{2\kappa\rho} \left[ \frac{\mu^3 \rho}{2} \cos(\kr) + \kappa^2 \left(-1 + e^{\mr} -\frac{\mr}{2} \right) \cos(\kr) + \kappa\mu (1+\mr) \sin(\kr) \right] \\
%\nonumber &= \frac{e^{-\mr} \mu^3\rho}{4\kr} \cos(\kr) + \frac{\kappa^2}{2} \left[-e^{-\mr} + 1 - \frac{\mr}{2} e^{-\mr}\right] \frac{\cos(\kr)}{\kr} + \frac{e^{-\mr}}{2\kr} \kappa\mu (1 + \mr)\sin(\kr) \\
 %&= \frac{e^{-\mr} \mu^3\rho}{4\kr} \cos(\kr) + \frac{\kappa^2}{2} \left[1 - e^{-\mr}\left(1 + \frac{\mu}{2} \rho \right) \right] \frac{\cos(\kr)}{\kr} + \frac{e^{-\mr}}{2\kr} \kappa\mu (1 + \mr)\sin(\kr)
%\label{eq:laplacianshield}
%\end{align}
%
%\noindent The second term here cancels the $\displaystyle \frac{1}{2}\kappa^2$ in parentheses in \cref{eq:LC2}.  Now we have
%\begin{align}
%\mathcal{L}C_0 = \: & \Phi_{Ps}(r_{12}) \Phi_H(r_3) \sqrt{\frac{2\kappa}{4\pi}} \left\{ \left(\frac{2}{r_1} - \frac{2}{r_2} - \frac{2}{r_{13}} + \frac{2}{r_{23}} \right) \frac{\cos(\kr)}{\kr} \left[1 - e^{-\mr} \left(1 + \frac{\mu}{2}\rho \right) \right] \right. \nonumber\\
%& + \left. \frac{e^{-\mr} \mu^3 \rho}{4} \frac{\cos(\kr)}{\kr} + \frac{e^{-\mr}}{2} \kappa \mu (1+\mu\rho) \frac{\sin(\kr)}{\kr} \right\}.
%\label{eq:LC}
%\end{align}
Combining \cref{eq:LCMathS,eq:LC2} and the permuted versions gives
\begin{align}
\mathcal{L}\bar{C}_0 = \:\: &\frac{1}{\sqrt{2}} \mathcal{L}(C_0 \pm C_0^\prime) = \frac{1}{\sqrt{2}} (\mathcal{L}C_0 \pm \mathcal{L}C_0^\prime) \nonumber \\
= \:\: &\frac{1}{\sqrt{8\pi}} \Phi_{Ps}(r_{12}) \Phi_H(r_3) \sqrt{2\kappa} \nonumber  \\
&\times \left\{ \frac{\kappa\mu}{2} e^{-\mr} (1+\mr) \frac{\sin(\kr)}{\kr} + \frac{\mu^3 \rho}{4} e^{-\mr} \frac{\cos(\kr)}{\kr} \right. \nonumber \\
&+ \left. \left(\frac{2}{r_1} - \frac{2}{r_2} - \frac{2}{r_{13}} + \frac{2}{r_{23}}\right) \frac{\cos(\kr)}{\kr} \left[1 - e^{-\mr} \left(1 + \frac{\mu}{2}\rho\right)\right] \right\} \nonumber \\
\pm &\frac{1}{\sqrt{8\pi}} \Phi_{Ps}(r_{13}) \Phi_H(r_2) \sqrt{2\kappa} \nonumber  \\
&\times \left\{ \frac{\kappa\mu}{2} e^{-\mrp} (1+\mrp) \frac{\sin(\krp)}{\krp} + \frac{\mu^3 \rhop}{4} e^{-\mrp} \frac{\cos(\krp)}{\krp} \right. \nonumber \\
&+ \left. \left(\frac{2}{r_1} - \frac{2}{r_3} - \frac{2}{r_{12}} + \frac{2}{r_{23}}\right) \frac{\cos(\krp)}{\krp} \left[1 - e^{-\mrp} \left(1 + \frac{\mu}{2}\rhop\right)\right] \right\}.
\label{eq:LCBar}
\end{align}

\subsection{\texorpdfstring{$(\bar{C}_0,\mathcal{L}\bar{C}_0)$}{CLC} Matrix Element}
Using \cref{eq:PermPropFull,eq:SCBarDef},
\begin{align}
\left(\bar{C}_0,\mathcal{L}\bar{C}_0\right) = \frac{1}{2}\left[2(C_0,\mathcal{L}C_0) \pm 2(C_0',\mathcal{L}C_0)\right] 
 = (C_0,\mathcal{L}C_0) \pm (C_0',\mathcal{L}C_0).
 \label{eq:CLC1}
\end{align}

Substitute \cref{eq:GenCDef,eq:LCBar} in \cref{eq:CLC1} and simplify to get
%\begin{align}
%\left(\bar{C}_0,\mathcal{L}\bar{C}_0\right) = &(C_0,\mathcal{L}C_0) \pm (C_0',\mathcal{L}C_0) = \left((C_0 \pm C_0'),\mathcal{L}C_0 \right) \nonumber\\
 %= &\left((C_0 \pm C_0') \Phi_{Ps}(r_{12}) \Phi_H(r_3) \sqrt{\frac{2\kappa}{4\pi}} \left\{ \left(\frac{2}{r_1} - \frac{2}{r_2} - \frac{2}{r_{13}} + \frac{2}{r_{23}} \right) \frac{\cos(\kr)}{\kr} \right.\right.  \nonumber \\
   %&\times \left[1 - e^{-\mr}\left(1 + \frac{\mu}{2}\rho \right) \right] + \left.\left. \frac{e^{-\mr} \mu^3 \rho}{4} \frac{\cos(\kr)}{\kr} + \frac{e^{-\mr}}{2} \kappa\mu (1+\mr) \frac{\sin(\kr)}{\kr} \right\} \right).
 %\label{eq:CBarLCBar1}
%\end{align}
%
%These can be split up to simplify:
\begin{align}
\left(\bar{C}_0,\mathcal{L}\bar{C}_0\right) %= &\sqrt{\frac{2\kappa}{4\pi}} \frac{\kappa\mu}{2} \left((C_0 \pm C_0^\prime) e^{-\mr}(1+\mr) \frac{\sin(\kr)}{\kr} \Phi_{Ps}(r_{12}) \Phi_H(r_3) \right)  \nonumber\\
% &+ \sqrt{\frac{2\kappa}{4\pi}} \frac{\mu^3}{4} \left((C_0 \pm C_0') e^{-\mr} \frac{\cos(\kr)}{\kr} \Phi_{Ps}(r_{12}) \Phi_H(r_3) \right) \nonumber\\
% &+ \sqrt{\frac{2\kappa}{4\pi}} \left((C_0 \pm C_0') \left(\frac{2}{r_1} - \frac{2}{r_3} - \frac{2}{r_{12}} + \frac{2}{r_{23}}\right) \frac{\cos(\kr)}{\kr} \right.  \nonumber \\
% &\times \left. \left[1 - e^{-\mr}\left(1 + \frac{\mu}{2}\rho \right)\right] \Phi_{Ps}(r_{12}) \Phi_H(r_3) \right) \nonumber\\
= &\frac{\kappa\mu}{2} \left((C_0 \pm C_0^\prime) e^{-\mr}(1+\mr) S_0 \right) \nonumber\\
 &+ \sqrt{\frac{2\kappa}{4\pi}} \frac{\mu^3}{4} \left((C_0 \pm C_0^\prime) e^{-\mr} \frac{\cos(\kr)}{\kr} \Phi_{Ps}(r_{12}) \Phi_H(r_3) \right) \nonumber\\
 &+ \left((C_0 \pm C_0^\prime) \left(\frac{2}{r_1} - \frac{2}{r_3} - \frac{2}{r_{12}} + \frac{2}{r_{23}}\right) C_0 \right).
 \label{eq:CBarLCBar2}
\end{align}

Looking at just the last term:
\begin{align}
&\left((C_0 \pm C_0') \left(\frac{2}{r_1} - \frac{2}{r_3} - \frac{2}{r_{12}} + \frac{2}{r_{23}}\right) C_0 \right) \nonumber \\
 &\phantom{Space} = \left( \left(\frac{2}{r_1} - \frac{2}{r_3} - \frac{2}{r_{12}} + \frac{2}{r_{23}}\right) C_0^2 \right) \pm \left( \left(\frac{2}{r_1} - \frac{2}{r_3} - \frac{2}{r_{12}} + \frac{2}{r_{23}}\right) C_0' C_0 \right).
\end{align}

\noindent The first set of parentheses has the same form as \cref{eq:SbarLSbar}. These terms in parentheses are antisymmetric with respect to the $1 \leftrightarrow 2$ permutation. Also, $C_0$ is symmetric with respect to this permutation. So the first set of parentheses is 0.  Thus,
\beq
\left((C_0 \pm C_0') \left(\frac{2}{r_1} - \frac{2}{r_3} - \frac{2}{r_{12}} + \frac{2}{r_{23}}\right) C_0 \right) = \pm \left(C_0' \left(\frac{2}{r_1} - \frac{2}{r_3} - \frac{2}{r_{12}} + \frac{2}{r_{23}}\right) C_0 \right)
\eeq

\noindent We finally have that
\begin{align}
\left(\bar{C}_0,\mathcal{L}\bar{C}_0\right) = \: &\frac{\kappa\mu}{2} \left((C_0 \pm C_0') e^{-\mr}(1+\mr) S_0 \right) \nonumber\\
 &+ \sqrt{\frac{2\kappa}{4\pi}} \frac{\mu^3}{4} \left((C_0 \pm C_0') e^{-\mr} \frac{\cos(\kr)}{\kr} \Phi_{Ps}(r_{12}) \Phi_H(r_3) \right) \nonumber\\
 &\pm \left(C_0' \left(\frac{2}{r_1} - \frac{2}{r_3} - \frac{2}{r_{12}} + \frac{2}{r_{23}}\right) C_0 \right).
 \label{CBarLCBar}
\end{align}
This is the form that is used in the S-wave long-range code (\cref{chp:Programs}).% \cref{}.

\subsection{\texorpdfstring{$(\bar{\phi_i},\mathcal{L}\bar{C}_0)$}{PhiLC} Matrix Elements}
For the S-wave, following the work of Van Reeth \cite{VanReethThesis}, we 
chose to absorb both the $\frac{1}{\sqrt{2}}$ and
$Y_{0,0}(\theta_\rho,\varphi_\rho) = \frac{1}{\sqrt{4\pi}}$ into the $c_i$ 
constants of the short-range terms in \cref{eq:PhiDef}.

From \cref{eq:PermPropFull},
\begin{subequations}
\begin{align}
(\bar{\phi}_i, \mathcal{L}\bar{C}_0) &= \frac{2}{\sqrt{2}} \left[(\phi_i,\mathcal{L}C_0) \pm (\phi_i',\mathcal{L}C_0)\right] \label{PhiBarLCBar2a} \\
 &= \frac{2}{\sqrt{2}} \left[(\phi_i,\mathcal{L}C_0) \pm (\phi_i,\mathcal{L}C_0')\right].  \label{PhiBarLCBar2b}
\end{align}
\end{subequations}

\noindent Using \cref{eq:LCBar} in the above gives the results for $(\bar{\phi}_i, \mathcal{L}\bar{C}_0)$.
\begin{subequations}
\begin{align}
(\bar{\phi}_i, \mathcal{L}\bar{C}_0) = &\sqrt{\frac{\kappa}{\pi}} \left( (\phi_i \pm \phi_i') \Phi_{Ps}(r_{12}) \Phi_H(r_3) \left\{ \left(\frac{2}{r_1} - \frac{2}{r_2} - \frac{2}{r_{13}} + \frac{2}{r_{23}}\right) \frac{\cos(\kr)}{\kr} \right.\right. \nonumber\\
&\times \left[1 - e^{-\mr}\left(1 + \frac{\mu}{2}\rho\right)\right] + \left.\left.\frac{e^{-\mr}\mu^3\rho}{4} \frac{\cos(\kr)}{\kr} + \frac{e^{-\mr}}{2} \kappa\mu (1+\mr) \frac{\sin(\kr)}{\kr}  \right\}\right) \\
= \sqrt{\frac{\kappa}{\pi}} &\left( \phi_i \Phi_{Ps}(r_{12}) \Phi_H(r_3) \left\{ \left(\frac{2}{r_1} - \frac{2}{r_2} - \frac{2}{r_{13}} + \frac{2}{r_{23}}\right) \frac{\cos(\kr)}{\kr} \left[1 - e^{-\mr}\left(1 + \frac{\mu}{2}\rho\right)\right] \right.\right. \nonumber\\
&+ \left.\left.\frac{e^{-\mr}\mu^3\rho}{4} \frac{\cos(\kr)}{\kr} + \frac{e^{-\mr}}{2} \kappa\mu (1+\mr) \frac{\sin(\kr)}{\kr}  \right\}\right) \nonumber\\
\pm \sqrt{\frac{\kappa}{\pi}} &\left( \phi_i \Phi_{Ps}(r_{12}) \Phi_H(r_3) \left\{ \left(\frac{2}{r_1} - \frac{2}{r_3} - \frac{2}{r_{12}} + \frac{2}{r_{23}}\right) \frac{\cos(\krp)}{\krp} \left[1 - e^{-\mrp}\left(1 + \frac{\mu}{2}\rhop\right)\right] \right.\right. \nonumber\\
&+ \left.\left.\frac{e^{-\mr}\mu^3\rhop}{4} \frac{\cos(\krp)}{\krp} + \frac{e^{-\mrp}}{2} \kappa\mu (1+\mrp) \frac{\sin(\krp)}{\krp}  \right\}\right)
\end{align}
\end{subequations}
Either of these forms can be used. We use the second form, since this only has
$\phi_i$, not $\phi_i$ and $\phi_i'$.


\section{Results}

%The primary quantity computed with the Kohn variational method is the phase 
%shift. From this, other quantities can be calculated, such as resonance 
%parameters (\cref{sec:SWaveResonances}), scattering lengths (\cref{chp:ERT}), 
%effective ranges (\cref{chp:ERT}), and cross sections (\cref{chp:CrossSections}).

\subsection{Phase Shifts}
All runs here were performed using the set of integration points described in 
\cref{sec:QuadraturePoints}. The number of terms used for $^1$S was determined
using the procedure in \cref{sec:MaxMu}, and for $^3$S, the method in
\cref{sec:CompPhase}. The phase shifts are calculated with the programs 
described in \cref{chp:Programs}. \Cref{tab:SWavePhase} shows the $^{1,3}$S phase shifts 
calculated using the $S$-matrix complex Kohn at regular intervals of $\kappa$, which we 
compare to the results from other groups in \cref{tab:SWaveComparisons}. 
The extrapolations in the fourth and fifth columns are performed using the 
technique described in \cref{sec:Extrapolations}. The last two columns show 
the percent difference given by \cref{eq:PercentDiff}.

\begin{table}
\centering
\begin{tabular}{c c c c c c c c}
\toprule
$\kappa$ & $\delta^+ (\omega = 7)$ & $\delta^- (\omega = 7)$ & $\delta^+ (\omega \rightarrow \infty)$ & $\delta^- (\omega \rightarrow \infty)$ & \% Diff$^+$ & \% Diff$^-$ \\
\midrule
0.1 & $-0.427$ & $-0.215$ & $-0.426$ & $-0.214$ & $0.223\%$ & $0.120\%$ \\
0.2 & $-0.820$ & $-0.431$ & $-0.819$ & $-0.431$ & $0.010\%$ & $0.063\%$ \\
0.3 & $-1.161$ & $-0.645$ & $-1.161$ & $-0.645$ & $0.040\%$ & $0.094\%$ \\
0.4 & $-1.446$ & $-0.850$ & $-1.446$ & $-0.849$ & $0.022\%$ & $0.130\%$ \\
0.5 & $-1.678$ & $-1.041$ & $-1.677$ & $-1.040$ & $0.031\%$ & $0.166\%$ \\
0.6 & $-1.858$ & $-1.217$ & $-1.857$ & $-1.214$ & $0.040\%$ & $0.273\%$ \\
0.7 & $-1.964$ & $-1.375$ & $-1.963$ & $-1.372$ & $0.045\%$ & $0.250\%$ \\
\bottomrule
\end{tabular}
\caption[$^{1,3}$S complex Kohn phase shifts]{$^{1,3}$S phase shifts using the $S$-matrix complex Kohn. \% Diff$^+$ and \% Diff$^-$ are the percent differences between the
 current complex Kohn $\omega = 7$ and $\omega \rightarrow \infty$ results.}
\label{tab:SWavePhase}
\end{table}

\Cref{tab:SWaveComparisons} gives comparisons between the complex Kohn phase 
shifts and phase shifts calculated elsewhere in the literature. The $\omega = 
7$ phase shifts from this table are the same as Van Reeth and 
Humberston's results for $\omega = 6$ \cite{VanReeth2003}, with some 
exceptions in the last digit. This indicates that the prior Kohn variational 
method S-wave phase shifts were well converged, despite only using 721
short-range terms. 
We use a larger basis set here, which brings the phase shifts up slightly,
but the larger set of integration points for the long-range terms (see
\cref{sec:SelQuadPoints}) tended to bring the phase shifts down slightly.
\Cref{fig:SWavePhase} has the more complete set of phase shifts plotted with 
respect to the incoming Ps energy, $E_{\bm \kappa}$. This compares the 
complex Kohn phase shifts to that of the $^1$S CC \cite{Walters2004} and the $
^3$S CC \cite{Blackwood2002}, along with the $^{1,3}$S CVM \cite{Zhang2012}. 

\begin{figure}
	\centering
	\includegraphics[width=\textwidth]{swave-phases}
	\caption[$^{1,3}$S phase shifts]{$^{1,3}$S complex Kohn phase shifts. The $^1S$ CC phase shifts
\cite{Walters2004} are given by \mbox{\textcolor{blue}{$\times$}}, and the
$^3S$ CC phase shifts \cite{Blackwood2002} are given by
\mbox{\textcolor{red}{\textbf{+}}}. The CVM $^1S$ and $^3S$ phase shifts
\cite{Zhang2012} are blue and red circles,
respectively. Vertical dashed lines denote the complex rotation resonance
positions \cite{Yan1999,Yan1998a,Ho1998}. An interactive version of this
figure is available online \cite{Plotly}.}
	\label{fig:SWavePhase}
\end{figure}

As seen in \cref{tab:SWavePhase,fig:SWavePhase}, the $^1$S CC phase 
shifts are slightly below the complex Kohn phase shifts, with a larger 
difference at higher $\kappa$. The complex Kohn $^3$S results are almost 
exactly the same as the prior Kohn \cite{VanReeth2003}, and Van Reeth and 
Humberston noted then that the CC $^3$S phase shifts were higher. For 
scattering problems, there is no rigorous bound, but the phase shifts are
typically empirically bound. In the inset in
\cref{fig:SWavePhase} however, we see that the recent CVM results line up 
well with the complex Kohn for both $^1$S and $^3$S, potentially indicating 
that complex Kohn phase shifts are more accurate than the CC.



\begin{figure}
	\centering
	\includegraphics[width=5.25in]{swave-comparisons}
	\caption[Comparison of $^1S$ and $^3S$ phase shifts]{Comparison of $^1S$ (a) and $^3S$ (b) phase shifts
with results from other groups. Results are ordered according to year of
publication. This work -- solid curves;
\mbox{\textcolor{blue}{$\times$} -- CC \cite{Walters2004};}
\mbox{$\CIRCLE$ -- Kohn \cite{VanReeth2003};}
\mbox{\textcolor{red}{\textbf{+}} -- CC \cite{Blackwood2002};}
\mbox{$\blacktriangle$ -- DMC \cite{Chiesa2002};} 
\mbox{$\triangledown$ -- SVM 2002 \cite{Ivanov2002};} 
\mbox{\textcolor[RGB]{0,127,0}{$\blacktriangle$} -- T-matrix \cite{Biswas2002a};} 
\mbox{$\Circle$ -- SVM 2001 \cite{Ivanov2001};} 
\mbox{\textcolor[RGB]{0,127,0}{$\triangledown$} -- 2 channel / static exchange with model exchange \cite{Biswas2001};} 
\mbox{\textcolor{red}{$\vartriangle$} -- 6-state CC \cite{Sinha2000};} 
\mbox{$\blacksquare$ -- 5-state CC \cite{Adhikari1999};} 
\mbox{$\square$ -- Coupled-pseudostate \cite{Campbell1998};} 
\mbox{$\vartriangle$ -- 3-state CC \cite{Sinha1997};} 
\mbox{\textcolor[RGB]{0,127,0}{$\bigstar$} -- Static-exchange \cite{Ray1997};} 
\mbox{$\triangleright$ -- Stabilization \cite{Drachman1976};} 
\mbox{\textcolor{red}{$\blacklozenge$} -- Stabilization \cite{Drachman1975};}
\mbox{\textcolor{blue}{$\lozenge$} -- Static-exchange \cite{Hara1975};}
\mbox{$\blacktriangledown$ -- Static-exchange \cite{Fraser1961}.}}
	\label{fig:SWaveComparisons}
\end{figure}

%\todoi{Add \cite{Fraser1961} $\delta_0^-$ and \cite{Biswas2002a} $\delta_0^+$}

\Cref{fig:SWaveComparisons} shows comparisons of the complex Kohn phase 
shifts to that of other groups for calculations of $^1$S and $^3$S. The
different Kohn-type methods agree to the accuracy given after methods with
Schwartz singularities are removed. The current $S$-matrix complex Kohn results are extremely close to Van
Reeth and Humberston's results \cite{VanReeth2003}, so they 
follow along the solid line as well. Several groups have results that cluster 
very closely to the current $S$-matrix complex Kohn phase shifts, namely Blackwood et al. \cite{Blackwood2002}, Walters 
et al. \cite{Walters2004}, Chiesa et al. \cite{Chiesa2002} and Ivanov et al. \cite{Ivanov2002}. 
Ivanov et al.\ \cite{Ivanov2002} discuss that the higher phase shifts of
Adhikari and Biswas \cite{Adhikari1999} are likely to be in error. Likewise, 
the further $^1$S calculations by Biswas et al. \cite{Biswas2002a} are near 
that of Adhikari and Biswas \cite{Adhikari1999}. The Biswas et al. \cite{Biswas2001} phase shifts 
agree relatively well with the current complex Kohn and that of the accurate 
Refs.~\cite{Blackwood2002,VanReeth2003,Walters2004} for $^1$S but seem to 
overestimate the $^3$S phase shifts.
%, like that of Adhikari and Biswas
%\cite{Adhikari1999}.


\begin{table}
\centering
\setlength{\tabcolsep}{-2pt}
\footnotesize
\begin{tabular}{@{\hskip 0.1cm}l . . . . . . .}
\toprule
Method & \multicolumn{1}{c}{\phantom{1}0.1} & \multicolumn{1}{c}{\phantom{1}0.2} & \multicolumn{1}{c}{\phantom{1}0.3} & \multicolumn{1}{c}{\phantom{1}0.4} & \multicolumn{1}{c}{\phantom{1}0.5} & \multicolumn{1}{c}{\phantom{1}0.6} & \multicolumn{1}{c}{\phantom{1}0.7} \\
\midrule
This work $\omega = 7$ $\delta_0^+$									& -0.427   & -0.820   & -1.161   & -1.446  & -1.678  & -1.858  & -1.964 \\
This work $\omega \to \infty$ $\delta_0^+$							& -0.426   & -0.819   & -1.161   & -1.446  & -1.677  & -1.857  & -1.963 \\
Kohn $\omega = 6$ \cite{VanReeth2003} $\delta_0^+$					& -0.427   & -0.820   & -1.161   & -1.446  & -1.677  & -1.857  & -1.964 \\
Kohn $\omega \rightarrow \infty$ \cite{VanReeth2003} $\delta_0^+$	& -0.425   & -0.817   & -1.158   & -1.443  & -1.674  & -1.852  & -1.959 \\
CVM \cite{Zhang2012} $\delta_0^+$									& -0.42636 & -0.81973 & $---$    & $---$   & $---$   & $---$   & $---$  \\
CC 14Ps14H+H$^-$ \cite{Walters2004} $\delta_0^+$					& -0.428   & -0.825   & -1.167   & -1.453  & -1.685  & -1.867  & -1.992 \\
CC 14Ps14H \cite{Blackwood2002} $\delta_0^+$						& -0.434   & -0.834   & -1.178   & -1.467  & -1.704  & -1.890  & -2.018 \\
T-matrix \cite{Biswas2002a} $\delta_0^+$							& -0.38269 & -0.73419 & -1.03799 & -1.2924 & -1.5014 & -1.6667 & $---$  \\
2 channel ME \cite{Biswas2001} $\delta_0^+$							& -0.532   & -0.966   & -1.294   & -1.546  & -1.746  & -1.910  & -2.048 \\
3-state CC \cite{Sinha1997} $\delta_0^+$							& -0.68    & -1.20    & -1.59    & -1.89   & -2.13   & -2.32   & -2.48  \\
SE \cite{Ray1997} $\delta_0^+$										& -0.692   & -1.212   & -1.592   & -1.902  & -2.142  & -2.362  & -2.512 \\
5-state CC \cite{Adhikari1999} $\delta_0^+$							& -0.362   & -0.702   & -1.002   & -1.252  & -1.462  & -1.622  & -1.712 \\
SE \cite{Hara1975} $\delta_0^+$										& -0.68649 & -1.2147  & -1.6029  & -1.9026 & -2.144  & -2.344  & -2.511 \\
\midrule                                                            
This work $\omega = 7$ $\delta_0^-$									& -0.215   & -0.431   & -0.645   & -0.850  & -1.041  & -1.217  & -1.375  \\
This work $\omega \to \infty$ $\delta_0^-$							& -0.214   & -0.431   & -0.645   & -0.849  & -1.040  & -1.214  & -1.372  \\
Kohn $\omega = 6$ \cite{VanReeth2003} $\delta_0^-$ 					& -0.215   & -0.432   & -0.645   & -0.850  & -1.040  & -1.215  & -1.373  \\
Kohn $\omega \rightarrow \infty$ \cite{VanReeth2003} $\delta_0^-$	& -0.214   & -0.431   & -0.645   & -0.849  & -1.038  & -1.211  & -1.366  \\
CVM \cite{Zhang2012} $\delta_0^-$ 									& -0.21464 & -0.43159 & $---$    & $---$   & $---$   & $---$   & $---$   \\
CC 14Ps14H \cite{Blackwood2002} $\delta_0^-$ 						& -0.206   & -0.414   & -0.624   & -0.838  & -1.037  & -1.213  & -1.367  \\
SE ME \cite{Biswas2001} $\delta_0^-$ 								& -0.145   & -0.283   & -0.410   & -0.521  & -0.613  & -0.683  & -0.731  \\
3-state CC \cite{Sinha1997} $\delta_0^+$							& -0.24    & -0.44    & -0.69    & -0.89   & -1.08   & -1.23   & -1.362  \\
SE \cite{Ray1997} $\delta_0^-$ 										& -0.252   & -0.502   & -0.722   & -0.942  & -1.142  & -1.332  & -1.502  \\
5-state CC \cite{Adhikari1999} $\delta_0^-$ 						& -0.167   & -0.327   & -0.474   & -0.602  & -0.706  & -0.784  & -0.833  \\
SE \cite{Hara1975} $\delta_0^-$										& -0.2469  & -0.4888  & -0.7211  & -0.9402 & -1.1435 & -1.3300 & -1.4996 \\
\bottomrule
\end{tabular}
\caption[Comparison of $^{1,3}$S phase shifts]{Comparison of $^{1,3}$S phase shifts between complex Kohn results and those from other groups. Values in the header are $\kappa$ in a.u.}
\label{tab:SWaveComparisons}
\end{table}

An extensive comparison of the S-wave phase shifts with calculations from other
groups is also shown in \cref{tab:SWaveComparisons}. Fewer groups have attempted this
problem than the PsH bound state problem in \cref{tab:BoundEnergyOther}. We see
that the accurate complex Kohn results are very similar to the prior Kohn
\cite{VanReeth2003} and agree extremely well with the CVM results
\cite{Zhang2012}. The CC phase shifts \cite{Walters2004,Blackwood2002} also
agree relatively well, with closer agreement for $^1$S than $^3$S.



\subsection{Resonance Parameters}
\label{sec:SWaveResonances}

Before the Ps(n=2) threshold at \SI{5.102}{eV}, 
there are two very clear resonances for $^1$S scattering, which can be seen 
in \cref{fig:SWavePhase}. %The first resonance is associated with the
%$2s$ state \cite{DiRienzi2002b}. The $1s$ state is associated with the PsH bound 
%state in \cref{chp:PsHBound}.
The trial wavefunction we use cannot be 
extended past this threshold without modifications to take into account the
Ps(n=2) channel. We use the numerical methods described in
\cref{sec:ResonanceFit} to accurately determine the resonance parameters.

\setlength{\abovecaptionskip}{6pt}   % 0.5cm as an example
\setlength{\belowcaptionskip}{6pt}   % 0.5cm as an example
\begin{table}
\footnotesize
\centering
\begin{tabular}{l l l l l}
\toprule
Method & \thead{$^1E_R \text{ (eV)}$} & \thead{$^1\Gamma \text{ (eV)}$} & \thead{$^2E_R \text{ (eV)}$} & \thead{$^2\Gamma \text{ (eV)}$} \\
\midrule
%Current work: Average $\pm$ standard deviation & $4.0065 \pm 0.0001$ & $0.0955 \pm 0.0001$ & $5.0272 \pm 0.0029$ & $0.0608 \pm 0.0007$ \\
%Current work: $S$-matrix complex Kohn & $4.0065$ & $0.0955$ & $5.0278$ & $0.0608$ \\
Current work: &  &  &  & \\
\ Average $\pm$ standard deviation & $4.0065 \pm 0.0001$ & $0.0955 \pm 0.0001$ & $5.0272 \pm 0.0029$ & $0.0608 \pm 0.0007$ \\
Current work: &  &  &  & \\
\ $S$-matrix complex Kohn & $4.0065$ & $0.0955$ & $5.0278$ & $0.0608$ \\
CC (9Ps9H + H$^-$) \cite{Walters2004} & $4.149$ & $0.103$ & $4.877$ & $0.0164$ \\
Kohn variational \cite{VanReeth2004} & $4.0072 \pm 0.0020$ & $0.0956 \pm 0.010$ & $5.0267 \pm 0.0020$ & $0.0597 \pm 0.0010$ \\
Stabilization \cite{Yan2003} & $4.007$ & $0.0969$ & $4.953$ & $0.0574$ \\
CC (22Ps1H + H$^-$) \cite{Blackwood2002b} & $4.141$ & $0.071$ & $4.963$ & $0.033$ \\
CC (9Ps9H) \cite{Blackwood2002} & $4.37$ & $0.10$ & --- & --- \\
Optical potential \cite{DiRienzi2002b} & $4.021$ & $0.0259$ & --- & --- \\
T-matrix \cite{Biswas2002a} & $4.06$ & --- & --- & --- \\
CC \cite{Biswas2002} & $4.04$ & --- & --- & --- \\
Five-state CC \cite{Adhikari2001e} & $4.01$ & $0.15$ & --- & --- \\
Complex rotation \cite{Yan1999} & $4.0058 \pm 0.0005$ & $0.0952 \pm 0.0011$ & $4.9479 \pm 0.0014$ & $0.0585 \pm 0.0027$ \\
Coupled-pseudostate \cite{Campbell1998} & $4.55$ & $0.084$ & --- & --- \\
Complex rotation \cite{Ho1978} & $4.013 \pm 0.014$ & $0.075 \pm 0.027$ & --- & --- \\
Complex rotation \cite{Drachman1975} & $4.455 \pm 0.010$ & $0.062 \pm 0.015$ & --- & --- \\
Stabilization \cite{Hazi1970} & $5.8366$ & $0.2693$ & --- & --- \\
\bottomrule
\end{tabular}
\caption{S-wave resonance parameters}
\label{tab:SWaveResonancesOther}
\end{table}

The current complex Kohn resonance parameters in
\cref{tab:SWaveResonancesOther} are very similar to that of the previous Kohn 
calculations \cite{VanReeth2004}, but the first resonance position matches 
better with the complex rotation of Yan and Ho \cite{Yan1999}. The complex
rotation can be considered one of the best calculations for these 
resonances, and the stabilization method in Ref.~\cite{Yan2003} from the same 
authors agrees well with the complex rotation. Both the complex Kohn and 
prior Kohn calculations give a second resonance position that is at a higher 
energy than the complex rotation, but the current results agree relatively 
well with the complex rotation for most parameters.

The CC calculations of Ref.~\cite{Walters2004} tabulates resonance parameters 
through the F-wave singlet. Their resonance parameters are close to the 
complex Kohn and CC results, but there is a significant difference in the 
second resonance width, which is much smaller than the other calculations. 
Other calculations shown in this table agree approximately on the position 
and width of the resonances, with the exception of the much earlier 
stabilization by Hazi and Taylor \cite{Hazi1970}.

%\todoi{Create comparison graph like in \cite{Ho1998}?}

Ray \cite{Ray2006} reports a $^3$S resonance in a 3-state CC approximation. 
This has not been reported by any other groups, and it appears that this may 
not be a true resonance. The Belfast group notes in Ref.~\cite{Campbell1998}, 
stating that ``Since H$^-$ is in a spin singlet state, the resonances must 
have total electronic spin zero. Accordingly, we find no resonances in our 
triplet partial waves.'' \Cref{fig:triplet-false-resonance} shows that for a 
small number of terms, we get a Schwartz singularity in the triplet. As the 
number of terms increases, this singularity disappears, which can illustrate 
how critical it is to have basis sets with a large number of terms for Ps-H 
scattering.

%\todoi{From \cite{Walters2004}, ``The event line for triplet
%scattering is the same as Fig. 1 except that the PsH
%bound state and the H$^-$ and Ps$^-$ channels are
%omitted. The absence of these formation channels
%means also a corresponding absence of resonance
%structure.''}

\begin{figure}
	\centering
	\includegraphics[width=5in]{triplet-false-resonance}
	\caption{$^3$S plot showing Schwartz singularity at low $N$}
	\label{fig:triplet-false-resonance}
\end{figure}


\section{Summary}
\label{sec:SummaryS}

The Kohn-type variational methods have provided highly accurate phase shifts 
and reliable resonance parameters for the $^{1,3}$S-wave. The $S$-matrix
complex Kohn phase shifts and resonance parameters compare well with those
of accurate calculations from other groups \cite{Blackwood2002,Walters2004,Zhang2012}.



\biblio
\end{document}