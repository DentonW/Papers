\documentclass[Dissertation.tex]{subfiles} 
\begin{document}


\chapter{S-Wave}
\label{chp:SWave}

\section{Wavefunction}
Our trial wavefunction for the s-wave partial wave of positronium scattering from hydrogen is
\begin{equation}
\Psi_t^\pm = \bar{S} + \tan \eta_t^\pm \, \bar{C} + \sum_{i=1}^N c_i \bar{\phi}_i^t ,
\label{eq:SWaveTrialSimple}
\end{equation}

\noindent where

\begin{subequations}\label{SCphiBarDef}
\begin{align}
\bar{S} &= \frac{\left( 1 \pm P_{23} \right) S}{\sqrt{2}} \label{SBarDef} \\
\bar{C} &= \frac{\left( 1 \pm P_{23} \right) C}{\sqrt{2}} \label{CBarDef} \\
\bar{\phi}_i^t &= \left( 1 \pm P_{23} \right) \phi_i \label{PhiBarDef}
\end{align}
\end{subequations}

\noindent and

\begin{subequations}\label{eq:SCPhiDef}
\begin{align}
S &= Y_{0,0}\!\left( \theta_\rho, \phi_\rho \right) \Phi_{Ps}\left(r_{12}\right) \Phi_H\left(r_3\right) \sqrt{2\kappa} \,j_0\!\left(\kappa\rho\right) \label{eq:SDef} \\
C &= -Y_{0,0}\!\left( \theta_\rho, \phi_\rho \right) \Phi_{Ps}\left(r_{12}\right) \Phi_H\left(r_3\right) \sqrt{2\kappa} \,n_0\!\left(\kappa\rho\right) \left[1 - \exp(-\mu \rho) (1+\frac{\mu}{2}\rho)\right] \label{eq:CDef} \\
\phi_i &= e^{-\left(\alpha r_1 + \beta r_2 + \gamma r_3 \right)} r_1^{k_i} r_2^{l_i} r_{12}^{m_i} r_3^{n_i} r_{13}^{p_i} r_{23}^{q_i}. \label{eq:PhiDef}
\end{align}
\end{subequations}

\noindent The $\frac{1}{\sqrt{2}}$ and $Y_{0,0}\left( \theta_\rho, \phi_\rho \right) = \frac{1}{\sqrt{4\pi}}$ for $\bar{\phi}_i^t$ are included in the $c_i$ coefficients in (\ref{eq:SWaveTrialSimple}).

$\rho$ and $\rho'$ are defined as (refer to figure \ref{fig:PsHCoords})
\begin{subequations}
\begin{align}
\vec{\rho} &= \frac{1}{2}\left(\vec{r_1} + \vec{r_2}\right) \label{eq:RhoDef}\\
\vec{\rho}^\prime &= \frac{1}{2}\left(\vec{r_1} + \vec{r_3}\right) \label{eq:RhopDef}.
\end{align}
\end{subequations}

\noindent As a shortcut, we can also write
\begin{equation}
S^\prime = P_{23} S \text{ and } C^\prime = P_{23} C.
\label{eq:SCprime}
\end{equation}

The short-range Hylleraas-type terms are given by the $\phi_i$ in equation \ref{eq:SWaveTrialSimple}. 

The Hamiltonian for the fundamental Coulombic system is
\begin{align}
\label{eq:Hamiltonian1}
H = -\frac{1}{2} \nabla_{r_1}^2 - \frac{1}{2} \nabla_{r_2}^2 - \frac{1}{2} \nabla_{r_3}^2 + \frac{1}{r_1} - \frac{1}{r_2} - \frac{1}{r_3} - \frac{1}{r_{12}} -\frac {1}{r_{13}} + \frac{1}{r_{23}}.
\end{align}

\noindent The Hamiltonian can also be expressed in terms of other variables as
\begin{align}
H = -\frac{1}{4} \nabla_{\rho}^2 - \frac{1}{2} \nabla_{r_3}^2 - \nabla_{r_{12}}^2 + \frac{1}{r_1} - \frac{1}{r_2} - \frac{1}{r_3} - \frac{1}{r_{12}} - \frac{1}{r_{13}} + \frac{1}{r_{23}}
\label{eq:Hamiltonian2}
\end{align}
and
\begin{align}
H = -\frac{1}{4} \nabla_{\rho^\prime}^2 - \frac{1}{2} \nabla_{r_2}^2 - \nabla_{r_{13}}^2 + \frac{1}{r_1} - \frac{1}{r_2} - \frac{1}{r_3} - \frac{1}{r_{12}} - \frac{1}{r_{13}} + \frac{1}{r_{23}}.
\label{eq:Hamiltonian3}
\end{align}


\section{General Wavefunction}

The applications of the Kohn, inverse Kohn, complex Kohns and the generalized Kohn to our S-, P- and D-wave functions are all very similar in form. The wavefunctions for the various partial waves (equations \ref{eq:SWaveTrialSimple}, \ref{eq:PWaveSimple} and \ref{eq:DWaveSimple}) can be written in a general form as

\beq
\Psi_t^\pm = \tilde{S} + \mathcal{L}_l^t \, \tilde{C} + \sum_{i=1}^N c_i \tilde{\phi}_i^t.
\label{eq:GeneralWaveTrial}
\eeq

The short-range $\tilde{\phi}_i^t$ terms represent terms of different symmetries, such as the $\bar{\phi}_{i1}$ and $\bar{\phi}_{j2}$ of the P-wave in equation \ref{eq:PWaveSimple}. In addition to letting the $\tilde{S}$ and $\tilde{C}$ represent the $\bar{S}$ and $\bar{C}$ for the different partial waves, we can define them in such a way as to use multiple Kohn methods (Kohn, inverse Kohn, etc.). We begin by defining a $2\times 2$ matrix $u$ which satisfies

\beq
\label{eq:GenSCMatrix}
\begin{bmatrix}
\tilde{S} \\
\tilde{C}
\end{bmatrix}
=
u
\begin{bmatrix}
\bar{S} \\
\bar{C}
\end{bmatrix}
=
\begin{bmatrix}
u_{00} & u_{01} \\
u_{10} & u_{11}
\end{bmatrix}
\begin{bmatrix}
\bar{S} \\
\bar{C}
\end{bmatrix}.
\eeq

\noindent This notation is equivalent to that of Lucchese \cite{Lucchese1989}. From this, it can easily be seen that
\begin{subequations}
\begin{align}
\tilde{S} &= u_{00} \bar{S} + u_{01} \bar{C} \\
\tilde{C} &= u_{10} \bar{S} + u_{11} \bar{C}.
\end{align}
\end{subequations}

Generally,
\beq
\bar{S} = \frac{1}{\sqrt{2}}(S \pm S^\prime) \text{ and } \bar{C} = \frac{1}{\sqrt{2}}(C \pm C^\prime),
\eeq
where $S^\prime = P_{23}S$ and $C^\prime = P_{23}C$.
The general form for the long-range terms $S$ and $C$ is
\begin{subequations}
\label{eq:GenSandC}
\begin{align}
S = &Y_{l,0}\left( \theta_\rho, \phi_\rho \right) \Phi_{Ps}\left(r_{12}\right) \Phi_H\left(r_3\right) \sqrt{2\kappa} \,j_l\!\left(\kappa\rho\right) \label{eq:GenSDef} \\
C = -&Y_{l,0}\left( \theta_\rho, \phi_\rho \right) \Phi_{Ps}\left(r_{12}\right) \Phi_H\left(r_3\right) \sqrt{2\kappa} \,n_l\!\left(\kappa\rho\right) \label{eq:GenCDef}.
\end{align}
\end{subequations}


\noindent The $u$ and $\mathcal{L}_l$ for the various Kohn methods are described now. Note that for each of these, $\det(u) = 1$.

\subsection{Kohn}
\beq
u =
\begin{bmatrix}
1 & 0 \\
0 & 1 
\end{bmatrix}
\label{eq:uKohn}
\eeq

\beq
\mathcal{L}_l = \lambda_t = K_t
\label{eq:LKohn}
\eeq


\subsection{Inverse Kohn}
\beq
u =
\begin{bmatrix}
0 & 1 \\
-1 & 0 
\end{bmatrix}
\label{eq:uInvKohn}
\eeq

\beq
\mathcal{L}_l = -\mu_t = -K^{-1}_t = -\bar{K}_t
\label{eq:LInvKohn}
\eeq


\subsection{Generalized Kohn}
\beq
u =
\begin{bmatrix}
\cos\tau & \sin\tau \\
-\sin\tau & \cos\tau 
\end{bmatrix}
\label{eq:uGenKohn}
\eeq

\beq
\mathcal{L}_l = a_t
\label{eq:LGenKohn}
\eeq

\noindent The generalized Kohn method is described by Cooper et al.\ \cite{Cooper2009, Cooper2010}.  When $\tau = 0$ is substituted in \ref{eq:uGenKohn}, the $u$-matrix for the Kohn method is generated (equation \ref{eq:uKohn}). Similarly, when $\tau = \frac{\pi}{2}$, the $u$-matrix for the inverse Kohn method is generated (equation \ref{eq:uInvKohn}). As to be expected, the Kohn and inverse Kohn methods are special cases of the generalized Kohn method.


\subsection{Complex Kohn T-matrix}
\beq
u =
\begin{bmatrix}
1 & 0 \\
\ii & 1
\end{bmatrix}
\label{eq:uCompTKohn}
\eeq

\beq
\mathcal{L}_l = T
\label{eq:LCompTKohn}
\eeq

\noindent Lucchese denotes this as $\mathcal{L}_l = -\pi T$ \cite{Lucchese1989}, but to be consistent with our definition of the $T$-matrix, we must use equation \ref{eq:LCompTKohn}.


\subsection{Complex Kohn S-matrix}
\beq
u =
\begin{bmatrix}
-\ii & 1 \\
-\frac{1}{2} & \frac{\ii}{2}
\end{bmatrix}
\label{eq:uCompSKohn}
\eeq

\beq
\mathcal{L}_l = 2 \ii S_l
\label{eq:LCompSKohn}
\eeq

\noindent The Lucchese version of $u$ differs from ours, since he uses a different definition for the $S$-matrix  \cite{Lucchese1989}. The form of the $S$-matrix we are using with our wavefunction is related to the $K$-matrix by
\beq
K_l = \frac{\ii(1-S_l)}{1+S_l},
\eeq
which is satisfied by the above $u$-matrix.


\section{Derivation of the General Kohn Principle}

The version of (\ref{eq:IlDefU}) for the full wavefunction is
\beq
I_l\left[\Psi_l^t\right] \equiv \left<{\Psi_l^t}^* | L | \Psi_l^t \right> = \int \Psi_l^t(\vec{r}) \mathcal{L}_l \Psi_l^t(\vec{r}) \,d\vec{r}.
\label{eq:IlDefPsiGen}
\eeq

\subsection{General Kohn Principle}
\label{sec:GenKohnPrinciple}


In this work, we only consider the first three partial waves, i.e. $l = 0, 1$ and $2$ for the S-, P- and D-waves, respectively. There is no general proof of this (\textbf{@TODO: Is there?}), but for these first few partial waves, a relation between $S$ and $C$ is seen by taking the gradient of each with respect to $\rho$. For the S-wave, $l = 0$, the gradient of the spherical Bessel function is

\beq
\frac{\partial}{\partial \rho} j_0(\kappa\rho) = \frac{\partial}{\partial \rho} \frac{\sin(\kappa\rho)}{\kappa\rho} = \frac{\cos(\kappa\rho)}{\rho} - \frac{\sin(\kappa\rho)}{\kappa \rho^2}.
\eeq

\noindent The leading term in $\rho$ for this case is $\frac{\cos(\kappa\rho)}{\rho}$, which is equal to $\kappa n_0(\kappa\rho)$. Using the definitions from equation \ref{eq:GenSandC} for the first few partial waves considered, and only for the leading terms, it can easily be seen that
\begin{subequations}
\label{eq:GenSCGrad}
\begin{align}
\nabla_\rho S &\approx \kappa C \text{ and}\\
\nabla_\rho C &\approx \kappa S.
\end{align}
\end{subequations}

\noindent Generalizing this to $\tilde{S}$ and $\tilde{C}$ gives
\begin{subequations}
\label{eq:GenGenSCGrad}
\begin{align}
\nabla_{\rho/\rho^\prime} \tilde{S} &= \frac{1}{\sqrt{2}} \left[u_{00} \left(\nabla_\rho S \pm \nabla_{\rho^\prime} S^\prime \right) + u_{01} \left(\nabla_\rho C \pm \nabla_{\rho^\prime} C^\prime \right) \right] \approx \kappa \left[ u_{00} \bar{C} - u_{01} \bar{S} \right] \text{ and}\\
\nabla_{\rho/\rho^\prime} \tilde{C} &= \frac{1}{\sqrt{2}} \left[u_{10} \left(\nabla_\rho S \pm \nabla_{\rho^\prime} S^\prime \right) + u_{11} \left(\nabla_\rho C \pm \nabla_{\rho^\prime} C^\prime \right) \right] \approx \kappa \left[ u_{10} \bar{C} - u_{11} \bar{S} \right].
\end{align}
\end{subequations}

\beq
\label{eq:GenKohnPrinciple}
\mathcal{L}_v = \mathcal{L}_t - I_l\left[\Psi_l^t\right]
\eeq



\section{Application of the Kohn Methods}

We use the general Kohn functional (equation \ref{eq:GenKohnPrinciple}) with our trial function to get
\beq
\label{eq:GenKohnApplied}
\mathcal{L}_v = \mathcal{L}_t - (\psi_t,\mathcal{L} \psi_t) = \mathcal{L}_t - \Big((\bar{\bar{S}} + \mathcal{L}_t \tilde{C} + \sum_i c_i \tilde{\phi}_i), L (\tilde{S} + \mathcal{L}_t \tilde{C} + \sum_j c_j \tilde{\phi}_j )\Big).
\eeq

The property of the Kohn functional that it is stationary with respect to variations in the linear parameters (\ref{Partials1}) can be written in our case as
\beq
\frac{\partial \mathcal{L}_v}{\partial \mathcal{L}_t} = 0  \text{ and } \frac{\partial \mathcal{L}_v}{\partial c_i} = 0 \text{ where $i = 1,\ldots,N$}.
\label{eq:KohnStationary}
\eeq

Performing the first variation gives
\beq
0 = \frac{\partial \mathcal{L}_v}{\partial \mathcal{L}_t} = 1 - \Big[(\tilde{S},\mathcal{L}\tilde{C}) + (\tilde{C},\mathcal{L}\tilde{S}) + \frac{\partial}{\partial \mathcal{L}_t}(\mathcal{L}_t \tilde{C},\mathcal{L} \mathcal{L}_t \tilde{C}) + (\tilde{C}, L \sum_i c_i \tilde{\phi}_i) + (\sum_i c_i \tilde{\phi}_i, L \tilde{C}) \Big].
\label{eq:PdLambda1}
\eeq

\noindent The third term in brackets becomes
\beq
\frac{\partial}{\partial \mathcal{L}_t} (\mathcal{L}_t \tilde{C},\mathcal{L} \mathcal{L}_t \tilde{C}) = (\tilde{C},\mathcal{L} \tilde{C}) \frac{\partial}{\partial \mathcal{L}_t} \mathcal{L}_t^2 = 2(\tilde{C},\mathcal{L}\tilde{C}) \mathcal{L}_t.
\eeq

\noindent The last two terms of (\ref{eq:PdLambda1}) are equal to each other, and we can use (\ref{eq:SLCandCLSBar}) to rewrite this.
\beq
0 = -(\tilde{C},\mathcal{L}\tilde{S}) - (\tilde{C},\mathcal{L}\tilde{S}) - 2 \mathcal{L}_t (\tilde{C},\mathcal{L}\tilde{C}) - 2 \sum_i c_i (\tilde{C},\mathcal{L}\tilde{\phi}_i)
\eeq

\noindent Rearranging gives
\beq
-(\tilde{C},\mathcal{L}\tilde{S}) = \mathcal{L}_t (\tilde{C},\mathcal{L}\tilde{C}) + \sum_i c_i (\tilde{C},\mathcal{L}\tilde{\phi}_i)
\label{eq:PdLambda}
\eeq

Now we perform the variation with respect to a general $c_k$ as in (\ref{eq:KohnStationary}).
\beq
0 = \frac{\partial \mathcal{L}_v}{\partial c_k} = -\Big[ (\tilde{S},\mathcal{L} \tilde{\phi}_k) + \mathcal{L}_t (\tilde{C},\mathcal{L} \tilde{\phi}_k) + (\tilde{\phi}_k,\mathcal{L} \tilde{S}) + \mathcal{L}_t (\tilde{\phi}_k,\mathcal{L} \tilde{C}) + \frac{\partial}{\partial c_k} (\sum_i c_i \tilde{\phi}_i, L \sum_j c_j \tilde{\phi}_j) \Big]
\label{eq:PdCk1}
\eeq

If $c_i \ne c_j$,
\begin{subequations}
\begin{align}
\frac{\partial}{\partial c_i} (c_i \tilde{\phi}_i, L \sum_{j \ne i} c_j \tilde{\phi}_j) &= \sum_{j \ne i} c_j (\tilde{\phi}_i, L \tilde{\phi}_j) \text{ and} \\
\frac{\partial}{\partial c_j} (\sum_{i \ne j} c_i \tilde{\phi}_i, L c_j \tilde{\phi}_j) &= \sum_{i \ne j} c_i (\tilde{\phi}_i, L \tilde{\phi}_j).
\end{align}
\end{subequations}

\noindent These two equations are equivalent, since $\left( \tilde{\phi}_i, L \tilde{\phi}_j \right) = \left( \tilde{\phi}_j, L \tilde{\phi}_i \right)$ by (\ref{PhiLPhiPerm}).

If $c_i = c_j$,
\beq
\frac{\partial}{\partial c_i} \left( c_i \tilde{\phi}_i, L c_j \tilde{\phi}_j \right) = \frac{\partial}{\partial c_i} \left( c_i \tilde{\phi}_i, L c_i \tilde{\phi}_i \right) = \frac{\partial}{\partial c_i} c_i^2 \left( \tilde{\phi}_i, L \tilde{\phi}_j \right) = 2 \, c_i \left( \tilde{\phi}_i, L \tilde{\phi}_j \right).
\eeq

\noindent We can also use (\ref{eq:PhiLSPerm}) and (\ref{eq:PhiLCPerm}) to reduce (\ref{eq:PdCk1}) to
\beq
0 = -\Big[ 2 (\tilde{\phi}_k, L \tilde{S}) + 2 \mathcal{L}_t (\tilde{\phi}_k, L \tilde{C}) + 2 \sum_i (\tilde{\phi}_k, L c_i \tilde{\phi}_i) \Big].
\eeq

\noindent
Rearranging gives
\beq
-\left( \tilde{\phi}_k, L \tilde{S} \right) = \mathcal{L}_t \left( \tilde{\phi}_k, L \tilde{C} \right) + \sum_i \left( \tilde{\phi}_k, L c_i \tilde{\phi}_i \right).
\label{eq:PdCk}
\eeq

The set of linear equations in (\ref{eq:PdLambda}) and (\ref{eq:PdCk}) can be written in matrix form as
\begin{equation}
\label{eq:GeneralKohnMatrix}
\begin{bmatrix} 
 (\tilde{C},\mathcal{L}\tilde{C}) & (\tilde{C},\mathcal{L}\tilde{\phi}_1) & \cdots & (\tilde{C},\mathcal{L}\tilde{\phi}_j) & \cdots\\
 (\tilde{\phi}_1,\mathcal{L}\tilde{C}) & (\tilde{\phi}_1,\mathcal{L}\tilde{\phi}_1) & \cdots & (\tilde{\phi}_1,\mathcal{L}\tilde{\phi}_j) & \cdots\\
 \vdots & \vdots & \ddots & \vdots \\
 (\tilde{\phi}_i,\mathcal{L}\tilde{C}) & (\tilde{\phi}_i,\mathcal{L}\tilde{\phi}_1) & \cdots & (\tilde{\phi}_i,\mathcal{L}\tilde{\phi}_j) & \cdots\\
 \vdots & \vdots & & \vdots & \\
\end{bmatrix}
\begin{bmatrix}
\mathcal{L}_t\\
c_1\\
\vdots\\
c_i\\
\vdots
\end{bmatrix}
= -
\begin{bmatrix}
(\tilde{C},\mathcal{L}\tilde{S}) \\
(\tilde{\phi}_1,\mathcal{L}\tilde{S}) \\
\vdots \\
(\tilde{\phi}_i,\mathcal{L}\tilde{S}) \\
\vdots
\end{bmatrix}.
\end{equation}

\noindent This matrix equation can be rewritten as
\beq
\label{eq:GenKohnMatrixAXB}
\textbf{\emph{AX = -B}}.
\eeq

\noindent Solving this for $\textbf{\emph{X}}$ gives
\beq
\textbf{\emph{X = $-A^{-1}$B}}.
\eeq

To obtain $\mathcal{L}_v$ from this matrix equation, we must next expand equation \ref{eq:GenKohnApplied}.

\begin{align}
\nonumber \mathcal{L}_v = \mathcal{L}_t - \Big[ & (\tilde{S},\mathcal{L}\tilde{S}) + \mathcal{L}_t (\tilde{S},\mathcal{L}\tilde{C}) + \sum_i c_i (\tilde{S},\tilde{\phi}_i) + \mathcal{L}_t (\tilde{C},\mathcal{L}\tilde{S}) + \mathcal{L}_t^2 (\tilde{C},\mathcal{L}\tilde{C}) + \mathcal{L}_t \sum_i c_i (\tilde{C},\mathcal{L} \tilde{\phi}_i) \\
& + \sum_i c_i (\tilde{\phi}_i, L\tilde{S}) + \mathcal{L}_t \sum_i c_i (\tilde{\phi}_i, L\tilde{C}) + \sum_i \sum_j c_i c_j (\tilde{\phi}_i, L\tilde{\phi}_j) \Big]
\end{align}

\noindent By substituting equation \ref{eq:GenSLCandCLS} in for $(\tilde{S},\mathcal{L}\tilde{C})$, the first $L_t$ above is cancelled, leaving

\begin{align}
\label{eq:GenKohnApplied2}
\nonumber \mathcal{L}_v = - \Big[ & (\tilde{S},\mathcal{L}\tilde{S}) + \mathcal{L}_t (\tilde{C},\mathcal{L}\tilde{S}) + \sum_i c_i (\tilde{S},\tilde{\phi}_i) + \mathcal{L}_t (\tilde{C},\mathcal{L}\tilde{S}) + \mathcal{L}_t^2 (\tilde{C},\mathcal{L}\tilde{C}) + \mathcal{L}_t \sum_i c_i (\tilde{C},\mathcal{L} \tilde{\phi}_i) \\
& + \sum_i c_i (\tilde{\phi}_i, L\tilde{S}) + \mathcal{L}_t \sum_i c_i (\tilde{\phi}_i, L\tilde{C}) + \sum_i \sum_j c_i c_j (\tilde{\phi}_i, L\tilde{\phi}_j) \Big].
\end{align}

Using the following definitions of
\beq
D = 
\begin{bmatrix}
\mathcal{L}_t & c_1 & \cdots & c_N
\end{bmatrix}
\text{ and}
\eeq
\beq
\label{eq:GenFandD}
F =
\begin{bmatrix}
(\boldsymbol{\tilde{C},\mathcal{L}\tilde{C}}) & (\boldsymbol{\tilde{C},\mathcal{L}\tilde{\phi}}) & (\boldsymbol{\tilde{C},\mathcal{L}\tilde{S}}) \\
(\boldsymbol{\tilde{\phi},\mathcal{L}\tilde{C}}) & (\boldsymbol{\tilde{\phi},\mathcal{L}\tilde{\phi}}) & (\boldsymbol{\tilde{\phi},\mathcal{L}\tilde{S}}) \\
(\boldsymbol{\tilde{C},\mathcal{L}\tilde{S}}) & (\boldsymbol{\tilde{S},\mathcal{L}\tilde{\phi}}) & (\boldsymbol{\tilde{S},\mathcal{L}\tilde{S}})
\end{bmatrix},
\eeq
equation \ref{eq:GenKohnApplied2} can be rewritten as the following matrix equation:

\beq
\label{eq:GenDFDT}
\mathcal{L}_v = - D F D^T.
\eeq

\noindent Using equations \ref{eq:GenKohnMatrix} and \ref{eq:GenKohnMatrix} in \ref{eq:GenDFDT} and expanding gives
\begin{align}
\label{eq:GenDFDT2}
\nonumber \mathcal{L}_v &= - 
\begin{bmatrix}
\boldsymbol{X^T} & 1 
\end{bmatrix}
\begin{bmatrix}
\boldsymbol{A} & \boldsymbol{B} \\
\boldsymbol{B^T} & \boldsymbol{(\tilde{S},\mathcal{L}\tilde{S})}
\end{bmatrix}
\begin{bmatrix}
\boldsymbol{X} \\
1
\end{bmatrix}
= -
\begin{bmatrix}
\boldsymbol{X^T} & 1 
\end{bmatrix}
\begin{bmatrix}
0 \\
\boldsymbol{B^T X} + (\tilde{S},\mathcal{L}\tilde{S})
\end{bmatrix} \\
&= -\boldsymbol{B^T X} - (\tilde{S},\mathcal{L}\tilde{S}),
\end{align}
where
\beq
\boldsymbol{B^T X} = \mathcal{L}_t (\tilde{C},\mathcal{L}\tilde{S}) + \sum_i c_i (\tilde{\phi}_i, L\tilde{S}).
\eeq

\noindent A more compact way of writing equation \ref{eq:GenDFDT2} is by
\beq
\mathcal{L}_v = -\left( \Psi^{t,0},\mathcal{L} \tilde{S} \right).
\eeq
$\Psi^{t,0}$ is the full general wavefunction in equation \ref{eq:GeneralWaveTrial} with its nonlinear parameters optimized.



\section{Matrix Elements}
In this section, we examine the matrix elements of equation \ref{eq:GeneralKohnMatrix}. The three types of matrix elements are short-range -- short-range, short-range -- long-range and long-range -- long-range.

\subsection{Total Energy and H/Ps Wavefunctions}
The total energy is needed to evaluate all the matrix elements. From (\cite{}), the wavenumber is defined as
\beq
\kappa = \frac{\sqrt{2 m E}}{\hbar}
\label{eq:Wavenumber}
\eeq

From equation \ref{eq:Wavenumber}, since positronium has twice the mass of a positron, $m \to 2m$, the kinetic energy of the positronium atom is $T = \frac{1}{4} \kappa^2$, where $\kappa$ is the wavenumber of Ps.  Including the dissociation energy of H and Ps to get the total energy gives
\beq
E_T = E_H + E_{Ps} + \frac{1}{4} \kappa^2.
\eeq
For the ground states of H and Ps,
\beq
E_T = -\frac{1}{2} - \frac{1}{4} + \frac{1}{4} \kappa^2 = -\frac{3}{4} + \frac{1}{4} \kappa^2 \:\: \text{(in a.u.)}.
\label{eq:EnergyTotal}
\eeq

Some important cancellations are made in the matrix element equations by using the H and Ps ground state wavefunctions. Separately, the H and Ps equations are respectively (for large values of $\rho$)
\beq
\left(-\frac{1}{2}\nabla_{r_3}^2 - \frac{1}{r_3}\right) \Phi_H(r_3) = E_H \Phi_H(r_3)
\label{eq:HEqn}
\eeq
\beq
\left(-\nabla_{r_{12}}^2 - \frac{1}{r_{12}}\right) \Phi_{Ps}(r_{12}) = E_{Ps} \Phi_{Ps}(r_{12})
\label{eq:PsEqn}
\eeq

\subsection{Matrix Elements Involving Long-Range Terms}
\label{sec:MatrixLong}
The short-range--long-range and long-range--long-range matrix elements have a similar analysis. For all of these, the effect of the $\mathcal{L} = 2(H-E)$ operator on the long-range terms must be considered, and then integrations over the external angles (see Appendix \ref{chp:AngularInt}) are performed. The remaining 6-dimensional integral is then numerically integrated as described in sections \ref{sec:LongLongInt} and \ref{sec:ShortLongInt}.

\subsubsection{Elements with \texorpdfstring{$\mathcal{L}S$}{LS}}
\label{sec:LSElements}
The $(\tilde{C},\mathcal{L}\tilde{S})$ and $(\tilde{\phi}_i,\mathcal{L}\tilde{S})$ matrix elements have the $L$-operator acting on $S$. While not explicitly in the matrix given by \ref{eq:GeneralKohnMatrix}, $(\tilde{S},\mathcal{L}\tilde{S})$ is required in the final equation \ref{eq:GenDFDT2}. Also from the general $\tilde{S}$ and $\tilde{C}$, equation \ref{eq:GenSCMatrix} gives that the evaluation of $(\tilde{S},\mathcal{L}\tilde{S})$ is needed. We first look at $L$ operating on $S$ and then determine the form of these matrix elements.

\begin{align}
\nonumber \mathcal{L}S &= \left(-\frac{1}{2}\nabla_\rho^2 - \nabla_{r_3}^2 - 2\nabla_{r_{12}}^2 + \frac{2}{r_1} - \frac{2}{r_2} - \frac{2}{r_3} - \frac{2}{r_{12}} - \frac{2}{r_{13}} + \frac{2}{r_{23}} - 2 E_T\right) \Phi_{Ps}(r_{12}) \Phi_H(r_3) \frac{\sin(\kappa\rho)}{\kappa\rho} \sqrt{\frac{2\kappa}{4\pi}} \\
&= \left(-\frac{1}{2}\nabla_\rho^2 - \nabla_{r_3}^2 - 2\nabla_{r_{12}}^2 + \frac{2}{r_1} - \frac{2}{r_2} - \frac{2}{r_3} - \frac{2}{r_{12}} - \frac{2}{r_{13}} + \frac{2}{r_{23}} - 2 E_H - 2 E_{Ps} - \frac{1}{2}\kappa^2 \right) \Phi_{Ps}(r_{12}) \Phi_H(r_3) \frac{\sin(\kappa\rho)}{\kappa\rho} \sqrt{\frac{2\kappa}{4\pi}}.
\end{align}

Since $\Phi_H$ and $\frac{\sin(\kappa\rho)}{\kappa\rho}$ are independent of $r_3$, we can use (\ref{eq:HEqn}) to replace the $\left(-\frac{1}{2}\nabla_{r_3}^2 - \frac{1}{r_3}\right)$ with $E_H$, which gives a cancellation with the $-2 E_H$:
\beq
\mathcal{L}S = \left(-\frac{1}{2}\nabla_\rho^2 - 2\nabla_{r_{12}}^2 + \frac{2}{r_1} - \frac{2}{r_2} - \frac{2}{r_{12}} - \frac{2}{r_{13}} + \frac{2}{r_{23}} - 2 E_{Ps} - \frac{1}{2}\kappa^2 \right) \Phi_{Ps}(r_{12}) \Phi_H(r_3) \frac{\sin(\kappa\rho)}{\kappa\rho} \sqrt{\frac{2\kappa}{4\pi}}
\eeq

\noindent We can do the same with (\ref{eq:PsEqn}):
\beq
\mathcal{L}S = \left(-\frac{1}{2}\nabla_\rho^2 + \frac{2}{r_1} - \frac{2}{r_2} - \frac{2}{r_{13}} + \frac{2}{r_{23}} - \frac{1}{2}\kappa^2 \right) \Phi_{Ps}(r_{12}) \Phi_H(r_3) \frac{\sin(\kappa\rho)}{\kappa\rho} \sqrt{\frac{2\kappa}{4\pi}}
\label{LS1}
\eeq

The only part that explicitly depends on $\rho$ is $j_0(\kappa\rho)$.  We can see that $j_0(\kappa\rho)$ is an eigenfunction of $\nabla_\rho^2$:
From the file 'PVR Thesis.nb',
\beq
\nabla_\rho^2 \: j_0(\kappa\rho) = \frac{1}{\rho^2 } \frac{\partial}{\partial\rho} \left[ \rho^2 \frac{\partial}{\partial\rho} \left( \frac{\sin(\kappa\rho)}{\kappa\rho} \right)\right] = -\kappa^2 \, \frac{\sin(\kappa\rho)}{\kappa\rho} = -\kappa^2 j_0(\kappa\rho)
\label{eigenj0}
\eeq

\noindent Similarly,
\beq
\nabla_\rho^2 \: n_0(\kappa\rho) = -\frac{1}{\rho^2 } \frac{\partial}{\partial\rho} \left[ \rho^2 \frac{\partial}{\partial\rho} \left( \frac{\cos(\kappa\rho)}{\kappa\rho} \right)\right] = \kappa^2 \, \frac{\cos(\kappa\rho)}{\kappa\rho} = -\kappa^2 n_0(\kappa\rho)
\label{eigenn0}
\eeq

\noindent So $j_0(\kr)$ is an eigenfunction of $-\nabla_\rho^2$ with eigenvalue $-\kappa^2$.  Now using (\ref{eigenj0}) in (\ref{LS1}) gives
\begin{align}
\label{eq:LS2}
\nonumber \mathcal{L}S &= \left( \frac{2}{r_1} - \frac{2}{r_2} - \frac{2}{r_{13}} + \frac{2}{r_{23}} \right) \Phi_{Ps}(r_{12}) \Phi_H(r_3) \frac{\sin(\kr)}{\kr} \sqrt{\frac{2\kappa}{4\pi}} \\
&= \left( \frac{2}{r_1} - \frac{2}{r_2} - \frac{2}{r_{13}} + \frac{2}{r_{23}} \right) S
\end{align}

\beq
(S,\mathcal{L}S) = \left( \left[ \frac{2}{r_1} - \frac{2}{r_2} - \frac{2}{r_{13}} + \frac{2}{r_{23}} \right] S^2 \right)
\label{eq:SLS}
\eeq

The potential energy terms in brackets are antisymmetric, i.e. replacing $1 \leftrightarrow 2$.  Also notice from (\ref{eq:SCPhiDef}) that S and C are symmetric.  So under the $1 \leftrightarrow 2$ permutation, $(S,\mathcal{L}S)$ and $(C,\mathcal{L}S)$ are antisymmetric.  Thus, we have
\beq
(S,\mathcal{L}S) = 0 \text{ and } (C,\mathcal{L}S) = 0.
\eeq

From (\ref{SCphiBarDef}),
\beq
(\bar{S},\mathcal{L}\bar{S}) = \frac{1}{2} \left((S \pm S^\prime),\mathcal{L}(S \pm S^\prime)\right) = \frac{1}{2} \left[(S,\mathcal{L}S) \pm (S,\mathcal{L}S^\prime) \pm (S^\prime,\mathcal{L}S) + (S^\prime,\mathcal{L} S^\prime) \right]
\eeq

\noindent From the symmetry of these elements (as in (\ref{})),
\beq
(S,\mathcal{L}S) = (S^\prime,\mathcal{L}S^\prime) = 0.
\eeq

\noindent Also,
\beq
(S,\mathcal{L}S^\prime) = (S^\prime,\mathcal{L}S).
\eeq

\beq
(\bar{S},\mathcal{L}\bar{S}) = \pm \left(S^\prime,\mathcal{L}S\right) = \pm \left(S^\prime, \left[ \frac{2}{r_1} - \frac{2}{r_2} - \frac{2}{r_{13}} + \frac{2}{r_{23}} \right] S\right)
\label{eq:SbarLSbar}
\eeq

The same computation applies for $\bar{C}$:
\beq
(\bar{C},\mathcal{L}\bar{S}) = \pm \left(C^\prime,\mathcal{L}S\right) = \pm \left(C^\prime, \left[ \frac{2}{r_1} - \frac{2}{r_2} - \frac{2}{r_{13}} + \frac{2}{r_{23}} \right] S\right)
\label{eq:CbarLSbar}
\eeq

We can use (\ref{eq:SLCandCLS}) to find $(\bar{S},\mathcal{L}\bar{C})$ from this:
\beq
\left(\bar{S},\mathcal{L}\bar{C}\right) = \left(\bar{C},\mathcal{L}\bar{S}\right) + 1
\label{eq:SLCandCLSBar}
\eeq

$\mathcal{L}S'$ is constructed by performing the $2 \leftrightarrow 3$ permutation of (\ref{eq:LS2}):
\beq
\label{eq:LSP2}
\mathcal{L}S' = \left( \frac{2}{r_1} - \frac{2}{r_3} - \frac{2}{r_{12}} + \frac{2}{r_{23}} \right) S'.
\eeq

So $(\bar{\phi}_i, \mathcal{L}\bar{S})$ is a straightforward calculation.
\beq
\label{eq:PhiBarLSBar1}
(\bar{\phi}_i, \mathcal{L}\bar{S}) = \left( (\phi_i \pm \phi_i') L \frac{(S \pm S')}{\sqrt{2}}\right)
= \frac{1}{\sqrt{2}} \left(\phi_i L S \pm \phi_i L S' \pm \phi_i' L S + \phi_i' L S'\right)
\eeq

Again, from the properties of the $P_{23}$ permutation operator,
\beq
(\phi_i,\mathcal{L}S) = (\phi_i',\mathcal{L}S') \text{ and } (\phi_i,\mathcal{L}S') = (\phi_i',\mathcal{L}S).
\eeq

Equation (\ref{eq:PhiBarLSBar1}) becomes
\begin{subequations}
\label{eq:PhiBarLSBar2}
\begin{align}
(\bar{\phi}_i, \mathcal{L}\bar{S}) = \frac{1}{\sqrt{2}} \left[2(\phi_i,\mathcal{L}S) \pm 2(\phi_i',\mathcal{L}S)\right] &= \frac{2}{\sqrt{2}} \left[(\phi_i,\mathcal{L}S) \pm (\phi_i',\mathcal{L}S)\right] \label{eq:PhiBarLSBar2a} \\
 &= \frac{2}{\sqrt{2}} \left[(\phi_i,\mathcal{L}S) \pm (\phi_i,\mathcal{L}S')\right]  \label{eq:PhiBarLSBar2b}
\end{align}
\end{subequations}

Notice that (\ref{eq:PhiBarLSBar2a}) and (\ref{eq:PhiBarLSBar2b}) are equivalent ways of writing this expression.  Either could be used, depending on the form desired for the computation.
From (\ref{eq:LS2}) and (\ref{eq:LSP2}),
\begin{align}
(\bar{\phi}_i, \mathcal{L}\bar{S}) &= \frac{2}{\sqrt{2}} \left[\left( \phi_i \left( \frac{2}{r_1} - \frac{2}{r_2} - \frac{2}{r_{13}} + \frac{2}{r_{23}} \right) S\right) \pm \left( \phi_i' \left( \frac{2}{r_1} - \frac{2}{r_2} - \frac{2}{r_{13}} + \frac{2}{r_{23}} \right) S\right)\right] \\
 &= \frac{2}{\sqrt{2}} \left[\left( \phi_i \left( \frac{2}{r_1} - \frac{2}{r_2} - \frac{2}{r_{13}} + \frac{2}{r_{23}} \right) S\right) \pm \left( \phi_i \left( \frac{2}{r_1} - \frac{2}{r_3} - \frac{2}{r_{12}} + \frac{2}{r_{23}} \right) S'\right)\right]
\end{align}

\noindent We also have that
\beq
(\bar{\phi}_i, \mathcal{L}\bar{S}) = (\bar{S}, \mathcal{L}\bar{\phi}_i).
\label{eq:PhiLSPerm}
\eeq


\subsubsection{Elements with \texorpdfstring{$\mathcal{L}C$}{LC}}
\label{sec:LCElements}
Now we need to do the same analysis for $L\bar{C}$ as we just did for $L\bar{S}$. The shielding factor, $f_{sh}(\rho)$, complicates the derivatives slightly.
\begin{align}
\mathcal{L}C = & \left(-\frac{1}{2}\nabla_\rho^2 - \nabla_{r_3}^2 - 2\nabla_{r_{12}}^2 + \frac{2}{r_1} - \frac{2}{r_2} - \frac{2}{r_3} - \frac{2}{r_{12}} - \frac{2}{r_{13}} + \frac{2}{r_{23}} - 2 E_H - 2 E_{Ps} - \frac{1}{2}\kappa^2 \right) \nonumber \\
 & \times \Phi_{Ps}(r_{12}) \Phi_H(r_3) \frac{\sin(\kappa\rho)}{\kappa\rho} \sqrt{\frac{2\kappa}{4\pi}} \left[1 + e^{-\mu\rho} \left(1 + \frac{\mu}{2} \rho \right) \right]
\label{LC1}
\end{align}

\noindent Again, we use (\ref{eq:HEqn}) and (\ref{eq:PsEqn}) to simplify this expression.
\begin{align}
\mathcal{L}C = & \left(-\frac{1}{2}\nabla_\rho^2 + \frac{2}{r_1} - \frac{2}{r_2} - \frac{2}{r_3} - \frac{2}{r_{12}} - \frac{2}{r_{13}} + \frac{2}{r_{23}}  - \frac{1}{2}\kappa^2\right) \nonumber \\
 & \times \Phi_{Ps}(r_{12}) \Phi_H(r_3) \frac{\sin(\kappa\rho)}{\kappa\rho} \sqrt{\frac{2\kappa}{4\pi}} \left[1 + e^{-\mu\rho} \left(1 + \frac{\mu}{2} \rho \right) \right]
\label{LC2}
\end{align}

From (\ref{eigenn0}), $n_0(\kr)$ is an eigenfunction of $\nabla_\rho^2$ with eigenvalue $-\kappa^2$.  Then $\displaystyle -\frac{1}{2} \nabla_\rho^2 \frac{\cos(\kr)}{\kr}$ could be replaced by $\displaystyle \frac{1}{2}\kappa^2\frac{\cos(\kr)}{\kr}$ if the shielding term could be ignored.  However, it also depends on $\rho$, so we have to calculate this separately.

\begin{align}
\nonumber -\frac{1}{2} \nabla_\rho^2 & \left\lbrace \frac{\cos(\kr)}{\kr} \left[1 - e^{-\mr} \left(1 + \frac{\mu}{2} \rho \right)\right]\right\rbrace \\
\nonumber &= \frac{e^{-\mr}}{2\kappa\rho} \left[ \frac{\mu^3 \rho}{2} \cos(\kr) + \kappa^2 \left(-1 + e^{\mr} -\frac{\mr}{2} \right) \cos(\kr) + \kappa\mu (1+\mr) \sin(\kr) \right] \\
\nonumber &= \frac{e^{-\mr} \mu^3\rho}{4\kr} \cos(\kr) + \frac{\kappa^2}{2} \left[-e^{-\mr} + 1 - \frac{\mr}{2} e^{-\mr}\right] \frac{\cos(\kr)}{\kr} + \frac{e^{-\mr}}{2\kr} \kappa\mu (1 + \mr)\sin(\kr) \\
 &= \frac{e^{-\mr} \mu^3\rho}{4\kr} \cos(\kr) + \frac{\kappa^2}{2} \left[1 - e^{-\mr}\left(1 + \frac{\mu}{2} \rho \right) \right] \frac{\cos(\kr)}{\kr} + \frac{e^{-\mr}}{2\kr} \kappa\mu (1 + \mr)\sin(\kr)
\label{laplacianshield}
\end{align}

\noindent The second term here cancels the $\displaystyle \frac{1}{2}\kappa^2$ in parentheses in (\ref{LC2}).  Now we have
\begin{align}
\mathcal{L}C = \: & \Phi_{Ps}(r_{12}) \Phi_H(r_3) \sqrt{\frac{2\kappa}{4\pi}} \left\{ \left(\frac{2}{r_1} - \frac{2}{r_2} - \frac{2}{r_{13}} + \frac{2}{r_{23}} \right) \frac{\cos(\kr)}{\kr} \left[1 - e^{-\mr} \left(1 + \frac{\mu}{2}\rho \right) \right] \right. \nonumber\\
& + \left. \frac{e^{-\mr} \mu^3 \rho}{4} \frac{\cos(\kr)}{\kr} + \frac{e^{-\mr}}{2} \kappa \mu (1+\mu\rho) \frac{\sin(\kr)}{\kr} \right\}
\label{eq:LC}
\end{align}

To find $\mathcal{L}\bar{C}$, we also have to analyze $\mathcal{L}C^\prime$.  From (\ref{eq:SCprime}) and (\ref{eq:SCPhiDef}),
\beq
C^\prime = P_{23} C = -Y_{0,0}(\theta_{\rho^\prime}, \varphi_{\rho^\prime}) \Phi_{Ps}(r_{13}) \Phi_H(r_2) \sqrt{2\kappa} n_0(\kr) \left[1 - e^{-\mu\rho^\prime}\left(1 + \frac{\mu}{2} \rho^\prime \right) \right].
\eeq

\noindent The calculation is exactly the same as that for $LC$, except $\rho\to\rho^\prime$ and $2\leftrightarrow3$.

\begin{align}
\mathcal{L}C^\prime = & \left(-\frac{1}{2}\nabla_{\rhop}^2 - \nabla_{r_2}^2 - 2\nabla_{r_{13}}^2 + \frac{2}{r_1} - \frac{2}{r_2} - \frac{2}{r_3} - \frac{2}{r_{12}} - \frac{2}{r_{13}} + \frac{2}{r_{23}} - 2 E_H - 2 E_{Ps} - \frac{1}{2}\kappa^2 \right) \nonumber \\
 & \times \Phi_{Ps}(r_{13}) \Phi_H(r_2) \frac{\sin(\kappa\rhop)}{\kappa\rhop} \sqrt{\frac{2\kappa}{4\pi}} \left[1 + e^{-\mu\rhop} \left(1 + \frac{\mu}{2} \rhop \right) \right]
\label{LCP1}
\end{align}

We can modify (\ref{eq:HEqn}) and (\ref{eq:PsEqn}) to simplify this:
\begin{align}
&\left(-\frac{1}{2}\nabla_{r_2}^2 - \frac{1}{r_2}\right) \Phi_H(r_2) = E_H \Phi_H(r_2) \text{ and} \\
&\left(-\nabla_{r_{13}}^2 - \frac{1}{r_{13}}\right) \Phi_{Ps}(r_{13}) = E_{Ps} \Phi_{Ps}(r_{13}).
\end{align}

\beq
\mathcal{L}C^\prime = \left(-\frac{1}{2}\nabla_{\rhop}^2 + \frac{2}{r_1} - \frac{2}{r_3} - \frac{2}{r_{12}}  + \frac{2}{r_{23}} - \frac{1}{2}\kappa^2 \right) \Phi_{Ps}(r_{13}) \Phi_H(r_2) \frac{\sin(\kappa\rhop)}{\kappa\rhop} \sqrt{\frac{2\kappa}{4\pi}} \left[1 + e^{-\mu\rhop} \left(1 + \frac{\mu}{2} \rhop \right) \right]
\label{eq:LCP2}
\eeq

Modifying the result from (\ref{laplacianshield}) to use $\rhop$ instead of $\rho$ yields
\begin{align}
\nonumber -\frac{1}{2} \nabla_\rhop^2 & \left\lbrace \frac{\cos(\krp)}{\krp} \left[1 - e^{-\mrp} \left(1 + \frac{\mu}{2} \rhop \right)\right]\right\rbrace \\
 &= \frac{e^{-\mrp} \mu^3\rhop}{4\kr} \cos(\krp) + \frac{\kappa^2}{2} \left[1 - e^{-\mrp}\left(1 + \frac{\mu}{2} \rhop \right) \right] \frac{\cos(\krp)}{\krp} + \frac{e^{-\mrp}}{2\krp} \kappa\mu (1 + \mrp)\sin(\krp)
\end{align}

\noindent Similar to (\ref{eq:LC}), (\ref{eq:LCP2}) becomes
\begin{align}
\mathcal{L}C^\prime = \: & \Phi_{Ps}(r_{13}) \Phi_H(r_2) \sqrt{\frac{2\kappa}{4\pi}} \left\{ \left(\frac{2}{r_1} - \frac{2}{r_3} - \frac{2}{r_{12}} + \frac{2}{r_{23}} \right) \frac{\cos(\krp)}{\krp} \left[1 - e^{-\mrp} \left(1 + \frac{\mu}{2}\rhop \right) \right] \right. \nonumber\\
& + \left. \frac{e^{-\mrp} \mu^3 \rhop}{4} \frac{\cos(\krp)}{\krp} + \frac{e^{-\mrp}}{2} \kappa \mu (1+\mu\rhop) \frac{\sin(\krp)}{\krp} \right\}
\label{eq:LCP}
\end{align}

\noindent Combining (\ref{eq:LC}) and (\ref{eq:LCP}) gives
\begin{align}
\mathcal{L}\bar{C} = \:\: &\frac{1}{\sqrt{2}} \mathcal{L}(C \pm C^\prime) = \frac{1}{\sqrt{2}} (\mathcal{L}C \pm \mathcal{L}C^\prime) \nonumber \\
= \:\: &\frac{1}{\sqrt{8\pi}} \Phi_{Ps}(r_{12}) \Phi_H(r_3) \sqrt{2\kappa} \nonumber  \\
&\times \left\{ \frac{\kappa\mu}{2} e^{-\mr} (1+\mr) \frac{\sin(\kr)}{\kr} + \frac{\mu^3 \rho}{4} e^{-\mr} \frac{\cos(\kr)}{\kr} \right. \nonumber \\
&+ \left. \left(\frac{2}{r_1} - \frac{2}{r_2} - \frac{2}{r_{13}} + \frac{2}{r_{23}}\right) \frac{\cos(\kr)}{\kr} \left[1 - e^{-\mr} \left(1 + \frac{\mu}{2}\rho\right)\right] \right\} \nonumber \\
\pm &\frac{1}{\sqrt{8\pi}} \Phi_{Ps}(r_{13}) \Phi_H(r_2) \sqrt{2\kappa} \nonumber  \\
&\times \left\{ \frac{\kappa\mu}{2} e^{-\mrp} (1+\mrp) \frac{\sin(\krp)}{\krp} + \frac{\mu^3 \rhop}{4} e^{-\mrp} \frac{\cos(\krp)}{\krp} \right. \nonumber \\
&+ \left. \left(\frac{2}{r_1} - \frac{2}{r_3} - \frac{2}{r_{12}} + \frac{2}{r_{23}}\right) \frac{\cos(\krp)}{\krp} \left[1 - e^{-\mrp} \left(1 + \frac{\mu}{2}\rhop\right)\right] \right\}
\label{LCBar}
\end{align}

The properties of the permutation operator give that
\beq
(C,\mathcal{L}C) = (C',\mathcal{L}C') \text{ and } (C,\mathcal{L}C') = (C',\mathcal{L}C)
\eeq

Using these two relations gives that
\begin{align}
\left(\bar{C},\mathcal{L}\bar{C}\right) &= \frac{1}{2}\left(\left(C \pm C'\right),\mathcal{L}(C \pm C')\right) \nonumber\\
 &= \frac{1}{2}\left[(C,\mathcal{L}C) \pm (C,\mathcal{L}C') \pm (C',\mathcal{L}C) + (C',\mathcal{L}C')\right] = \frac{1}{2}\left[2(C,\mathcal{L}C) \pm 2(C',\mathcal{L}C)\right] \nonumber\\
 &= (C,\mathcal{L}C) \pm (C',\mathcal{L}C).
 \label{CLC1}
\end{align}

Substitute (\ref{eq:CDef}) and (\ref{eq:LC}) in (\ref{CLC1}) to get
\begin{align}
\left(\bar{C},\mathcal{L}\bar{C}\right) = &(C,\mathcal{L}C) \pm (C',\mathcal{L}C) = \left((C \pm C'),\mathcal{L}C\right) \nonumber\\
 = &\left((C \pm C') \Phi_{Ps}(r_{12}) \Phi_H(r_3) \sqrt{\frac{2\kappa}{4\pi}} \left\{ \left(\frac{2}{r_1} - \frac{2}{r_2} - \frac{2}{r_{13}} + \frac{2}{r_{23}} \right) \frac{\cos(\kr)}{\kr} \left[1 - e^{-\mr}\left(1 + \frac{\mu}{2}\rho \right) \right] \right.\right. \\
   &+ \left.\left. \frac{e^{-\mr} \mu^3 \rho}{4} \frac{\cos(\kr)}{\kr} + \frac{e^{-\mr}}{2} \kappa\mu (1+\mr) \frac{\sin(\kr)}{\kr} \right\} \right).
 \label{CBarLCBar1}
\end{align}

These can be split up to simplify:
\begin{align}
\left(\bar{C},\mathcal{L}\bar{C}\right) = &\sqrt{\frac{2\kappa}{4\pi}} \frac{\kappa\mu}{2} \left((C \pm C') e^{-\mr}(1+\mr) \frac{\sin(\kr)}{\kr} \Phi_{Ps}(r_{12}) \Phi_H(r_3) \right)  \nonumber\\
 &+ \sqrt{\frac{2\kappa}{4\pi}} \frac{\mu^3}{4} \left((C \pm C') e^{-\mr} \frac{\cos(\kr)}{\kr} \Phi_{Ps}(r_{12}) \Phi_H(r_3) \right) \nonumber\\
 &+ \sqrt{\frac{2\kappa}{4\pi}} \left((C \pm C') \left(\frac{2}{r_1} - \frac{2}{r_3} - \frac{2}{r_{12}} + \frac{2}{r_{23}}\right) \frac{\cos(\kr)}{\kr} \left[1 - e^{-\mr}\left(1 + \frac{\mu}{2}\rho \right)\right] \Phi_{Ps}(r_{12}) \Phi_H(r_3) \right) \nonumber\\
= &\frac{\kappa\mu}{2} \left((C \pm C') e^{-\mr}(1+\mr) S\right) \nonumber\\
 &+ \sqrt{\frac{2\kappa}{4\pi}} \frac{\mu^3}{4} \left((C \pm C') e^{-\mr} \frac{\cos(\kr)}{\kr} \Phi_{Ps}(r_{12}) \Phi_H(r_3) \right) \nonumber\\
 &+ \left((C \pm C') \left(\frac{2}{r_1} - \frac{2}{r_3} - \frac{2}{r_{12}} + \frac{2}{r_{23}}\right) C \right).
 \label{CBarLCBar2}
\end{align}

Looking at just the last term:
\beq
\left((C \pm C') \left(\frac{2}{r_1} - \frac{2}{r_3} - \frac{2}{r_{12}} + \frac{2}{r_{23}}\right) C \right) = \left( \left(\frac{2}{r_1} - \frac{2}{r_3} - \frac{2}{r_{12}} + \frac{2}{r_{23}}\right) C^2 \right) \pm \left( \left(\frac{2}{r_1} - \frac{2}{r_3} - \frac{2}{r_{12}} + \frac{2}{r_{23}}\right) C'C \right)
\eeq

\noindent The first set of parentheses has the same form as (\ref{eq:SLS}).  These terms in parentheses are antisymmetric with respect to the $1 \leftrightarrow 2$ permutation.  Also, $C$ is symmetric with respect to this permutation.  So the first set of parentheses is 0.  Thus,
\beq
\left((C \pm C') \left(\frac{2}{r_1} - \frac{2}{r_3} - \frac{2}{r_{12}} + \frac{2}{r_{23}}\right) C \right) = \pm \left(C' \left(\frac{2}{r_1} - \frac{2}{r_3} - \frac{2}{r_{12}} + \frac{2}{r_{23}}\right) C \right)
\eeq

\noindent We finally have that
\begin{align}
\left(\bar{C},\mathcal{L}\bar{C}\right) = \: &\frac{\kappa\mu}{2} \left((C \pm C') e^{-\mr}(1+\mr) S\right) \nonumber\\
 &+ \sqrt{\frac{2\kappa}{4\pi}} \frac{\mu^3}{4} \left((C \pm C') e^{-\mr} \frac{\cos(\kr)}{\kr} \Phi_{Ps}(r_{12}) \Phi_H(r_3) \right) \nonumber\\
 &\pm \left(C' \left(\frac{2}{r_1} - \frac{2}{r_3} - \frac{2}{r_{12}} + \frac{2}{r_{23}}\right) C \right).
 \label{CBarLCBar}
\end{align}

\textbf{@TODO:} Put this elsewhere.
We chose to absorb both the $\frac{1}{\sqrt{2}}$ and $Y_{0,0}(\theta_\rho,\varphi_\rho) = \frac{1}{\sqrt{4\pi}}$ into the $c_i$ constants of the short-range terms (\ref{eq:PhiDef}).


The analysis for $(\bar{\phi}_i, \mathcal{L}\bar{C})$ and $(\bar{C}, L\bar{\phi}_i)$ is similar to the analysis in section \ref{sec:LSElements}.
\begin{subequations}
\begin{align}
(\bar{\phi}_i, \mathcal{L}\bar{C}) &= \frac{2}{\sqrt{2}} \left[(\phi_i,\mathcal{L}C) \pm (\phi_i',\mathcal{L}C)\right] \label{PhiBarLCBar2a} \\
 &= \frac{2}{\sqrt{2}} \left[(\phi_i,\mathcal{L}C) \pm (\phi_i,\mathcal{L}C')\right]  \label{PhiBarLCBar2b}
\end{align}
\end{subequations}

Using (\ref{eq:LC}) and (\ref{eq:LCP}) in the above gives our final result for $(\bar{\phi}_i, \mathcal{L}\bar{C})$.
\begin{subequations}
\begin{align}
(\bar{\phi}_i, \mathcal{L}\bar{C}) = \sqrt{\frac{\kappa}{\pi}} &\left( (\phi_i \pm \phi_i') \Phi_{Ps}(r_{12}) \Phi_H(r_3) \left\{ \left(\frac{2}{r_1} - \frac{2}{r_2} - \frac{2}{r_{13}} + \frac{2}{r_{23}}\right) \frac{\cos(\kr)}{\kr} \left[1 - e^{-\mr}\left(1 + \frac{\mu}{2}\rho\right)\right] \right.\right. \nonumber\\
&+ \left.\left.\frac{e^{-\mr}\mu^3\rho}{4} \frac{\cos(\kr)}{\kr} + \frac{e^{-\mr}}{2} \kappa\mu (1+\mr) \frac{\sin(\kr)}{\kr}  \right\}\right) \\
= \sqrt{\frac{\kappa}{\pi}} &\left( \phi_i \Phi_{Ps}(r_{12}) \Phi_H(r_3) \left\{ \left(\frac{2}{r_1} - \frac{2}{r_2} - \frac{2}{r_{13}} + \frac{2}{r_{23}}\right) \frac{\cos(\kr)}{\kr} \left[1 - e^{-\mr}\left(1 + \frac{\mu}{2}\rho\right)\right] \right.\right. \nonumber\\
&+ \left.\left.\frac{e^{-\mr}\mu^3\rho}{4} \frac{\cos(\kr)}{\kr} + \frac{e^{-\mr}}{2} \kappa\mu (1+\mr) \frac{\sin(\kr)}{\kr}  \right\}\right) \nonumber\\
\pm \sqrt{\frac{\kappa}{\pi}} &\left( \phi_i \Phi_{Ps}(r_{12}) \Phi_H(r_3) \left\{ \left(\frac{2}{r_1} - \frac{2}{r_3} - \frac{2}{r_{12}} + \frac{2}{r_{23}}\right) \frac{\cos(\krp)}{\krp} \left[1 - e^{-\mrp}\left(1 + \frac{\mu}{2}\rhop\right)\right] \right.\right. \nonumber\\
&+ \left.\left.\frac{e^{-\mr}\mu^3\rhop}{4} \frac{\cos(\krp)}{\krp} + \frac{e^{-\mrp}}{2} \kappa\mu (1+\mrp) \frac{\sin(\krp)}{\krp}  \right\}\right)
\end{align}
\end{subequations}

Similar to (\ref{eq:PhiLSPerm}), we have
\beq
(\bar{\phi}_i, \mathcal{L}\bar{C}) = (\bar{C}, L\bar{\phi}_i).
\label{eq:PhiLCPerm}
\eeq

\subsection{Matrix Elements Involving Only Short-Range Terms}
\label{sec:MatrixShort}
The short-range -- short-range interactions are calculated in a separate document ``3.24 Using Spherical Coordinates''.

\beq
\label{eq:SWaveShortShort}
\left(\bar{\phi}_i, L \bar{\phi}_j\right) = \int \left[ \sum_{l=1}^3 \boldsymbol{\nabla}_{\!\mathbf{r}_l} \bar{\phi}_i \boldsymbol{\cdot} \boldsymbol{\nabla}_{\!\mathbf{r}_l} \bar{\phi}_j + \left( \frac{2}{r_1} - \frac{2}{r_2} - \frac{2}{r_3} - \frac{2}{r_{12}} - \frac{2}{r_{13}} + \frac{2}{r_{23}} - 2 E_T \right) \bar{\phi}_i \bar{\phi}_j \right] d\tau.
\eeq

Lastly, we need to prove that $\left(\bar{\phi}_i, L \bar{\phi}_j\right) = \left(\bar{\phi}_j, L \bar{\phi}_i\right)$.  Consider the functional
\beq
F = (\phi_i, L \phi_j) - (\phi_j L \phi_i).
\eeq

Using (\ref{eq:Hamiltonian1}),
\begin{align}
	F=&\left({-\phi_i,{\frac {1}{2}{\nabla }_{{r}_{1}}^{2}\phi_j}}\right)+\left({\phi_j,{\frac {1}{2}{\nabla }_{{r}_{1}}^{2}\phi_i}}\right)+
	\left({-\phi_i,{\frac {1}{2}{\nabla }_{{r}_{2}}^{2}\phi_j}}\right)+\left({\phi_j,{\frac {1}{2}{\nabla }_{{r}_{2}}^{2}\phi_i}}\right) \nonumber \\
	&+\left({-\phi_i,{\frac {1}{2}{\nabla }_{{r}_{3}}^{2}\phi_j}}\right)+\left({\phi_j,{\frac {1}{2}{\nabla }_{{r}_{3}}^{2}\phi_i}}\right).
\end{align}

From (\ref{eq:PhiDef}), the $\bar{\phi}_i$ (and $\bar{\phi}_j$) have a decaying exponential dependence in $r_1$, $r_2$ and $r_3$, so the surface terms in the integrations will vanish.  Combining this fact with (\ref{PhiBarDef}) gives that $F = 0$.  Then
\beq
\left(\bar{\phi}_i, \mathcal{L} \bar{\phi}_j\right) = \left(\bar{\phi}_j, \mathcal{L} \bar{\phi}_i\right).
\label{PhiLPhiPerm}
\eeq

The claims in (\ref{eq:PhiLSPerm}) and (\ref{eq:PhiLCPerm}) are proven using the same method.  Both (\ref{eq:PhiLSPerm}) and (\ref{eq:PhiLCPerm}) include $\bar{\phi}_i$, so these terms do have a decaying exponential dependence, making the claims true.

%(\bar{\phi}_i, \mathcal{L}\bar{C}) = \frac{\sqrt{2\kappa}}{4\pi} &\left( (\phi_i \pm \phi_i') \Phi_{Ps}(r_{12}) \Phi_H(r_3) \left\{ \left(\frac{2}{r_1} - \frac{2}{r_2} - \frac{2}{r_{13}} + \frac{2}{r_{23}}\right) \frac{\cos(\kr)}{\kr} \left[1 - e^{-\mr}\left(1 + \frac{\mu}{2}\rho\right)\right] \right.\right. \nonumber\\
%&+ \frac{e^{-\mr}\mu^3\rho}{4} \frac{\cos(\kr)}{\kr} + \frac{e^{-\mr}}{2} \kappa\mu (1+\mr) \frac{\sin(\kr)} \left.\left. \right\}\right)


\subsection{Matrix Element Symmetry}
The claims in equations \ref{eq:PhiLSPerm} and \ref{eq:PhiLCPerm} can be proven by starting with the functional
\begin{align}
	F \equiv \left(g,\mathcal{L}f \right)-\left(f,\mathcal{L}g \right).
\end{align}

Only the first three terms of the above functional have to be evaluated, as the other terms go to 0 with the subtraction.
\begin{align}
	F=&\left({-g,\frac{1}{2} \nabla_{r_1}^2 f}\right)+\left({f,\frac{1}{2} \nabla_{r_1}^2 g}\right)+
	\left({-g,\frac{1}{2} \nabla_{r_2}^2 f}\right)+\left({f,\frac{1}{2} \nabla_{r_2}^2 g}\right) \nonumber \\
	&+\left({-g,\frac{1}{2} \nabla_{r_3}^2 f}\right)+\left({f,\frac{1}{2} \nabla_{r_3}^2 g}\right)
\end{align}

We can use Green's theorem on each pair of terms:

\todoi{Is it really only 2 volumes, not 3?}

\begin{align}
	{\int _{V_2}{{\int_{V_1}{\left[{u{\nabla }_1^2 v-v{\nabla }_1^2 u}\right]{d \tau_1}}}{d \tau_2}}}
	= \int_{V_2} \int_{S_1} \left[u \vec{\nabla}_1 v - v \vec{\nabla}_1 u \right] \cdot d\vec{\sigma}_1 d\tau_2
\end{align}

\textbf{@TODO:} These do not immediately follow from the above line. Expand this.
\begin{subequations}
\label{eq:elem_symm}
\begin{align}
\left(\bar{\phi}_i, \mathcal{L} \bar{\phi}_j \right) &= \left(\bar{\phi}_j, \mathcal{L} \bar{\phi}_i \right) \\
\left(\bar{\phi}_i, \mathcal{L} \bar{S} \right) &= \left(\bar{S}, \mathcal{L} \bar{\phi}_i \right) \label{eq:SLPhi} \\
\left(\bar{\phi}_i, \mathcal{L} \bar{C} \right) &= \left(\bar{C}, \mathcal{L} \bar{\phi}_i \right) \label{eq:CLPhi}
\end{align}
\end{subequations}

\noindent Equations \ref{eq:SLPhi} and \ref{eq:CLPhi} allow us to avoid having to operate $\mathcal{L}$ on the $\phi$ terms. From equation \ref{}, the form of $\mathcal{L}\phi_i$ is very complicated, so our 

We just need to consider the long-range--long-range matrix elements. Examining the functional
 $F = \left(\bar{S}, \mathcal{L}\bar{C}\right) - \left(\bar{C}, \mathcal{L}\bar{S}\right)$:
\begin{equation*}
F = \left(\frac{1}{\sqrt{2}} \left[S \pm S^\prime \right], \mathcal{L} \frac{1}{\sqrt{2}}\left[C \pm C^\prime \right] \right) -
    \left(\frac{1}{\sqrt{2}} \left[C \pm C^\prime \right], \mathcal{L} \frac{1}{\sqrt{2}}\left[S \pm S^\prime \right] \right)
\end{equation*}
\begin{equation}
= \frac{1}{2}\left[(S,\mathcal{L}C) \pm (S,\mathcal{L}C^\prime) \pm (S^\prime,\mathcal{L}C) + (S^\prime,\mathcal{L}C^\prime) - (C,\mathcal{L}S) \mp (C,\mathcal{L}S^\prime) \mp (C^\prime,\mathcal{L}S) - (C^\prime,\mathcal{L}S^\prime)\right]
\end{equation}

From (\ref{}), $(S,\mathcal{L}C) = (S^\prime,\mathcal{L}C^\prime)$, $(S^\prime,\mathcal{L}C) = (S,\mathcal{L}C^\prime)$, $(C,\mathcal{L}S) = (S^\prime,\mathcal{L}C^\prime)$ and $(C^\prime,\mathcal{L}S) = (C,\mathcal{L}S^\prime)$, causing the above to reduce to
\beqs
F = \left[ (S,\mathcal{L}C) - (C,\mathcal{L}S)\right] \pm \left[ (S,\mathcal{L}C^\prime) - (C^\prime,\mathcal{L}S)\right].
\eeqs
\beq
\equiv G \pm G^\prime
\label{GBarDef}
\eeq

By Green's theorem,
\begin{align}
\nonumber
G = &\int\limits_{V_3} \int\limits_{V_{12}} \int\limits_{S_\rho} \left[ S \frac{\nabla_\rho}{2} C - C \frac{\nabla}{2} S\right] \cdot d\vec{\sigma}_\rho d\tau_{12} d\tau_3
  + \int\limits_{V_\rho} \int\limits_{V_{12}} \int\limits_{S_3} \left[ S \nabla_{r_3} C - C \nabla_{r_3} S\right] \cdot d\vec{\sigma}_3 d\tau_{12} d\tau_{\rho} \\
  + &\int\limits_{V_\rho} \int\limits_{V_{13}}\int\limits_{S_{12}} \left[ S \: 2 \nabla_{r_{12}} C - C \: 2 \nabla_{r_{12}} S\right] \cdot d\vec{\sigma}_{12} d\tau_{13} d\tau_\rho
  \label{GDef}
\end{align}
The $\frac{1}{2}$ in the first term, the 1 in the second and the 2 in the third are from the appropriate Hamiltonian (\ref{Hamiltonian2}).
The positronium and hydrogen functions have an exponential dependence on $r_3$ and $r_{12}$, respectively.  So the second term goes to 0 due to the $r_3$ dependence, and so does the third term ($r_{12}$ dependence).

The surface elements under consideration are normal to $\hat{\rho}$, so we can ignore the angular dependence in $\vec{\nabla}_\rho$.  Then (\ref{Gdef}) becomes
\beq
G = \frac{1}{2} \int\limits_{V_3} \int\limits_{V_{12}} \left[\int\limits_{S_\rho} \left(C \frac{\partial S}{\partial \rho} - S \frac{\partial C}{\partial \rho} \right) \rho^2 \sin \theta_\rho d\theta_\rho d\varphi_\rho \right] d\tau_{12} d\tau_3
\label{GDef2}
\eeq

Using (\ref{eq:SCPhiDef}), we have
\begin{align}
\nonumber\frac{\partial S}{\partial \rho} & \asymplim{\rho} Y_{0,0} \left(\theta_\rho,\varphi_\rho \right) \Phi_{Ps}(r_{12}) \Phi_{H}(r_{3}) \sqrt{2\kappa} \frac{\partial}{\partial\rho} \frac{\sin(\kappa\rho)}{\kappa\rho} \\
\nonumber & \;\: = \frac{1}{\sqrt{4\pi}} \Phi_{Ps}(r_{12}) \Phi_{H}(r_{3}) \sqrt{2\kappa} \left(\frac{\cos(\kappa\rho)}{\rho} - \frac{\sin(\kappa\rho)}{\kappa\rho^2}\right) \\
& \;\: =\frac{1}{\sqrt{4\pi}} \Phi_{Ps}(r_{12}) \Phi_{H}(r_{3}) \sqrt{2\kappa} \left(\frac{\kappa\rho\cos(\kappa\rho) - \sin(\kappa\rho)}{\kappa\rho^2} \right)
\label{SPartial}
\end{align}

For $\displaystyle \frac{\partial C}{\partial \rho}$, notice that the shielding factor $(1-e^{-\mu\rho})(1+\frac{\mu}{2}\rho)$ goes to 1 as $\rho\to\infty$.
\begin{align}
\nonumber\frac{\partial S}{\partial \rho} & \asymplim{\rho} Y_{0,0} \left(\theta_\rho,\varphi_\rho \right) \Phi_{Ps}(r_{12}) \Phi_{H}(r_{3}) \sqrt{2\kappa} \frac{\partial}{\partial\rho} \frac{\cos(\kappa\rho)}{\kappa\rho} \\
\nonumber & \;\: = \frac{1}{\sqrt{4\pi}} \Phi_{Ps}(r_{12}) \Phi_{H}(r_{3}) \sqrt{2\kappa} \left(\frac{\sin(\kappa\rho)}{\rho} + \frac{\cos(\kappa\rho)}{\kappa\rho^2}\right) \\
& \;\: =\frac{1}{\sqrt{4\pi}} \Phi_{Ps}(r_{12}) \Phi_{H}(r_{3}) \sqrt{2\kappa} \left(\frac{\kappa\rho\sin(\kappa\rho) + \cos(\kappa\rho)}{\kappa\rho^2} \right)
\label{CPartial}
\end{align}

Substituting (\ref{SPartial}) and (\ref{CPartial}) into the parentheses of (\ref{GDef2}) yields
\begin{align}
\nonumber & \left(C \frac{\partial S}{\partial \rho} - S \frac{\partial C}{\partial \rho} \right) \\
\nonumber & \asymplim{\rho} \frac{1}{4\pi} \Phi_{Ps}(r_{12})^2 \Phi_{H}(r_{3})^2 \:2 \kappa
\left[\frac{\cos(\kappa\rho)}{\kappa\rho}\left(\frac{\kappa\rho\cos(\kappa\rho) - \sin(\kappa\rho)}{\kappa\rho^2}\right)
+ \frac{\sin(\kappa\rho)}{\kappa\rho}\left(\frac{\kappa\rho\sin(\kappa\rho) + \cos(\kappa\rho)}{\kappa\rho^2}\right) \right] \\
\nonumber & \;\: = \Phi_{Ps}(r_{12})^2 \Phi_{H}(r_{3})^2 \left[\frac{\kappa \rho \cos^2(\kappa\rho) - \cos(\kappa\rho)\sin(\kappa\rho) + \kappa\rho \sin^2(\kappa\rho) + \sin(\kappa\rho)\cos(\kappa\rho)}{2 \pi \kappa \rho^3} \right] \\
& \;\: = \frac{\Phi_{Ps}(r_{12})^2 \Phi_{H}(r_{3})^2}{2\pi\rho^2}
\end{align}

Substituting this into equation (\ref{GDef2}) gives
\begin{align}
\nonumber G &= \frac{1}{4\pi} \int\limits_{V_3} \int\limits_{V_{12}} \left[ \int\limits_{S_\rho} \Phi_{Ps}(r_{12})^2 \Phi_{H}(r_{3})^2 \rho^2 \sin \theta_\rho d\theta_\rho d\varphi_\rho \right] d\tau_{12} d\tau_3 \\
&= \frac{1}{4\pi} \int\limits_{V_3} \int\limits_{V_{12}} \Phi_{Ps}(r_{12})^2 \Phi_{H}(r_{3})^2 \left[ \int\limits_{S_\rho} \rho^2 \sin \theta_\rho d\theta_\rho d\varphi_\rho \right] d\tau_{12} d\tau_3
\end{align}

Performing the integration in brackets simply yields $4\pi$, so
\beq
G = \int\limits_{V_3} \int\limits_{V_{12}} \Phi_{Ps}(r_{12})^2 \Phi_{H}(r_{3})^2 d\tau_{12} d\tau_3
\eeq

Since $\Phi_{Ps}(r_{12})$ is only dependent on $r_{12}$ and $\Phi_{H}(r_{3})$ is only dependent on $r_3$, plus the fact that they are both orthonormal, the integration gives 1.  So we finally have
\beq
G = 1.
\eeq

From (\ref{GBarDef}), $G = (S,\mathcal{L}C) - (C,\mathcal{L}S)$, so
\beq
(S,\mathcal{L}C) = (C,\mathcal{L}S) + 1.
\label{eq:SLCandCLS}
\eeq

\noindent Writing this in terms of $\bar{S}$ and $\bar{C}$, we have the final relation of
\beq
\left(\bar{S},\mathcal{L}\bar{C}\right) = \left(\bar{C},\mathcal{L}\bar{S}\right) + 1.
\eeq



\subsection{General \texorpdfstring{$(\tilde{S}\mathcal{L}\tilde{C})$ and $(\tilde{C}\mathcal{L}\tilde{S})$}{SLC and CLS} Relation}
\label{sec:GenSLCandCLS}

\begin{align}
\nonumber (\tilde{S},\mathcal{L}\tilde{C}) &= \left((u_{00}\bar{S} + u_{01}\bar{C}),\mathcal{L}(u_{10}\bar{S} + u_{11}\bar{C})\right) \\
&= u_{00} u_{10} (\bar{S},\mathcal{L}\bar{S}) + u_{00} u_{11} (\bar{S},\mathcal{L}\bar{C}) + u_{01} u_{10} (\bar{C},\mathcal{L}\bar{S}) + u_{01} u_{11} (\bar{C},\mathcal{L}\bar{C})
\label{eq:GenSLC}
\end{align}

Likewise,
\begin{align}
\nonumber (\tilde{C},\mathcal{L}\tilde{S}) &= \left((u_{10}\bar{S} + u_{11}\bar{C}),\mathcal{L}(u_{00}\bar{S} + u_{01}\bar{C})\right) \\
&= u_{10} u_{00} (\bar{S},\mathcal{L}\bar{S}) + u_{10} u_{01} (\bar{S},\mathcal{L}\bar{C}) + u_{11} u_{00} (\bar{C},\mathcal{L}\bar{S}) + u_{11} u_{01} (\bar{C},\mathcal{L}\bar{C})
\label{eq:GenCLS}
\end{align}

\noindent Combining equations \ref{eq:GenSLC} and \ref{eq:GenCLS} gives
\begin{align}
\nonumber (\tilde{S},\mathcal{L}\tilde{C}) - (\tilde{C},\mathcal{L}\tilde{S}) = \,\, &[u_{00} u_{10} - u_{10} u_{00}] (\bar{S},\mathcal{L}\bar{S}) + [u_{00} u_{11} - u_{10} u_{01}] (\bar{S},\mathcal{L}\bar{C}) \\
\nonumber + &[u_{01} u_{10} - u_{11} u_{00}] (\bar{C},\mathcal{L}\bar{S}) + [u_{01} u_{11} - u_{11} u_{01}] (\bar{C},\mathcal{L}\bar{C}) \\
\nonumber = \,\, &[u_{00} u_{11} - u_{10} u_{01}] [(\bar{S},\mathcal{L}\bar{C}) - (\bar{C},\mathcal{L}\bar{S})] \\
\nonumber = \,\, & \det(u) [(\bar{S},\mathcal{L}\bar{C}) - (\bar{C},\mathcal{L}\bar{S})]
\end{align}

\noindent Each of our $u$ matrices satisfies $\det(u) = 1$, making this
\beq
(\tilde{S},\mathcal{L}\tilde{C}) - (\tilde{C},\mathcal{L}\tilde{S}) = (\bar{S},\mathcal{L}\bar{C}) - (\bar{C},\mathcal{L}\bar{S}).
\eeq

\noindent This can finally be written as the general form of equation \ref{eq:SLCandCLS}, giving
\beq
(\tilde{S},\mathcal{L}\tilde{C}) = (\tilde{C},\mathcal{L}\tilde{S}) + 1.
\label{eq:GenSLCandCLS}
\eeq

\subsection{Summary of Matrix Elements}
\label{SWaveMatrixSummary}
This section 



\section{Results}

\subsection{Singlet}
All runs here were performed using Peter's set, with an extra 10 points in all coordinates, except for the 2nd, 3rd and 4th (which gained an extra 5 points instead).  The phase shift is given at the last term.  The value in brackets after the extrapolated phase shift is the starting value of $\omega$ for the extrapolation.

The following table shows some sample runs for one value of $\kappa$, $0.1$.  The extrapolation becomes less stable for higher $\kappa$, as is discussed in section (\ref{sec:Extrapolation}).  The highlighted line shows the choice of parameters that are used, since this leads to a more well-converged set of results.

\begin{center}
\begin{tabular}{|c|c|c|c|c|c|c|}
\hline
Description & $\alpha$, $\beta$ and $\gamma$ & $\omega$ & $\kappa$ & Terms & Phase Shift & Extrapolated\\
\hline
\rowcolor{LightCyan} Double, $10^{-6}$, with asymptotic expansion & 0.5, 0.6, 1.1 & 7 & 0.1 & 1450 & -0.426666 & -0.4254 (3) \\
& & & & & & -0.4257 [4] \\
Quadruple, $10^{-6}$, with asymptotic expansion & 0.5, 0.6, 1.1 & 7 & 0.1 & 1595 & -0.42647 & -0.42517 (3) \\
Double, $10^{-6}$, with asymptotic expansion & 0.5, 0.6, 1.1 & 8 & 0.1 & 1614 & -0.426646 & -0.4254 (3) \\
Double, $10^{-6}$, with asymptotic expansion & 0.6, 0.6, 1.0 & 8 & 0.1 & 1925 & -0.42662 & -0.4258 (3) \\
& & & & & & -0.4259 (4) \\
& & & & & & -0.4259 (5) \\
\hline
\end{tabular}
\end{center}



\subsection{Phase Shifts}

\begin{table}[H]
\centering
\begin{tabular}{c c c c c c c c}
\toprule
$\kappa$ & $\delta^+ (\omega = 7)$ & $\delta^- (\omega = 7)$ & $\delta^+ (\omega \rightarrow \infty)$ & $\delta^- (\omega \rightarrow \infty)$ & \% Diff$^+$ & \% Diff$^-$ \\
\midrule
0.1 & -0.427 & -0.215 & -0.426 & -0.214 & 0.223\% & 0.120\% \\
0.2 & -0.820 & -0.431 & -0.819 & -0.432 & 0.019\% & 0.063\% \\
0.3 & -1.161 & -0.645 & -1.161 & -0.645 & 0.040\% & 0.094\% \\
0.4 & -1.446 & -0.850 & -1.446 & -0.850 & 0.022\% & 0.130\% \\
0.5 & -1.678 & -1.041 & -1.677 & -1.040 & 0.031\% & 0.166\% \\
0.6 & -1.858 & -1.217 & -1.857 & -1.215 & 0.040\% & 0.273\% \\
0.7 & -1.964 & -1.375 & -1.963 & -1.373 & 0.045\% & 0.250\% \\
\bottomrule
\end{tabular}
\caption{S-Wave complex Kohn phase shifts}
\label{tab:SWavePhase}
\end{table}

Table \ref{tab:SWavePhase} contains the phase shifts for regular intervals of $\kappa$, which we can compare to the results from other groups in Tables \ref{tab:SWaveSingletOther} and \ref{tab:SWaveTripletOther} starting on page \pageref{tab:SWaveSingletOther}.  Our phase shifts at $\omega = 7$ from this table are exactly the same as Van Reeth and Humberston's results for $\omega = 6$ \cite{VanReeth2003}, with some exceptions in the last digit.  Figure \ref{fig:SWavePhaseOmega=7} has the fuller set of phase shifts plotted with respect to the positronium momentum, $\kappa$.

\textbf{@TODO:} What are our results for $\omega = 6$?

As mentioned in Section \ref{sec:Extrapolation}, extrapolation to $\omega = \infty$ has been problematic with larger numbers of terms.

\begin{figure}[H]
	\centering
	\includegraphics[width=7in]{swave-phases}
	\caption{$^{1,3}$S phase shifts}
	\label{fig:SWavePhase}
\end{figure}


\begin{table}[H]
\begin{center}
\begin{tabular}{c c c c c c c}
\toprule
 & $(\omega = 6)$ & $(\omega \rightarrow \infty)$ &  &  &  & \\
$\kappa$ & $\delta^+$ \cite{VanReeth2003} & $\delta^+$ \cite{VanReeth2003} & $\delta^+$ \cite{Blackwood2002} & $\delta^+$ \cite{Walters2004} & $\delta^+$ \cite{Ray1997} & $\delta^+$ \cite{Adhikari1999} \\
\midrule
$0.05$ & --- & --- & $-0.219$ & --- & --- & --- \\
$0.1$ & $-0.427$ & $-0.425$ & $-0.434$ & $-0.428$ & $-0.692$ & $-0.362$ \\
$0.2$ & $-0.820$ & $-0.817$ & $-0.834$ & $-0.825$ & $-1.212$ & $-0.702$ \\
$0.3$ & $-1.161$ & $-1.158$ & $-1.178$ & $-1.167$ & $-1.592$ & $-1.002$ \\
$0.4$ & $-1.446$ & $-1.443$ & $-1.467$ & $-1.453$ & $-1.902$ & $-1.252$ \\
$0.5$ & $-1.677$ & $-1.674$ & $-1.704$ & $-1.685$ & $-2.142$ & $-1.462$ \\
$0.6$ & $-1.857$ & $-1.852$ & $-1.890$ & $-1.867$ & $-2.362$ & $-1.622$ \\
$0.7$ & $-1.964$ & $-1.959$ & $-2.018$ & $-1.992$ & $-2.512$ & $-1.712$ \\
$0.8$ &    --- &    --- &    --- &    --- & $-2.652$ & $-2.163$ \\
\bottomrule
\end{tabular}
\caption{S-Wave Singlet Results from Other Groups}
\label{tab:SWaveSingletOther}
\end{center}
\end{table}


\begin{table}[H]
\begin{center}
\begin{tabular}{c c c c c c}
\toprule
 & $(\omega = 6)$ & $(\omega \rightarrow \infty)$ &  &  &   \\
$\kappa$ & $\delta^-$ \cite{VanReeth2003} & $\delta^-$ \cite{VanReeth2003} & $\delta^-$ \cite{Blackwood2002} & $\delta^-$ \cite{Ray1997} & $\delta^-$ \cite{Adhikari1999} \\
\midrule
$0.05$ & --- & --- & $-0.103$ & --- & --- \\
$0.1$ & $-0.215$ & $-0.214$ & $-0.206$ & $-0.252$ & $-0.167$ \\
$0.2$ & $-0.432$ & $-0.431$ & $-0.414$ & $-0.502$ & $-0.327$ \\
$0.3$ & $-0.645$ & $-0.645$ & $-0.624$ & $-0.722$ & $-0.474$ \\
$0.4$ & $-0.850$ & $-0.849$ & $-0.838$ & $-0.942$ & $-0.602$ \\
$0.5$ & $-1.040$ & $-1.038$ & $-1.037$ & $-1.142$ & $-0.706$ \\
$0.6$ & $-1.215$ & $-1.211$ & $-1.213$ & $-1.332$ & $-0.784$ \\
$0.7$ & $-1.373$ & $-1.366$ & $-1.367$ & $-1.502$ & $-0.833$ \\
$0.8$ &    --- &    --- &    --- & $-1.652$ & $-0.851$ \\
\bottomrule
\end{tabular}
\caption{S-Wave Triplet Results from Other Groups}
\label{tab:SWaveTripletOther}
\end{center}
\end{table}


\begin{figure}[H]
	\centering
	\includegraphics[width=5.25in]{swave-comparisons}
	\caption[Comparison of S-wave phase shifts]{Comparison of $^1$S (a) and $^3$S (b) phase shifts with results from other groups. Results are ordered according to year of publication. This work -- solid curves; \mbox{\textcolor{blue}{$\times$} -- CC \cite{Walters2004};} \mbox{$\CIRCLE$ -- Kohn \cite{VanReeth2003};} \mbox{\textcolor{red}{\textbf{+}} -- CC \cite{Blackwood2002};} \mbox{$\blacktriangle$ -- DMC \cite{Chiesa2002};} \mbox{$\triangledown$ -- SVM 2002 \cite{Ivanov2002};} \mbox{$\Circle$ -- SVM 2001 \cite{Ivanov2001};} \mbox{\textcolor{red}{$\vartriangle$} -- 6-state CC \cite{Sinha2000};} \mbox{$\blacksquare$ -- 5-state CC \cite{Adhikari1999};} \mbox{$\square$ -- Coupled-pseudostate \cite{Campbell1998};} \mbox{$\vartriangle$ -- 3-state CC \cite{Sinha1997};} \mbox{\textcolor[RGB]{0,127,0}{$\bigstar$} -- CC \cite{Ray1997};} \mbox{$\triangleright$ -- Stabilization \cite{Drachman1976};} \mbox{\textcolor{red}{$\blacklozenge$} -- Stabilization \cite{Drachman1975};} \mbox{\textcolor{blue}{$\lozenge$} -- Static-exchange \cite{Hara1975};} \mbox{$\blacktriangledown$ -- Static-exchange \cite{Fraser1961}.}}
	\label{fig:SWaveComparisons}
\end{figure}

Figure \ref{fig:SWaveComparisons} shows the results from various groups for calculations of the singlet S-wave.  Our results are extremely close to Van Reeth's \cite{VanReeth2003}, so they would follow along the solid line as well.  Several groups have results that cluster very closely to his and ours, namely Blackwood \cite{Blackwood2002}, Walters \cite{Walters2004}, Chiesa \cite{Chiesa2002} and Ivanov \cite{Ivanov2002}.






\section{S-Wave Resonances}
\label{sec:SWaveResonances}

There are two very obvious resonances before the Ps(n=2) channel threshold for $^1$S scattering, which can be seen in figure \ref{fig:SWavePhase}.  This threshold is at an incoming Ps energy of approximately 5.102 eV.  

\subsection{Resonance Model}
\label{sec:ResonanceModel}
The resonance positions and widths can be calculated to high accuracy by fitting the data to the following standard curve:

\beq
\label{eq:ResonanceCurve}
\delta(E) = A + BE + CE^2 + \arctan\left[ \frac{^1\Gamma}{2\left(^1E_R-E\right)} \right] + \arctan\left[ \frac{^2\Gamma}{2\left(^2E_R-E\right)} \right]
\eeq
The polynomial part of the above equation corresponds to hard-sphere scattering. The arctangent parts correspond to the first and second Fano resonances \cite{Fano1961,Macek1970,Hazi1979}, with $^1E_R$ and $^2E_R$ as the positions of the resonances and $^1\Gamma$ and $^2\Gamma$ as the widths of the respective resonances.

\textbf{@TODO:} Refer back to the Bransden and Joachain book and/or the Ray paper about decomposing this equation into the different parts. Refer to the discussions in Peter's paper and Yan/Ho's paper about the resonances corresponding to 2S and 3S.  Discussion on energy levels elsewhere for H and Ps, along with what the threshold is. Ray has an analysis about the resonance width versus the lifetime.
\subsection{Resonance Graphs}
\label{sec:ResonanceGraphs}

\begin{figure}[H]
	\centering
	\includegraphics[width=7in]{ResOmega=6,pg1}
	\caption{First set of resonance fitting graphs for $\omega = 6$}
	\label{fig:ResOmega=6,pg1}
\end{figure}

\begin{figure}[H]
	\centering
	\includegraphics[width=7in]{ResOmega=6,pg2}
	\caption{Second set of resonance fitting graphs for $\omega = 6$}
	\label{fig:ResOmega=6,pg2}
\end{figure}

\begin{figure}[H]
	\centering
	\includegraphics[width=7in]{ResOmega=6,Residuals}
	\caption{Graphs of residuals for the resonance fittings for $\omega = 6$}
	\label{fig:ResOmega=6,Residuals}
\end{figure}

\begin{figure}[H]
	\centering
	\includegraphics[width=7in]{ResOmega=7,pg1}
	\caption{First set of resonance fitting graphs for $\omega = 7$}
	\label{fig:ResOmega=7,pg1}
\end{figure}

\begin{figure}[H]
	\centering
	\includegraphics[width=7in]{ResOmega=7,pg2}
	\caption{Second set of resonance fitting graphs for $\omega = 7$}
	\label{fig:ResOmega=7,pg2}
\end{figure}

\begin{figure}[H]
	\centering
	\includegraphics[width=7in]{ResOmega=7,Residuals}
	\caption{Graphs of residuals for the resonance fittings for $\omega = 7$}
	\label{fig:ResOmega=7,Residuals}
\end{figure}


\subsection{Resonance Values and Errors}
\label{sec:ResonanceErrors}

\begin{figure}[H]
	\centering
	\includegraphics[width=5in]{KohnFitting}
	\caption{Kohn fitting ($\tau = 0.0$) with different weighting functions for $\omega = 7$}
	\label{fig:KohnFitting}
\end{figure}

\begin{figure}[H]
	\centering
	\includegraphics[width=5in]{07Fitting}
	\caption{Fitting for $\tau = 0.7$ with different weighting functions for $\omega = 7$}
	\label{fig:07Fitting}
\end{figure}

\begin{figure}[H]
	\centering
	\includegraphics[width=5in]{pi4Fitting}
	\caption{Fitting for $\frac{\pi}{4}$ with different weighting functions for $\omega = 7$}
	\label{fig:pi4Fitting}
\end{figure}

The results for the resonance parameters normally differ slightly for each of the Kohn methods.  To determine these, we gathered the data for all $\kappa$ values below the Ps(2s)+H(1s) channel threshold for a given Kohn method.  An example of the resultant graph for $\tau = 0.0$ (Kohn) can be seen in figure \ref{fig:KohnFitting}.  As described in section \ref{sec:ResonanceFit}, MATLAB's nonlinear fitting routine, nlinfit, is used with 8 different weighting functions to try various fits.  All 8 weighting functions give at least respectable fits to the data, but some fits are obviously worse than others.  For our data, the Fair, Logistic and Huber do not fit well for the second resonance.  The Cauchy will fit it close but not as close as the Andrews, Bisquare, Talwar and Welsch.  Out of the closest fittings, the Talwar is usually the worst of the four.

The closeness of the fits can be determined somewhat by plotting the residuals or looking at the values of the residual sum of squares, but these can be misleading near the resonances.  The MATLAB script that does the fittings also generates graphs for each fit with the data points superimposed.  The quality of the fitting is easily evaluated visually.  Due to the narrowness of the second resonance, the poor fits will usually overestimate its width and/or give the wrong position, causing the curve to not intersect the data points.

An example of a good set of fits is given in figure \ref{fig:KohnFitting}.  The graphs in this section are plotted with respect to the positronium energy, unlike the graphs in section \ref{sec:ResonanceGraphs}, which are plotted with respect to the momentum, $\kappa$.  All three of the plotted fitting curves follow the data nicely, with the exception of the extreme points near $\delta^+ = 0$.  This is not a failure of the fitting but instead a side effect of using the model given by \ref{eq:ResonanceCurve}.  The range of $\arctan$ is $(-\frac{\pi}{2},\frac{\pi}{2})$, so the $\arctan$ parts of this model cannot bring the phase shift from the background (near 2.0) all the way to 0.0, as it can only add up to $\frac{\pi}{2}$.  Regardless of this, the fits are nearly perfect in this example.

For some values of $\tau$, we have what appear to be Schwartz singularities.  Section \ref{sec:SchwartzSing} has more discussion on how these arise.  Figure \ref{fig:07Fitting} has a series of data points to the right of the second singularity that do not agree with runs for other values of $\tau$.  MATLAB attempts to fit with these, leading to incorrect parameters for the second resonance.  This type of behavior is observed for $\kappa = 0.7$ and $0.8$, so these runs are not included in the final results.  A similar problem is seen in figure \ref{fig:pi4Fitting}.  One data point at 5 eV is obviously shifted from where it ``should'' be.  The different weighting functions assign different importance to this point, so all three weighting functions gives different positions and widths of the second resonance.  This run for $\kappa = \frac{\pi}{4}$ was also discarded. 

The resonance parameters are calculated separately for each Kohn method (Kohn, inverse Kohn, etc.).  The ``good'' fits for all Kohn methods are all compiled in an Excel spreadsheet, where the mean, mode and errors are determined for each of the four resonance parameters.

The error for each parameter could be determined one of three ways.  The errors given in table \ref{tab:SWaveResonances} are simply the standard deviation.  We could have also used the standard error, which is given as $SE = \sigma / \sqrt{n}$.  The standard error is typically too small, since we are using approximately 100 tests.  For instance, using $\omega = 7$, for $^1E_R$, we get a standard error of $0.00001$, which implies a greater precision and agreement than our results have.  Another possible method for determining the error that we have looked at is to find the maximum deviation from the mean.  However, this overemphasizes the importance of outliers.  Using $\omega = 6$, the maximum deviation from the mean is $0.00586$ for $^2E_R$, which is $4.0$ times as large as the standard deviation. 



\subsection{Resonance Parameters}

\setlength{\abovecaptionskip}{6pt}   % 0.5cm as an example
\setlength{\belowcaptionskip}{6pt}   % 0.5cm as an example
\begin{table}[H]
\footnotesize
\centering

\begin{tabular}{l l l l l}
\toprule
Method & $^1E_R \text{ (eV)}$ & $^1\Gamma \text{ (eV)}$ & $^2E_R \text{ (eV)}$ & $^2\Gamma \text{ (eV)}$ \\
\midrule
This work & $4.0065 \pm 0.0001$ & $0.0955 \pm 0.0001$ & $5.0277 \pm 0.0018$ & $0.0608 \pm 0.0005$ \\
Complex rotation \cite{Drachman1975} & $4.455 \pm 0.010$ & $0.062 \pm 0.015$ & --- & --- \\
Complex rotation \cite{Ho1978} & $4.013 \pm 0.014$ & $0.075 \pm 0.027$ & --- & --- \\
Coupled-pseudostate \cite{Campbell1998} & $4.55$ & $0.084$ & --- & --- \\
Complex rotation \cite{Yan1999} & $4.0058 \pm 0.0005$ & $0.0952 \pm 0.0011$ & $4.9479 \pm 0.0014$ & $0.0585 \pm 0.0027$ \\
Five-state CC \cite{Adhikari2001e} & $4.01$ & $0.15$ & --- & --- \\
Optical potential \cite{DiRienzi2002b} & $4.021$ & $0.0259$ & --- & --- \\
CC (9Ps9H) \cite{Blackwood2002} & $4.37$ & $0.10$ & --- & --- \\
CC (22Ps1H + H$^-$) \cite{Blackwood2002b} & $4.141$ & $0.071$ & $4.963$ & $0.033$ \\
Stabilization \cite{Yan2003} & $4.007$ & $0.0969$ & $4.953$ & $0.0574$ \\
Kohn variational \cite{VanReeth2004} & $4.0072 \pm 0.0020$ & $0.0956 \pm 0.010$ & $5.0267 \pm 0.0020$ & $0.0597 \pm 0.0010$ \\
CC (9Ps9H + H$^-$) \cite{Walters2004} & $4.149$ & $0.103$ & $4.877$ & $0.0164$ \\
\bottomrule
\end{tabular}
\caption{S-wave resonance parameters} % title of Table
\label{tab:SWaveResonancesOther}
\end{table}

\textbf{@TODO:} Create comparison graph like in \cite{Ho1998}.
\textbf{@TODO:} Discussion of Peter's previous calculations and omitted terms.





\section{Scattering Length and Effective Range}
\label{sec:ScatteringLength}

\textbf{@TODO: Explanation of what scattering length is}

\subsection{Scattering Length Definition}
The scattering length is defined as \citep[pg. 589]{Bransden2003}
\beq
\label{eq:ScatLen}
a^\pm = -\lim_{\kappa \to 0} \frac{\tan{\delta^\pm}}{\kappa}.
\eeq

%\noindent We follow a similar procedure to that of Van Reeth and Humberston to determine the scattering length \cite{VanReeth2003}.  The scattering program is run for values of $\kappa = 0.001$ and $\kappa = 0.0001$ for both the singlet and triplet states.  The scattering lengths calculated for each are nearly identical, differing only in the fifth decimal place.  

\noindent If this definition is used for decreasing (but positive) values of $\kappa$, there is a clear convergence.  Table \ref{tab:ScatLenDef} has the scattering length calculated for $\kappa$ values of $0.001$ and $0.0001$.  To the quoted accuracy, the values of the scattering length agree exactly for the two different values of $\kappa$.  This is the method used by Van Reeth and Humberston for finding the scattering length \cite{VanReeth2003}.

\textbf{@TODO: Put in graph showing convergence.}

\begin{table}[H]
\begin{center}
\begin{tabular}{c c c c c}
\toprule
$\omega$ & $a^+ (\kappa = 0.001)$ & $a^+ (\kappa = 0.0001)$ & $a^- (\kappa = 0.001)$ & $a^- (\kappa = 0.0001)$ \\
\midrule
6 & 4.3364 & 4.3364 & 2.1415 & 2.1415 \\
7 & 4.3306 & 4.3306 & 2.1363 & 2.1363 \\
\bottomrule
\end{tabular}
\caption{Scattering Length from Definition}
\label{tab:ScatLenDef}
\end{center}
\end{table}

\noindent Van Reeth and Humberston also extrapolate their scattering length to $\omega \rightarrow \infty$ using the following equation \cite{VanReeth2003}:
\beq
\label{eq:ScatLenExtrap}
a^\pm(\omega) = a^\pm(\omega \rightarrow \infty) + \frac{c}{\omega^p}.
\eeq

\textbf{@TODO: Extrapolation}


\subsection{Effective Range}
For short-range potentials, the effective range is given by \cite{Bethe1949,Blatt1949,Drake2006}

\beq
\label{eq:EffectiveRangeShort}
\kappa \cot\eta_0^\pm = -\frac{1}{a^\pm} + \frac{1}{2} r_0^\pm \kappa^2 + \mathcal{O}(\kappa^4).
\eeq

\noindent This equation will be referred to as the standard fit. Phase shifts for low values of $\kappa$ are fitted to this equation to determine the effective range and scattering length, and the results are shown in table \ref{tab:ScatLenStandard}. This is a subset of tables \ref{tab:ScatLenSinglet} and \ref{tab:ScatLenTriplet}. For each range other than $\kappa = 0.1 - 0.6$, we used 10 equidistant values of $\kappa$. This fitting is not carried out to $\kappa$ higher than 0.6, due to the resonance structure described in section \ref{sec:SWaveResonances}.

\textbf{@TODO: Graphs}

\begin{table}[H]
\begin{center}
\begin{tabular}{c l c c c c}
\toprule
$\omega$ & Range & $a^+$ & $r_0^+$ & $a^-$ & $r_0^-$ \\
\midrule
6 & $0.1 - 0.6$ & 4.3168 & 2.2755 & 2.1701 & 1.3447 \\
  & $0.01 - 0.09$ & 4.3364 & 2.1968 & 2.1417 & 1.8562 \\
  & $0.001 - 0.009$ & 4.3364 & 2.1882 & 2.1415 & 1.9288 \\
  & $0.0001 - 0.0009$ & 4.3364 & 2.1920 & 2.1415 & 1.9302 \\
\midrule
7 & $0.1 - 0.6$ & 4.3132 & 2.2745 & 2.1677 & 1.3419 \\
  & $0.01 - 0.09$ & 4.3306 & 2.2012 & 2.1365 & 1.9345 \\
  & $0.001 - 0.009$ & 4.3306 & 2.1927 & 2.1363 & 2.0374 \\
  & $0.0001 - 0.0009$ & 4.3306 & 2.2070 & 2.1363 & 1.5457 \\
\bottomrule
\end{tabular}
\caption{Scattering Length and Effective Range for Standard Fit}
\label{tab:ScatLenStandard}
\end{center}
\end{table}

This fitting works well enough for the singlet, but the triplet data does not fit exactly.  This can be seen in figure \textbf{ADD FIGURE}, which is similar to figure 5 of Van Reeth \cite{VanReeth2003}.  The long-range $R^{-6}$ van der Waals potential appears to be more dominant for the triplet. \textbf{Physical Reason?}

\textbf{ADD FIGURE}

For small values of $\kappa$ (less than 0.009), the scattering length agrees with the definition in equation \ref{eq:ScatLen}. These values also agree well with the scattering length found by other recent calculations, shown in table \ref{tab:SWaveScatLenOther}. The effective range for the $\kappa = 0.1 - 0.6$ entry for the standard fit also agrees relatively well with the results from other groups. As can easily be seen, when smaller $\kappa$ values are used however, the value of $r_0^\pm$ changes drastically for the triplet and somewhat for the singlet. Previous work using the Kohn variational method found $r_0^- = 1.39$ \cite{VanReeth2003}, which is close to our value of $r_0^- = 1.3438$.

All other calculations for $r_0$ in the literature have used only the standard fit for both the singlet and triplet. These calculations also only use a fairly high set of $\kappa$ values, such as $0.1 - 0.6$. The discrepancy between calculated effective ranges with equation \ref{eq:EffectiveRangeShort} led us to consider other effective range theory models that include the van der Waals interaction.

\subsection{van der Waals Terms}
\label{sec:vanderWaalsERT}

The van der Waals potential is given as

\beq
\label{eq:VanderWaals}
V(R) = -\frac{C}{R^6}.
\eeq

\textbf{@TODO: Discuss why this potential is used.}

\noindent For the Ps-H system, $C$ has been calculated in atomic units by Martin and Fraser to be $34.78473$ \cite{Martin1980}. When the long-range potential is taken into account, the effective range equation becomes (dropping the $\pm$ for clarity) \cite[pg. 669]{Drake2006}

\beq
\label{eq:EffectiveRangeLong}
\kappa \cot\eta_0 = -\frac{1}{a} + \frac{1}{2} r_0 \kappa^2 - \frac{\pi}{15 a^2} \left(\frac{2 M C}{\hbar^2}\right) \kappa^3 - \frac{4}{15 a} \left(\frac{2 M C}{\hbar^2}\right) \kappa^4 \ln \left(\kappa a_0 \right) + \mathcal{O}(\kappa^4).
\eeq

\noindent In atomic units (see section \ref{sec:Units}), $M = 2$ (mass of Ps), $\hbar = 1$ (Planck's constant) and $a_0 = 1$ (Bohr radius), simplifying this equation as
\beq
\label{eq:EffectiveRangeLongAu}
\kappa \cot\eta_0 = -\frac{1}{a} + \frac{1}{2} r_0 \kappa^2 - \frac{4 \pi C}{15 a^2} \kappa^3 - \frac{16 C}{15 a} \kappa^4 \ln \left(\kappa \right) + \mathcal{O}(\kappa^4).
\eeq

In tables \ref{tab:ScatLenSinglet} and \ref{tab:ScatLenTriplet}, the $\kappa^2$ column fits only to the first two terms of \ref{eq:EffectiveRangeLongAu}, which makes this the same as the standard fit, equation \ref{eq:EffectiveRangeShort}.  The $\kappa^3$ column fits to the first three terms, and the $\kappa^4 \ln$ column fits to all four terms.  Entries in these tables are given as $a^\pm / r_0^\pm$. The scattering length, $a^\pm$, is used as a fitting parameter, not a fixed value determined by \ref{eq:ScatLen} in table \ref{tab:ScatLenDef}.


\begin{table}[H]
\begin{center}
\begin{tabular}{c l c c c}
\toprule
$\omega$ & Range & $\kappa^2$ & $\kappa^3$ & $\kappa^4 \ln$ \\
\midrule
5 & $0.1 - 0.6$ & 4.3233/2.2834 & 3.9143/4.5169 & 4.6162/0.8488 \\
  & $0.01 - 0.09$ & 4.3459/2.1995 & 4.3442/2.4770 & 4.3469/2.1502 \\
  & $0.001 - 0.009$ & 4.3460/2.1957 & 4.3460/2.2234 & 4.3460/2.2171 \\
  & $0.0001 - 0.0009$ & 4.3460/2.1890 & 4.3460/2.1918 & 4.3460/2.1917 \\
\midrule
6 & $0.1 - 0.6$ & 4.3162/2.2768 & 3.9060/4.5202 & 4.6070/0.8429 \\
  & $0.01 - 0.09$ & 4.3348/2.1994 & 4.3330/2.4784 & 4.3358/2.1507 \\
  & $0.001 - 0.009$ & 4.3348/2.1953 & 4.3348/2.2232 & 4.3348/2.2168 \\
  & $0.0001 - 0.0009$ & 4.3348/2.2085 & 4.3348/2.2113 & 4.3348/2.2112 \\
\midrule
7 & $0.1 - 0.6$ & 4.3132/2.2746 & 3.9023/4.5225 & 4.6030/0.8411 \\
  & $0.01 - 0.09$ & 4.3289/2.2046 & 4.3271/2.4844 & 4.3299/2.1562 \\
  & $0.001 - 0.009$ & 4.3289/2.2006 & 4.3289/2.2285 & 4.3289/2.2221 \\
  & $0.0001 - 0.0009$ & 4.3289/2.4926 & 4.3289/2.4954 & 4.3289/2.4953 \\
\bottomrule
\end{tabular}
\caption{Singlet Scattering Length and Effective Range}
\label{tab:ScatLenSinglet}
\end{center}
\end{table}


\begin{table}[H]
\begin{center}
\begin{tabular}{c l c c c}
\toprule
$\omega$ & Range & $\kappa^2$ & $\kappa^3$ & $\kappa^4 \ln$ \\
\midrule
5 & $0.1 - 0.6$ & 2.1763/1.3614 & 2.1586/8.0998 & 2.1832/2.0213 \\
  & $0.01 - 0.09$ & 2.1533/1.7514 & 2.1516/2.8828 & 2.1530/2.2222 \\
  & $0.001 - 0.009$ & 2.1532/1.7959 & 2.1532/1.9089 & 2.1532/1.8961 \\
  & $0.0001 - 0.0009$ & 2.1532/1.7950 & 2.1532/1.8063 & 2.1532/1.8061 \\
\midrule
6 & $0.1 - 0.6$ & 2.1705/1.3490 & 2.1530/8.1222 & 2.1760/2.0351 \\
  & $0.01 - 0.09$ & 2.1425/1.8521 & 2.1407/2.9950 & 2.1421/2.3311 \\
  & $0.001 - 0.009$ & 2.1423/1.9225 & 2.1423/2.0367 & 2.1423/2.0238 \\
  & $0.0001 - 0.0009$ & 2.1423/1.9236 & 2.1423/1.9350 & 2.1423/1.9349 \\
\midrule
7 & $0.1 - 0.6$ & 2.1680/1.3438 & 2.1505/8.1323 & 2.1728/2.0415 \\
  & $0.01 - 0.09$ & 2.1371/1.9309 & 2.1353/3.0796 & 2.1367/2.4140 \\
  & $0.001 - 0.009$ & 2.1369/2.0326 & 2.1369/2.1473 & 2.1369/2.1345 \\
  & $0.0001 - 0.0009$ & 2.1369/1.9969 & 2.1369/2.0083 & 2.1369/2.0081 \\  
\bottomrule
\end{tabular}
\caption{Triplet Scattering Length and Effective Range}
\label{tab:ScatLenTriplet}
\end{center}
\end{table}
 



\subsubsection{Hinckelmann Equation}
Equation \ref{eq:EffectiveRangeLong} is derived by starting with the expression given in Hinckelmann and Spruch, then inverting and performing an expansion. Hinckelmann gives this in terms of $\tan\eta_0$ instead of the now more commonly used $\cot\eta_0$ \cite{Hinckelmann1971}.

\beq
\label{eq:HinckelmannEqn}
\tan\eta_0 = -a \kappa - \frac{1}{2} r_0 a^2 \kappa^3 + \frac{1}{15} C^4 \kappa^4 + \frac{4}{15} C^4 \kappa^5 \ln|2\kappa d| + \mathcal{O}(\kappa^5)
\eeq

\begin{table}[H]
\begin{center}
\begin{tabular}{c l c c c c}
\toprule
$\omega$ & Range & $\kappa^3$ & $\kappa^4$ & $\kappa^5 \ln$ \\
\midrule
7 & $0.1 - 0.6$ & 13.6676/-0.7050 & 13.3858/-0.6984 & Undefined \\
  & $0.01 - 0.09$ & 4.3269/2.3156 & 4.3267/2.3234 & Undefined \\
  & $0.001 - 0.009$ & 4.3289/2.2014 & 4.3289/2.2015 & 4.3289/2.2062 \\
  & $0.0001 - 0.0009$ & 4.3289/2.3566 & 4.3289/2.3566 & 4.3289/2.3569 \\  
\bottomrule
\end{tabular}
\caption{Hinckelmann Singlet Scattering Length and Effective Range}
\label{tab:HinckScatLenSinglet}
\end{center}
\end{table}


\begin{table}[H]
\begin{center}
\begin{tabular}{c l c c c c}
\toprule
$\omega$ & Range & $\kappa^3$ & $\kappa^4$ & $\kappa^5 \ln$ \\
\midrule
7 & $0.1 - 0.6$ & 1.7293/5.0014 & 1.4474/10.2684 & Undefined \\
  & $0.01 - 0.09$ & 2.1370/1.9457 & 2.1371/1.9773 & 2.1369/1.9886 \\
  & $0.001 - 0.009$ & 2.1369/2.0323 & 2.1369/2.0326 & 2.1368/2.0381 \\
  & $0.0001 - 0.0009$ & 2.1369/1.9906 & 2.1369/1.9907 & 2.1368/1.9912 \\  
\bottomrule
\end{tabular}
\caption{Hinckelmann Triplet Scattering Length and Effective Range}
\label{tab:HinckScatLenTriplet}
\end{center}
\end{table}

Tables \ref{tab:HinckScatLenSinglet} and \ref{tab:HinckScatLenTriplet} have entries for the $\kappa^5 \ln$ term where the fitting cannot be used, since the fitting attempts to use a negative value of $d$, forcing the natural logarithm to return a complex value. The variable $d$ is given in their paper as the distance $r < d$ at which the van der Waals potential is negligible. We have to fit to $d$ in our problem, making this particular model not fit as well. The Flannery equation, equation \ref{eq:EffectiveRangeLong}, is a modified version of this and is easier to fit to our type of problem.

\subsection{Analytic Solution}
Arriola gives an analytic solution for the van der Waals effective range \cite{Arriola2010}:
\beq
\label{eq:EffRangeAnalytic}
\frac{r_0}{R} = 1.395 - 1.333 \frac{R}{a_0} + 0.6373 \frac{R^2}{a_0^2} \,\,,
\eeq
where the van der Waals range, $R$, is
\beq
R = \left(\frac{M C}{\hbar^2}\right)^{\frac{1}{4}}.
\eeq

\noindent Using the values from table \ref{tab:ScatLenDef} for $a^\pm$, this equation produces the results in table \ref{tab:EffRangeArriola}.

\begin{table}[H]
\begin{center}
\begin{tabular}{c c c c c}
\toprule
$\omega$ & $r_0^+$ & $r_0^-$ \\
\midrule
5 & 2.2833 & 2.1764 \\
6 & 2.2809 & 2.1839 \\
7 & 2.2796 & 2.1878 \\
\bottomrule
\end{tabular}
\caption{Effective Range from Analytic Equation}
\label{tab:EffRangeArriola}
\end{center}
\end{table}

Equation \ref{eq:EffRangeAnalytic} is derived using a semiclassical approach and not a full effective range theory treatment; it nonetheless returns values for the effective range similar to that returned from the other models.


\subsection{Gao Model}
Gao's effective range theory treatment is the most complicated of the models tried. Gao solves the Schr\"{o}dinger equation for an attractive $r^{-6}$ potential to find an expression relating the phase shifts to a quantity he refers to as $K_l^0(\epsilon)$ \cite{Gao1998}.

\beq
\label{eq:GaoZEqn}
\tan\delta_l = [Z_{ff} - K_l^0(\epsilon) Z_{gf}]^{-1} [K_l^0(\epsilon) Z_{gg} - Z_{fg}]
\eeq

\noindent The Z functions in this equation are complicated but described fully in his paper. The phase shifts are fitted to equation \ref{eq:GaoZEqn} to determine $K_l^0(\epsilon)$ for each $\kappa$ value. $K_l^0(\epsilon)$ must be expanded in a Taylor series as
\beq
\label{eq:GaoKTaylor}
K_l^0(\epsilon) = K_l^0(0) + \xi_1 \epsilon + \xi_2 \epsilon^2 + \ldots.
\eeq
From this, $K_l^0(0)$ and ${K_l^0}^\prime(0)$ are determined. In another paper \cite{Gao1998a}, he performs an expansion of this expression for low $\kappa$ to generate expressions for $a$ and $r_0$:

\beq
\label{eq:GaoScatLen}
a = \frac{2\pi}{[\Gamma(1/4)]^2} \frac{K_{l=0}^0(0) - 1}{K_{l=0}^0(0)} \beta_6
\eeq

\beq
\label{eq:GaoEffRange}
r_0 = \frac{[\Gamma(1/4)]^2}{3\pi} \frac{[K_{l=0}^0(0)]^2 + 1}{[K_{l=0}^0(0) - 1]^2} \beta_6 + \frac{[\Gamma(1/4)]^2}{\pi} \frac{{K_{l=0}^0}^\prime(0)^2(\hbar^2/2\mu)(1/\beta_6)^2}{[K_{l=0}^0(0) - 1]^2} \beta_6.
\eeq

\noindent $\beta_6$ is related to $C$ by $\beta_6 = (2\mu C/\hbar^2)^{1/4}$.

Table \ref{tab:GaoResults} is found by taking equation \ref{eq:GaoKTaylor} out to the second term, which includes the $K_l^0$ derivative. Two values of $\kappa$ are used to solve the equations for the unknowns $K_{l=0}^0(0)$ and ${K_{l=0}^0}^\prime(0)$. These values are in the table and are used to determine $a$ and $r_0$ via equations \ref{eq:GaoScatLen} and \ref{eq:GaoEffRange}.

\begin{table}[H]
\begin{center}
\begin{tabular}{c c c c c c}
\toprule
\toprule
& $\kappa$ & $K_l^0$ & $K_l^{0\prime}$ & $a$ & $r_0$ \\
\midrule
$^1S$ & $0.1, 0.2$ & -0.611384 & -1.63317 & 4.32678 & 2.34283 \\
 & $0.2, 0.3$ & -0.608792 & -1.89231 & 4.33821 & 2.31404 \\
 & $0.3, 0.4$ & -0.597468 & -2.39563 & 4.38932 & 2.26125 \\
 & $0.4, 0.5$ & -0.569266 & -3.10066 & 4.52544 & 2.19206 \\
 & $0.5, 0.6$ & -0.500657 & -4.19842 & 4.92064 & 2.09247 \\
\midrule
$^1S$ & 0.001, 0.002 & -0.610903 & -2.62093 & 4.32890 & 2.22722 \\
 & 0.002, 0.003 & -0.610902 & -3.07837 & 4.32890 & 2.17353 \\
 & 0.003, 0.004 & -0.610903 & -2.58573 & 4.32889 & 2.23135 \\
 & 0.004, 0.005 & -0.610903 & -2.72993 & 4.32890 & 2.21442 \\
 & 0.005, 0.006 & -0.610904 & -2.58189 & 4.32889 & 2.23180 \\
 & 0.006, 0.007 & -0.610903 & -2.70708 & 4.32890 & 2.21711 \\
 & 0.007, 0.008 & -0.610905 & -2.50404 & 4.32889 & 2.24093 \\
 & 0.008, 0.009 & -0.610903 & -2.65887 & 4.32890 & 2.22276 \\
\midrule
\midrule
$^3S$ & $0.1, 0.2$ & -3.33351 & -20.9463 & 2.13412 & 2.74986 \\
 & $0.2, 0.3$ & -3.29935 & -24.3621 & 2.13921 & 2.67874 \\
 & $0.3, 0.4$ & -3.15084 & -30.9623 & 2.16267 & 2.49085 \\
 & $0.4, 0.5$ & -2.62982 & -43.9880 & 2.26589 & 1.86113 \\
 & $0.5, 0.6$ & -0.686168 & -75.0864 & 4.03413 & -5.56551 \\
\midrule
$^3S$ & 0.001, 0.002 & -3.31507 & -64.1297 & 2.13686 & 2.03551 \\
 & 0.002, 0.003 & -3.31507 & -61.5075 & 2.13686 & 2.07840 \\
 & 0.003, 0.004 & -3.31507 & -59.8674 & 2.13685 & 2.10523 \\
 & 0.004, 0.005 & -3.31507 & -59.2409 & 2.13685 & 2.11548 \\
 & 0.005, 0.006 & -3.31508 & -59.0905 & 2.13685 & 2.11794 \\
 & 0.006, 0.007 & -3.31510 & -56.2482 & 2.13685 & 2.16445 \\
 & 0.007, 0.008 & -3.31509 & -57.1905 & 2.13685 & 2.14903 \\
 & 0.008, 0.009 & -3.31512 & -55.5620 & 2.13685 & 2.17569 \\
\bottomrule
\bottomrule
\end{tabular}
\caption{Full Gao Model Scattering Length and Effective Range}
\label{tab:GaoResults}
\end{center}
\end{table}

The values for $a$ are well-converged and compare well with the values in table \ref{tab:ScatLenDef} from the definition in equation \ref{eq:ScatLen}. The values of $r_0$ are not well-converged, but they compare well with the fittings in section \ref{sec:vanderWaalsERT}. We also attempted to carry the expansion in equation \ref{eq:GaoKTaylor} to the third term with the second derivative, but numerical inaccuracy proved to be a problem. The $\epsilon^2$ is proportional to $\kappa^4$, so this term is vanishingly small for most $\kappa$ values considered, making determining $\xi_2$ difficult.


\subsection{Comparison of Effective Range Theories}

\begin{figure}[H]
	\centering
	\includegraphics[width=5in]{ERT-Comparisons-Singlet-01.pdf}
	\caption{Comparison of Effective Range Theories for $^1S$ Ps-H}
	\label{fig:ERT-Comparisons-Singlet-01}
\end{figure}

\begin{figure}[H]
	\centering
	\includegraphics[width=5in]{ERT-Comparisons-Singlet-001.pdf}
	\caption{Comparison of Effective Range Theories for $^1S$ Ps-H}
	\label{fig:ERT-Comparisons-Singlet-001}
\end{figure}

\begin{figure}[H]
	\centering
	\includegraphics[width=5in]{ERT-Comparisons-Triplet-01.pdf}
	\caption{Comparison of Effective Range Theories for $^3S$ Ps-H}
	\label{fig:ERT-Comparisons-Triplet-01}
\end{figure}

\begin{figure}[H]
	\centering
	\includegraphics[width=5in]{ERT-Comparisons-Triplet-001.pdf}
	\caption{Comparison of Effective Range Theories for $^3S$ Ps-H}
	\label{fig:ERT-Comparisons-Triplet-001}
\end{figure}

To obtain the plots in figures \ref{fig:ERT-Comparisons-Singlet-01} - \ref{fig:ERT-Comparisons-Triplet-001}, after obtaining fits for each of the models, we solved the respective equations for the phase shifts.  This provides a reliable way to determine how well each model fits the phase shift data. When $\kappa$ is small, the different models agree extremely well, as seen in figures \ref{fig:ERT-Comparisons-Singlet-001} and \ref{fig:ERT-Comparisons-Triplet-001}. Higher order terms in each of the model equations become negligible as $\kappa$ gets smaller. Figures \ref{fig:ERT-Comparisons-Singlet-01}, \ref{fig:ERT-Comparisons-Singlet-001} and \ref{fig:ERT-Comparisons-Triplet-01} do not include the Hinckelmann results since, as seen in tables \ref{tab:HinckScatLenSinglet} and \ref{tab:HinckScatLenTriplet}, this expression does not generate valid results due to the need to fit to the $d$ parameter in equation \ref{eq:HinckelmannEqn}.



\subsubsection{Other Groups' Results}
\begin{table}[H]
\begin{center}
\begin{tabular}{l l l l l}
\toprule
Method & \multicolumn{1}{c}{$a^+$} & \multicolumn{1}{c}{$r_0^+$} & \multicolumn{1}{c}{$a^-$} & \multicolumn{1}{c}{$r_0^-$}\\
\midrule
Static-exchange (Hara \emph{et al} 1975) \cite{Hara1975} & 7.275 & \,\,--- & 2.476 & \,\,--- \\
Kohn 35 terms (Page 1976) \cite{Page1976} & 5.844 & 2.90 & 2.319 & \,\,--- \\
Stabilization (Drachman \emph{et al} 1975) \cite{Drachman1975} & 5.33 & 2.54 & \,\,--- & \,\,--- \\
Stabilization (Drachman \emph{et al} 1976) \cite{Drachman1976} & \,\,--- & \,\,--- & 2.36 & 1.31 \\
9-state R-matrix (Campbell \emph{et al} 1998) \cite{Campbell1998} & 5.51 & 2.63 & 2.45 & 1.33 \\
22-state R-matrix (Campbell \emph{et al} 1998) \cite{Campbell1998} & 5.20 & 2.52 & 2.45 & 1.32 \\
5-state (Adhikari \emph{et al} 1999) \cite{Adhikari1999} & 3.72 & 1.67 & --- & --- \\
6-state close coupling (Sinha \emph{et al} 2000) \cite{Sinha2000} & 5.90 & 2.73 & 2.32 & 1.29 \\
Variational basis-set (Adhikari \emph{et al} 2001) \cite{Adhikari2001b} & 3.49 & \,\,--- & 2.46 & \,\,--- \\
Stochastic variational (Ivanov \emph{et al} 2001) \cite{Ivanov2001} & 4.3 & --- & 2.2 & --- \\
Diffusion Monte Carlo (Chiesa \emph{et al} 2002) \cite{Chiesa2002} & 4.375 & 2.228 & 2.246 & 1.425 \\
Stochastic variational (Ivanov \emph{et al} 2002) \cite{Ivanov2002} & 4.34 & 2.39 & 2.22 & 1.29 \\
R-matrix 14Ps14H (Blackwood \emph{et al} 2002) \cite{Blackwood2002} & 4.41 & 2.19 & 2.06 & 1.47 \\
Kohn 721 terms (Van Reeth \emph{et al} 2003) \cite{VanReeth2003} & 4.334 & \,\,--- & 2.143 & \,\,--- \\
Kohn extrapolated (Van Reeth \emph{et al} 2003) \cite{VanReeth2003} & 4.311 & 2.27 & 2.126 & 1.39 \\
R-matrix 14Ps14H+H$^-$ (Walters \emph{et al} 2004) \cite{Blackwood2002} & 4.327 & \,\,--- & \,\,--- & \,\,--- \\
\bottomrule
\end{tabular}
\caption{S-Wave Scattering Length and Effective Range}
\label{tab:SWaveScatLenOther}
\end{center}
\end{table}




\end{document}