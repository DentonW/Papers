%\RequirePackage[l2tabu, orthodox]{nag}  % Warns of obsolete or incorrect LaTeX usage

%\documentclass[10pt,openany]{book}
\documentclass[12pt,openany]{book}


%\documentclass[draft]{}
%\usepackage[nobysame]{amsrefs}
%\usepackage[demo]{graphicx}
\usepackage{graphicx}
\usepackage{amsmath, fullpage, subfiles, mathrsfs}
%\usepackage[Lenny]{fncychap}
\usepackage[Bjornstrup]{fncychap}
%\usepackage[colorlinks=true]{hyperref}  % hyperref must be loaded after fncychap.
\usepackage{srcltx, tabularx, pdfsync, multirow, float, fancyhdr}
\usepackage[table]{xcolor}  % For table coloring - can have option clash if loaded after other packages
\usepackage{color, colortbl, rotate, listings, rotating}
\usepackage{caption, booktabs, tocvsec2, longtable}
\usepackage[numbers]{natbib}
%\usepackage[nottoc]{tocbibind}
\usepackage{tocbibind}
\usepackage{subfig}
\usepackage{todonotes}
\usepackage{bm}  % Bold-faced math
\usepackage{lettrine}  % Dropped capitals
\usepackage{cool}  % Typesetting for various equation types
%\usepackage[grey,utopia]{quotchap}  % Decorative chapter headings
\usepackage{tikz}
%\PassOptionsToPackage{table}{xcolor}
\usepackage{braket}
\usepackage[pagebackref=true, plainpages=false, bookmarks, bookmarksnumbered, colorlinks, linktocpage=true, linkcolor=blue, citecolor=blue, filecolor=black,urlcolor=blue, %
			pdflang={en}, pdftitle={Variational Calculations of Positronium Scattering with Hydrogen}, pdfauthor={Denton Woods}, %
			pdfsubject={Denton Woods's Dissertation at the University of North Texas}, pdfkeywords={positronium, hydrogen, scattering, positronium hydride, Denton Woods}]{hyperref}
%\usepackage{doi}
\usepackage[activate={true,nocompatibility},final,tracking=true,kerning=true,spacing=true,factor=1100,stretch=10,shrink=10]{microtype}  % http://www.khirevich.com/latex/microtype/
\microtypecontext{spacing=nonfrench}  % To get rid of that warning
\usepackage{siunitx}  % For correct typesetting of units
\sisetup{per-mode=symbol}
%\sisetup{range-phrase = \text{--}}
\usepackage{wasysym}  % Simply for \CIRCLE and \Circle (could use \circ for the latter)
\usepackage{pdflscape}  % For landscape page
\usepackage{dcolumn}
%\usepackage[arrowdel,notrig]{physics}  % Unfortunately conflicts with other packages
\usepackage[makeroom]{cancel}  % For crossing out items in equations
\usepackage{tocstyle}  % For spacing out table of contents numbers from ellipses

\usepackage{cleveref}
\crefname{equation}{Equation}{Equations}
\crefname{table}{Table}{Tables}
\crefname{figure}{Figure}{Figures}
\crefname{chapter}{Chapter}{Chapters}
%\usepackage{showlabels}

% For dcolumn
\newcolumntype{d}[1]{D{.}{\cdot}{#1}}
\newcolumntype{.}{D{.}{.}{-1}}

% To center table head (when more than one decimal point)
\newcommand*{\thead}[1]{\multicolumn{1}{c}{#1}}

% Look at http://texblog.wordpress.com/tag/pagestyle/
%\pagestyle{fancy}

% From http://latex-alive.tumblr.com/post/698006466
\usepackage{mathpazo}

%\usepackage[table]{xcolor}  % May need to be loaded after other packages
%\PassOptionsToPackage{table}{xcolor}

%\documentclass[draft]{}
%\usepackage[nobysame]{amsrefs}

%\usepackage[dvips,letterpaper,margin=0.75in,bottom=0.75in]{geometry}  % For smaller page margins
\usepackage{setspace}
%\singlespacing
%\onehalfspacing
%\doublespacing

% If we want the equations, figures and tables to be numbered sequentially instead of by chapter/section
%\usepackage{chngcntr}
%\counterwithout{equation}{chapter}
%\counterwithout{figure}{chapter}
%\counterwithout{table}{chapter}

% From http://texblog.org/2013/07/16/how-to-add-extra-space-to-the-table-of-contents-list-of-figures-and-tables/
%\usepackage{etoolbox, tocloft}
%\preto\figure{%
%  \ifnum\value{figure}=0\addtocontents{lof}{{\bfseries Chapter \thechapter\vskip10pt}}\fi
%}

% If using amsrefs
%\renewcommand{\biblistfont}{%
%  \normalfont
%  \normalsize
%}

\providecommand{\e}[1]{\ensuremath{\times 10^{#1}}}
\newcommand{\ee} {\,\text{e}}
\newcommand{\beq}{\begin{equation}}
\newcommand{\eeq}{\end{equation}}
\newcommand{\beqs}{\begin{equation*}}
\newcommand{\eeqs}{\end{equation*}}
\newcommand{\barr}{\begin{array}}
\newcommand{\earr}{\end{array}}
\newcommand{\bce}{\begin{center}}
\newcommand{\ece}{\end{center}}
\newcommand{\asymplim}[1]{\: {\underset{\scriptstyle{{#1}\to\infty}}{\displaystyle{\sim}}} \:}
\newcommand{\kr} {\kappa\rho}
\newcommand{\krp} {\kappa\rhop}
\newcommand{\mr} {\mu\rho}
\newcommand{\mrp} {\mu\rho'}
\newcommand{\rhop} {\rho'}
\newcommand{\ii}{{\textrm{i}}}
% Clebsch-Gordan
\newcommand{\cg}[6]{
  \left<
  \begin{array}{cc|c}
  #1 & #3 & #5 \\
  #2 & #4 & #6
  \end{array}
  \right>
}
% 3J Symbol
\newcommand{\tj}[6]{ \begin{pmatrix}
  #1 & #2 & #3 \\
  #4 & #5 & #6 
 \end{pmatrix}}
% 6J Symbol
\newcommand{\Gj}[6]{ \begin{Bmatrix}
  #1 & #2 & #3 \\
  #4 & #5 & #6 
 \end{Bmatrix}}
% 9J Symbol
\newcommand{\gj}[9]{ \begin{Bmatrix}
  #1 & #2 & #3 \\
  #4 & #5 & #6 \\
  #7 & #8 & #9
 \end{Bmatrix}}
\newcommand{\rtoinfty} {{\xrightarrow[\scriptscriptstyle {r\rightarrow \infty}]{}}}

% For creating nice boxes
\tikzstyle{nicebox}=[draw=gray!100, fill=blue!10, very thick,rounded corners, rectangle, inner sep=4pt, inner ysep=16pt]
\tikzstyle{niceboxtitle}=[draw=gray!100, fill=white, text=black,rounded corners, very thick, rectangle]
\newcommand\nicebox[2]{
	{\centering
	\begin{tikzpicture}
		\node [nicebox](box){
			\begin{minipage}{0.95\textwidth}\centering
			\begin{minipage}{0.95\textwidth}
						#2
			\end{minipage}\end{minipage}};
		\node[niceboxtitle, right=10pt] at (box.north west)
			{\small\textbf{#1}};
	\end{tikzpicture}\par}
}


% If you need a copy without the equations (word cloud, etc.), enable these lines.
%\usepackage{comment}
%\excludecomment{figure}
%\let\endfigure\relax
%\excludecomment{equation}
%\let\endequation\relax
%\excludecomment{lstlisting}
%\let\endlstlisting\relax
%\excludecomment{alignat}
%\let\endalignat\relax
%\excludecomment{align}
%\let\endalign\relax
%\excludecomment{subequations}
%\let\endsubequations\relax
%\excludecomment{tabular}
%\let\endtabular\relax
%\excludecomment{longtable}
%\let\endlongtable\relax
%\let\beq\iffalse
%\let\eeq\fi

\newcommand{\todoi}{\todo[inline]}

% From http://www.latex-community.org/forum/viewtopic.php?f=50&t=10320,
%  to get references working properly in citations.
\def\biblio{\bibliographystyle{UNTamsplain}\bibliography{Dissertation}}


% From http://latex-alive.tumblr.com/post/698006466
\DeclareSymbolFont{euler}{U}{eur}{m}{n}
\DeclareMathSymbol \uppi \mathalpha {euler} {"19}
%\DeclareMathSymbol \ee \mathalpha {euler} {`\e}

%\input{UNTdissertation.sty}
% Modified from http://www.dfcd.net/articles/latex/latex.html

% ***********************************************************
% ******************* PHYSICS HEADER ************************
% ***********************************************************
% Version 2
%\documentclass[11pt]{article} 
%\usepackage{amsmath} % AMS Math Package
%\usepackage{amsthm} % Theorem Formatting
%\usepackage{amssymb}	% Math symbols such as \mathbb
%\usepackage{graphicx} % Allows for eps images
%\usepackage{multicol} % Allows for multiple columns
%\usepackage[dvips,letterpaper,margin=0.75in,bottom=0.5in]{geometry}
 % Sets margins and page size
%\pagestyle{empty} % Removes page numbers
\makeatletter % Need for anything that contains an @ command 
\renewcommand{\maketitle} % Redefine maketitle to conserve space
{ \begingroup \vskip 10pt \begin{center} \large {\bf \@title}
	\vskip 10pt \large \@author \hskip 20pt \@date \end{center}
  \vskip 10pt \endgroup \setcounter{footnote}{0} }
\makeatother % End of region containing @ commands
\renewcommand{\labelenumi}{(\alph{enumi})} % Use letters for enumerate
% \DeclareMathOperator{\Sample}{Sample}
\let\vaccent=\v % rename builtin command \v{} to \vaccent{}
\renewcommand{\v}[1]{\ensuremath{\mathbf{#1}}} % for vectors
\newcommand{\gv}[1]{\ensuremath{\mbox{\boldmath$ #1 $}}} 
% for vectors of Greek letters
\newcommand{\uv}[1]{\ensuremath{\mathbf{\hat{#1}}}} % for unit vector
\newcommand{\abs}[1]{\left| #1 \right|} % for absolute value
\newcommand{\avg}[1]{\left< #1 \right>} % for average
\let\underdot=\d % rename builtin command \d{} to \underdot{}
\renewcommand{\d}[2]{\frac{d #1}{d #2}} % for derivatives
\newcommand{\dd}[2]{\frac{d^2 #1}{d #2^2}} % for double derivatives
\newcommand{\pd}[2]{\frac{\partial #1}{\partial #2}} 
% for partial derivatives
\newcommand{\pdd}[2]{\frac{\partial^2 #1}{\partial #2^2}} 
% for double partial derivatives
%%%\newcommand{\pdc}[3]{\left( \frac{\partial #1}{\partial #2}
%%% \right)_{#3}} % for thermodynamic partial derivatives
%\newcommand{\ket}[1]{\left| #1 \right>} % for Dirac bras
%\newcommand{\bra}[1]{\left< #1 \right|} % for Dirac kets
%\newcommand{\braket}[2]{\left< #1 \vphantom{#2} \right|
% \left. #2 \vphantom{#1} \right>} % for Dirac brackets
\newcommand{\matrixel}[3]{\left< #1 \vphantom{#2#3} \right|
 #2 \left| #3 \vphantom{#1#2} \right>} % for Dirac matrix elements
\newcommand{\grad}[1]{\gv{\nabla} #1} % for gradient
\let\divsymb=\div % rename builtin command \div to \divsymb
\renewcommand{\div}[1]{\gv{\nabla} \cdot #1} % for divergence
\newcommand{\curl}[1]{\gv{\nabla} \times #1} % for curl
\let\baraccent=\= % rename builtin command \= to \baraccent
\renewcommand{\=}[1]{\stackrel{#1}{=}} % for putting numbers above =
\newtheorem{prop}{Proposition}
\newtheorem{thm}{Theorem}[section]
\newtheorem{lem}[thm]{Lemma}
%\theoremstyle{definition}
%\newtheorem{dfn}{Definition}
%\theoremstyle{remark}
%\newtheorem*{rmk}{Remark}

% ***********************************************************
% ********************** END HEADER *************************
% ***********************************************************  % Much already in the physics package

\definecolor{Gray}{gray}{0.9}
\definecolor{LightCyan}{rgb}{0.88,1,1}
\definecolor{startcolor}{HTML}{B32018}

%\renewcommand{\showlabelfont}{\scriptsize}

\graphicspath{{IPython/}{Images/}}

\setlength{\extrarowheight}{3pt}
%\numberwithin{equation}{section}
% Only use if the above line is disabled and sequential equation numbering is desired.
%\usepackage{chngcntr}
%\counterwithout{equation}{chapter}
%\counterwithout{figure}{chapter}
%\counterwithout{table}{chapter}

%%% This is the main style file.
%\input{UNTdissertation.sty}

\begin{document}
\def\biblio{}

%\title{Variational Calculations of Positronium Scattering with Hydrogen}
%\author{Denton Woods}
%%\degree{Dissertation Prepared for the Degree of\\ Doctor of Philosophy}
%%\degreedate{May 2015}
%%\approved{Wild Bill Hickock, Major Professor\\
%%  Calamity Jane, Minor Professor\\
%%  Billy the Kid, Committee Mamber and\\
%%  \hspace{2em}  Famous Outlaw\\
%%  Robert B. Toulouse, Erstwhile Dean \\
%%  \hspace{2em} of the Graduate School}


%\clearpage
%\phantomsection
\pdfbookmark[1]{Title Page}{title} % Sets a PDF bookmark for the title page
%\maketitle
%\newpage

% Basic idea from https://www.sharelatex.com/blog/2013/08/09/thesis-series-pt5.html

\begin{titlepage}
    \begin{center}
        \vspace*{1cm}
        
        \LARGE
        \textbf{Variational Calculations of Positronium Scattering with Hydrogen}
        \vspace{0.5cm}
		
        \Large     
        \textbf{Denton Woods}
        
		\iftoggle{UNT}{
			\vspace{2.5cm}
		}{
			\vspace{4cm}
		}
        
		\normalsize
        Dissertation Prepared for the Degree of\\
        DOCTOR OF PHILOSOPHY
        
		\iftoggle{UNT}{
			\vspace{2.5cm}
		}{
			\vspace{4cm}
		}
		
        \normalsize
		UNIVERSITY OF NORTH TEXAS\\
		June 2015
    \end{center}

	\vfill
  
	\hspace{8cm} APPROVED:
	\vspace{0.1cm}
	
	\hspace{8cm} S.J. Ward, Major Professor

	\hspace{8cm} Peter Van Reeth, Collaborator

	\hspace{9cm} 	and Minor Professor

	\hspace{8cm} Duncan Weathers, Committee Member

	\hspace{8cm} David Shiner, Committee Member

	\hspace{8cm} Carlos Ordonez, Committee Member

	\hspace{8cm} David Schultz, Department Chair

\end{titlepage}

\thispagestyle{plain}
\begin{center}
    \large
    \textbf{Abstract}
\end{center}

Lorem ipsum dolor...

% From http://tex.stackexchange.com/questions/45809/acknowledgement-dedication-problem
\clearpage
\thispagestyle{plain}
\begin{center}
    \large
    \textbf{Acknowledgements}
\end{center}

Many people helped me through this project. Special thanks goes to Ryan 
Bosca, who helped me at times and let me bounce ideas off of him. Keri Ward
and Lauren Murphy have been part of a great support network that helped me
survive this process.

I received several travel grants to help me go to and present at conferences, 
including from DAMOP, UNT's Student Government Association, and UNT's College 
of Arts and Sciences. Dr. Ward received a grant from the National Science 
Foundation under grant no. PHYS-0968638 and another from UNT through the UNT 
faculty research grant GA9150, both of which supported my research.

Computational resources were provided by UNT’s High Performance Computing 
Services, a project of Academic Computing and User Services division of the 
University Information Technology with additional support from UNT Office of 
Research and Economic Development.


%%%%%%\setcounter{tocdepth}{2}
\hypersetup{linktocpage}
\hypersetup{
    colorlinks,
    citecolor=black,
    filecolor=black,
    linkcolor=black,
    urlcolor=black
}
%\pagestyle{fancy}
%\headheight 35pt
%\pdfpagewidth 8.5in
%\pdfpageheight 11in

\pagenumbering{roman}
\tableofcontents
\newpage
\listoffigures
\listoftables
\newpage
\pagenumbering{arabic}


\clearpage
\chapter*{Nomenclature}
\label{chp:nomenclature}
\addcontentsline{toc}{chapter}{Nomenclature}
\begin{table*}
	\begin{tabular}{l l}
short-short & short-range--short-range \\
long-short & long-range--short-range \\
long-long & long-range--long-range \\
$^1S$, $^3S$ & S-wave singlet, S-wave triplet  \\
CC & close coupling                \\
SVM & stochastic variational method  \\
DMC & diffusion Monte Carlo \\
CVM & confined variational method \\
CI & configuration interaction \\
SE & static-exchange \\
Ps & positronium \\
PsH & positronium hydride \\
	\end{tabular}
\end{table*}

\subfile{Introduction.tex}
\subfile{BoundState.tex}
\subfile{Theory.tex}
\subfile{Computation.tex}
%%%%%%\settocdepth{subsection}
\subfile{S-Wave.tex}
%%%%%%\settocdepth{section}
\subfile{P-Wave.tex}
\subfile{D-Wave.tex}
\subfile{General.tex}
\subfile{HigherWaves.tex}
\subfile{CrossSections.tex}
\subfile{EffectiveRangeTheories.tex}
\subfile{Exponential.tex}
\subfile{Conclusion.tex}

\appendix
\subfile{AngularIntegration.tex}
\subfile{ExtraDerivations.tex}
\subfile{RPowersCoeffs.tex}
\subfile{WLimits.tex}
\subfile{ExtraNumerics.tex}
\subfile{Programs.tex}

%\addcontentsline{toc}{chapter}{References}
\newpage
%\bibliographystyle{h-physrev3}
%\bibliographystyle{h-physrev}
%\bibliographystyle{h-elsevier}
%\bibliographystyle{iopart-num}
%\bibliographystyle{plainnat}
%\bibliographystyle{plainurl}
%\bibliographystyle{jphysicsB}
%\bibliographystyle{unsrt}
%\bibliographystyle{utphysLSB}
\bibliographystyle{UNTamsplain}

\bibliography{Dissertation}

\end{document}