% -*- root: Dissertation.tex -*-
\documentclass[Dissertation.tex]{subfiles} 
\begin{document}


\chapter{P-Wave}
\label{chp:PWave}

Like the S-wave in \cref{chp:SWave}, the general formalism in
\cref{chp:WaveKohn} was developed much later than the P-wave derivations and code 
were developed. So though the analysis in \cref{chp:WaveKohn} will work 
properly for the P-wave, it is worthwhile to show the exact formulas that are 
used in the P-wave long-range code.

The part of $\mathcal{L}C_1$ calculated by the \emph{Mathematica} code in \cref{fig:LCMath} is
\begin{align}
\frac{1}{2} & \left(\Laplacian_\rho + \kappa^2\right) \SphericalHarmonicY{1}{0}{\theta_\rho}{\varphi_\rho} n_1(\kappa\rho) f_1(\rho)  \nonumber \\
&= -\frac{\rho  f^{\prime\prime}(\rho ) \left[\kappa  \rho  \sin (\kappa  \rho )+\cos (\kappa  \rho )\right]+2 f^\prime(\rho ) \left[\left(\kappa ^2 \rho ^2-1\right) \cos (\kappa  \rho )-\kappa  \rho  \sin (\kappa \rho )\right]}{2 \kappa ^2 \rho ^3}
\end{align}

\todoi{How it relates to the following}

\section{P-Wave Wavefunction}
\label{sec:PWaveFn}
We include two short-range symmetries for the P-wave, so that \cref{eq:GeneralWaveTrial} becomes
\beq
\label{eq:PWaveTrial}
\Psi_1^{\pm,t} = \widetilde{S}_1 + L_1^{\pm,t} \, \widetilde{C}_1 + \sum_{i=1}^{N(\omega)} c_i \bar{\phi}_{1i} + \sum_{j=1}^{N(\omega)} d_j \bar{\phi}_{2j}.
\eeq
Like for the S-wave and as mentioned in \cref{sec:KohnApplied}, we compute only matrix elements with the barred versions of $S_1$ and $C_1$, 

\todoi{Describe $N'(\omega)$}
\todoi{Pairing terms - how really 2x}

\noindent To simplify the notation, $\widetilde{S}_1$ and $\widetilde{C}_1$
can be written as
\begin{subequations}
\label{eq:PWaveSandCBar}
\begin{align}
\bar{S}_1 &= \frac{Y_{10}(\theta_\rho,\varphi_\rho)S_{22} \pm Y_{10}(\theta_{\rho^\prime},\varphi_{\rho'})S_{23} }{\sqrt{2}} \label{eq:PWaveSBar} \\
\bar{C}_1 &= \frac{Y_{10}(\theta_\rho,\varphi_\rho)C_{22} \pm Y_{10}(\theta_{\rho^\prime},\varphi_{\rho'})C_{23} }{\sqrt{2}} \label{eq:PWaveCBar} 
\end{align}
\end{subequations}
with
\begin{subequations}
\label{eq:PWaveSandC}
\begin{alignat}{2}
S_{22} &={}&\Phi_{Ps}\left(r_{12}\right) \Phi_H\left(r_3\right) &\sqrt{2\kappa} \,j_1\!\left(\kappa\rho\right) \label{eq:PWaveS22Def} \\
S_{23} &={}&\Phi_{Ps}\left(r_{13}\right) \Phi_H\left(r_2\right) &\sqrt{2\kappa} \,j_1\!\left(\kappa\rho^\prime\right) \label{eq:PWaveS23Def} \\
C_{22} &={}-&\Phi_{Ps}\left(r_{12}\right) \Phi_H\left(r_3\right) &\sqrt{2\kappa} \,n_1\!\left(\kappa\rho\right) f_1(\rho) \label{eq:PWaveC22Def} \\
C_{23} &={}-&\Phi_{Ps}\left(r_{13}\right) \Phi_H\left(r_2\right) &\sqrt{2\kappa} \,n_1\!\left(\kappa\rho^\prime\right) f_1(\rho^\prime). \label{eq:PWaveC23Def}
\end{alignat}
\end{subequations}

\noindent The shielding factor, $f_1(\rho)$, is given by
\beq
f_1(\rho) = \left[1 - \ee^{-\mu \rho} \left(1+\frac{\mu}{2}\rho\right)\right]^3.
\label{eq:PWaveShielding}
\eeq
The first and second derivatives of this shielding factor are needed in the derivations of the matrix elements, and it is worthwhile to work them out separately.
\begin{subequations}
\label{eq:PWaveShieldingDer}
\begin{align}
f_1^\prime(\rho) &= \frac{3}{2} \ee^{-\mu \rho} \mu \left[1 - \ee^{-\mu \rho} \left(1+\frac{\mu}{2}\rho\right)\right]^2 (1 + \mu\rho) \label{eq:PWaveShielding1} \\
f_1^{\prime\prime}(\rho) &= \frac{3}{2} \ee^{-\mu \rho} \mu^2 \left[1 - \ee^{-\mu \rho} \left(1+\frac{\mu}{2}\rho\right)\right] \left[-\mu\rho + \ee^{-\mu \rho} \left(1+3\mu\rho+\frac{3}{2}\mu^2\rho^2\right)\right] \label{eq:PWaveShielding2}
\end{align}
\end{subequations}

\noindent The short-range terms are given by
\begin{subequations}
\label{eq:PWavePhiBar}
\begin{align}
\bar{\phi}_{1i} &= \left(1 \pm P_{23}\right) Y_{10}(\theta_1) r_1 \phi_i \label{eq:PWavePhi1i}\\
\bar{\phi}_{2j} &= \left(1 \pm P_{23}\right) Y_{10}(\theta_2) r_2 \phi_j \label{eq:PWavePhi2j},
\end{align}
\end{subequations}
where $\phi_i$ and $\phi_j$ are given by (\ref{eq:PhiDef}).  We also use the shortcuts
\begin{subequations}
\label{eq:PWavePhi}
\begin{align}
\phi_{1i} &= r_1 \phi_i \\
\phi_{2j} &= r_2 \phi_j.
\end{align}
\end{subequations}
We refer to the short-range Hylleraas-type terms $\phi_{1i}$ and $\phi_{2j}$ as
the first and second symmetries, respectively.

The cosine factors present in many of the matrix element equations are easily expressed in terms of $r_i$ and $r_{ij}$ by using the law of cosines \cite[p.174]{CRC1978}:
\beq
\label{eq:Cosines}
\cos\theta_{12} = \frac{r_1^2 + r_2^2 - r_{12}^2}{2 r_1 r_2}, \ \ \ \ \cos\theta_{13} = \frac{r_1^2 + r_3^2 - r_{13}^2}{2 r_1 r_3} \ \ \ \text{and}  \ \ \cos\theta_{23} = \frac{r_2^2 + r_3^2 - r_{23}^2}{2 r_2 r_3}.
\eeq

\noindent The $j_1(\kappa\rho)$ and $n_1(\kappa\rho)$ are the spherical Bessel and Neumann functions given by \cite[p.729]{Arfken2005}
\begin{subequations}
\label{eq:PWaveBessel}
\begin{align}
j_1(x) & = \frac{\sin x}{x^2} - \frac{\cos x}{x} \label{eq:Bessel1} \\
n_1(x) & = -\frac{\cos x}{x^2} - \frac{\sin x}{x}. \label{eq:Neumann1}
\end{align}
\end{subequations}

\noindent The $Y_{10}(\theta)$ are an abbreviated form of the spherical harmonics, since there is azimuthal symmetry. For the P-wave, this is
\beq
\label{eq:PWaveSpherHarm}
Y_{10}(\theta) = Y_{10}(\theta,\varphi) = \sqrt{\frac{3}{4\pi}} \cos\theta.
\eeq


\section{First Formalism}


\subsection{P-Wave Matrix Equation}
\label{sec:PWaveMatrix}

Since the P-wave trial wavefunction has two sets of short-range terms, the matrix equation can be seen from \cref{eq:GeneralKohnMatrix} to be
{\small
\begin{equation}
\label{eq:PWaveKohnMatrix}
	\begin{bmatrix} 
	 (\widetilde{C},\mathcal{L}\widetilde{C}) & (\widetilde{C},\mathcal{L}\bar{\phi}_{11}) & \cdots & (\widetilde{C},\mathcal{L}\bar{\phi}_{1N}) & (\widetilde{C},\mathcal{L}\bar{\phi}_{21}) & \cdots & (\widetilde{C},\mathcal{L}\bar{\phi}_{2N})\\
	 (\bar{\phi}_{11},\mathcal{L}\widetilde{C}) & (\bar{\phi}_{11},\mathcal{L}\bar{\phi}_{11}) & \cdots & (\bar{\phi}_{11},\mathcal{L}\bar{\phi}_{1N}) & (\bar{\phi}_{11},\mathcal{L}\bar{\phi}_{21}) & \cdots & (\bar{\phi}_{11},\mathcal{L}\bar{\phi}_{2N})\\
	 \vdots & \vdots & \ddots & \vdots & \vdots & \ddots & \vdots \\
	 (\bar{\phi}_{1N},\mathcal{L}\widetilde{C}) & (\bar{\phi}_{1N},\mathcal{L}\bar{\phi}_{11}) & \cdots & (\bar{\phi}_{1N},\mathcal{L}\bar{\phi}_{1N}) & (\bar{\phi}_{1N},\mathcal{L}\bar{\phi}_{21}) & \cdots & (\bar{\phi}_{1N},\mathcal{L}\bar{\phi}_{2N})\\
	 (\bar{\phi}_{21},\mathcal{L}\widetilde{C}) & (\bar{\phi}_{21},\mathcal{L}\bar{\phi}_{11}) & \cdots & (\bar{\phi}_{21},\mathcal{L}\bar{\phi}_{1N}) & (\bar{\phi}_{21},\mathcal{L}\bar{\phi}_{21}) & \cdots & (\bar{\phi}_{21},\mathcal{L}\bar{\phi}_{2N})\\
	 \vdots & \vdots & \ddots & \vdots & \vdots & \ddots & \vdots \\
	 (\bar{\phi}_{2N},\mathcal{L}\widetilde{C}) & (\bar{\phi}_{2N},\mathcal{L}\bar{\phi}_{11}) & \cdots & (\bar{\phi}_{2N},\mathcal{L}\bar{\phi}_{1N}) & (\bar{\phi}_{2N},\mathcal{L}\bar{\phi}_{21}) & \cdots & (\bar{\phi}_{2N},\mathcal{L}\bar{\phi}_{2N})\\
	\end{bmatrix}
	\begin{bmatrix}
  L_\ell^t\\
	c_1\\
	\vdots\\
	c_N\\
	d_1\\
	\vdots\\
	d_N\\
	\end{bmatrix}
	= -
	\begin{bmatrix}
	(\widetilde{C},\mathcal{L}\widetilde{S}) \\
	(\bar{\phi}_{11},\mathcal{L}\widetilde{S}) \\
	\vdots \\
	(\bar{\phi}_{1N},\mathcal{L}\widetilde{S}) \\
	(\bar{\phi}_{21},\mathcal{L}\widetilde{S}) \\
	\vdots \\
	(\bar{\phi}_{2N},\mathcal{L}\widetilde{S}) \\
	\end{bmatrix}.
\end{equation}
}
The analysis in \cref{sec:KohnApplied} applies directly to the P-wave.


\subsection{P-Wave Short-Range -- Short-Range Integrals}
\label{sec:PWaveShortShort}

Due to the non-constant spherical harmonics, the P-wave short-short integrals
are more complicated than the S-wave. Using the results in \cref{sec:AngularInt},
these integrals with both symmetries can be written as shown in
\cref{eq:PWavePhi1Phi1,eq:PWavePhi2Phi2,eq:PWavePhi1Phi2,eq:PWavePhi2Phi1}.
Each of these equations has re-expressed the Laplacian as the gradient-gradient,
allowing us to use \cref{eq:GradGradShort}.

Full derivations are available from the Research Wiki \cite{Wiki}.
Tables showing the coefficients and powers of $r_i$ and $r_{ij}$ for each of 
these equations is shown in \cref{sec:RPowersCoeffsP}. These tables allow a 
direct correspondence between the derivations and the code described in
\cref{chp:Programs}. I wrote a \emph{Mathematica} notebook
(see \cref{chp:RPowersCoeffs}) that transforms
these equations into a form more suitable for direct use in the C++ code.


\begin{align}
\label{eq:PWavePhi1Phi1}
\left(\bar{\phi}_{1i},\mathcal{L} \bar{\phi}_{1j}\right) = &2 \cdot 2\pi \int{ \Bigg\{ \sum_{k=1}^3 \left[ \boldsymbol{\nabla}_{\!\mathbf{r}_k} \nonumber \phi_{1i} \boldsymbol{\cdot} \boldsymbol{\nabla}_{\!\mathbf{r}_k} \phi_{1j} \pm \boldsymbol{\nabla}_{\!\mathbf{r}_k} \phi_{1i} \boldsymbol{\cdot} \boldsymbol{\nabla}_{\!\mathbf{r}_k} \phi_{1j}^\prime \right] } \\
\nonumber  &+ \left. \left[\frac{2}{r_1} - \frac{2}{r_2} - \frac{2}{r_3} - \frac{2}{r_{12}} - \frac{2}{r_{13}} + \frac{2}{r_{23}} - 2 E_H - 2 E_{Ps} - \frac{1}{2}\kappa^2 + \frac{2}{r_1^2} \right] \right. \\
 &\;\;\;\;\; \times \left(\phi_{1i} \phi_{1j} \pm \phi_{1i} \phi_{1j}^\prime \right) \Bigg\} d\tau_{int}
\end{align}

\begin{align}
\label{eq:PWavePhi2Phi2}
\left(\bar{\phi}_{2i},\mathcal{L} \bar{\phi}_{2j}\right) = 2 & \cdot 2\pi \int \Bigg\{ \sum_{k=1}^3 \left[ \boldsymbol{\nabla}_{\!\mathbf{r}_k} \nonumber \phi_{2i} \boldsymbol{\cdot} \boldsymbol{\nabla}_{\!\mathbf{r}_k} \phi_{2j} \pm \cos\theta_{23} \boldsymbol{\nabla}_{\!\mathbf{r}_k} \phi_{2i} \boldsymbol{\cdot} \boldsymbol{\nabla}_{\!\mathbf{r}_k} \phi_{2j}^\prime \right]  + \frac{2}{r_2^2}\phi_{2i}\phi_{2j} \\
 \nonumber &\mp \phi_{2i} \phi_{2j}^\prime \left[p_i \frac{r_1}{r_3 r_{13}^2} (\cos\theta_{12} - \cos\theta_{23} \cos\theta_{13}) + m_j^\prime \frac{r_1}{r_2 r_{12}^2}(\cos\theta_{13}-\cos\theta_{23} \cos\theta_{12})\right.\\
 \nonumber & \left. \;\;\;\;\;  + \sin^2\theta_{23} \left(q_i \frac{r_2}{r_3 r_{23}^2} + q_j^\prime \frac{r_3}{r_2 r_{23}^2} \right) \right] \\
 \nonumber &+ \left. \left[\frac{2}{r_1} - \frac{2}{r_2} - \frac{2}{r_3} - \frac{2}{r_{12}} - \frac{2}{r_{13}} + \frac{2}{r_{23}} - 2 E_H - 2 E_{Ps} - \frac{1}{2}\kappa^2 \right] \right. \\
 &\;\;\;\;\; \times \left(\phi_{2i} \phi_{2j} \pm \cos\theta_{23} \phi_{2i} \phi_{2j}^\prime \right) \Bigg\} d\tau_{int}
\end{align}

\begin{align}
\label{eq:PWavePhi1Phi2}
\left(\bar{\phi}_{1i},\mathcal{L} \bar{\phi}_{2j}\right) = 2 & \cdot 2\pi \int \Bigg\{ \sum_{k=1}^3 \left[ \cos\theta_{12} \boldsymbol{\nabla}_{\!\mathbf{r}_k} \nonumber \phi_{1i} \boldsymbol{\cdot} \boldsymbol{\nabla}_{\!\mathbf{r}_k} \phi_{2j} \pm \cos\theta_{13} \boldsymbol{\nabla}_{\!\mathbf{r}_k} \phi_{1i} \boldsymbol{\cdot} \boldsymbol{\nabla}_{\!\mathbf{r}_k} \phi_{2j}^\prime \right] \\
 \nonumber &\mp \phi_{1i} \phi_{2j} \left[q_i \frac{r_3}{r_2 r_{23}^2} (\cos\theta_{13} - \cos\theta_{12} \cos\theta_{23}) + p_j \frac{r_3}{r_1 r_{13}^2}(\cos\theta_{23}-\cos\theta_{12} \cos\theta_{13})\right.\\
 \nonumber & \left. \;\;\;\;\;  + \sin^2\theta_{12} \left(m_i \frac{r_1}{r_2 r_{12}^2} + m_j \frac{r_2}{r_1 r_{12}^2} \right) \right] \\
 \nonumber &\mp \phi_{1i} \phi_{2j}^\prime \left[q_i \frac{r_2}{r_3 r_{23}^2} (\cos\theta_{12} - \cos\theta_{13} \cos\theta_{23}) + m_j^\prime \frac{r_2}{r_1 r_{12}^2}(\cos\theta_{23}-\cos\theta_{12} \cos\theta_{13})\right.\\
 \nonumber & \left. \;\;\;\;\;  + \sin^2\theta_{13} \left(p_i \frac{r_1}{r_3 r_{13}^2} + p_j^\prime \frac{r_3}{r_1 r_{13}^2} \right) \right] \\
 \nonumber &+ \left. \left[\frac{2}{r_1} - \frac{2}{r_2} - \frac{2}{r_3} - \frac{2}{r_{12}} - \frac{2}{r_{13}} + \frac{2}{r_{23}} - 2 E_H - 2 E_{Ps} - \frac{1}{2}\kappa^2 \right] \right. \\
 &\;\;\;\;\; \times \left(\cos\theta_{12} \phi_{1i} \phi_{2j} \pm \cos\theta_{13} \phi_{1i} \phi_{2j}^\prime \right) \Bigg\} d\tau_{int}
\end{align}

\begin{align}
\label{eq:PWavePhi2Phi1}
\left(\bar{\phi}_{2i},\mathcal{L} \bar{\phi}_{1j}\right) = 2 & \cdot 2\pi \int \Bigg\{ \sum_{k=1}^3 \cos\theta_{12} \left[ \boldsymbol{\nabla}_{\!\mathbf{r}_k} \nonumber \phi_{2i} \boldsymbol{\cdot} \boldsymbol{\nabla}_{\!\mathbf{r}_k} \phi_{1j} \pm \boldsymbol{\nabla}_{\!\mathbf{r}_k} \phi_{2i} \boldsymbol{\cdot} \boldsymbol{\nabla}_{\!\mathbf{r}_k} \phi_{1j}^\prime \right] \\
 \nonumber &\mp \phi_{2i} \phi_{1j} \left[p_i \frac{r_3}{r_1 {r_{13}}^2} (\cos\theta_{23} - \cos\theta_{12} \cos\theta_{13}) + q_j \frac{r_3}{r_2 {r_{23}}^2}(\cos\theta_{13}-\cos\theta_{12} \cos\theta_{23})\right.\\
 \nonumber & \left. \;\;\;\;\;  + \sin^2\theta_{12} \left(m_i \frac{r_2}{r_1 {r_{12}}^2} + m_j \frac{r_1}{r_2 {r_{12}}^2} \right) \right] \\
 \nonumber &\mp \phi_{2i} \phi_{1j}^\prime \left[p_i \frac{r_3}{r_1 r_{13}^2} (\cos\theta_{23} - \cos\theta_{12} \cos\theta_{13}) + q_j^\prime \frac{r_3}{r_2 {r_{23}}^2}(\cos\theta_{13}-\cos\theta_{12} \cos\theta_{23})\right.\\
 \nonumber & \left. \;\;\;\;\;  + \sin^2\theta_{12} \left(m_i \frac{r_2}{r_1 {r_{12}}^2} + m_j^\prime \frac{r_1}{r_2 {r_{12}}^2} \right) \right] \\
 \nonumber &+ \left. \left[\frac{2}{r_1} - \frac{2}{r_2} - \frac{2}{r_3} - \frac{2}{r_{12}} - \frac{2}{r_{13}} + \frac{2}{r_{23}} - 2 E_H - 2 E_{Ps} - \frac{1}{2}\kappa^2 \right] \right. \\
 &\;\;\;\;\; \times \cos\theta_{12} \left(\phi_{2i} \phi_{12j} \pm \phi_{2i} \phi_{1j}^\prime \right) \Bigg\} d\tau_{int}
\end{align}


\subsection{P-Wave Short-Range -- Long-Range Integrals}
\label{sec:PWaveShortLong}

Like the S-wave in \cref{sec:LCElements}, we derived these and wrote code using these results before we developed a general formalism described in \cref{chp:WaveKohn}. They are equivalent, but it is easier to compare these to the P-wave long-range code.

\begin{align}
\label{eq:PWavePhi1SBar}
\nonumber \left(\bar{\phi}_{1i},\mathcal{L} \bar{S}_1\right) = \sqrt{2} \pi & \int \left[ \frac{r_1 + r_2 \cos\theta_{12}}{\rho} \left(\phi_{1i} \pm \phi_{1i}^\prime \right) \left(\frac{2}{r_1} - \frac{2}{r_2} - \frac{2}{r_{13}} + \frac{2}{r_{23}} \right) S_{22} \right] d\tau_{int} \\
\nonumber = \sqrt{2} \pi & \int \phi_{1i} \left[ \frac{r_1 + r_2 \cos\theta_{12}}{\rho} \left( \frac{2}{r_1} - \frac{2}{r_2} - \frac{2}{r_{13}} + \frac{2}{r_{23}} \right) S_{22} \right. \\
& \pm \left. \frac{r_1 + r_3 \cos\theta_{13}}{\rho^\prime} \left( \frac{2}{r_1} - \frac{2}{r_3} - \frac{2}{r_{12}} + \frac{2}{r_{23}} \right) S_{23} \right]  d\tau_{int}
\end{align}

\todoi{Get these to fit on fully on the page now that they're 12 point.}

\begin{align}
\label{eq:PWavePhi1CBar}
\nonumber \left(\bar{\phi}_{1i},\mathcal{L} \bar{C}_1\right) = -2 \pi &\sqrt{\kappa} \int \phi_{1i} \left\{ \frac{r_1 + r_2 \cos\theta_{12}}{\rho} \Phi_{Ps}(r_{12}) \Phi_H(r_3) \left[ \left( \frac{2}{r_1} - \frac{2}{r_2} - \frac{2}{r_{13}} + \frac{2}{r_{23}} \right) n_1(\kappa\rho) f_1(\rho) \right. \right. \\
\nonumber & + \left.\left. \left[f_1^\prime(\rho) \frac{1}{\rho} \left( n_1(\kappa\rho) + \cos(\kappa\rho) \right) - \frac{1}{2} f_1^{\prime\prime}(\rho) n_1(\kappa\rho) \right]\right] \pm \frac{r_1 + r_3 \cos\theta_{13}}{\rho^\prime}  \Phi_{Ps}(r_{13}) \Phi_H(r_2) \right. \\
\nonumber & \times \left. \left[ \left( \frac{2}{r_1} - \frac{2}{r_3} - \frac{2}{r_{12}} + \frac{2}{r_{23}} \right) n_1(\kappa\rho^\prime) f_1(\rho^\prime) + \left[f_1^\prime(\rho^\prime) \frac{1}{\rho^\prime} \left( n_1(\kappa\rho^\prime) + \cos(\kappa\rho^\prime) \right) - \frac{1}{2} f_1^{\prime\prime}(\rho^\prime) n_1(\kappa\rho^\prime) \right]\right]\right\} d\tau_{int} \\
\nonumber = -2 \pi &\sqrt{\kappa} \int \left\{ \left( \phi_{1i} \pm \phi_{1i}^\prime \right) \frac{r_1 + r_2 \cos\theta_{12}}{\rho} \Phi_{Ps}(r_{12}) \Phi_H(r_3) \left[ \left( \frac{2}{r_1} - \frac{2}{r_2} - \frac{2}{r_{13}} + \frac{2}{r_{23}} \right) n_1(\kappa\rho) f_1(\rho) \right.\right. \\
& + \left.\left. \left[ f_1^\prime (\rho) \frac{1}{\rho} \left( n_1(\kappa\rho) + \cos(\kappa\rho) \right) - \frac{1}{2} f_1^{\prime\prime}(\rho) n_1(\kappa\rho) \right] \right] \right\} d\tau_{int}
\end{align}

\begin{align}
\label{eq:PWavePhi2SBar}
\nonumber \left(\bar{\phi}_{2j},\mathcal{L} \bar{S}_1\right) = \sqrt{2} \pi & \int \left\{ \frac{1}{\rho} \left[ \phi_{2j} \left(r_2 + r_1 \cos\theta_{12}\right) \pm \phi_{2j}^\prime \left(r_1 \cos\theta_{13} + r_2 \cos\theta_{23} \right) \right] \left(\frac{2}{r_1} - \frac{2}{r_2} - \frac{2}{r_{13}} + \frac{2}{r_{23}} \right) S_{22} \right\} d\tau_{int} \\
\nonumber = \sqrt{2} \pi & \int \phi_{2j} \left[ \frac{1}{\rho} \left(r_2 + r_1 \cos\theta_{12}\right) \left( \frac{2}{r_1} - \frac{2}{r_2} - \frac{2}{r_{13}} + \frac{2}{r_{23}} \right) S_{22} \right. \\
& \pm \left. \frac{1}{\rho^\prime} \left(r_1 \cos\theta_{12} + r_3 \cos\theta_{23}\right) \left( \frac{2}{r_1} - \frac{2}{r_3} - \frac{2}{r_{12}} + \frac{2}{r_{23}} \right) S_{23} \right]  d\tau_{int}
\end{align}

\begin{align}
\label{eq:PWavePhi2CBar}
\nonumber \left(\bar{\phi}_{2j},\mathcal{L} \bar{C}_1\right) = -2 \uppi &\sqrt{\kappa} \int \phi_{2j} \left\{ \frac{r_2 + r_1 \cos\theta_{12}}{\rho} \Phi_{Ps}(r_{12}) \Phi_H(r_3) \left[ \left( \frac{2}{r_1} - \frac{2}{r_2} - \frac{2}{r_{13}} + \frac{2}{r_{23}} \right) n_1(\kappa\rho) f_1(\rho) \right. \right. \\
\nonumber & + \left.\left. \left[f_1^\prime(\rho) \frac{1}{\rho} \left( n_1(\kappa\rho) + \cos(\kappa\rho) \right) - \frac{1}{2} f_1^{\prime\prime}(\rho) n_1(\kappa\rho) \right]\right] \pm \frac{r_1 \cos\theta_{12} + r_3 \cos\theta_{23}}{\rho^\prime}  \Phi_{Ps}(r_{13}) \Phi_H(r_2) \right. \\
\nonumber & \times \left. \left[ \left( \frac{2}{r_1} - \frac{2}{r_3} - \frac{2}{r_{12}} + \frac{2}{r_{23}} \right) n_1(\kappa\rho^\prime) f_1(\rho^\prime) + \left[f_1^\prime(\rho^\prime) \frac{1}{\rho^\prime} \left( n_1(\kappa\rho^\prime) + \cos(\kappa\rho^\prime) \right) - \frac{1}{2} f_1^{\prime\prime}(\rho^\prime) n_1(\kappa\rho^\prime) \right]\right]\right\} \\
\nonumber = -2 \uppi &\sqrt{\kappa} \int \left( \phi_{2j} \frac{r_2 + r_1 \cos\theta_{12}}{\rho} \pm \phi_{2j}^\prime \frac{r_1 \cos\theta_{13} + r_2 \cos\theta_{23}}{\rho} \right) \\
& \times \Phi_{Ps}(r_{12}) \Phi_H(r_3) \left\{ \left( \frac{2}{r_1} - \frac{2}{r_2} - \frac{2}{r_{13}} + \frac{2}{r_{23}} \right) n_1(\kappa\rho) f_1(\rho) \right. \\
& + \left. \left[ f_1^\prime (\rho) \frac{1}{\rho} \left( n_1(\kappa\rho) + \cos(\kappa\rho) \right) - \frac{1}{2} f_1^{\prime\prime}(\rho) n_1(\kappa\rho) \right] \right\} d\tau_{int}
\end{align}


\subsection{P-Wave Long-Range -- Long-Range Integrals}
\label{sec:PWaveLongLong}

\begin{align}
\label{eq:PWaveSBarSBar}
\left(\bar{S}_1,L\bar{S}_1\right) = \pm \frac{\uppi}{2} \int \frac{1}{\rho\rho^\prime} & \left[S_{22} S_{23} \left(\frac{2}{r_1} - \frac{2}{r_2} - \frac{2}{r_{13}} + \frac{2}{r_{23}} \right) \right.  \nonumber \\
& \left. \times \left(r_1^2 + r_1 r_2 \cos\theta_{12} + r_1 r_3 \cos\theta_{13} + r_2 r_3 \cos\theta_{23} \right) \right] d\tau_{int}
\end{align}

\begin{align}
\label{eq:PWaveCBarSBar}
\left(\bar{C}_1,L\bar{S}_1\right) = \pm \frac{\uppi}{2} \int \frac{1}{\rho\rho^\prime} & \left[C_{23} S_{22} \left(\frac{2}{r_1} - \frac{2}{r_2} - \frac{2}{r_{13}} + \frac{2}{r_{23}} \right) \right.  \nonumber \\
& \left. \times \left(r_1^2 + r_1 r_2 \cos\theta_{12} + r_1 r_3 \cos\theta_{13} + r_2 r_3 \cos\theta_{23} \right) \right] d\tau_{int}
\end{align}

\begin{align}
\label{eq:PWaveCBarCBar}
\nonumber \left(\bar{C}_1,L\bar{C}_1\right) = 2 \uppi \kappa & \int \Phi_{Ps}(r_{12}) \Phi_H(r_3) \Bigg\{ 2 \Phi_{Ps}(r_{12}) \Phi_H(r_3) n_1(\kappa\rho) f_1(\rho) \\
\nonumber & \times \left(\frac{1}{\rho} f_1^\prime(\rho) \left(n_1(\kappa\rho) + \cos(\kappa\rho)\right) - \frac{1}{2} f_1^{\prime\prime}(\rho) n_1(\kappa\rho)\right) \\
\nonumber \pm & \frac{1}{2\rho\rho^\prime} (r_1^2 + r_1 r_2 \cos\theta_{12} + r_1 r_3 \cos\theta_{13} + r_2 r_3 \cos\theta_{23}) \\
\nonumber & \times \Phi_{Ps}(r_{13}) \Phi_H(r_2) n_1(\kappa\rho^\prime) f_1(\rho^\prime) \\
\nonumber & \times \left[ n_1(\kappa\rho) f_1(\kappa\rho) \left(\frac{2}{r_1} - \frac{2}{r_2} - \frac{2}{r_{13}} + \frac{2}{r_{23}} \right) \right. \\
& \; \; + \left. \left(\frac{1}{\rho} f_1^\prime(\rho) \left(n_1(\kappa\rho) + \cos(\kappa\rho)\right) - \frac{1}{2} f_1^{\prime\prime}(\rho)  n_1(\kappa\rho)\right) \right]\Bigg\} d\tau_{int}
\end{align}


\section{P-Wave Second Formalism}
\label{sec:PWave2Formalism}

Van Reeth and Humberston had difficulty with convergence of the $^3$P phase shifts \cite{VanReeth2004}. They specifically state:
\begin{quote}
The present triplet p-wave phase shifts are not yet fully converged; at very low energies we
estimate them to be $\approx 20\%$ below the fully converged values and $\approx 3\%$ in the higher energy range. This relatively poor convergence is due to the fact
that we have not yet included in the wavefunction a set of terms for which the unit of angular momentum is on the electron in the H atom.
\end{quote}
Due to this, we looked at a second formalism that includes these short-range terms instead of what we call the first formalism in \cref{sec:PWaveFn}. 

\subsection{Wavefunction}
The trial wavefunction that we use for the second formalism is similar to the first formalism in \cref{sec:PWaveFn}.
\begin{equation}
\Psi_1^{\pm,t} = \widetilde{S}_1 + L_1^{\pm,t} \, \widetilde{C}_1 + \sum_{i=1}^{N'(\omega)} c_i \bar{\phi}_{\rho i} + \sum_{j=1}^{N'(\omega)} d_j \bar{\phi}_{3j}
\label{eq:PWave2ndWavefn}
\end{equation}

\noindent The new short-range terms are given by
\begin{subequations}
\label{eq:PWave2ndPhiBar}
\begin{align}
\bar{\phi}_{\rho i} &= \left(1 \pm P_{23}\right) Y_{10}(\theta_\rho) \rho \phi_i \label{eq:PWave2ndPhi1i}\\
\bar{\phi}_{3j} &= \left(1 \pm P_{23}\right) Y_{10}(\theta_3) r_3 \phi_j \label{eq:PWave2ndPhi2j}.
\end{align}
\end{subequations}
These place the angular momentum on the Ps and on the electron of H, instead of on the positron and the electron in the Ps. In the first formalism, some amount of the angular momentum is placed on the second electron through the $P_{23}$ permutation operator.

The short-long terms can be directly computed by changing the short-range terms in the code, but the integration routines (see \cref{sec:ShortInt}) for the short-short terms cannot handle the $\rho$ factor of $\bar{\phi}_{\rho 1}$ directly. The following relations must be used:
\begin{subequations}
\label{eq:P2rhoY10}
\begin{align}
\rho Y_{10}(\theta_\rho) &= \frac{1}{2}\left[ r_1 Y_{10}(\theta_1) + r_2 Y_{10}(\theta_2) \right] \\
\rho^\prime Y_{10}(\theta_{\rho^\prime}) &= \frac{1}{2}\left[ r_1 Y_{10}(\theta_1) + r_3 Y_{10}(\theta_3) \right].
\end{align}
\end{subequations}
The first of these is obtained by substituting
\begin{equation}
\label{eq:CosRho}
\cos\theta_\rho = \frac{r_1 \cos\theta_1 + r_2 \cos\theta_2}{2\rho}
\end{equation}
in the definition of \cref{eq:PWaveSpherHarm}. The second is found by using the $P_{23}$ swap.
\todoi{Does this show up elsewhere?}


\subsection{Short-Short Integrals}
\label{sec:PWave2ndShortShort}
In terms of the first formalism,
\beq
\left(\bar{\phi}_{\rho i},\mathcal{L} \bar{\phi}_{\rho j}\right) = \frac{1}{4} \left[ \left(\bar{\phi}_{1i},\mathcal{L} \bar{\phi}_{1j}\right)_{1st} + \left(\bar{\phi}_{1i},\mathcal{L} \bar{\phi}_{2j}\right)_{1st} + \left(\bar{\phi}_{2i},\mathcal{L} \bar{\phi}_{1j}\right)_{1st} + \left(\bar{\phi}_{2i},\mathcal{L} \bar{\phi}_{2j}\right)_{1st} \right].
\eeq

The other types of terms cannot be expressed as combinations of the first formalism results, but these calculations are similar to the first formalism.

\begin{align}
\nonumber \left(\bar{\phi}_{\rho i},\mathcal{L} \bar{\phi}_{3j}\right) = &(Y_{10}(\theta_1) r_1 \phi_i, \mathcal{L} Y_{10}(\theta_3) r_3 \phi_j) + (Y_{10}(\theta_2) r_2 \phi_i, \mathcal{L} Y_{10}(\theta_3) r_3 \phi_j) \\
\pm &(Y_{10}(\theta_1) r_1 \phi_i, \mathcal{L} Y_{10}(\theta_2) r_2 \phi_j^\prime) \pm (Y_{10}(\theta_2) r_2 \phi_i, \mathcal{L} Y_{10}(\theta_2) r_2 \phi_j^\prime)
\end{align}

\begin{align}
\nonumber \left(\bar{\phi}_{\rho i},\mathcal{L} \bar{\phi}_{3j}\right) = \pm &(Y_{10}(\theta_2) r_2 \phi_i^\prime, \mathcal{L} Y_{10}(\theta_1) r_1 \phi_j) \pm (Y_{10}(\theta_2) r_2 \phi_i^\prime, \mathcal{L} Y_{10}(\theta_2) r_2 \phi_j) \\
+ &(Y_{10}(\theta_2) r_2 \phi_i^\prime, \mathcal{L} Y_{10}(\theta_1) r_1 \phi_j^\prime) + (Y_{10}(\theta_2) r_2 \phi_i^\prime, \mathcal{L} Y_{10}(\theta_3) r_3 \phi_j^\prime)
\end{align}

\beq
\nonumber \left(\bar{\phi}_{\rho i},\mathcal{L} \bar{\phi}_{3j}\right) = 2 \left[ \pm(Y_{10}(\theta_2) r_2 \phi_i, \mathcal{L} Y_{10}(\theta_3) r_3 \phi_j) + (Y_{10}(\theta_2) r_2 \phi_i^\prime, \mathcal{L} Y_{10}(\theta_2) r_2 \phi_j^\prime) \right]
\eeq

\subsection{Short-Long Integrals}
\label{sec:PWave2ndShortLong}
In terms of the first formalism,
\begin{subequations}
\begin{align}
\left(\bar{\phi}_{\rho i},\mathcal{L} \bar{S}\right) &= \frac{1}{2} \left[ \left(\bar{\phi}_{1i},\mathcal{L} \bar{S}\right)_{1st} + \left(\bar{\phi}_{2i},\mathcal{L} \bar{S}\right)_{1st} \right] \\
\left(\bar{\phi}_{\rho i},\mathcal{L} \bar{C}\right) &= \frac{1}{2} \left[ \left(\bar{\phi}_{1i},\mathcal{L} \bar{C}\right)_{1st} + \left(\bar{\phi}_{2i},\mathcal{L} \bar{C}\right)_{1st} \right].
\end{align}
\end{subequations}
This is a straightforward adaptation of the code for the first formalism.


\subsection{Phase Shifts}
\label{sec:PWave2ndPhase}

\begin{table}[H]
\centering
\begin{tabular}{c c c c c}
\toprule
$\kappa$ & $1^{st}$ Formalism $\delta_1^+$ & $2^{nd}$ Formalism $\delta_1^+$ & $1^{st}$ Formalism $\delta_1^-$ & $2^{nd}$ Formalism $\delta_1^-$ \\
\midrule
0.1 & $2.26^{-2}$ & $2.27^{-2}$ & $-1.79^{-3}$ & $-1.80^{-3}$ \\
0.2 & $1.91^{-1}$ & $1.91^{-1}$ & $-1.68^{-2}$ & $-1.67^{-2}$ \\
0.3 & $6.08^{-1}$ & $6.08^{-1}$ & $-5.54^{-2}$ & $-5.52^{-2}$ \\
0.4 & $9.93^{-1}$ & $9.93^{-1}$ & $-1.15^{-1}$ & $-1.15^{-1}$ \\
0.5 & $1.14$      & $1.14$      & $-1.84^{-1}$ & $-1.84^{-1}$ \\
0.6 & $1.16$      & $1.16$      & $-2.49^{-1}$ & $-2.48^{-3}$ \\
0.7 & $1.15$      & $1.15$      & $-2.93^{-1}$ & $-2.93^{-1}$ \\
\bottomrule
\end{tabular}
\caption[Comparison of $^{1,3}$P phase shifts for the first and second formalisms]{Comparison of $^{1,3}$P phase shifts for the first and second formalisms $(\omega = 6)$. The $2^{nd}$ formalism for $\delta_1^-$ uses 889 terms, while the others use 924 terms.}
\label{tab:PWaveFormalPhase}
\end{table}

The phase shifts for $^1$P and $^3$P in \cref{tab:PWaveFormalPhase} were not 
improved by using the second formalism and were even lower in some cases. The 
one exception is $\kappa = 0.3$ for $^3$P, where the second formalism phase shift
is slightly higher. We note that this is the same results as the $\omega = 7$
case with 1000 terms as presented in \cref{tab:PWaveComparisons}.
The difficulty with the second formalism is that linear dependence becomes more
of an issue, forcing us to only use 889 terms for $\omega = 6$ and unable to
do an $\omega = 7$ run. I tried an $\omega = 7$ run with the Todd energy
program, but it could only select 767 terms, leading to worse phase shifts than
the $\omega = 6$ runs.

The energy eigenvalues in \cref{tab:SimplexPWaveSingOpt,tab:SimplexPWaveTripOpt}
were also lower for the first formalism, so we did not pursue using the second
formalism any further. The second formalism was not tried for the D-wave due to
its difficulty. Refer to \cref{sec:DSecondForm} for a discussion of this for
the D-wave.



\section{Results}

\subsection{Phase Shifts}

\Cref{tab:PWaveComparisons} compares the current complex Kohn results for the first P-wave formalism with that of other groups. We use $\omega = 4 - 7$ phase shifts to do the extrapolations for $\omega \rightarrow \infty$. The \% Diff entries give an estimate of the error for our results. The singlet results are converged very well. The triplet results are also converged well, but there is more possible error at $\kappa = 0.1$ and $\kappa = 0.7$. Even with these slightly larger percentages, our $^3$P phase shifts appear to be well converged.

\begin{table}[H]
\centering
\setlength{\tabcolsep}{-2pt}
\footnotesize
\begin{tabular}{@{\hskip 0.1cm}l . . . . . . .}
\toprule
Method & \multicolumn{1}{c}{\phantom{1}0.1} & \multicolumn{1}{c}{\phantom{1}0.2} & \multicolumn{1}{c}{\phantom{1}0.3} & \multicolumn{1}{c}{\phantom{1}0.4} & \multicolumn{1}{c}{\phantom{1}0.5} & \multicolumn{1}{c}{\phantom{1}0.6} & \multicolumn{1}{c}{\phantom{1}0.7} \\
\midrule
This work $(\omega = 7)$ $\delta_1^+$ 					& 0.226^{-1} & 0.191 & 0.609 & 0.994 & 1.140 & 1.162 & 1.152 \\
This work $(\omega \to \infty)$ $\delta_1^+$			& 0.227^{-1} & 0.192 & 0.611 & 0.996 & 1.142 & 1.163 & 1.154 \\
\% Diff$^+$												& 0.465\% & 0.306\% & 0.314\% & 0.205\% & 0.140\% & 0.137\% & 0.181\% \\
\arrayrulecolor[RGB]{220,220,220}\midrule\arrayrulecolor{black}
Kohn $(\omega = 6)$ \cite{VanReethPrivate} $\delta_1^+$	& 0.226^{-1} & 0.192 & 0.612 & 0.997 & 1.143 & 1.165 & 1.155 \\
CC 14Ps14H+H$^-$ \cite{Walters2004} $\delta_1^+$		& 0.221^{-1} & 0.183 & 0.580 & 0.956 & 1.106 & 1.134 & 1.133 \\
CC 14Ps14H \cite{Blackwood2002} $\delta_1^+$			& 0.213^{-1} & 0.175 & 0.545 & 0.908 & 1.068 & 1.103 & 1.099 \\
3-state CC \cite{Sinha1997} $\delta_1^+$				& 8.14^{-3} & 6.27^{-2} & 0.190 & 0.358 & 0.489 & 0.551 & 0.556 \\
SE \cite{Ray1997} $\delta_1^+$ 							& 0.798^{-2} & 0.614^{-1} & 0.186 & 0.349 & 0.477 & 0.536 & 0.538 \\
5-state CC \cite{Adhikari1999} $\delta_1^+$				& 0.477^{-2} & 0.370^{-1} & 0.116 & 0.239 & 0.372 & 0.478 & 0.541 \\
SE \cite{Hara1975} $\delta_1^+$							& 0.79^{-2}  & 0.611^{-1} & 0.1853 & 0.3487 & 0.4772 & 0.5361 & 0.5388 \\
\midrule                                                
This work $(\omega = 7)$ $\delta_1^-$					& -0.178^{-2} & -0.167^{-1} & -0.552^{-1} & -0.115 & -0.183 & -0.248 & -0.292 \\
This work $(\omega \to \infty)$ $\delta_1^-$			& -0.172^{-2} & -0.165^{-1} & -0.540^{-1} & -0.114 & -0.182 & -0.246 & -0.288 \\
\% Diff$^-$												& 3.176\% & 0.993\% & 0.749\% & 0.698\% & 0.749\% & 0.896\% & 1.237\% \\
\arrayrulecolor[RGB]{220,220,220}\midrule\arrayrulecolor{black}
CC 14Ps14H \cite{Blackwood2002} $\delta_1^-$			& -0.953^{-3} & -0.122^{-1} & -0.456^{-1} & -0.104 & -0.178 & -0.247 & -0.295 \\
3-state CC \cite{Sinha1997} $\delta_1^-$				& -4.43^{-3} & -3.08^{-2} & -8.51^{-2} & -0.159 & -0.236 & -0.302 & -0.332 \\
SE \cite{Ray1997} $\delta_1^-$							& -0.503^{-2} & -0.352^{-1} & -0.980^{-1} & -0.186 & -0.287 & -0.390 & -0.488 \\
5-state CC \cite{Adhikari1999} $\delta_1^-$				& -0.233^{-2} & -0.167^{-1} & -0.476^{-1} & -0.918^{-1} & -0.142 & -0.190 & -0.228 \\
SE \cite{Hara1975} $\delta_1^-$							& -0.50^{-2}  & -0.350^{-1} & -0.978^{-1} & -0.1860 & -0.2872 & -0.3906 & -0.4882 \\
\bottomrule
\end{tabular}
\caption[$^{1,3}$P comparisons]{$^{1,3}$P comparisons. \% Diff$^\pm$ is the percent difference
between the current complex Kohn $\omega = 7$ and $\omega \rightarrow \infty$
results. Values in the header are $\kappa$ in au. Exponents denote powers of 10.}
\label{tab:PWaveComparisons}
\end{table}

\Cref{tab:PWaveComparisons,fig:PWavePhase} give good comparisons with the CC results of the Belfast group \cite{Blackwood2002,Walters2004}. The complex Kohn phase shifts are higher for $^1$P, but they are slightly lower for much of the range for $^3$P. We note that this was also the case for the S-wave (see \cref{tab:SWaveComparisons} on \pageref{tab:SWaveComparisons}), but the CVM results seem to confirm the complex Kohn $^3$S phase shifts. Both the complex Kohn and the CC phase shifts of the Belfast group appear to give the most accurate sets of phase shifts, and the current complex Kohn can potentially be viewed as benchmark results.


%\begin{table}[H]
%\centering
%\begin{tabular}{c | c c c c c c c}
%\toprule
%$\kappa$ & $\delta^+ (\omega = 6)$ & $\delta^+ (\omega = 7)$ & $\delta^+$ \cite{VanReethPrivate} & $\delta^+$ \cite{Blackwood2002} & $\delta^+$ \cite{Walters2004} & $\delta^+$ \cite{Ray1997} & $\delta^+$ \cite{Adhikari1999} \\
%\midrule
%0.1 & $2.255^{-2}$ & $2.256^{-2}$ & $2.26^{-2}$ & $2.13^{-2}$ & $2.21^{-2}$ & $7.98^{-3}$ & $4.77^{-3}$ \\
%0.2 & $1.909^{-1}$ & $1.913^{-1}$ & $1.92^{-1}$ & $1.75^{-1}$ & $1.83^{-1}$ & $6.14^{-2}$ & $3.70^{-2}$ \\
%0.3 & $6.082^{-1}$ & $6.093^{-1}$ & $6.12^{-1}$ & $5.45^{-1}$ & $5.80^{-1}$ & $1.86^{-1}$ & $1.16^{-1}$ \\
%0.4 & $9.930^{-1}$ & $9.939^{-1}$ & $9.97^{-1}$ & $9.08^{-1}$ & $9.56^{-1}$ & $3.49^{-1}$ & $2.39^{-1}$ \\
%0.5 & $1.140$ & $1.140$ & $1.143$ & $1.068$ & $1.106$ & $4.77^{-1}$ & $3.72^{-1}$ \\
%0.6 & $1.161$ & $1.162$ & $1.165$ & $1.103$ & $1.134$ & $5.36^{-1}$ & $4.78^{-1}$ \\
%0.7 & $1.151$ & $1.152$ & $1.155$ & $1.099$ & $1.133$ & $5.38^{-1}$ & $5.41^{-1}$ \\
%\bottomrule
%\end{tabular}
%\caption{P-Wave Singlet Results}
%\label{tab:PWaveSinglet}
%\end{table}

%\begin{table}[H]
%\centering
%\begin{tabular}{c | c c c c c}
%\toprule
%$\kappa$ & $\delta^- (\omega = 6)$ & $\delta^- (\omega = 7)$ & $\delta^-$ \cite{Blackwood2002} & $\delta^-$ \cite{Ray1997} & $\delta^-$ \cite{Adhikari1999} \\
%\midrule
%0.1 & $-1.785^{-3}$ & $-1.775^{-3}$ & $-9.53^{-4}$ & $-5.03^{-3}$ & $-2.33^{-3}$ \\
%0.2 & $-1.676^{-2}$ & $-1.669^{-2}$ & $-1.22^{-2}$ & $-3.52^{-2}$ & $-1.67^{-2}$ \\
%0.3 & $-5.536^{-2}$ & $-5.518^{-2}$ & $-4.56^{-2}$ & $-9.80^{-2}$ & $-4.76^{-2}$ \\
%0.4 & $-1.149^{-1}$ & $-1.146^{-1}$ & $-1.04^{-1}$ & $-1.86^{-1}$ & $-9.18^{-2}$ \\
%0.5 & $-1.840^{-1}$ & $-1.834^{-1}$ & $-1.78^{-1}$ & $-2.87^{-1}$ & $-1.42^{-1}$ \\
%0.6 & $-2.486^{-1}$ & $-2.477^{-1}$ & $-2.47^{-1}$ & $-3.90^{-1}$ & $-1.90^{-1}$ \\
%0.7 & $-2.932^{-1}$ & $-2.917^{-1}$ & $-2.95^{-1}$ & $-4.88^{-1}$ & $-2.28^{-1}$ \\
%\bottomrule
%\end{tabular}
%\caption{P-Wave Triplet Results}
%\label{tab:PWaveTriplet}
%\end{table}


%\begin{table}[H]
%\centering
%\begin{tabular}{c c c c}
%\toprule
%$\kappa$ & $\delta^+ (\omega \rightarrow \infty) (3-6)$ & $\delta^+ (\omega \rightarrow \infty) (3-7)$ & $\delta^+ (\omega \rightarrow \infty) (4-7)$\\
%\midrule
%0.1 & $2.30^{-2}$ & $2.28^{-2}$ & $2.27^{-1}$ \\
%0.2 & $1.93^{-1}$ & $0.192^{-1}$ & $1.92^{-1}$ \\
%0.3 & $6.13^{-1}$ & $6.12^{-1}$ & $6.11^{-1}$ \\
%0.4 & $9.99^{-1}$ & $9.97^{-1}$ & $9.96^{-1}$ \\
%0.5 & $1.146$ & $1.144$ & $1.142$ \\
%0.6 & $1.170$ & $1.167$ & $1.163$ \\
%0.7 & $1.164$ & $1.159$ & $1.154$ \\
%\bottomrule
%\end{tabular}
%\caption{$^1$P extrapolation results}
%\label{tab:PWaveSingletExtrap}
%\end{table}
%
%\begin{table}[H]
%\centering
%\begin{tabular}{c c c c}
%\toprule
%$\kappa$ & $\delta^- (\omega \rightarrow \infty) (3-6)$ & $\delta^- (\omega \rightarrow \infty) (3-7)$ & $\delta^- (\omega \rightarrow \infty) (4-7)$\\
%\midrule
%0.1 & $-1.447^{-3}$ & $-1.606^{-3}$ & $-1.720^{-3}$ \\
%0.2 & $-1.586^{-2}$ & $-1.619^{-2}$ & $-1.652^{-2}$ \\
%0.3 & $-5.323^{-2}$ & $-5.477^{-2}$ & $-5.401^{-2}$ \\
%0.4 & $-1.111^{-1}$ & $-1.124^{-1}$ & $-1.138^{-1}$ \\
%0.5 & $-1.771^{-1}$ & $-1.797^{-1}$ & $-1.821^{-1}$ \\
%0.6 & $-2.373^{-1}$ & $-2.417^{-1}$ & $-2.455^{-1}$ \\
%0.7 & $-2.755^{-1}$ & $-2.822^{-1}$ & $-2.881^{-1}$ \\
%\bottomrule
%\end{tabular}
%\caption{$^3$P extrapolation results}
%\label{tab:PWaveTripletExtrap}
%\end{table}

\begin{figure}[H]
	\centering
	\includegraphics[width=\textwidth]{pwave-phases}
	\caption[$^{1,3}$P phase shifts]{$^{1,3}$P complex Kohn phase shifts. The $^1S$ CC phase shifts
\cite{Walters2004} are given by \mbox{\textcolor{blue}{$\times$}}, and the
$^3S$ CC phase shifts \cite{Blackwood2002} are given by
\mbox{\textcolor{red}{\textbf{+}}}. Vertical dashed lines denote the complex rotation resonance
positions \cite{Yan1999}.}
	\label{fig:PWavePhase}
\end{figure}

\begin{figure}[H]
	\centering
	\includegraphics[width=5.25in]{pwave-comparisons}
	\caption[Comparison of P-wave phase shifts]{Comparison of $^1$P (a) and $^3$P (b) phase shifts with results from other groups. Results are ordered according to year of publication. This work -- solid curves; \mbox{\textcolor{blue}{$\times$} -- CC \cite{Walters2004};} \mbox{$\CIRCLE$ -- Kohn \cite{VanReeth2003};} \mbox{\textcolor{red}{\textbf{+}} -- CC \cite{Blackwood2002};}\mbox{$\triangledown$ -- SVM 2002 \cite{Ivanov2002};} \mbox{\textcolor{red}{$\vartriangle$} -- 6-state CC \cite{Sinha2000};} \mbox{$\blacksquare$ -- 5-state CC \cite{Adhikari1999};} \mbox{$\vartriangle$ -- 3-state CC \cite{Sinha1997};} \mbox{\textcolor[RGB]{0,127,0}{$\bigstar$} -- CC \cite{Ray1997};} \mbox{\textcolor{blue}{$\lozenge$} -- Static-exchange \cite{Hara1975}.}}
	\label{fig:PWaveComparisons}
\end{figure}

\Cref{fig:PWaveComparisons} gives a detailed comparison to phase shifts from many other groups, similar to the S-wave comparisons in \cref{fig:SWaveComparisons}. The 5-state CC method of Adhikari and Biswas \cite{Adhikari1999} again gives phase shifts that do not agree well with other methods, and Ivanov et al. \cite{Ivanov2002} has a detailed discussion of this. The SVM results of Ivanov et al. \cite{Ivanov2002} follow along well with the current complex Kohn and the CC \cite{Blackwood2002,Walters2004} but are below throughout the energy range. We would not expect the SE \cite{Hara1975,Ray1997} and smaller CC calculations \cite{Sinha1997} to match our phase shifts well.


\subsection{Resonance Parameters}
\label{sec:PWaveResonances}

Similar to the S-wave in \cref{sec:SWaveResonances}, the P-wave has two 
resonances before the inelastic threshold, as seen in \cref{fig:PWavePhase}. 
Di Rienzi and Drachman found that the first $^1$P resonance is associated 
with the $3p$ state, not the $2p$ state as might be expected
\cite{DiRienzi2002b}. These resonances are shifted to higher energies than the
S-wave, and we use the numerical fittings described in \cref{sec:ResonanceFit}
to determine the resonance parameters given in \cref{tab:PWaveResonancesOther}.

\setlength{\abovecaptionskip}{6pt}   % 0.5cm as an example
\setlength{\belowcaptionskip}{6pt}   % 0.5cm as an example
\begin{table}[H]
\footnotesize
\centering
\begin{tabular}{l l l l l}
\toprule
Method & \thead{$^1E_R \text{ (eV)}$} & \thead{$^1\Gamma \text{ (eV)}$} & \thead{$^2E_R \text{ (eV)}$} & \thead{$^2\Gamma \text{ (eV)}$} \\
\midrule
Current work: Average $\pm$ standard deviation & $4.2856 \pm 0.0001$ & $0.0445 \pm 0.0001$ & $5.0577 \pm 0.0004$ & $0.0459 \pm 0.0005$ \\
Current work: $S$-matrix complex Kohn & $4.2856$ & $0.0445$ & $5.0579$ & $0.0459$ \\
Kohn variational \cite{VanReeth2004} & $4.29 \pm 0.01$ & $0.042 \pm 0.005$ & --- & --- \\
CC (9Ps9H + H$^-$) \cite{Walters2004} & $4.475$ & $0.0827$ & $4.905$ & $0.0043$ \\
Stabilization \cite{Yan2003} & $4.287$ & $0.0446$ & $5.062$ & $0.0563$ \\
CC (22Ps1H + H$^-$) \cite{Blackwood2002b} & $4.401$ & $0.029$ & $5.108$ & $0.017$ \\
Optical potential \cite{DiRienzi2002b} & $4.472$ & $0.082$ & --- & --- \\
Five-state CC \cite{Adhikari2001e} & $5.08$ & $0.004$ & --- & --- \\
CC (9Ps9H) \cite{Blackwood2002} & $4.66$ & $0.084$ & --- & --- \\
Complex rotation \cite{Yan1999} & $4.2850 \pm 0.0014$ & $0.0435 \pm 0.0027$ & $5.0540 \pm 0.0027$ & $0.0585 \pm 0.0054$ \\
Coupled-pseudostate \cite{Campbell1998} & $4.88$ & $0.058$ & --- & --- \\
\bottomrule
\end{tabular}
\caption{P-wave resonance parameters}
\label{tab:PWaveResonancesOther}
\end{table}

\Cref{tab:PWaveResonancesOther} has the P-wave resonances calculated in this work and compared to that of calculations from other groups. The complex rotation result of Yan and Ho \cite{Yan1999} is one of the most accurate calculations of these resonance parameters. The positions of the first and resonances using the complex Kohn agree very well with the complex rotation, as does the width of the first resonance. We find a width for the second resonance about half that of the complex rotation. A more recent result of Yan and Ho uses a stabilization method \cite{Yan2003}, which brings the second resonance width more in line with the complex Kohn.

The CC results of the Belfast group \cite{Blackwood2002,Blackwood2002b,Walters2004} show again that the H$^-$ channel is important for the resonances. Going from the 9Ps9H approximation to the 9Ps9H+H$^-$ significantly improves the first resonance position. The $^1E_R$ that they calculate is still higher than this work and that of Yan and Ho, but it is roughly comparable. Including the H$^-$ channel also allowed them to compute the second resonance's parameters. The $^2E_R$ they calculate is lower than the complex Kohn, stabilization and complex rotation, although it is also close. The CC does not seem to capture the width of the second resonance well.


\biblio
\end{document}