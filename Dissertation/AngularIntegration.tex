% -*- root: Dissertation.tex -*-
\documentclass[Dissertation.tex]{subfiles} 
\begin{document}

\clearpage
\pagebreak
\newpage

\chapter{Angular Integrations}
\label{chp:AngularInt}

\section{Chebyshev-Gauss Quadrature}
\label{sec:ChebyshevGauss}

We need to prove equation 5.4 of Peter Van Reeth's thesis \cite[p.79]{VanReethThesis}. Our integration is over $(0,2\pi)$.

\begin{equation}
\Int{ D(\cos \varphi_{23})}{\varphi_{23},0,2\pi} \approx \frac{2\pi}{n}\sum_{i=1}^n D\left[\cos\left(\frac{2i-1}{2n}\pi\right)\right].
\label{eq:PVRAng}
\end{equation}


The form that we will need is
\begin{equation}
\Int{ D(\cos \varphi_{23}) }{\varphi_{23},0,\pi} \approx \frac{\pi}{n}\sum_{i=1}^n D\left[\cos\left(\frac{2i-1}{2n}\pi\right)\right].
\label{eq:PVRAngNew}
\end{equation}
Both of these equations are variations on Gaussian quadratures. The
Chebyshev-Gauss quadrature is given by
\cite{Abramowitz1965,MathworldChebyshevGauss}
\todoi{Mathworld has a specific citation at end of page}
\begin{equation}
\Int{ \frac{f\left( x \right)}{\sqrt{1 - x^2}} }{x,-1,1} \approx \sum_{i = 1}^n {w_i}f\left( {{x_i}} \right),
\label{eq:ChebyshevGauss}
\end{equation}
with $w_i = \frac{\pi}{n}$ and $x_i = \cos\left(\frac{2i-1}{2n}\pi\right)$.

Starting from the left side of \cref{eq:PVRAngNew}, we have
\begin{align}
\Int{ D\left( {\cos \varphi } \right) }{\varphi,0,\pi} &= \Int{ \frac{1}{{\sqrt {1 - {{\cos }^2}\varphi } }}\sqrt {1 - {{\cos }^2}\varphi } \: D\!\left( {\cos \varphi } \right) }{\varphi,0,\pi} \\
& = \Int{ \frac{1}{{\sqrt {1 - {{\cos }^2}\varphi } }} D\!\left( {\cos \varphi } \right)\sin \varphi \: }{\varphi,0,\pi}.
\label{eq:PVRAngNew2}
\end{align}

Using the substitution $x = \cos \varphi$ and $dx = -\sin \varphi \:d\varphi$
changes the limits to $x_{min} \!\!=\! \cos 0 \! =\! 1$ and
$x_{max} = \cos \pi = -1$. \Cref{eq:PVRAngNew2} becomes
\begin{equation}
\int_0^\pi  D\left( {\cos \varphi } \right){\textrm{d}}\varphi = - \int_1^{ - 1} \frac{1}{{\sqrt {1 - {x^2}} }}D\left( x \right){\textrm{d}}x = \int_{ - 1}^1 \frac{1}{{\sqrt {1 - {x^2}} }}D\left( x \right){\textrm{d}}x.
\label{eq:PVRAngNew3}
\end{equation}
This is the exact form needed for Gauss-Chebyshev quadrature (with f = D):
\begin{equation}
\int_0^\pi  D\left( {\cos \varphi } \right){\textrm{d}}\varphi \approx \frac{\pi }{n} \sum_{i = 1}^n D\!\left[ {\cos \left( {\frac{{2i - 1}}{{2n}}\pi } \right)} \right]
\label{eq:AngQuadrature}
\end{equation}

This proves only form \cref{eq:PVRAngNew}. To prove \cref{eq:PVRAng}, we split up the integration into two parts:
\begin{equation}
\int_0^{2\pi } D\left( {\cos {\varphi_{23}}} \right){\textrm{d}}{\varphi_{23}} = \int_0^\pi  D\left( {\cos {\varphi_{23}}} \right){\textrm{d}}{\varphi_{23}} + \int _\pi ^{2\pi } D\left( {\cos {\varphi _{23}}} \right){\textrm{d}}{\varphi _{23}}.
\label{eq:AngSplit}
\end{equation}
The first integration is just \cref{eq:PVRAngNew}. The only difference between the first and second integration is the limits. Defining $y = \varphi - \pi$ gives
\begin{equation}
\int _\pi ^{2\pi } D\left( {\cos {\varphi}} \right){\textrm{d}}{\varphi} = \int_0^\pi D\left[\cos(\varphi)\right] dy = \int_0^\pi D\left[\cos(y+\pi)\right] dy.
\end{equation}
If we also define $z = \cos \varphi = \cos(y+\pi)$ and $dz = \sin y \,dy$, we get an expression the same as \cref{eq:PVRAngNew3}:
\begin{equation}
\int_0^\pi D\left(\cos\varphi\right) dy = \int_0^\pi \frac{1}{\sqrt{1-z^2}} D(z)\, sin\varphi\, dy = \int_{-1}^1 \frac{1}{\sqrt{1-z^2}} D(z) dz
\end{equation}
Since this is the same as \cref{eq:PVRAngNew3}, we just combine this with \cref{eq:AngSplit,eq:AngQuadrature} to get \cref{eq:PVRAng}, proving the first form.


\section{External Angular Integrations}
\label{sec:AngularInt}

We perform rotations and then integrations over the 3 
external angles to reduce our 9-dimensional integrations to 6 dimensions. The
procedure described here is the same as in Van Reeth's thesis
\cite{VanReethThesis}.

The integrals needed in
\cref{eq:BoundHFull,eq:GeneralKohnMatrix,eq:FourBody,eq:ShortIntGen} have
volume elements of
\begin{equation}
\label{eq:dTau1}
d\tau = d\textbf{r}_1 d\textbf{r}_2 d\textbf{r}_3.
\end{equation}
In spherical coordinates, this becomes
\begin{equation}
\label{eq:dTau2}
d\tau = r_1^2 dr_1 \sin\theta_1 d\theta_1 d\varphi_1 r_2^2 dr_2 \sin\theta_2
  d\theta_2 d\varphi_2 r_3^2 dr_3 \sin\theta_3 d\theta_3 d\varphi_3.
\end{equation}
These coordinates are given in an arbitrary coordinate system as in
\cref{fig:CoordinateSystemOriginal}.

\begin{figure}
	\centering
	\includegraphics[width=4.5in]{CoordinateSystemOriginal}
	\caption{Ps-H original coordinate system}
	\label{fig:CoordinateSystemOriginal}
\end{figure}

We can rotate the coordinate system multiple ways in order to integrate over 
the external angles. First consider rotating the coordinate system so that the
z-axis is along $\textbf{r}_1$. If we then choose to perform another rotation
so that $\textbf{r}_2$ is in the x$'$-z$'$ plane, from
\cref{fig:CoordinateSystemRotated}, the volume element becomes
\begin{equation}
\label{eq:dTau3}
d\tau = r_1^2 dr_1 \sin\theta_1 d\theta_1 d\varphi_1 r_2^2 dr_2 \sin\theta_{12}
  d\theta_{12} d\phi'_2 r_3^2 dr_3 \sin\theta_{13} d\theta_{13} d\varphi_{23}.
\end{equation}
\begin{figure}
	\centering
	\includegraphics[width=4.5in]{CoordinateSystemRotated}
	\caption{Ps-H rotated coordinate system}
	\label{fig:CoordinateSystemRotated}
\end{figure}
\noindent The angles $\theta_{12}$, $\theta_{13}$, and $\varphi_{23}$ are 
internal angles. The external angle $\phi'_2$ is measured from the x$'$-axis 
before the rotation into $\textbf{r}_2$. The angles $\theta_1$ and $\varphi_1$
are also external angles.
As Peter Van Reeth points out in his thesis \cite{VanReethThesis} on page 78, 
$\varphi_{23}$ is the angle between the planes of the triangles
$(r_1,r_2,r_{12})$ and $(r_1,r_3,r_{13})$. This angle can range between
$\varphi_{23} = 0$ and $\varphi_{23} = 2 \pi$.

For the PsH bound state and S-wave Ps-H scattering, integrating over the
external angles gives
\begin{equation}
\label{eq:dTauS1}
d\tau = 8 \uppi^2 r_1^2 dr_1 r_2^2 dr_2 \sin\theta_{12} d\theta_{12} r_3^2 dr_3
  \sin\theta_{13} d\theta_{13} d\varphi_{23}.
\end{equation}
We can transform this into integrations over $r_{12}$ and $r_{13}$ instead of
$d\theta_{12}$ and $d\theta_{13}$ by differentiating the following expression
from the law of cosines with respect to $\textbf{r}_{kl}$:
\begin{equation}
\label{eq:rkl}
r_{kl}^2 = r_k^2 + r_l^2 - 2 r_k r_l \cos\theta_{kl}.
\end{equation}
When differentiated, this gives
\begin{equation}
\label{eq:rklDer}
dr_{kl} r_{kl} = r_k r_l \sin\theta_{kl} d\theta_{kl}.
\end{equation}
Thus, we have
\begin{subequations}
\label{eq:rklDer}
\begin{align}
dr_{12} r_{12} &= r_1 r_2 \sin\theta_{12} d\theta_{12} \\
dr_{13} r_{13} &= r_1 r_3 \sin\theta_{13} d\theta_{13}.
\end{align}
\end{subequations}
Substituting these into \cref{eq:dTauS1}, we have
\begin{equation}
\label{eq:dTauS23}
d\tau = 8 \uppi^2 dr_1 r_2 dr_2 r_3 dr_3 r_{12} dr_{12} r_{13} dr_{13}
   d\varphi_{23}.
\end{equation}
This is the final form that we use in the short-range and long-range
integrations, described in \cref{sec:CompShort,sec:CompLong}.

If we do the rotations so that the z-axis is pointing in the direction of
$\textbf{r}_2$, then rotate so that $\textbf{r}_1$ is in the x$'$-z$'$ plane,
the volume element is instead
\begin{equation}
\label{eq:dTauS13}
d\tau = 8 \uppi^2 r_1 dr_1 dr_2 r_3 dr_3 r_{12} dr_{12} r_{23} dr_{23}
   d\varphi_{13}.
\end{equation}
We use this in the long-range integrations when there is an $r_{23}^{-1}$ term,
as in \cref{sec:LongLongInt,sec:ShortLongInt}.

There are 6 possible orderings to performing these rotations, but we only
consider 3 and only use 2 in the final code. The last one rotates the z-axis
into $\textbf{r}_3$, then performs another rotation where the x$'$-z$'$ plane
contains $\textbf{r}_2$, giving a volume element of
\begin{equation}
\label{eq:dTauS12}
d\tau = 8 \uppi^2 r_1 dr_1 r_2 dr_2 dr_3 r_{13} dr_{13} r_{23} dr_{23}
   d\varphi_{12}.
\end{equation}

For the P-wave and higher, we no longer have spherical symmetry, but there is
azimuthal symmetry for $m = 0$ of the spherical harmonics for all partial 
waves. For the P-wave,
\cref{tab:SphHarm} gives
\begin{equation}
Y_{10}(\theta,\varphi) = \sqrt{\frac{3}{4\pi}} \cos\theta.
\end{equation}






\todoi{Describe why these are needed}
These are derived in a \emph{Mathematica} notebook entitled ``General Angular Integrations.nb''.
\todoi{Describe how this notebook works, including ExtremeSimplify}



\subsection{Terms with Identical Spherical Harmonics}
\label{sec:AngSame}
We are not integrating over all angles, so the orthogonality relations of the
spherical harmonics do not hold. However, it can be seen that if we integrate
over all external angles, $d\tau_{ext}$, any spherical harmonic with itself
will give $2\pi$, i.e.
\beq
\label{eq:AngSame}
\Int{Y_\ell^0(\theta_i, \varphi_i) Y_\ell^0(\theta_i, \varphi_i)}{\tau_{ext}, \tau_{ext}} = 2 \pi,
\eeq
where $i$ = $1$, 2, 3, $\rho$ or $\rho^\prime$.


\subsection{Terms with \texorpdfstring{$Y_\ell^0(\theta_i, \varphi_i) Y_\ell^0(\theta_j, \varphi_j)$}{Yi-Yj} }
\label{sec:AngRiRj}

These terms are not completely necessary to generalize here, because they 
only appear in the short-short calculations. My derivations and code for the 
S-, P-, and D-wave short-short calculations use these. The general code
short-short derivations and code \cite{sec:} use a formalism from Drake and
Yan \cite{Yan1997} that does not do these external angular integrations. The 
general code is used for the F-, G-, and H-waves and can be used for 
arbitrary $\ell$. Derivations for all of these can be seen in the
\emph{Mathematica} notebook ``General Angular Integrations.nb''. Note that the
result of $2\pi$ in \cref{sec:AngSame} is recovered if $i=j$. The general 
result is
\beq
\label{eq:AngRiRj}
\Int{Y_\ell^0(\theta_i, \varphi_i) Y_\ell^0(\theta_j, \varphi_j)}{\tau_{ext}, \tau_{ext}} = 2 \pi \LegendreP{\ell, \cos\theta_{ij}}.
\eeq
where $i$ = $1$, 2, 3, $\rho$ or $\rho^\prime$. It takes some work to show that the results in the \emph{Mathematica} notebook ``General Angular Integrations.nb'' are equivalent to this.

These integrations for the first three partial waves are given below.
\begin{subequations}
\begin{align}
\Int{Y_0^0(\theta_i, \varphi_i) Y_0^0(\theta_j, \varphi_j)}{\tau_{ext}, \tau_{ext}} &= 2 \pi \\
\Int{Y_1^0(\theta_i, \varphi_i) Y_1^0(\theta_j, \varphi_j)}{\tau_{ext}, \tau_{ext}} &= 2 \pi \cos\theta_{ij} \\
\Int{Y_2^0(\theta_i, \varphi_i) Y_2^0(\theta_j, \varphi_j)}{\tau_{ext}, \tau_{ext}} &= 2 \pi \left(3 \cos^2 \theta_{ij} - 1 \right)
\end{align}
\end{subequations}

When performing the external angular integrations instead of using the general short-range integrals in \cref{sec:GeneralShort}, the very specific form of the short-short integrals has to be that of the four-body integrals in \cref{eq:FourBody}. From \cref{eq:AngRiRj}, the results here have a $\cos\theta_{ij}$, which does not appear in the four-body integrals. To use these results, we have to use the law of cosines to replace any $\cos\theta_{ij}$ terms by
\beq
\label{eq:LawCosines}
\cos\theta_{ij} = \frac{r_i^2 + r_j^2 - r_{ij}^2}{2 r_i r_j}.
\eeq
This allows us to split these into multiple integrations with only polynomial $r_i$ and $r_{ij}$ terms with the decaying exponentials. Most of the other external angular integrations in this Appendix also end up with $\cos\theta_{ij}$, but those only apply to the long-range code, which does not have this restriction.


\subsection{Terms with \texorpdfstring{$Y_\ell^0(\theta_1, \varphi_1) Y_\ell^0(\theta_{\rho}, \varphi_{\rho})$}{Y1-Yrho} and \texorpdfstring{$Y_\ell^0(\theta_1, \varphi_1) Y_\ell^0(\theta_{\rho^\prime}, \varphi_{\rho^\prime})$}{Y1-Yrho'}}
\label{sec:AngR1Rho}

After doing these integrations by hand for the P-, D-, and F-wave (the S-wave is just $2\pi$), I realized that it is possible to generalize these. This can be seen in the ``Vector Gaussian Integration.cpp'' file of the general long-range integration code. If we define
\beq
w_1 = \frac{r_1 + r_2 \cos\theta_{12}}{2 \rho},
\eeq
then
\beq
\Int{Y_\ell^0(\theta_1, \varphi_1) Y_\ell^0(\theta_{\rho}, \varphi_{\rho})}{\tau_{ext}, \tau_{ext}} = 2 \pi \LegendreP{\ell, w_1},
\eeq
where $P_\ell$ is the standard Legendre polynomial. This easily lets us calculate these terms in the C++ code. Likewise, for
\beq
w_1^\prime = \frac{r_1 + r_3 \cos\theta_{13}}{2 \rho^\prime},
\eeq
then
\beq
\Int{Y_\ell^0(\theta_1, \varphi_1) Y_\ell^0(\theta_{\rho^\prime}, \varphi_{\rho^\prime})}{\tau_{ext}, \tau_{ext}} = 2 \pi \LegendreP{\ell, w_1^\prime}.
\eeq


\subsection{Terms with \texorpdfstring{$Y_\ell^0(\theta_2, \varphi_2) Y_\ell^0(\theta_{\rho}, \varphi_{\rho})$}{Y2-Yrho} and \texorpdfstring{$Y_\ell^0(\theta_3, \varphi_3) Y_\ell^0(\theta_{\rho^\prime}, \varphi_{\rho^\prime})$}{Y3-Yrho'}}
\label{sec:AngR2Rho}

Similar to \cref{sec:AngR1Rho}, these integrals are also generalizable and are used in the C++ file ``Vector Gaussian Integration.cpp''. Defining
\beq
w_2 = \frac{r_2 + r_1 \cos\theta_{12}}{2 \rho},
\eeq
then
\beq
\Int{Y_\ell^0(\theta_2, \varphi_2) Y_\ell^0(\theta_{\rho}, \varphi_{\rho})}{\tau_{ext}, \tau_{ext}} = 2 \pi \LegendreP{\ell, w_2}.
\eeq
Since $\rho^\prime$ is constructed from $r_3$, we can do the same method for the next set. We define
\beq
w_3 = \frac{r_3 + r_1 \cos\theta_{13}}{2 \rho^\prime},
\eeq
which gives
\beq
\Int{Y_\ell^0(\theta_3, \varphi_3) Y_\ell^0(\theta_{\rho^\prime}, \varphi_{\rho^\prime})}{\tau_{ext}, \tau_{ext}} = 2 \pi \LegendreP{\ell, w_3}.
\eeq


\subsection{Terms with \texorpdfstring{$Y_\ell^0(\theta_2, \varphi_2) Y_\ell^0(\theta_{\rho^\prime}, \varphi_{\rho^\prime})$}{Y2-Yrho'} and \texorpdfstring{$Y_\ell^0(\theta_3, \varphi_3) Y_\ell^0(\theta_{\rho}, \varphi_{\rho})$}{Y3-Yrho}}
\label{sec:AngR2Rhop}

This set is more complicated than those in \cref{sec:AngR2Rho} but look similar. We first define
\beq
w_4 = \frac{r_1 \cos\theta_{12} + r_3 \cos\theta_{23}}{2 \rho^\prime},
\eeq
and then
\beq
\Int{Y_\ell^0(\theta_2, \varphi_2) Y_\ell^0(\theta_{\rho^\prime}, \varphi_{\rho^\prime})}{\tau_{ext}, \tau_{ext}} = 2 \pi \LegendreP{\ell, w_4}.
\eeq
For the permuted version of this, we use
\beq
w_5 = \frac{r_1 \cos\theta_{13} + r_2 \cos\theta_{23}}{2 \rho},
\eeq
giving
\beq
\Int{Y_\ell^0(\theta_3, \varphi_3) Y_\ell^0(\theta_{\rho}, \varphi_{\rho})}{\tau_{ext}, \tau_{ext}} = 2 \pi \LegendreP{\ell, w_5}.
\eeq
\todoi{The $w_4$ and $w_5$ look similar to equation (B.14) in Peter's thesis}


\subsection{Terms with \texorpdfstring{$Y_\ell^0(\theta_\rho, \varphi_\rho) Y_\ell^0(\theta_{\rho^\prime}, \varphi_{\rho^\prime})$}{Yrho-Yrho'}}

These are the most difficult angular integrations considered. After performing these integrations by hand through the D-wave, I came up with a way to compute them in \emph{Mathematica}. These are derived in a \emph{Mathematica} notebook entitled ``General Angular Integrations.nb''. Using this method, the H-wave integral ($\ell = 5$) took approximately 1 hour and 26 minutes to calculate in \emph{Mathematica}. An I-wave version of this integral was attempted on the Talon2 cluster (single processor only), but it was killed after approximately 2 weeks of runtime. Without this result, the I-wave and higher cannot be calculated using this method.

I discovered a way to generalize these integrals as well, but it was not until recently. The general long-range code still has the hardcoded values. Without realizing this relation, these integrals are very difficult to compute. Defining
\beq
y = \frac{4\left(\rho^2 + {\rho^\prime}^2\right) - r_{23}^2}{8 \rho \rho^\prime},
\eeq
these integrals can be generalized to
\begin{equation}
\Int{Y_\ell^0(\theta_\rho, \varphi_\rho) Y_\ell^0(\theta_{\rho^\prime}, \varphi_{\rho^\prime})}{\tau_{ext}, \tau_{ext}} = 2 \pi \LegendreP{\ell, y}.
\end{equation}
The results of these integrals for the first six partial waves follow.

\begin{align}
\Int{&Y_0^0(\theta_\rho, \varphi_\rho) Y_0^0(\theta_{\rho^\prime}, \varphi_{\rho^\prime})}{\tau_{ext}, \tau_{ext}} = 2 \pi
\end{align}

\begin{align}
\Int{&Y_1^0(\theta_\rho, \varphi_\rho) Y_1^0(\theta_{\rho^\prime}, \varphi_{\rho^\prime})}{\tau_{ext}, \tau_{ext}} = \frac{\pi}{4 \rho \rho^\prime}  \left[4 \left(\rho^2+{\rho^\prime}^2\right)-r_{23}^2\right]
\end{align}

\begin{align}
\Int{Y_2^0(\theta_\rho, \varphi_\rho) & Y_2^0(\theta_{\rho^\prime}, \varphi_{\rho^\prime})}{\tau_{ext}, \tau_{ext}} = \frac{\pi}{64 \rho ^2 {\rho^\prime}^2}  \left[16 \left(3 \rho ^4+2 \rho ^2 {\rho^\prime}^2 +3 {\rho^\prime}^4\right)  \right. \nonumber \\
& \left. +3 r_{23}^4-24 r_{23}^2 \left(\rho ^2+{\rho^\prime}^2\right)\right]
\end{align}

\begin{align}
\Int{&Y_3^0(\theta_\rho, \varphi_\rho) Y_3^0(\theta_{\rho^\prime}, \varphi_{\rho^\prime})}{\tau_{ext}, \tau_{ext}} = \frac{\pi}{512 \rho ^3 {\rho^\prime}^3} \left[64 \left(5 \rho ^6+3 \rho ^4 {\rho^\prime}^2+3 \rho ^2 {\rho^\prime}^4+5 {\rho^\prime}^6\right) \right.  \nonumber \\
& \left. -5 r_{23}^6 +60 r_{23}^4 \left(\rho ^2+{\rho^\prime}^2\right)-48 r_{23}^2 \left(5 \rho ^4+6 \rho ^2 {\rho^\prime}^2+5 {\rho^\prime}^4\right)\right]
\end{align}

\begin{align}
\Int{&Y_4^0(\theta_\rho, \varphi_\rho) Y_4^0(\theta_{\rho^\prime}, \varphi_{\rho^\prime})}{\tau_{ext}, \tau_{ext}} = \frac{\pi}{16384 \rho^4 {\rho^\prime}^4}  \left[8960 {\rho^\prime}^8+1280 {\rho^\prime}^6 \left(4 \rho^2-7 r_{23}^2\right) \right.  \nonumber \\
& -80 {\rho^\prime}^2 \left(r_{23}^2-4 \rho ^2\right)^2 \left(7 r_{23}^2-4 \rho ^2\right)+35 \left(r_{23}^2-4 \rho ^2\right)^4 \nonumber \\
& \left. +96 {\rho^\prime}^4 \left(48 \rho ^4+35 r_{23}^4-120 \rho ^2 r_{23}^2\right)\right]
\end{align}

\begin{align}
\Int{&Y_5^0(\theta_\rho, \varphi_\rho) Y_5^0(\theta_{\rho^\prime}, \varphi_{\rho^\prime})}{\tau_{ext}, \tau_{ext}} = \frac{\pi}{131072 \rho ^5 {\rho^\prime}^5} \left[64512 {\rho^\prime}^{10}-8960 {\rho^\prime}^8 \left(9 r_{23}^2-4 \rho ^2\right) \right.  \nonumber \\
& + 140 {\rho^\prime}^2 \left(r_{23}^2-4 \rho ^2\right)^3 \left(9 r_{23}^2-4 \rho ^2\right)-63 \left(r_{23}^2-4 \rho ^2\right)^5  \nonumber \\
& + 1920 {\rho^\prime}^6 \left(16 \rho ^4+21 r_{23}^4-56 \rho^2 r_{23}^2\right)  \nonumber \\
& \left. -480 {\rho^\prime}^4 (r_{23}-2 \rho ) (2 \rho +r_{23}) \left(16 \rho ^4+21 r_{23}^4-56 \rho ^2 r_{23}^2\right)\right]
\end{align}




\biblio
\end{document}