% -*- root: Dissertation.tex -*-
\documentclass[Dissertation.tex]{subfiles} 
\begin{document}


%\chapter{\chp{Scattering Theory}}
\chapter{\iftoggle{UNT}{SCATTERING THEORY}{Scattering Theory}}
\label{chp:WaveKohn}
\iftoggle{UNT}{As}{\lettrine{\textcolor{startcolor}{A}}{s}}
mentioned in the introduction, the 
Kohn variational method and its variants have been used for many types of 
systems. This chapter discusses the trial wavefunction used, derives the Kohn-
type variational methods, and applies them to this system.
%\todoi{Description of partial waves?}

\section{General Wavefunction}
\label{sec:GeneralWave}

The applications of the Kohn, inverse Kohn, complex Kohn for the
$S$-matrix and $T$-matrix, generalized Kohn, and generalized complex Kohn
for the $S$-matrix and $T$-matrix to the trial wavefunctions for each of the 
partial waves through the H-wave are all very similar in form. The trial 
wavefunctions for the partial waves
(\cref{eq:SWaveTrial,eq:PWaveTrial,eq:DWaveTrial}) can be written in a general
form as
\beq
\Psi_\ell^{\pm,t} = \widetilde{S}_\ell + L_\ell^{\pm,t} \, \widetilde{C}_\ell + \sum_{i=1}^{N(\omega)} c_i \bar{\phi}_i.
\label{eq:GeneralWaveTrial}
\eeq
We only consider the Ps(1s)+H(1s) system for energies up to the excitation
threshold of Ps(n=2)+H(1s), which is at an energy of $\tfrac{3}{16}$ a.u.
($5.102$ eV) \cite{Woods2015}.

To avoid confusion with $L_\ell^{\pm,t}$ here and $\mathcal{L}$ in
\cref{eq:LDef}, since the orbital angular momentum $\ell$ of the incoming Ps is
the same as the total angular momentum $L$, we use $\ell$ to indicate the partial 
wave.

The short-range $\bar{\phi}_i$ terms can represent terms of different 
symmetries, such as the $\bar{\phi}_{1i}$ and $\bar{\phi}_{2j}$ of the P-wave 
in \cref{eq:PWaveTrial}. The only requirement in this derivation is that 
these are Hylleraas-type short-range terms. In addition to letting the
$\widetilde{S}_\ell$ and $\widetilde{C}_\ell$ represent
the $\bar{S}_\ell$ and $\bar{C}_\ell$ for 
the different partial waves, we can define them in such a way as to use 
multiple Kohn methods (Kohn, inverse Kohn, etc.). We begin by defining a
matrix $\textbf{u}$ which satisfies
\beq
\label{eq:GenSCMatrix}
\begin{bmatrix}
\widetilde{S}_\ell \\
\widetilde{C}_\ell
\end{bmatrix}
=
\textbf{u}
\begin{bmatrix}
\bar{S}_\ell \\
\bar{C}_\ell
\end{bmatrix}
=
\begin{bmatrix}
u_{00} & u_{01} \\
u_{10} & u_{11}
\end{bmatrix}
\begin{bmatrix}
\bar{S}_\ell \\
\bar{C}_\ell
\end{bmatrix}.
\eeq

\noindent This notation is similar to that of Lucchese \cite{Lucchese1989}
and Cooper et al.~\cite{Cooper2010}. From this, it can easily be seen that
\begin{subequations}
\label{eq:TildeSCDef}
\begin{align}
\widetilde{S}_\ell &= u_{00} \bar{S}_\ell + u_{01} \bar{C}_\ell  \label{eq:TildeSDef} \\
\widetilde{C}_\ell &= u_{10} \bar{S}_\ell + u_{11} \bar{C}_\ell. \label{eq:TildeCDef}
\end{align}
\end{subequations}

We define
\beq
\label{eq:SCBarDef}
\bar{S}_\ell = \frac{1}{\sqrt{2}}(S_\ell \pm S_\ell^\prime) \text{ and } \bar{C}_\ell = \frac{1}{\sqrt{2}}(C_\ell \pm C_\ell^\prime),
\eeq
where
\beq
\label{eq:SCPrime}
S_\ell^\prime = P_{23}S_\ell \text{ and } C_\ell^\prime = P_{23}C_\ell.
\eeq
The general form for the long-range terms $S_\ell$ and $C_\ell$ is
\begin{subequations}
\label{eq:GenSandC}
\begin{align}
S_\ell = \,&\SphericalHarmonicY{\ell}{0}{\theta_\rho}{\varphi_\rho} \Phi_{Ps}\!\left(r_{12}\right) \Phi_H\!\left(r_3\right) \sqrt{2\kappa} \,j_\ell\!\left(\kappa\rho\right) \label{eq:GenSDef} \\
C_\ell = -&\SphericalHarmonicY{\ell}{0}{\theta_\rho}{\varphi_\rho} \Phi_{Ps}\!\left(r_{12}\right) \Phi_H\!\left(r_3\right) \sqrt{2\kappa} \,n_\ell\!\left(\kappa\rho\right) f_\ell(\rho) \label{eq:GenCDef}.
\end{align}
\end{subequations}
The $\SphericalHarmonicY{\ell}{0}{\theta_\rho}{\varphi_\rho}$ are the 
spherical harmonics, $j_\ell(\kappa\rho)$ are the spherical Bessel functions,
and $n_\ell(\kappa\rho)$ are the spherical Neumann functions. These are all
given in \cref{sec:SphericalFunc} through $\ell = 5$.
$\Phi_{\rm{Ps}}\!\left(r_{12}\right)$ and $\Phi_{\rm{H}}\!\left(r_3\right)$ are the Ps 
and H ground state wavefunctions given in \cref{eq:PsWave,eq:HWave}.

The shielding function, $f_\ell(\rho)$, removes the singularity at the origin 
due to the spherical Neumann function, $n_\ell$. The form that we have chosen 
for this is
\begin{equation}
  \label{eq:PartialWaveShielding}
  f_\ell(\rho) = \left[1 - \ee^{-\mu \rho} \left(1+\frac{\mu}{2}\rho\right)
  \right]^{m_\ell}.
\end{equation}
At a minimum, $m_\ell$ is chosen so that $C_\ell$ behaves like $S_\ell$ as
$\rho \to 0$. For more discussion of this, see \Cref{sec:ShieldingFunc}. The
values used for the different partial waves are given in \cref{tab:Nonlinear}.
Prior work \cite{VanReeth2003} used a slightly simpler shielding function 
for the S-wave of
\begin{equation}
\label{eq:OldShielding}
f(\rho) = (1 - \ee^{-\lambda \rho})^3.
\end{equation}
Note that their paper is missing the negative sign in the exponential.

%\todoi{Put in plots from ``Shielding Function Plots.nb''?}

The Hylleraas-type short-range terms are similar to that used
in \cref{eq:BoundWavefn}. These are chosen to have two symmetries, one 
with a prefactor of $r_1^\ell$ and the other with a prefactor of $r_2^\ell$.
The prefactors are included so that the correct asymptotic form of
$\Psi_\ell^{\pm,t} \sim r_k^\ell$ at the origin follows \cite[p.87]{BrownThesis}.
The first and second symmetries are given respectively by
\begin{subequations}
\label{eq:PhiDef}
\begin{align}
  \bar{\phi}_{1i} &= \left(1 \pm P_{23}\right) \SphericalHarmonicY{\ell}{0}{\theta_1}{\varphi_1}
  e^{-(\alpha r_1 + \beta r_2 + \gamma r_3)}
  r_1^{\ell} r_1^{k_i} r_2^{l_i} r_{12}^{m_i} r_3^{n_i} r_{13}^{p_i} r_{23}^{q_i} \label{eq:PhiDef1} \\
  \bar{\phi}_{2j} &= \left(1 \pm P_{23}\right) \SphericalHarmonicY{\ell}{0}{\theta_2}{\varphi_2}
  e^{-(\alpha r_1 + \beta r_2 + \gamma r_3)}
  r_2^{\ell} r_1^{k_j} r_2^{l_j} r_{12}^{m_j} r_3^{n_j} r_{13}^{p_j} r_{23}^{q_j}. \label{eq:PhiDef2}
\end{align}
\end{subequations}

From Refs.~\cite{Schwartz1961a,VanReethThesis}, the D-wave and higher can 
have additional symmetries where the angular momentum is shared between the 
Ps and H. From Ref.~\cite{Schwartz1961a}, we see that there are a possible
$\ell+1$ sets of short-range terms for each partial wave. We do not consider 
these mixed terms in this work (see \cref{sec:MixedTerms}). 

The S-wave has only a single symmetry, so $\bar{\phi}_i$ is a single set of
terms. Similar to \cref{sec:BoundWavefn}, the $\frac{1}{\sqrt{2}}$ is absorbed
into $c_{i0}$. The full S-wave trial wavefunction can be written as
\begin{equation}
  \label{eq:TrialWave}
  \Psi_0^{\pm,t} = \widetilde{S}_0 + L_0^{\pm,t} \, \widetilde{C}_0 + \sum_{i=1}^{N(\omega)} c_{i0} \bar{\phi}_{i1}.
\end{equation}
For the P-wave and higher ($\ell > 0$), 
\begin{equation}
  \label{eq:TrialWaveHigher}
  \Psi_\ell^{\pm,t} = \widetilde{S}_\ell + L^{\pm,t}_\ell \, \widetilde{C}_\ell
  + \sum_{i=1}^{N(\omega)} c_{i \ell} \bar{\phi}_{i1}
  + \!\!\!\sum_{i=N(\omega)+1}^{2N(\omega)} \!\! d_{i \ell} \bar{\phi}_{i2}.
\end{equation}

The symbols $\rho$ and $\rho'$ are defined as (refer to \cref{fig:PsHCoords,sec:RhoDef})
\begin{subequations}
\begin{align}
\vec{\rho} &= \frac{1}{2}\left(\vec{r_1} + \vec{r_2}\right) \label{eq:RhoDef}\\
\vec{\rho}^\prime &= \frac{1}{2}\left(\vec{r_1} + \vec{r_3}\right) \label{eq:RhopDef}.
\end{align}
\end{subequations}


\section{General Kohn Principle Derivation}
\label{sec:KohnDerivation}

Much of this derivation is similar to that in Peter Van Reeth's thesis \cite{
VanReethThesis} but is for single channel scattering and also generalized to 
a variety of Kohn-type variational methods. His thesis covers the Kohn and inverse 
Kohn methods for two channel e$^+$-He scattering. For this derivation, I will 
use \cref{eq:GeneralWaveTrial} but drop the short-range $\bar{\phi}_i^t$ 
terms. The derivation follows through the same with these terms, but it is 
clearer to ignore them here. Likewise, we only consider the direct terms 
here, unless otherwise specified. The final result of this section applies 
equally well to both the direct and exchanged terms.

The kinetic energy for Ps is
\beq
E_{\bm \kappa} = \frac{\hbar^2 \kappa^2}{2 m} = \frac{\kappa^2}{2 m} = \frac{1}{4} \kappa^2,
\label{eq:Wavenumber}
\eeq
where $\kappa$ is the momentum of the Ps atom.
Including this in the total energy with the ground-state energies of H and Ps, $E_H$ and $E_{Ps}$, gives
\beq
\label{eq:TotalEnergy}
E = E_H + E_{Ps} + E_{\bm \kappa} = -\frac{1}{2} - \frac{1}{4} + \frac{1}{4} \kappa^2 = -\frac{3}{4} + \frac{1}{4} \kappa^2.
\eeq

The functional $I_\ell$ is defined as \cite{Adhikari1998}
\begin{equation}
I[\Psi_\ell^t]\equiv \left<{\Psi_\ell^t}^\star | \mathcal{L} | \Psi_\ell^t \right> = \left(\Psi_\ell^t, \mathcal{L} \Psi_\ell^t \right) = \int \Psi_\ell^t \mathcal{L}
  \Psi_\ell^t \,d\tau,
\label{eq:IlDefPsi}
\end{equation}
where the operator $\mathcal{L}$ is given by
\beq
\label{eq:LDef}
\mathcal{L} = 2(H-E).
\eeq
Note that the exact wavefunction $\Psi_\ell$ solves the Schr\"{o}dinger equation, giving
\beq
\label{eq:Il0}
I[\Psi_\ell] = 0.
\eeq
\label{BraNote}Normally, the bra in bra-ket notation is conjugated,
but as noted by Refs.~\cite{Cooper2010,Lucchese1989,Zhang1988},
the bra is not conjugated for the Kohn-type variational methods.

The trial wavefunction is related to the exact solution by
\beq
\label{eq:PsilTrialRelation}
\Psi_\ell^t = \Psi_\ell + \delta \Psi_\ell.
\eeq
The variation of $I_\ell$ is
\begin{align}
\label{eq:IlPsiVariation1}
\nonumber \delta I_\ell &= I_\ell[\Psi_\ell^t] - I_\ell[\Psi_\ell] \\
\nonumber &= I_\ell[\Psi_\ell + \delta \Psi_\ell] - I_\ell[\Psi_\ell] \\
&= (\Psi_\ell, \mathcal{L} \Psi_\ell) + (\Psi_\ell, \mathcal{L} \,\delta\Psi_\ell) + (\delta\Psi_\ell, \mathcal{L} \Psi_\ell) + (\delta\Psi_\ell, \mathcal{L} \,\delta\Psi_\ell) - (\Psi_\ell, \mathcal{L} \Psi_\ell).
\end{align}
The first and last terms are equal to 0, by virtue of \cref{eq:Il0}. %We drop the $\ell$ subscript from this point.

The $(\delta\Psi_\ell, \mathcal{L} \,\delta\Psi_\ell)$ term is of second order in $\delta\Psi_\ell$, so this can be neglected. Denoting this approximation as $\delta I_\ell^\prime$ gives
\beq
\label{eq:IlPrimeDef}
\delta I_\ell^\prime = \delta I_\ell - (\delta\Psi_\ell, \mathcal{L} \,\delta\Psi_\ell).
\eeq
Since $\mathcal{L}\Psi_\ell = 0$,
\beq
(\delta\Psi_\ell, \mathcal{L} \Psi_\ell) = -(\delta\Psi_\ell, \mathcal{L} \Psi_\ell),
\eeq
which combined with the definition of $\mathcal{L}$ from \cref{eq:LDef}, allows us to write the above equation as
\beq
\delta I_\ell^\prime = 2 (\Psi_\ell^\star | H\!-\!E | \delta\Psi_\ell) - 2 (\delta\Psi_\ell^\star | H\!-\!E | \Psi_\ell).
\label{eq:IlPsiVariation2}
\eeq

The Hamiltonian for the fundamental Coulombic system is
\begin{align}
\label{eq:Hamiltonian1}
H = -\frac{1}{2} \Laplacian_{r_1} - \frac{1}{2} \Laplacian_{r_2} - \frac{1}{2} \Laplacian_{r_3} + \frac{1}{r_1} - \frac{1}{r_2} - \frac{1}{r_3} - \frac{1}{r_{12}} -\frac {1}{r_{13}} + \frac{1}{r_{23}}.
\end{align}

\noindent The Hamiltonian can also be expressed in terms of other variables in Jacobi coordinates as
\begin{align}
H = -\frac{1}{4} \Laplacian_\rho - \frac{1}{2} \Laplacian_{r_3} - \Laplacian_{r_{12}} + \frac{1}{r_1} - \frac{1}{r_2} - \frac{1}{r_3} - \frac{1}{r_{12}} - \frac{1}{r_{13}} + \frac{1}{r_{23}}
\label{eq:Hamiltonian2}
\end{align}
and for the permuted version,
\begin{align}
H = -\frac{1}{4} \Laplacian_{\rho^\prime} - \frac{1}{2} \Laplacian_{r_2} - \Laplacian_{r_{13}} + \frac{1}{r_1} - \frac{1}{r_2} - \frac{1}{r_3} - \frac{1}{r_{12}} - \frac{1}{r_{13}} + \frac{1}{r_{23}}.
\label{eq:Hamiltonian3}
\end{align}

\noindent Substituting the second form of $H$ in \cref{eq:IlPsiVariation2} and using the total energy from \cref{eq:TotalEnergy} gives
\begin{align}
\label{eq:IlPsiVariation3}
\delta I_\ell^\prime = 2 \int\limits_{V_{12}} \int\limits_{V_3} \int\limits_{V_\rho} \Psi_\ell &\left[-\frac{1}{4} \Laplacian_\rho - \frac{1}{2} \Laplacian_{r_3} - \Laplacian_{r_{12}} + \frac{1}{r_1} - \frac{1}{r_2} - \frac{1}{r_3} \right. \nonumber \\
  &- \left. \frac{1}{r_{12}} - \frac{1}{r_{13}} + \frac{1}{r_{23}} - E_H - E_{Ps} - \frac{1}{4}\kappa^2 \right] \delta \Psi_\ell \,d\tau_\rho d\tau_{r_3} d\tau_{r_{12}} \nonumber \\
 - 2 \int\limits_{V_{12}} \int\limits_{V_3} \int\limits_{V_\rho} \delta \Psi_\ell &\left[-\frac{1}{4} \Laplacian_\rho - \frac{1}{2} \Laplacian_{r_3} - \Laplacian_{r_{12}} + \frac{1}{r_1} - \frac{1}{r_2} - \frac{1}{r_3} \right. \nonumber \\
  &- \left. \frac{1}{r_{12}} - \frac{1}{r_{13}} + \frac{1}{r_{23}} - E_H - E_{Ps} - \frac{1}{4}\kappa^2 \right] \Psi_\ell \,d\tau_\rho d\tau_{r_3} d\tau_{r_{12}}.
\end{align}

%Some important cancellations are made in the matrix element equations in \cref{sec:??} by using the H and Ps ground state wave equations.
%Separately,
The H and Ps equations are respectively (for large values of $\rho$)
\begin{subequations}
\label{eq:HPsEqn}
\begin{align}
\left(-\frac{1}{2} \Laplacian_{r_3} - \frac{1}{r_3}\right) \Phi_H(r_3) &= E_H \Phi_H(r_3) \label{eq:HEqn} \\
\left(-\Laplacian_{r_{12}} - \frac{1}{r_{12}}\right) \Phi_{Ps}(r_{12}) &= E_{Ps} \Phi_{Ps}(r_{12}). \label{eq:PsEqn}
\end{align}
\end{subequations}
Realizing then that the Hamiltonians for H and Ps are given by
\begin{subequations}
\label{eq:HPsHamil}
\begin{align}
H_H =& -\frac{1}{2} \Laplacian_{r_3} - \frac{1}{r_3} \label{eq:HHamil} \\
H_{Ps} =& -\Laplacian_{r_{12}} - \frac{1}{r_{12}}, \label{eq:PsHamil}
\end{align}
\end{subequations}
and rearranging terms, \cref{eq:IlPsiVariation3} becomes
\begin{align}
\label{eq:IlPsiVariation4}
\delta I_\ell^\prime = -&\frac{1}{2} \int\limits_{V_{12}} \int\limits_{V_3} \int\limits_{V_\rho} \left[\Psi_\ell \Laplacian_\rho \,\delta\Psi_\ell - \delta\Psi_\ell \Laplacian_\rho \Psi_\ell \right] \,d\tau_\rho d\tau_{r_3} d\tau_{r_{12}} \nonumber \\
+ &2 \int\limits_{V_{12}} \int\limits_{V_3} \int\limits_{V_\rho} \Psi_\ell \left[H_H + H_{Ps} + \frac{1}{r_1} - \frac{1}{r_2} - \frac{1}{r_{13}} + \frac{1}{r_{23}} - E_H - E_{Ps} - \frac{1}{4}\kappa^2 \right] \delta\Psi_\ell \,d\tau_\rho d\tau_{r_3} d\tau_{r_{12}} \nonumber \\
- &2 \int\limits_{V_{12}} \int\limits_{V_3} \int\limits_{V_\rho} \delta\Psi_\ell \left[H_H + H_{Ps} + \frac{1}{r_1} - \frac{1}{r_2} - \frac{1}{r_{13}} + \frac{1}{r_{23}} - E_H - E_{Ps} - \frac{1}{4}\kappa^2 \right] \Psi_\ell \,d\tau_\rho d\tau_{r_3} d\tau_{r_{12}}.
\end{align}
Due to the exponential form of $\Phi_{Ps}(r_{12})$ and $\Phi_H(r_3)$ given in \cref{eq:PsWave,eq:HWave}, the last two lines cancel each other.

From Green's theorem,
\begin{align}
\label{eq:GreensThm}
\nonumber \Braket{\Psi_\ell^\star | \Laplacian_\rho | \delta\Psi_\ell } - \Braket{\delta\Psi_\ell^\star | \laplacian_\rho | \Psi_\ell}
&= \int\limits_{V_3} \int\limits_{V_{12}} \int\limits_{V_\rho} \left[ \Psi_\ell \laplacian_\rho \,\delta\Psi_\ell - \delta\Psi_\ell \laplacian_\rho \Psi_\ell \right] d\tau_\rho \, d\tau_{12} d\tau_3 \\
&= \int\limits_{V_3} \int\limits_{V_{12}} \int\limits_{S_\rho} \left[ \Psi_\ell \grad_\rho \delta\Psi - \delta\Psi_\ell \grad_\rho \Psi_\ell \right] \cdot d\bm{\sigma}_\rho d\tau_{12} d\tau_3.
\end{align}
$S_\rho$ is the surface at $\rho \rightarrow \infty$.

Since we are only considering the direct terms in \cref{eq:GeneralWaveTrial} so far, let us define a direct term only version of \cref{eq:TildeSCDef} with
\begin{subequations}
\label{eq:TildeSCDefDir}
\begin{align}
\widetilde{S}_d &= u_{00} S_\ell + u_{01} C_\ell \\
\widetilde{C}_d &= u_{10} S_\ell + u_{11} C_\ell.
\end{align}
\end{subequations}

From \cref{eq:PsilTrialRelation} and \cref{eq:GeneralWaveTrial},
\beq
\label{eq:DeltaPsi}
\delta \Psi = \Psi_\ell^t - \Psi_\ell = (\widetilde{S}_d + L_\ell^t \, \widetilde{C}_d) - (\widetilde{S}_d + L_\ell \, \widetilde{C}_d) = (L_\ell^t - L_\ell) \widetilde{C}_d \,.
\eeq
Substituting this into \cref{eq:IlPsiVariation4} with \cref{eq:GreensThm},
\beq
\label{eq:IlPsiVariation5}
\delta I_\ell^\prime = -\frac{1}{2} (L_\ell^t - L_\ell) \int\limits_{V_{12}} \int\limits_{V_3} \int\limits_{S_\rho} \left[(\widetilde{S}_d + L_\ell \, \widetilde{C}_d) \grad_\rho \widetilde{C}_d - \widetilde{C}_d \grad_\rho (\widetilde{S}_d + L_\ell \, \widetilde{C}_d) \right] \cdot d\bm{\sigma}_\rho d\tau_{12} d\tau_3 \;.
\eeq

From \cref{eq:SphBesDerRel,eq:GenSandC}, to first order, the gradient acting on $\widetilde{S}_d$ and $\widetilde{C}_d$ in \cref{eq:TildeSCDef} gives
\begin{subequations}
\label{eq:GradSC}
\begin{align}
\grad_\rho \widetilde{S}_d &\sim \kappa \left[u_{00} C_\ell - u_{01} S_\ell \right] \hat{\bm\rho} \\
\grad_\rho \widetilde{C}_d &\sim \kappa \left[u_{10} C_\ell - u_{11} S_\ell \right] \hat{\bm\rho} \,.
\end{align}
\end{subequations}
Substituting this into \cref{eq:IlPsiVariation5} and dropping the dot product, since the surface elements are in the same direction as $\hat{\bm\rho}$, this becomes
\begin{align}
\label{eq:IlPsiVariation6a}
\delta I_\ell^\prime \sim -\frac{1}{2} (L_\ell^t - L_\ell) & \int\limits_{V_{12}} \int\limits_{V_3} \int\limits_{S_\rho}  \left\{(\widetilde{S}_d + L_\ell \, \widetilde{C}_d) \kappa (u_{10} C_\ell - u_{11} S_\ell) \right. \nonumber \\
 &- \left. \widetilde{C}_d \kappa \left[(u_{00} C_\ell - u_{01} S_\ell) + L_\ell (u_{10} C_\ell - u_{11} S_\ell) \right] \right\} d\sigma_\rho d\tau_{12} d\tau_3.
\end{align}
Omitting terms quadratic in $L_\ell$ or $L_\ell^t$, including $L_\ell^t L_\ell$,
\begin{align}
\label{eq:IlPsiVariation6b}
\delta I_\ell^\prime \sim -\frac{1}{2} \kappa (L_\ell^t - L_\ell) & \int\limits_{V_{12}} \int\limits_{V_3} \int\limits_{S_\rho} \left[\widetilde{S}_d (u_{10} C_\ell - u_{11} S_\ell) - \widetilde{C}_d (u_{00} C_\ell - u_{01} S_\ell) \right] d\sigma_\rho d\tau_{12} d\tau_3 \nonumber \\
= -\frac{1}{2} \kappa (L_\ell^t - L_\ell) & \int\limits_{V_{12}} \int\limits_{V_3} \int\limits_{S_\rho} \left[\widetilde{S}_d u_{10} C_\ell - \widetilde{S}_d u_{11} S_\ell - \widetilde{C}_d u_{00} C_\ell + \widetilde{C}_d u_{01} S_\ell \right] d\sigma_\rho d\tau_{12} d\tau_3 \nonumber \\
= -\frac{1}{2} \kappa (L_\ell^t - L_\ell) & \int\limits_{V_{12}} \int\limits_{V_3} \int\limits_{S_\rho} \left[u_{00} u_{10} S_\ell C_\ell + u_{01} u_{10} C_\ell^2 - u_{00} u_{11} S_\ell^2 - u_{01} u_{11} C_\ell S_\ell\right. \nonumber \\
& \left. - u_{10} u_{00} S_\ell C_\ell - u_{11} u_{00} C_\ell^2 + u_{10} u_{01} S_\ell^2 + u_{11} u_{01} C_\ell S_\ell \right] d\sigma_\rho d\tau_{12} d\tau_3 \nonumber \\
= -\frac{1}{2} \kappa (L_\ell^t - L_\ell) & \int\limits_{V_{12}} \int\limits_{V_3} \int\limits_{S_\rho} (u_{10} u_{01} - u_{00} u_{11}) \left( S_\ell^2 + C_\ell^2 \right) d\sigma_\rho d\tau_{12} d\tau_3 \nonumber \\
= \frac{1}{2} \kappa (L_\ell^t - L_\ell) & \Det{\textbf{u}} \int\limits_{V_{12}} \int\limits_{V_3} \int\limits_{S_\rho} \left( S_\ell^2 + C_\ell^2 \right) d\sigma_\rho d\tau_{12} d\tau_3.
\end{align}

The rest of this derivation considers only the direct terms. The final result
applies as well when the exchanged terms are included. Since we are considering
the surface as $\rho \rightarrow \infty$, $f_\ell(\rho)$ in \cref{eq:GenCDef}
becomes 1. Then from \cref{eq:GenSandC},
\beq
S_\ell^2 + C_\ell^2 = \SphericalHarmonicY{\ell}{0}{\theta_\rho}{\varphi_\rho}^2 \Phi_{Ps}\left(r_{12}\right)^2 \Phi_H\left(r_3\right)^2 (2 \kappa) \left[j_\ell\!\left(\kappa\rho\right)^2 + n_\ell\!\left(\kappa\rho\right)^2\right].
\eeq
The asymptotic forms of $j_\ell$ and $n_\ell$ as $\rho \to \infty$ are given by \cite[p.729]{Arfken2005}
\begin{subequations}
\label{eq:AsymSphBes}
\begin{align}
j_\ell\!\left(\kappa\rho\right) &\sim \frac{1}{\kappa\rho} \sin\left(\kappa\rho - \frac{n \pi}{2}\right) \label{eq:AsymJl} \\
n_\ell\!\left(\kappa\rho\right) &\sim \frac{1}{\kappa\rho} \cos\left(\kappa\rho - \frac{n \pi}{2}\right). \label{eq:AsymNl}
\end{align}
\end{subequations}
As $\rho \to \infty$,
\beq
S_\ell^2 + C_\ell^2 \sim \SphericalHarmonicY{\ell}{0}{\theta_\rho}{\varphi_\rho}^2 \Phi_{Ps}\left(r_{12}\right)^2 \Phi_H\left(r_3\right)^2 (2 \kappa) \frac{1}{\kappa^2 \rho^2}.
\eeq

Substituting this in \cref{eq:IlPsiVariation6b} and expanding the
$d\sigma_\rho$ differential,
\begin{align}
\label{eq:IlPsiVariation7}
\delta I_\ell^\prime \sim& \; \kappa (L_\ell^t - L_\ell) \Det{\textbf{u}} \int\limits_{V_{12}} \int\limits_{V_3} \int\limits_{S_\rho} \SphericalHarmonicY{\ell}{0}{\theta_\rho}{\varphi_\rho}^2 \Phi_{Ps}\left(r_{12}\right)^2 \Phi_H\left(r_3\right)^2 \frac{1}{\kappa \rho^2} \rho^2 \sin\theta_\rho d\theta_\rho d\varphi_\rho d\tau_{12} d\tau_3 \nonumber \\
=& (L_\ell^t - L_\ell) \Det{\textbf{u}} \int\limits_{V_{12}} \int\limits_{V_3} \int\limits_{S_\rho} \SphericalHarmonicY{\ell}{0}{\theta_\rho}{\varphi_\rho}^2 \Phi_{Ps}\left(r_{12}\right)^2 \Phi_H\left(r_3\right)^2 \sin\theta_\rho d\theta_\rho d\varphi_\rho d\tau_{12} d\tau_3.
\end{align}
Since the Ps and H eigenfunctions are normalized, i.e.
\beq
\int\limits_{V_3}\! \left| \Phi_H(r_3) \right|^2 d\tau_3 = 1 \text{ and } \int\limits_{V_{12}}\! \left|\Phi_{Ps}(r_{12})\right|^2 d\tau_{12} = 1,
\label{eq:PsHNormalization}
\eeq
we now have
\beq
\label{eq:IlPsiVariation8}
\delta I_\ell^\prime = (L_\ell^t - L_\ell) \Det{\textbf{u}} \int\limits_{S_\rho} \SphericalHarmonicY{\ell}{0}{\theta_\rho}{\varphi_\rho}^2 \sin\theta_\rho d\theta_\rho d\varphi_\rho.
\eeq
The spherical harmonics are normalized so that \cite[p.788]{Arfken2005}
\beq
\label{eq:SphHarmNorm}
\int\limits_{S_\rho} \left| \SphericalHarmonicY{\ell}{0}{\theta_\rho}{\varphi_\rho} \right|^2 d\Omega = 1.
\eeq
This gives that
\beq
\label{eq:IlPsiVariation9}
\delta I_\ell^\prime = (L_\ell^t - L_\ell) \Det{\textbf{u}}.
\eeq

From \cref{eq:IlPrimeDef,eq:IlPsiVariation9},
\beq
\label{eq:KatoIdent}
\delta I_\ell^\prime = (L_\ell^t - L_\ell) \Det{\textbf{u}} + (\delta\Psi_\ell, \mathcal{L} \,\delta\Psi_\ell).
\eeq
This is the Kato identity \cite{Kato1951a}. For the Kohn-type variational methods, the last term is neglected, since it is quadratic in $\delta\Psi_\ell$. Using $\delta I_\ell^\prime \approx \delta I_\ell$, we have
\beq
\delta I_\ell = I_\ell[\Psi_\ell^t] - I_\ell[\Psi_\ell] \approx (L_\ell^t - L_\ell) \Det{\textbf{u}}.
\eeq
Replacing the exact $L_\ell$ by the variational $L_\ell^v$ and rearranging,
we finally get the general Kohn variational method of
\beq
\label{eq:GenKohn}
L_\ell^v = L_\ell^t - I_\ell[\Psi_\ell^t] / \! \Det{\textbf{u}},
\eeq
which is correct to second-order.
This was only derived using the direct terms, but the exchange terms follow 
the same steps with $\rho^\prime$ instead of $\rho$. This was also only shown 
for the long-range terms, but it applies equally as well to the full
wavefunction with the short-range terms.


\section{Application of the Kohn Methods}
\label{sec:KohnApplied}

We use the general Kohn variational method (\cref{eq:GenKohn}) with the full
trial wavefunction to get
\beq
\label{eq:GenKohnApplied}
L_\ell^v = L_\ell^t - \tfrac{1}{\Det{\textbf{u}}} \Big((\widetilde{S}_\ell + L_\ell^t \, \widetilde{C}_\ell + \sum_i c_i \bar{\phi}_i^t), \mathcal{L} (\widetilde{S}_\ell + L_\ell^t \, \widetilde{C}_\ell + \sum_j c_j \bar{\phi}_j^t )\Big).
\eeq
The property of the Kohn functional that it is stationary with respect to
variations in the linear parameters \cite{Joachain1979} can be written in
this case as
\beq
\frac{\partial L_\ell^v}{\partial L_\ell^t} = 0  \text{ and } \frac{\partial L_\ell^v}{\partial c_i} = 0 \text{, where $i = 1,\ldots,N$}.
\label{eq:KohnStationary}
\eeq

Performing the first variation gives
\begin{align}
\nonumber 0 &= \pderiv{L_\ell^v}{L_\ell^t} \\
&= 1 - \left[(\widetilde{S}_\ell,\mathcal{L}\widetilde{C}_\ell) + (\widetilde{C}_\ell,\mathcal{L}\widetilde{S}_\ell) + \frac{\partial}{\partial L_\ell^t}(L_\ell^t \widetilde{C}_\ell,\mathcal{L} L_\ell^t \widetilde{C}_\ell) + (\widetilde{C}_\ell, \mathcal{L} \sum_i c_i \bar{\phi}_i) + (\sum_i c_i \bar{\phi}_i, \mathcal{L} \widetilde{C}_\ell) \right].
\label{eq:PdLambda1}
\end{align}

\noindent The third term in brackets becomes
\beq
\pderiv{}{L_\ell^t} (L_\ell^t \widetilde{C}_\ell,\mathcal{L} L_\ell^t \widetilde{C}_\ell) = (\widetilde{C}_\ell,\mathcal{L} \widetilde{C}_\ell) \frac{\partial}{\partial L_\ell^t} {L_\ell^t}^2 = 2(\widetilde{C}_\ell,\mathcal{L}\widetilde{C}_\ell) L_\ell^t.
\eeq

\noindent The last two terms of \cref{eq:PdLambda1} are equal to each other, and we can use \cref{eq:GenSLCandCLS} to rewrite this.
\beq
0 = -(\widetilde{C}_\ell,\mathcal{L}\widetilde{S}_\ell) - (\widetilde{C}_\ell,\mathcal{L}\widetilde{S}_\ell) - 2 L_\ell^t (\widetilde{C}_\ell,\mathcal{L}\widetilde{C}_\ell) - 2 \sum_i c_i (\widetilde{C}_\ell,\mathcal{L}\bar{\phi}_i)
\eeq

\noindent Rearranging gives
\beq
-(\widetilde{C}_\ell,\mathcal{L}\widetilde{S}_\ell) = L_\ell^t (\widetilde{C}_\ell,\mathcal{L}\widetilde{C}_\ell) + \sum_i c_i (\widetilde{C}_\ell,\mathcal{L}\bar{\phi}_i)
\label{eq:PdLambda}
\eeq

Now we perform the variation with respect to a general $c_k$ as in \cref{eq:KohnStationary}.
\beq
0 = \frac{\partial \mathcal{L}_v}{\partial c_k} = -\!\!\left[ (\widetilde{S}_\ell,\mathcal{L} \bar{\phi}_k) + L_\ell^t (\widetilde{C}_\ell,\mathcal{L} \bar{\phi}_k) + (\bar{\phi}_k,\mathcal{L} \widetilde{S}_\ell) + L_\ell^t (\bar{\phi}_k,\mathcal{L} \widetilde{C}_\ell) + \frac{\partial}{\partial c_k} (\sum_i c_i \bar{\phi}_i, \mathcal{L} \sum_j c_j \bar{\phi}_j) \right]
\label{eq:PdCk1}
\eeq
%If $c_i \ne c_j$,
%\begin{subequations}
%\begin{align}
%\frac{\partial}{\partial c_i} (c_i \bar{\phi}_i, \mathcal{L} \sum_{j \ne i} c_j \bar{\phi}_j) &= \sum_{j \ne i} c_j (\bar{\phi}_i, \mathcal{L} \bar{\phi}_j) \text{ and} \\
%\frac{\partial}{\partial c_j} (\sum_{i \ne j} c_i \bar{\phi}_i, \mathcal{L} c_j \bar{\phi}_j) &= \sum_{i \ne j} c_i (\bar{\phi}_i, \mathcal{L} \bar{\phi}_j).
%\end{align}
%\end{subequations}
%These two equations are equivalent, since
%$\left( \bar{\phi}_i, \mathcal{L} \bar{\phi}_j \right) = \left( \bar{\phi}_j, \mathcal{L} \bar{\phi}_i \right)$
%by \cref{eq:ShortElemSymm}.
%
%If $c_i = c_j$,
%\beq
%\frac{\partial}{\partial c_i} \left( c_i \bar{\phi}_i, \mathcal{L} c_j \bar{\phi}_j \right) = \frac{\partial}{\partial c_i} \left( c_i \bar{\phi}_i, \mathcal{L} c_i \bar{\phi}_i \right) = \frac{\partial}{\partial c_i} c_i^2 \left( \bar{\phi}_i, \mathcal{L} \bar{\phi}_j \right) = 2 \, c_i \left( \bar{\phi}_i, \mathcal{L} \bar{\phi}_j \right).
%\eeq
%We can also use \cref{eq:ShortElemSymm} to reduce \cref{eq:PdCk1} to
%\beq
%0 = -\Big[ 2 (\bar{\phi}_k, \mathcal{L} \widetilde{S}_\ell) + 2 L_\ell^t (\bar{\phi}_k, \mathcal{L} \widetilde{C}_\ell) + 2 \sum_i (\bar{\phi}_k, \mathcal{L} c_i \bar{\phi}_i) \Big].
%\eeq
Rearranging gives
\beq
-\left( \bar{\phi}_k, \mathcal{L} \widetilde{S}_\ell \right) = L_\ell^t \left( \bar{\phi}_k, \mathcal{L} \widetilde{C}_\ell \right) + \sum_i \left( \bar{\phi}_k, \mathcal{L} c_i \bar{\phi}_i \right).
\label{eq:PdCk}
\eeq

The set of linear equations in \cref{eq:PdLambda,eq:PdCk} can be written in matrix form as
\begin{equation}
\label{eq:GeneralKohnMatrix}
\begin{bmatrix} 
 (\widetilde{C}_\ell,\mathcal{L}\widetilde{C}_\ell) & (\widetilde{C}_\ell,\mathcal{L}\bar{\phi}_1) & \cdots & (\widetilde{C}_\ell,\mathcal{L}\bar{\phi}_j) & \cdots\\
 (\bar{\phi}_1,\mathcal{L}\widetilde{C}_\ell) & (\bar{\phi}_1,\mathcal{L}\bar{\phi}_1) & \cdots & (\bar{\phi}_1,\mathcal{L}\bar{\phi}_j) & \cdots\\
 \vdots & \vdots & \ddots & \vdots \\
 (\bar{\phi}_i,\mathcal{L}\widetilde{C}_\ell) & (\bar{\phi}_i,\mathcal{L}\bar{\phi}_1) & \cdots & (\bar{\phi}_i,\mathcal{L}\bar{\phi}_j) & \cdots\\
 \vdots & \vdots & & \vdots & \\
\end{bmatrix}
\begin{bmatrix}
L_\ell^t\\
c_1\\
\vdots\\
c_i\\
\vdots
\end{bmatrix}
= -
\begin{bmatrix}
(\widetilde{C}_\ell,\mathcal{L}\widetilde{S}_\ell) \\
(\bar{\phi}_1,\mathcal{L}\widetilde{S}_\ell) \\
\vdots \\
(\bar{\phi}_i,\mathcal{L}\widetilde{S}_\ell) \\
\vdots
\end{bmatrix}.
\end{equation}

\noindent This matrix equation can be rewritten as
\beq
\label{eq:GenKohnMatrixAXB}
\textbf{\emph{AX}} = -\textbf{\emph{B}}.
\eeq

\noindent Solving this for $\textbf{\emph{X}}$ gives
\beq
\textbf{\emph{X}} = -\textbf{\emph{A}}^{-1}\textbf{\emph{B}}.
\eeq

To obtain $L_\ell^v$ from this matrix equation, we must next expand \cref{eq:GenKohnApplied}.

\begin{align}
\nonumber L_\ell^v = L_\ell^t - &\left[ (\widetilde{S}_\ell,\mathcal{L}\widetilde{S}_\ell) + L_\ell^t (\widetilde{S}_\ell,\mathcal{L}\widetilde{C}_\ell) + \sum_i c_i (\widetilde{S}_\ell,\mathcal{L} \bar{\phi}_i) + L_\ell^t (\widetilde{C}_\ell,\mathcal{L}\widetilde{S}_\ell) + {L_\ell^t}^2 (\widetilde{C}_\ell,\mathcal{L}\widetilde{C}_\ell)  \right. \\
& + \left. L_\ell^t \sum_i c_i (\widetilde{C}_\ell,\mathcal{L} \bar{\phi}_i) + \sum_i c_i (\bar{\phi}_i, \mathcal{L} \widetilde{S}_\ell) + L_\ell^t \sum_i c_i (\bar{\phi}_i, \mathcal{L} \widetilde{C}_\ell) + \sum_i \sum_j c_i c_j (\bar{\phi}_i, \mathcal{L} \bar{\phi}_j) \right]
\end{align}

\noindent By substituting \cref{eq:GenSLCandCLS} in for $(\widetilde{S}_\ell,\mathcal{L}\widetilde{C}_\ell)$, the first $L_\ell^t$ above is canceled, leaving

\begin{align}
\label{eq:GenKohnApplied2}
L_\ell^v = - & \left[ (\widetilde{S}_\ell,\mathcal{L}\widetilde{S}_\ell) + L_\ell^t (\widetilde{C}_\ell,\mathcal{L}\widetilde{S}_\ell) + \sum_i c_i (\widetilde{S}_\ell, \mathcal{L} \bar{\phi}_i) + L_\ell^t (\widetilde{C}_\ell,\mathcal{L}\widetilde{S}_\ell) + {L_\ell^t}^2 (\widetilde{C}_\ell,\mathcal{L}\widetilde{C}_\ell) \right.  \nonumber \\
& + \left. L_\ell^t \sum_i c_i (\widetilde{C}_\ell,\mathcal{L} \bar{\phi}_i)
+ \sum_i c_i (\bar{\phi}_i, \mathcal{L} \widetilde{S}_\ell) + L_\ell^t \sum_i c_i (\bar{\phi}_i, \mathcal{L} \widetilde{C}_\ell) + \sum_i \sum_j c_i c_j (\bar{\phi}_i, \mathcal{L} \bar{\phi}_j) \right].
\end{align}

Using the following definitions of
\beq
D = 
\begin{bmatrix}
L_\ell^t & c_1 & \cdots & c_N
\end{bmatrix}
\text{ and}
\eeq
\beq
\label{eq:GenFandD}
F =
\begin{bmatrix}
(\boldsymbol{\widetilde{C}_\ell,\mathcal{L}\widetilde{C}_\ell}) & (\boldsymbol{\widetilde{C}_\ell,\mathcal{L}\bar{\phi}}) & (\boldsymbol{\widetilde{C}_\ell,\mathcal{L}\widetilde{S}_\ell}) \\
(\boldsymbol{\bar{\phi},\mathcal{L}\widetilde{C}_\ell}) & (\boldsymbol{\bar{\phi},\mathcal{L}\bar{\phi}}) & (\boldsymbol{\bar{\phi},\mathcal{L}\widetilde{S}_\ell}) \\
(\boldsymbol{\widetilde{C}_\ell,\mathcal{L}\widetilde{S}_\ell}) & (\boldsymbol{\widetilde{S}_\ell,\mathcal{L}\bar{\phi}}) & (\boldsymbol{\widetilde{S}_\ell,\mathcal{L}\widetilde{S}_\ell})
\end{bmatrix},
\eeq
\cref{eq:GenKohnApplied2} can be rewritten as the following matrix equation:

\beq
\label{eq:GenDFDT}
L_\ell^v = - D F D^T.
\eeq

\noindent Using \cref{eq:GenKohnMatrixAXB} in \cref{eq:GenDFDT} and expanding gives
\begin{align}
\label{eq:GenDFDT2}
\nonumber L_\ell^v &= - 
\begin{bmatrix}
\boldsymbol{X^T} & 1 
\end{bmatrix}
\begin{bmatrix}
\boldsymbol{A} & \boldsymbol{B} \\
\boldsymbol{B^T} & \boldsymbol{(\widetilde{S}_\ell,\mathcal{L}\widetilde{S}_\ell)}
\end{bmatrix}
\begin{bmatrix}
\boldsymbol{X} \\
1
\end{bmatrix}
= -
\begin{bmatrix}
\boldsymbol{X^T} & 1 
\end{bmatrix}
\begin{bmatrix}
0 \\
\boldsymbol{B^T X} + (\widetilde{S}_\ell,\mathcal{L}\widetilde{S}_\ell)
\end{bmatrix} \\
&= -\boldsymbol{B^T X} - (\widetilde{S}_\ell,\mathcal{L}\widetilde{S}_\ell),
\end{align}
where
\beq
\boldsymbol{B^T X} = L_\ell^t (\widetilde{C}_\ell,\mathcal{L}\widetilde{S}_\ell) + \sum_i c_i (\bar{\phi}_i, \mathcal{L} \widetilde{S}_\ell).
\eeq

\noindent A more compact way of writing \cref{eq:GenDFDT2} is by
\beq
L_\ell^v = -\left( \Psi^{t,0},\mathcal{L} \widetilde{S}_\ell \right).
\eeq
$\Psi^{t,0}$ is the full general wavefunction in \cref{eq:GeneralWaveTrial} with its nonlinear parameters optimized.
Finally, to obtain the phase shifts, we use the relation given by Ref.~\cite{Lucchese1989} as
\begin{equation}
\label{eq:GenKohnL}
K_\ell = \tan \delta_\ell = (u_{01} + u_{11} L_\ell)(u_{00} + u_{10} L_\ell)^{-1}.
\end{equation}

The $\textbf{u}$ and $L^{\pm,t}_\ell$ for the various Kohn methods are 
described now. Note that for each of these, $\Det{\textbf{u}} = 1$, except
for the ones describing the $S$-matrix complex Kohn and generalized
$S$-matrix complex Kohn. When we 
create the matrix in \cref{eq:GeneralKohnMatrix}, we only calculate the 
matrix elements for the Kohn, along with $(\bar{S}_\ell,\mathcal{L}\bar{S}_\ell)$
and $(\bar{S}_\ell,\mathcal{L}\bar{C}_\ell)$. Then from the definitions in
\cref{eq:TildeSCDef}, this matrix can be changed to any of the other Kohn 
methods without recomputing any of the integrals. %This is described further 
%in \cref{sec:PhaseProgram}.

\subsubsection*{Kohn}
\label{sec:Kohn}
\beq
\textbf{u} =
\begin{bmatrix}
1 & 0 \\
0 & 1 
\end{bmatrix}
\label{eq:uKohn}
\eeq

\beq
L^{\pm,t}_\ell = \lambda_t = K_t
\label{eq:LKohn}
\eeq


\subsubsection*{Inverse Kohn}
\label{sec:InvKohn}
\beq
\textbf{u} =
\begin{bmatrix}
0 & 1 \\
-1 & 0 
\end{bmatrix}
\label{eq:uInvKohn}
\eeq

\beq
L^{\pm,t}_\ell = -\mu_t = -K^{-1}_t = -\bar{K}_t
\label{eq:LInvKohn}
\eeq


\subsubsection*{Generalized Kohn}
\label{sec:GenKohn}
\beq
\textbf{u} =
\begin{bmatrix}
\cos\tau & \sin\tau \\
-\sin\tau & \cos\tau 
\end{bmatrix}
\label{eq:uGenKohn}
\eeq
%\beq
%\mathcal{L}_l = a_t
%\label{eq:LGenKohn}
%\eeq
The generalized Kohn method is described by Cooper et al.\ \cite{
Cooper2009, Cooper2010}.  When $\tau = 0$ is substituted in \cref{eq:uGenKohn}
, the $\textbf{u}$-matrix for the Kohn method is generated (\cref{eq:uKohn}). 
Similarly, when $\tau = \frac{\pi}{2}$, the $\textbf{u}$-matrix for the 
inverse Kohn method is generated (\cref{eq:uInvKohn}).


\subsubsection*{Complex Kohn $T$-matrix}
\label{sec:ComplexTKohn}
\beq
\textbf{u} =
\begin{bmatrix}
1 & 0 \\
\ii & 1
\end{bmatrix}
\label{eq:uCompTKohn}
\eeq

\beq
L^{\pm,t}_\ell = T_\ell
\label{eq:LCompTKohn}
\eeq
Lucchese \cite{Lucchese1989} denotes this as $\mathcal{L}_\ell = -\pi T$,
but we use the definition of the $T$-matrix from Bransden \cite{Bransden1970}:
\begin{equation}
\label{eq:TMatrix}
K_\ell = \frac{T_\ell}{1 + \ii T_\ell}
\end{equation}

\subsubsection*{Complex Kohn $S$-matrix}
\label{sec:ComplexSKohn}
\beq
\textbf{u} =
\begin{bmatrix}
-\ii & 1 \\
\ii & 1
\end{bmatrix}
\label{eq:uCompSKohn}
\eeq

\beq
%\mathcal{L}_\ell = 2 \ii S_\ell
L^{\pm,t}_\ell = -S_\ell
\label{eq:LCompSKohn}
\eeq

The Lucchese \cite{Lucchese1989} version of $\textbf{u}$ differs from this, 
since he uses a different definition for the $S$-matrix. The form of the
$S$-matrix we are using is related to the $K$-matrix by
\cite{Bransden2003}
\begin{equation}
\label{eq:SMatrix}
K_\ell = \frac{\ii(1-S_\ell)}{1+S_\ell},
\end{equation}
which is satisfied by the above $\textbf{u}$-matrix. Also of note is that
$\det \textbf{u} = -2\ii$ instead of 1 like most of the other Kohn methods
presented here. Cooper et al.~\cite{Cooper2010} use the $T$-matrix but also
provide a relation between the two.

\subsubsection*{Generalized $T$-matrix Complex Kohn}
\label{sec:GenComplexTKohn}
\beq
\textbf{u} =
\begin{bmatrix}
\cos\tau & \sin\tau \\
-\sin\tau + \ii \cos\tau & \cos\tau + \ii \sin\tau
\end{bmatrix}
\label{eq:uGenTKohn}
\eeq
This is a generalized form of the $T$-matrix complex Kohn, similar to how the 
generalized Kohn works. When $\tau = 0$, this reduces to the $T$-matrix complex
Kohn. This is also a slightly different from than that of Cooper
et al.~\cite{Cooper2010}.

\subsubsection*{Generalized $S$-matrix Complex Kohn}
\label{sec:GenComplexSKohn}
\beq
\textbf{u} =
\begin{bmatrix}
-\ii \cos\tau - \sin\tau & -\ii \sin\tau + \cos\tau \\
\ii \cos\tau - \sin\tau & \ii \sin\tau + \cos\tau
\end{bmatrix}
\label{eq:uGenSKohn}
\eeq
This is a generalized form of the $S$-matrix complex Kohn. When $\tau = 0$, 
this reduces to the $S$-matrix complex Kohn.


\section{Matrix Elements}
In this section, we examine the matrix elements of \cref{eq:GeneralKohnMatrix}.
The three types of matrix elements are short-range--short-range (short-short),
short-range--long-range (short-long) and long-range--long-range (long-long).
The short-long and long-long matrix elements have a similar analysis. For 
these, the effect of the $\mathcal{L} = 2(H-E)$ operator on the long-range 
terms must be considered, and then integrations over the external angles (see 
\cref{chp:AngularInt}) are performed. The remaining 6-dimensional integral is 
then numerically integrated as described in \cref{sec:LongLongInt,sec:ShortLongInt}.

For all of these matrix elements, looking at the barred terms, we have 4
integrations to perform. Using a property of the $P_{23}$ permutation operator,
for a general $f$ and $g$,
\beq
\label{eq:PermProp}
(f,\mathcal{L}g) = (f^\prime,\mathcal{L}g^\prime) \text{ and } (f,\mathcal{L}g^\prime) = (f^\prime,\mathcal{L}g).
\eeq
The functions $f$ and $g$ are any of $\widetilde{S}_\ell$, $\widetilde{C}_\ell$
and $\bar{\phi}_i$. This relation allows us to reduce the number of integrations
needed by half by doing \cite{VanReethThesis}
\beq
\label{eq:PermPropFull}
(\bar{f},\mathcal{L}\bar{g}) = (f,\mathcal{L}g) \pm (f,\mathcal{L}g^\prime) \pm (f^\prime,\mathcal{L}g) + (f^\prime,\mathcal{L}g^\prime) = 2\left[(f,\mathcal{L}g) \pm (f,\mathcal{L}g^\prime)\right].
\eeq


\subsection{Matrix Element Symmetries}
\label{sec:Symmetries}

Not all matrix elements in \cref{eq:GeneralKohnMatrix} have to be calculated 
as presented. Some matrix elements are identical to other matrix elements, 
such as $(\widetilde{C}_\ell,\mathcal{L}\bar{\phi}_i) = (\bar{\phi}_i,\mathcal{L}\widetilde{C}_\ell)$.
In this particular case, it is much easier to 
calculate $(\bar{\phi}_i,\mathcal{L}\widetilde{C}_\ell)$ instead of
$(\widetilde{C}_\ell,\mathcal{L}\bar{\phi}_i)$, due to the complexity of 
operating $\mathcal{L}$ on $\bar{\phi}_i$ (see \cref{eq:GradGradShort,tab:LCList}).
We prove these claims in this section.

These arguments follow that of Appendix A of Peter Van Reeth's thesis \cite{VanReethThesis}.
We start with the functional
\begin{align}
	\label{eq:Fsymm}
	F \equiv \left(g,\mathcal{L}f \right)-\left(f,\mathcal{L}g \right) \; ,
\end{align}
with $\mathcal{L}$ given by \cref{eq:LDef}.
Using the Hamiltonian given by \cref{eq:Hamiltonian1}, only the first
three terms of the above functional have to be evaluated, as the other terms go to 0 with the subtraction.
\begin{align}
	F=&\left({-g,\frac{1}{2} \Laplacian_{r_1} f}\right)+\left({f,\frac{1}{2} \Laplacian_{r_1} g}\right)+
	\left({-g,\frac{1}{2} \Laplacian_{r_2} f}\right)+\left({f,\frac{1}{2} \Laplacian_{r_2} g}\right) \nonumber \\
	&+\left({-g,\frac{1}{2} \Laplacian_{r_3} f}\right)+\left({f,\frac{1}{2} \Laplacian_{r_3} g}\right)  \nonumber \\
=& \int\limits_{V_3} \int\limits_{V_2} \int\limits_{V_1} \left[
	-g  \Laplacian_{r_1} f + f \Laplacian_{r_1} g
	-g \Laplacian_{r_2} f + f \Laplacian_{r_2} g - g \Laplacian_{r_3} f + f \Laplacian_{r_3} g \right] \,d\tau_1 d\tau_2 d\tau_3
\end{align}
Using Green's theorem on each pair of terms,
\begin{align}
F = \int\limits_{V_3} \int\limits_{V_2} \int\limits_{S_1} & \left[-g \grad_{r_1} f + f \grad_{r_1} g \right] \cdot d\bm{\sigma}_1 d\tau_2 d\tau_3 + \int\limits_{V_3} \int\limits_{V_1} \int\limits_{S_2} \left[ -g \grad_{r_2} f + f \grad_{r_2} g \right] \cdot d\bm{\sigma}_2 d\tau_1 d\tau_3 \nonumber \\
	& + \int\limits_{V_1} \int\limits_{V_2} \int\limits_{S_3} \left[ -g \grad_{r_3} f + f \grad_{r_3} g \right] \cdot d\bm{\sigma}_3 d\tau_2 d\tau_1 \;.
\end{align}
For the first pair of terms, the differential surface element contains $r_1^2$.
The surface we are integrating over is at $r_1 \rightarrow \infty$, so if 
the integrand falls off faster than $r_1^{-2}$, the integral vanishes, the 
same as the argument in \cref{eq:IlPsiVariation4}. The same argument applies 
for $r_2^2$ and $r_3^2$ in the second and third pairs of terms, respectively. 
The short-range Hylleraas-type terms in \cref{eq:PhiDef} fulfill this 
requirement for $r_1$, $r_2$, and $r_3$, so if the matrix elements contain
$\bar{\phi}_i$, the right hand side is equal to 0. From \cref{eq:Fsymm}, this 
gives that matrix elements with short-range terms are symmetric, or that
\begin{subequations}
\label{eq:ShortElemSymm}
\begin{align}
\left(\bar{\phi}_i, \mathcal{L} \bar{\phi}_j \right) &= \left(\bar{\phi}_j, \mathcal{L} \bar{\phi}_i \right) \\
\left(\bar{\phi}_i, \mathcal{L} \widetilde{S}_\ell \right) &= \left(\widetilde{S}_\ell, \mathcal{L} \bar{\phi}_i \right) \label{eq:SLPhi} \\
\left(\bar{\phi}_i, \mathcal{L} \widetilde{C}_\ell \right) &= \left(\widetilde{C}_\ell, \mathcal{L} \bar{\phi}_i \right) \label{eq:CLPhi}.
\end{align}
\end{subequations}
From \cref{eq:LDef,eq:BoundHFull,eq:GradGradShort}, the form of $\mathcal{L}\phi_i$ is very complicated, but \cref{eq:SLPhi,eq:CLPhi} allow us to avoid having to operate $\mathcal{L}$ on the $\bar{\phi}_i$ terms.

Now if we let $g = \bar{S}_\ell$ and $f = \bar{C}_\ell$ in \cref{eq:Fsymm},
\begin{align}
F &= \left(\frac{1}{\sqrt{2}} \left[S_\ell \pm S_\ell^\prime \right], \mathcal{L} \frac{1}{\sqrt{2}}\left[C_\ell \pm C_\ell^\prime \right] \right) -
    \left(\frac{1}{\sqrt{2}} \left[C_\ell \pm C_\ell^\prime \right], \mathcal{L} \frac{1}{\sqrt{2}}\left[S_\ell \pm S_\ell^\prime \right] \right) \nonumber \\
& = \frac{1}{2}\left[(S_\ell,\mathcal{L}C_\ell) \pm (S_\ell,\mathcal{L}C_\ell^\prime) \pm (S_\ell^\prime,\mathcal{L}C_\ell) + (S_\ell^\prime,\mathcal{L}C_\ell^\prime) \right. \nonumber\\
& \left. \phantom{move} - (C_\ell,\mathcal{L}S_\ell) \mp (C_\ell,\mathcal{L}S_\ell^\prime) \mp (C_\ell^\prime,\mathcal{L}S_\ell) - (C_\ell^\prime,\mathcal{L}S_\ell^\prime)\right]
\end{align}
From the property of the permutation operators, $(S_\ell,\mathcal{L}C_\ell) = 
(S_\ell^\prime,\mathcal{L}C_\ell^\prime)$, $(S_\ell^\prime,\mathcal{L}C_\ell) 
= (S_\ell,\mathcal{L}C_\ell^\prime)$, $(C_\ell,\mathcal{L}S_\ell) = (S_\ell^
\prime,\mathcal{L}C_\ell^\prime)$ and
$(C_\ell^\prime,\mathcal{L}S_\ell) = (C_\ell,\mathcal{L}S_\ell^\prime)$,
causing the above to reduce to
\begin{align}
F = &\left[ (S_\ell,\mathcal{L}C_\ell) - (C_\ell,\mathcal{L}S_\ell)\right] \pm \left[ (S_\ell,\mathcal{L}C_\ell^\prime) - (C_\ell^\prime,\mathcal{L}S_\ell)\right] \nonumber \\
&\equiv G \pm G^\prime.
\label{GBarDef}
\end{align}
Using \cref{eq:Hamiltonian2} and Green's theorem,
\begin{align}
G &= \frac{1}{2} \int\limits_{V_3} \int\limits_{V_{12}} \int\limits_{S_\rho} \left[ S_\ell \grad_\rho C_\ell - C_\ell \grad_\rho S_\ell \right] \cdot d\bm{\sigma}_\rho d\tau_{12} d\tau_3  \nonumber \\
  &+ \int\limits_{V_\rho} \int\limits_{V_{12}} \int\limits_{S_3} \left[ S_\ell \grad_{r_3} C_\ell - C_\ell \grad_{r_3} S_\ell\right] \cdot d\bm{\sigma}_3 d\tau_{12} d\tau_{\rho} \nonumber \\
  &+ 2 \int\limits_{V_\rho} \int\limits_{V_{13}}\int\limits_{S_{12}} \left[ S_\ell \grad_{r_{12}} C_\ell - C_\ell \grad_{r_{12}} S_\ell \right] \cdot d\bm{\sigma}_{12} d\tau_{13} d\tau_\rho
  \label{eq:GDef}
\end{align}
As before, the Ps and H functions have an exponential dependence on $r_3$ and 
$r_{12}$, respectively, so the second and third terms go to 0. The surface 
elements under consideration at $\rho \to \infty$ are normal to $\hat{\rho}$, 
so we can ignore the angular dependence in $\grad_\rho$. Then \cref{eq:GDef} 
becomes
\beq
G = \frac{1}{2} \int\limits_{V_3} \int\limits_{V_{12}} \left[\int\limits_{S_\rho} \left(S_\ell \frac{\partial C_\ell}{\partial \rho} - C_\ell \frac{\partial S}{\partial \rho} \right) \rho^2 \sin \theta_\rho d\theta_\rho d\varphi_\rho \right] d\tau_{12} d\tau_3.
\label{eq:GDef2}
\eeq

Using \cref{eq:GenSandC} and realizing that $f_\ell \rightarrow 1$ as
$\rho \to \infty$, from the \emph{Mathematica} notebook ``SLC - CLS Proof.nb''
\cite{Wiki}, we have
\begin{equation}
\label{eq:CSderdiff}
\left(S_\ell \frac{\partial C_\ell}{\partial \rho} - C_\ell \frac{\partial S_\ell}{\partial \rho} \right) = \frac{2}{\rho^2} \SphericalHarmonicY{\ell}{0}{\theta_\rho}{\varphi_\rho}^2 \Phi_{Ps}\left(r_{12}\right)^2 \Phi_H\left(r_3\right)^2 .
\end{equation}
Substituting \cref{eq:CSderdiff} into \cref{eq:GDef2} yields
\begin{align}
G &= \cancel{\frac{1}{2}} \int\limits_{V_3} \int\limits_{V_{12}} \left[\int\limits_{S_\rho} \cancel{\frac{2}{\rho^2}} \SphericalHarmonicY{\ell}{0}{\theta_\rho}{\varphi_\rho}^2 \Phi_{Ps}\left(r_{12}\right)^2 \Phi_H\left(r_3\right)^2 \cancel{\rho^2} \sin \theta_\rho d\theta_\rho d\varphi_\rho \right] d\tau_{12} d\tau_3  \nonumber \\
&= \int\limits_{V_3} \int\limits_{V_{12}} \Phi_{Ps}(r_{12})^2 \Phi_{H}(r_{3})^2 \left[ \int\limits_{S_\rho} \SphericalHarmonicY{\ell}{0}{\theta_\rho}{\varphi_\rho}^2 \sin \theta_\rho d\theta_\rho d\varphi_\rho \right] d\tau_{12} d\tau_3  \nonumber \\
&= \int\limits_{V_3} \int\limits_{V_{12}} \Phi_{Ps}(r_{12})^2 \Phi_{H}(r_{3})^2 d\tau_{12} d\tau_3 = 1,
\label{eq:GDef2}
\end{align}
which follows from the orthonormality of the spherical harmonics and the
normalization of the Ps and H wavefunctions. These, when combined in
\cref{GBarDef}, give that
\beq
(S_\ell,\mathcal{L}C_\ell) = (C_\ell,\mathcal{L}S_\ell) + 1.
\label{eq:SLCandCLSdirect}
\eeq
As this also applies to the permuted versions, writing this in terms of $\bar{S}_\ell$ and $\bar{C}_\ell$ gives the final relation of
\beq
\label{eq:SLCandCLS}
\left(\bar{S}_\ell,\mathcal{L}\bar{C}_\ell\right) = \left(\bar{C}_\ell,\mathcal{L}\bar{S}_\ell\right) + 1.
\eeq

This can also be shown more generally for the $(\widetilde{S}_\ell,\mathcal{L}\widetilde{C}_\ell)$
and $(\widetilde{C}_\ell,\mathcal{L}\widetilde{S}_\ell)$ matrix elements.
From the definitions of $\widetilde{S}_\ell$ and $\widetilde{C}_\ell$ in \cref{eq:TildeSCDef},
\begin{align}
\nonumber (\widetilde{S}_\ell,\mathcal{L}\widetilde{C}_\ell) &= \left((u_{00}\bar{S}_\ell + u_{01}\bar{C}_\ell),\mathcal{L}(u_{10}\bar{S}_\ell + u_{11}\bar{C}_\ell)\right) \\
&= u_{00} u_{10} (\bar{S}_\ell,\mathcal{L}\bar{S}_\ell) + u_{00} u_{11} (\bar{S}_\ell,\mathcal{L}\bar{C}_\ell) + u_{01} u_{10} (\bar{C}_\ell,\mathcal{L}\bar{S}_\ell) + u_{01} u_{11} (\bar{C}_\ell,\mathcal{L}\bar{C}_\ell) .
\label{eq:GenSLC}
\end{align}
Likewise,
\begin{align}
\nonumber (\widetilde{C}_\ell,\mathcal{L}\widetilde{S}_\ell) &= \left((u_{10}\bar{S}_\ell + u_{11}\bar{C}_\ell),\mathcal{L}(u_{00}\bar{S}_\ell + u_{01}\bar{C}_\ell)\right) \\
&= u_{10} u_{00} (\bar{S}_\ell,\mathcal{L}\bar{S}_\ell) + u_{10} u_{01} (\bar{S}_\ell,\mathcal{L}\bar{C}_\ell) + u_{11} u_{00} (\bar{C}_\ell,\mathcal{L}\bar{S}_\ell) + u_{11} u_{01} (\bar{C}_\ell,\mathcal{L}\bar{C}_\ell).
\label{eq:GenCLS}
\end{align}

\noindent Combining \cref{eq:GenSLC,eq:GenCLS} gives
\begin{align}
\nonumber (\widetilde{S}_\ell,\mathcal{L}\widetilde{C}_\ell) - (\widetilde{C}_\ell,\mathcal{L}\widetilde{S}_\ell) = \,\, &[u_{00} u_{10} - u_{10} u_{00}] (\bar{S}_\ell,\mathcal{L}\bar{S}_\ell) + [u_{00} u_{11} - u_{10} u_{01}] (\bar{S}_\ell,\mathcal{L}\bar{C}_\ell) \\
\nonumber + &[u_{01} u_{10} - u_{11} u_{00}] (\bar{C}_\ell,\mathcal{L}\bar{S}_\ell) + [u_{01} u_{11} - u_{11} u_{01}] (\bar{C}_\ell,\mathcal{L}\bar{C}_\ell) \\
\nonumber = \,\, &[u_{00} u_{11} - u_{10} u_{01}] [(\bar{S}_\ell,\mathcal{L}\bar{C}_\ell) - (\bar{C}_\ell,\mathcal{L}\bar{S}_\ell)] \\
\nonumber = \,\, & \Det{\textbf{u}} [(\bar{S}_\ell,\mathcal{L}\bar{C}_\ell) - (\bar{C}_\ell,\mathcal{L}\bar{S}_\ell)]
\end{align}

%\noindent Each of our $\textbf{u}$ matrices satisfies $\Det{\textbf{u}} = 1$, making this
%\beq
%(\widetilde{S}_\ell,\mathcal{L}\widetilde{C}_\ell) - (\widetilde{C}_\ell,\mathcal{L}\widetilde{S}_\ell) = (\bar{S}_\ell,\mathcal{L}\bar{C}_\ell) - (\bar{C}_\ell,\mathcal{L}\bar{S}_\ell).
%\eeq
This is finally written as the general form of \cref{eq:SLCandCLS}, giving
\beq
(\widetilde{S}_\ell,\mathcal{L}\widetilde{C}_\ell) = (\widetilde{C}_\ell,\mathcal{L}\widetilde{S}_\ell) + \Det{\textbf{u}}.
\label{eq:GenSLCandCLS}
\eeq

This relation can let us obtain the
$(\widetilde{S}_\ell,\mathcal{L}\widetilde{C}_\ell)$ matrix element from
$(\widetilde{C}_\ell,\mathcal{L}\widetilde{S}_\ell)$, but it also gives us a 
nice numerical check. I calculate these two matrix elements separately in my 
code and calculate their difference for the Kohn. \label{GenSLCandCLS}If it is close to 1, this
gives us confidence that the long-long integrations are accurate.


\subsection{Matrix Elements Involving Long-Range Terms}
\label{sec:MatrixLong}
The short-long and long-long matrix elements have a similar analysis. For all 
of these, the effect of the $\mathcal{L} = 2(H-E)$ operator on the long-range 
terms must be considered, and then integrations over the external angles (see 
\cref{chp:AngularInt}) are performed. The remaining 6-dimensional integral is 
then numerically integrated as described in \cref{sec:LongLongInt,sec:ShortLongInt}.

\subsubsection{\texorpdfstring{$\mathcal{L}\bar{S}_\ell$}{LS} Terms}
\label{sec:LSTerms}
The matrix elements in \cref{eq:GeneralKohnMatrix} require us to first determine
$\mathcal{L}\bar{S}_\ell$. We start with examining $\mathcal{L}S_\ell$ first.
Using \cref{eq:GenSDef,eq:LDef}, 
\begin{align}
\label{eq:LS1}
\mathcal{L}S_\ell = &\left(-\frac{1}{2} \Laplacian_\rho - \Laplacian_{r_3} - 2 \Laplacian_{r_{12}} + \frac{2}{r_1} - \frac{2}{r_2} - \frac{2}{r_3} - \frac{2}{r_{12}} - \frac{2}{r_{13}} + \frac{2}{r_{23}} - 2 E_H - 2 E_{Ps} - \frac{1}{2} \kappa^2\right) \nonumber \\
& \times \SphericalHarmonicY{\ell}{0}{\theta_\rho}{\varphi_\rho} \Phi_{Ps}\left(r_{12}\right) \Phi_H\left(r_3\right) \sqrt{2\kappa} \,j_\ell\!\left(\kappa\rho\right).
\end{align}
Since $S_\ell$ is independent of $r_3$ and $r_{12}$ except for the $\Phi_H$ and $\Phi_{Ps}$ functions, respectively, using \cref{eq:HPsEqn,eq:HPsHamil} simplifies this to
\begin{align}
\label{eq:LS2}
\mathcal{L}S_\ell = \left(-\frac{1}{2} \Laplacian_\rho + \frac{2}{r_1} - \frac{2}{r_2} - \frac{2}{r_{13}} + \frac{2}{r_{23}} - \frac{1}{2} \kappa^2\right) \SphericalHarmonicY{\ell}{0}{\theta_\rho}{\varphi_\rho} \Phi_{Ps}\left(r_{12}\right) \Phi_H\left(r_3\right) \sqrt{2\kappa} \,j_\ell\!\left(\kappa\rho\right).
\end{align}
From \cref{sec:SphBess2}, we find that $\SphericalHarmonicY{\ell}{0}{\theta_\rho}{\varphi_\rho} j_\ell(\kappa\rho)$ is an eigenfunction of $\Laplacian_\rho$ with eigenvalue $-\kappa^2$:
\begin{equation}
\Laplacian_\rho \left[\SphericalHarmonicY{\ell}{0}{\theta_\rho}{\varphi_\rho} j_\ell(\kappa\rho) \right] = \frac{(-\kappa^2 \rho^2) \LegendreP{\ell, \cos\theta} } {\rho^2 \LegendreP{\ell, \cos\theta}}
= -\kappa^2 \, \SphericalHarmonicY{\ell}{0}{\theta_\rho}{\varphi_\rho} j_\ell(\kappa\rho).
\end{equation}
Then \cref{eq:LS2} reduces down to
\begin{equation}
\label{eq:LS3}
\mathcal{L}S_\ell = \left(\frac{2}{r_1} - \frac{2}{r_2} - \frac{2}{r_{13}} + \frac{2}{r_{23}}\right) \SphericalHarmonicY{\ell}{0}{\theta_\rho}{\varphi_\rho} \Phi_{Ps}\left(r_{12}\right) \Phi_H\left(r_3\right) \sqrt{2\kappa} \,j_\ell\!\left(\kappa\rho\right)
\end{equation}
or
\begin{equation}
\label{eq:LSFinal}
\mathcal{L}S_\ell = \left(\frac{2}{r_1} - \frac{2}{r_2} - \frac{2}{r_{13}} + \frac{2}{r_{23}}\right) S_\ell.
\end{equation}

$\mathcal{L}S_\ell^\prime$ is simply the same as \cref{eq:LSFinal} but with $2 \leftrightarrow 3$ due to the permutation operator, or
\begin{equation}
\label{eq:LSPrimeFinal}
\mathcal{L}S_\ell^\prime = \left(\frac{2}{r_1} - \frac{2}{r_3} - \frac{2}{r_{12}} + \frac{2}{r_{23}}\right) S_\ell^\prime.
\end{equation}


\subsubsection{\texorpdfstring{$\mathcal{L}\bar{C}_\ell$}{LC} Terms}
\label{sec:LCTerms}
To calculate the matrix elements in \cref{eq:GeneralKohnMatrix} that include $\bar{C}_\ell$ in the ket, we start by writing a general form of $\mathcal{L}C_\ell$ using \cref{eq:LDef,eq:GenCDef}:
\begin{align}
\label{eq:LC1}
\mathcal{L}C_\ell = -&\left(-\frac{1}{2} \Laplacian_\rho - \Laplacian_{r_3} - 2 \Laplacian_{r_{12}} + \frac{2}{r_1} - \frac{2}{r_2} - \frac{2}{r_3} - \frac{2}{r_{12}} - \frac{2}{r_{13}} + \frac{2}{r_{23}} - 2 E_H - 2 E_{Ps} - \frac{1}{2} \kappa^2\right) \nonumber \\
& \times \SphericalHarmonicY{\ell}{0}{\theta_\rho}{\varphi_\rho} \Phi_{Ps}\left(r_{12}\right) \Phi_H\left(r_3\right) \sqrt{2\kappa} \,n_\ell\!\left(\kappa\rho\right) f_\ell(\rho) 
\end{align}
Similar to \cref{eq:LS2}, using \cref{eq:HPsEqn,eq:HPsHamil} reduces this to
\begin{align}
\label{eq:LC2}
\mathcal{L}C_\ell = -&\left(-\frac{1}{2} \Laplacian_\rho + \frac{2}{r_1} - \frac{2}{r_2} - \frac{2}{r_{13}} + \frac{2}{r_{23}} - 2 E_H - 2 E_{Ps} - \frac{1}{2} \kappa^2\right) \nonumber \\
& \times \SphericalHarmonicY{\ell}{0}{\theta_\rho}{\varphi_\rho} \Phi_{Ps}\left(r_{12}\right) \Phi_H\left(r_3\right) \sqrt{2\kappa} \,n_\ell\!\left(\kappa\rho\right) f_\ell(\rho).
\end{align}
Unlike with $\mathcal{L}S_\ell$ in \cref{sec:LSTerms}, there is not a cancellation 
with the $\Laplacian_\rho$ and $\kappa^2$ terms. The combination of these 
terms was calculated in the ``First Partial Waves LC.nb'' \emph{Mathematica} 
notebook \cite{GitHub,Wiki} using the code given in \cref{fig:LCMath}. This is for the F-wave, 
and replacing the $\ell$-value of 3 in \texttt{SphericalBesselY} allows this 
to be used for any partial wave. The results of these derivations is given in 
each partial wave chapter through the D-wave. The
$\frac{1}{2} \left(\Laplacian_\rho + \kappa^2\right) \SphericalHarmonicY{\ell}{0}{\theta_\rho}{\varphi_\rho} n_\ell(\kappa\rho) f_\ell(\rho)$
given in this table is substituted in \cref{eq:LC2} to find the full $\mathcal{L}C_\ell$.

\begin{figure}
	\centering
	\includegraphics[width=6.5in]{LC}
	\caption[\emph{Mathematica} code to calculate part of LC]{Listing of \emph{Mathematica} code in ``First Partial Waves LC.nb'' to calculate part of $\mathcal{L}C_\ell$ for the F-wave}
	\label{fig:LCMath}
\end{figure}

%{
%\renewcommand{\arraystretch}{3}  % To space out rows - could also look at the array environment.
%\begin{table}
%\centering
%\centerline{
%\begin{tabular}{c l}
%\toprule \\[-2.7cm]
%Partial & \\[-1.3cm]
%Wave & $\frac{1}{2} \left(\Laplacian_\rho + \kappa^2\right) \SphericalHarmonicY{\ell}{0}{\theta_\rho}{\varphi_\rho} n_\ell(\kappa\rho) f_\ell(\rho)$ \\
%\midrule
%S-Wave & {$\!\begin{aligned} % http://tex.stackexchange.com/q/98482/16595 
               %\frac{2 \kappa  f_\ell^\prime(\rho ) \sin (\kappa  \rho )-f_\ell^{\prime\prime}(\rho) \cos (\kappa \rho )}{2 \kappa  \rho } \\    % http://tex.stackexchange.com/q/78788/16595
           %\end{aligned}$} \\
%P-Wave & {$\!\begin{aligned}
				%-\frac{\rho  f_\ell^{\prime\prime}(\rho ) \left[\kappa  \rho  \sin (\kappa \rho)+\cos (\kappa \rho )\right]+2 f_\ell^\prime(\rho ) \left[\left(\kappa ^2 \rho ^2-1\right) \cos (\kappa  \rho )-\kappa  \rho  \sin (\kappa \rho )\right]}{2 \kappa ^2 \rho ^3}
			%\end{aligned}$} \\
%D-Wave & {$\!\begin{aligned}
		%-\frac{1}{2 \kappa ^3 \rho ^4} &\left\{\rho  f_\ell^{\prime\prime}(\rho ) \left[\left(3-\kappa ^2 \rho ^2\right) \cos (\kappa  \rho )+3 \kappa  \rho  \sin (\kappa  \rho )\right] \right. \\
		%& \left.+2 f_\ell^\prime(\rho ) \left[\kappa  \rho  \left(\kappa ^2 \rho ^2-6\right) \sin (\kappa  \rho )+3 \left(\kappa ^2 \rho ^2-2\right) \cos (\kappa  \rho )\right] \right\}
		%\end{aligned}$} \\[0.6cm]
%F-Wave & {$\!\begin{aligned}
	%\frac{1}{2 \kappa ^4 \rho ^5} &\left\{\rho  f_\ell^{\prime\prime}(\rho ) \left[\kappa  \rho  \left(\kappa ^2 \rho ^2-15\right) \sin (\kappa  \rho )+3 \left(2 \kappa ^2 \rho ^2-5\right) \cos (\kappa  \rho )\right] \right. \\
	%& \left. +2 f_\ell^\prime(\rho ) \left[3 \kappa  \rho  \left(15-2 \kappa ^2 \rho ^2\right) \sin (\kappa  \rho )+\left(\kappa ^4 \rho ^4-21 \kappa ^2 \rho ^2+45\right) \cos (\kappa  \rho )\right]\right\}
	%\end{aligned}$} \\[1cm] 
%G-Wave & {$\!\begin{aligned}
	%\frac{1}{2 \kappa ^5 \rho ^6} & \left\{2 f_\ell^\prime(\rho ) \left[\kappa  \rho  \left(\kappa ^4 \rho ^4-55 \kappa ^2 \rho ^2+420\right) \sin (\kappa  \rho ) \right. \right. \\
	%& \left. +5 \left(2 \kappa ^4 \rho ^4-39 \kappa ^2 \rho ^2+84\right) \cos (\kappa  \rho )\right]-\rho  f_\ell^{\prime\prime}(\rho ) \left[5 \kappa  \rho  \left(21-2 \kappa ^2 \rho ^2\right) \sin (\kappa  \rho ) \right. \\
	%& \left. \left. +\left(\kappa ^4 \rho ^4-45 \kappa ^2 \rho ^2+105\right) \cos (\kappa  \rho )\right] \right\}
	%\end{aligned}$} \\[1.5cm]
%H-Wave & {$\!\begin{aligned}
	%-\frac{1}{2 \kappa ^6 \rho ^7} & \left\{\rho  f_\ell^{\prime\prime}(\rho ) \left[\kappa  \rho  \left(\kappa ^4 \rho ^4-105 \kappa ^2 \rho ^2+945\right) \sin (\kappa  \rho ) \right. \right. \\
	%& \left. +15 \left(\kappa ^4 \rho ^4-28 \kappa ^2 \rho ^2+63\right) \cos (\kappa  \rho )\right] \\
	%& + 2 f_\ell^\prime(\rho ) \left[\left(\kappa ^6 \rho ^6-120 \kappa ^4 \rho ^4+2205 \kappa ^2 \rho ^2-4725\right) \cos (\kappa  \rho ) \right. \\
	%& \left. \left. -15 \kappa  \rho  \left(\kappa ^4 \rho ^4-42 \kappa ^2 \rho ^2+315\right) \sin (\kappa  \rho )\right]\right\}
	%\end{aligned}$} \\
%\bottomrule
%\end{tabular}
%}
%\caption{Part of $\mathcal{L}C_\ell$ derived for each partial wave}
%\label{tab:LCList}
%\end{table}
%}

Matrix elements involving $\mathcal{L}C_\ell^\prime$ look similar but have the 2 and 3 coordinates swapped. In other words, $\rho \leftrightarrow \rho^\prime$.

From \cref{eq:PartialWaveShielding}, the general shielding function for $\widetilde{C}_\ell$ to keep it regular at the origin is given by
\begin{equation}
  %\label{eq:PartialWaveShielding}
  f_\ell(\rho) = \left[1 - \ee^{-\mu \rho} \left(1+\frac{\mu}{2}\rho\right)
  \right]^{m_\ell}.
\end{equation}
From \cref{tab:LCList}, the derivatives $f_\ell^\prime(\rho)$ and $f_\ell^{\prime\prime}(\rho)$ are needed. In the \emph{Mathematica} notebook ``Shielding Factor.nb'' \cite{GitHub,Wiki}, I found out that the derivatives can be written generally as
\beq
\label{eq:Shielding1Der}
f_\ell^\prime(\rho) = -\frac{\mu m_\ell (\mu  \rho +1) \left[1-\frac{1}{2} e^{-\mu  \rho } (\mu  \rho +2)\right]^{m_\ell}}{\mu  \rho -2 e^{\mu  \rho }+2}
\eeq
and
\beq
\label{eq:Shielding2Der}
f_\ell^{\prime\prime}(\rho) = \frac{\mu^2 m_\ell \left[-2 \mu  \rho  e^{\mu  \rho }+ m_\ell (\mu  \rho +1)^2-1\right] \left[1-\frac{1}{2} e^{-\mu  \rho } (\mu  \rho +2)\right]^{m_\ell}} {\left(\mu  \rho -2 e^{\mu  \rho }+2\right)^2}.
\eeq


\subsubsection{\texorpdfstring{$(\bar{S}_\ell,\mathcal{L}\bar{S}_\ell)$}{SLS} and \texorpdfstring{$(\bar{C}_\ell,\mathcal{L}\bar{S}_\ell)$}{CLS} Matrix Elements}
\label{sec:SLSandCLS}

From \cref{eq:TildeSCDef}, we see that, in general, any matrix element in 
\cref{eq:GeneralKohnMatrix} containing only long-range terms will contain 
both $(\bar{S}_\ell,\mathcal{L}\bar{S}_\ell)$ and
$(\bar{C}_\ell,\mathcal{L}\bar{S}_\ell)$, along with the other two combinations
given in \cref{sec:LCTerms}.
When $(\bar{S}_\ell,\mathcal{L}\bar{S}_\ell)$ is expanded,

\beq
(\bar{S}_\ell,\mathcal{L}\bar{S}_\ell) = \frac{1}{2} \left[ (S_\ell,\mathcal{L}S_\ell) \pm (S_\ell^\prime,\mathcal{L}S_\ell) \pm (S_\ell,\mathcal{L}S_\ell^\prime) \pm (S_\ell^\prime,\mathcal{L}S_\ell^\prime) \right].
\eeq
The properties of the permutation operator give that
\beq
(S_\ell,\mathcal{L}S_\ell) = (S_\ell^\prime,\mathcal{L}S_\ell^\prime) \text{ and } (S_\ell^\prime,\mathcal{L}S_\ell) = (S_\ell,\mathcal{L}S_\ell^\prime),
\eeq
so this becomes
\beq
(\bar{S}_\ell,\mathcal{L}\bar{S}_\ell) = (S_\ell,\mathcal{L}S_\ell) \pm (S_\ell^\prime,\mathcal{L}S_\ell).
\eeq

From \cref{eq:LSFinal,eq:LSPrimeFinal}, the Laplacian on $S_\ell$ and
$S_\ell^\prime$ leaves only some of the potential terms. The potential terms in
\cref{eq:LSFinal} are antisymmetric upon the $1 \leftrightarrow 2$ swap, and the 
terms in \cref{eq:LSPrimeFinal} are antisymmetric upon the $1 \leftrightarrow 3$
swap. $S_\ell$ and $S_\ell^\prime$ are symmetric with the $1 \leftrightarrow 2$
and $1 \leftrightarrow 3$ swaps, respectively. When these 
are integrated over these coordinates, the combination of symmetric with 
antisymmetric functions causes the integral to be 0:
\beq
\label{eq:SLS0Test}
(S_\ell,\mathcal{L}S_\ell) = (S_\ell^\prime,\mathcal{L}S_\ell^\prime) = 0.
\eeq
Therefore, using \cref{eq:LSFinal,eq:LSPrimeFinal},
\begin{subequations}
\label{eq:SbarLSbar}
\begin{align}
(\bar{S}_\ell,\mathcal{L}\bar{S}_\ell) &= \pm (S_\ell^\prime,\mathcal{L}S_\ell) = \pm \left(S_\ell^\prime, \left[ \frac{2}{r_1} - \frac{2}{r_2} - \frac{2}{r_{13}} + \frac{2}{r_{23}} \right] S_\ell\right)  \label{eq:SbarLSbar1} \\
& = \pm (S_\ell,\mathcal{L}S_\ell^\prime) = \pm \left(S_\ell, \left[ \frac{2}{r_1} - \frac{2}{r_3} - \frac{2}{r_{12}} + \frac{2}{r_{23}} \right] S_\ell^\prime \right) . \label{eq:SbarLSbar2}
\end{align}
\end{subequations}
Either form can be used to calculate $(\bar{S}_\ell,\mathcal{L}\bar{S}_\ell)$ in the long-range code.


Since $C_\ell$ and $C_\ell^\prime$ are also symmetric with their respective swaps, the $(\bar{C}_\ell,\mathcal{L}\bar{S}_\ell)$ matrix element is a similar form given by
\begin{subequations}
\label{eq:CbarLSbar}
\begin{align}
(\bar{C}_\ell,\mathcal{L}\bar{S}_\ell) &= \pm (C_\ell^\prime,\mathcal{L}S_\ell) = \pm \left(C_\ell^\prime, \left[ \frac{2}{r_1} - \frac{2}{r_2} - \frac{2}{r_{13}} + \frac{2}{r_{23}} \right] S_\ell\right)  \label{eq:CbarLSbar1} \\
& = \pm (C_\ell,\mathcal{L}S_\ell^\prime) = \pm \left(C_\ell, \left[ \frac{2}{r_1} - \frac{2}{r_3} - \frac{2}{r_{12}} + \frac{2}{r_{23}} \right] S_\ell^\prime \right) . \label{eq:CbarLSbar2}
\end{align}
\end{subequations}


\subsubsection{\texorpdfstring{$(\bar{\phi}_i, \mathcal{L}\bar{S}_\ell)$}{phiLS} and \texorpdfstring{$(\bar{\phi}_i, \mathcal{L}\bar{C}_\ell)$}{phiLC} Matrix Elements}
\label{sec:phiLSC}

The $(\bar{\phi}_i,\mathcal{L}\widetilde{S}_\ell)$ and $(\bar{\phi}_i,\mathcal{L}\widetilde{C}_\ell)$ in \cref{eq:GeneralKohnMatrix} have combinations of $(\bar{\phi}_i, \mathcal{L}\bar{S})$ and $(\bar{\phi}_i, \mathcal{L}\bar{C})$, as seen in \cref{eq:TildeSCDef}.

Let us investigate $(\bar{\phi}_i, \mathcal{L}\bar{S}_\ell)$ first.
\beq
\label{eq:PhiBarLSBar1}
(\bar{\phi}_i, \mathcal{L}\bar{S}_\ell) = \left( (\phi_i \pm \phi_i'), \mathcal{L} \frac{(S_\ell \pm S_\ell^\prime)}{\sqrt{2}}\right)
= \frac{1}{\sqrt{2}} \left[ (\phi_i, \mathcal{L} S_\ell) \pm (\phi_i, \mathcal{L} S_\ell^\prime) \pm (\phi_i^\prime, \mathcal{L} S_\ell) + (\phi_i^\prime, \mathcal{L} S_\ell^\prime) \right]
\eeq
Again, from the properties of the $P_{23}$ permutation operator,
\beq
(\phi_i,\mathcal{L}S_\ell) = (\phi_i^\prime,\mathcal{L}S_\ell^\prime) \text{ and } (\phi_i,\mathcal{L}S_\ell^\prime) = (\phi_i^\prime,\mathcal{L}S_\ell).
\eeq

\noindent \Cref{eq:PhiBarLSBar1} becomes
\begin{subequations}
\label{eq:PhiBarLSBar2}
\begin{align}
(\bar{\phi}_i, \mathcal{L}\bar{S}_\ell) = \frac{1}{\sqrt{2}} \left[2(\phi_i,\mathcal{L}S_\ell) \pm 2(\phi_i^\prime,\mathcal{L}S_\ell)\right] &= \frac{2}{\sqrt{2}} \left[(\phi_i,\mathcal{L}S_\ell) \pm (\phi_i',\mathcal{L}S_\ell)\right] \label{eq:PhiBarLSBar2a} \\
 &= \frac{2}{\sqrt{2}} \left[(\phi_i,\mathcal{L}S_\ell) \pm (\phi_i,\mathcal{L}S_\ell^\prime)\right]  \label{eq:PhiBarLSBar2b}
\end{align}
\end{subequations}

\noindent Notice that \cref{eq:PhiBarLSBar2a,eq:PhiBarLSBar2b}) are equivalent ways of writing this expression. Either could be used, depending on the form desired for the computation.
From \cref{eq:LSFinal,eq:LSPrimeFinal},
\begin{align}
(\bar{\phi}_i, \mathcal{L}\bar{S}_\ell) &= \frac{2}{\sqrt{2}} \left[\left( \phi_i \left( \frac{2}{r_1} - \frac{2}{r_2} - \frac{2}{r_{13}} + \frac{2}{r_{23}} \right) S_\ell\right) \pm \left( \phi_i^\prime \left( \frac{2}{r_1} - \frac{2}{r_2} - \frac{2}{r_{13}} + \frac{2}{r_{23}} \right) S_\ell\right)\right] \\
 &= \frac{2}{\sqrt{2}} \left[\left( \phi_i \left( \frac{2}{r_1} - \frac{2}{r_2} - \frac{2}{r_{13}} + \frac{2}{r_{23}} \right) S_\ell\right) \pm \left( \phi_i \left( \frac{2}{r_1} - \frac{2}{r_3} - \frac{2}{r_{12}} + \frac{2}{r_{23}} \right) S_\ell^\prime \right)\right]
\end{align}




\subsection{Matrix Elements Involving Only Short-Range Terms}
\label{sec:MatrixShort}
%The short-range -- short-range interactions are calculated in a separate document ``3.25 Using Spherical Coordinates.pdf''.

Using \cref{eq:BoundHFull} with \cref{eq:LDef} and realizing that the bra is not conjugated in the Kohn-type variational methods, the short-short integrals are of the form
\beq
\label{eq:SWaveShortShort}
\left(\bar{\phi}_i, \mathcal{L} \bar{\phi}_j\right) = \int \left[ \sum_{l=1}^3 \boldsymbol{\nabla}_{\!\mathbf{r}_l} \bar{\phi}_i \boldsymbol{\cdot} \boldsymbol{\nabla}_{\!\mathbf{r}_l} \bar{\phi}_j + \left( \frac{2}{r_1} - \frac{2}{r_2} - \frac{2}{r_3} - \frac{2}{r_{12}} - \frac{2}{r_{13}} + \frac{2}{r_{23}} - 2 E \right) \bar{\phi}_i \bar{\phi}_j \right] d\tau.
\eeq
Again, the $\bar{\phi}$ represents any of the short-range terms given in
\cref{eq:PhiDef} and could also represent any other Hylleraas-type terms, such as 
the mixed terms (see \cref{sec:MixedTerms}) or the second formalism of the
P-wave (see \cref{sec:PWave2Formalism}). These matrix elements are numerically 
integrated using the methods described in \cref{sec:ShortInt}.



%{
%\renewcommand{\arraystretch}{2}
%\begin{table}
%\centering
%\begin{tabular}{l l l l}
%\toprule\\[-1.5cm]
%Partial Wave & $\SphericalHarmonicY{\ell}{0}{\theta_\rho}{\varphi_\rho}$ & $j_\ell(x)$ & $n_\ell(x)$ \\
%\midrule
%S-Wave & $\frac{1}{\sqrt{4\pi}}$ & & \\
%P-Wave & $\sqrt{\frac{3}{4\pi}} \cos \theta$ & & \\
%D-Wave & $\sqrt{\frac{5}{16\pi}} \left(3 \cos^2\theta - 1 \right)$ &  &  \\
%F-Wave & $\sqrt{\frac{7}{16\pi}} \left(5 \cos^3\theta - 3 \cos\theta \right)$ &  &  \\
%G-Wave & $\sqrt{\frac{9}{256\pi}} \left(35 \cos^4\theta - 30 \cos^2\theta + 3 \right)$ &  &  \\
%H-Wave & $\sqrt{\frac{11}{256\pi}} \left(63 \cos^5\theta - 70 \cos^3\theta + 15 \cos\theta \right)$ &  &  \\
%\bottomrule
%\end{tabular}
%\caption{Spherical harmonics, spherical Bessel, and spherical Neumann functions for partial waves through $\ell = 5$}
%\label{tab:SphHarmBessel}
%\end{table}
%}


\section{Schwartz Singularities}
\label{eq:SchwartzSing}

A disadvantage of the Kohn-type variational methods is the presence of spurious 
singularities in the phase shifts. Looking at \cref{eq:GeneralKohnMatrix}, if 
$\textbf{\emph{A}}$ becomes near-singular
(i.e. $\Det{\textbf{\emph{A}}} \approx 0$), solving this matrix equation
will yield incorrect phase shifts. 
These ``Schwartz singularities'' were described first by Schwartz
\cite{Schwartz1961} and analyzed by others \cite{Nesbet1968,Nesbet1969}.
These singularities do not make the Kohn-type variational methods 
unusable however. These singularities are often easily noticeable, because 
they do not follow the pattern of other phase shifts.

\begin{figure}
	\centering
	\includegraphics[width=\textwidth]{schwartz-singularity}
	\caption[Example of Schwartz singularities for $^1$S]{Example of Schwartz singularities for $^1$S at $\omega = 7$.
The generalized Kohn method with $\tau = 1.4$ is shown in (a). The $S$-matrix complex Kohn is shown in (b).}
	\label{fig:schwartz-singularity}
\end{figure}

As an example, refer to \cref{fig:schwartz-singularity}(a). This shows an 
example of Schwartz singularities for the generalized Kohn method with
$\tau = 1.4$. A clear Schwartz singularity exists at $\kappa = 0.851$ or $E = \SI{4.927}{eV}$,
which can be seen as a point on the graph that is far away from 
the red fitting curve. There is also a Schwartz singularity at $\kappa = 0.85$
or $E = \SI{4.915}{eV}$, seen as a slight deviation from the fitting curve.

The same calculation as in \cref{fig:schwartz-singularity}(a) is performed 
using the $S$-matrix complex Kohn in \cref{fig:schwartz-singularity}(b), but 
no Schwartz singularities are evident. Normally, if one Kohn variational 
method described in the previous section (\cref{sec:KohnApplied}) has a 
Schwartz singularity, other Kohn methods will not. This gives us a strategy 
of simply rejecting Kohn methods that have obvious Schwartz singularities.



Additionally, the complex Kohn methods in
\cref{eq:uCompTKohn,eq:uCompSKohn,eq:uGenTKohn,eq:uGenSKohn} are far less likely to 
suffer from these singularities. A complex-valued Kohn variational method was 
first proposed by Miller et al. \cite{Miller1987} and used the same year by 
McCurdy et al.~\cite{McCurdy1987}. Subsequent work by
Lucchese~\cite{Lucchese1989} showed that the complex Kohn methods can indeed have
Schwartz singularities, but they are not likely to show up in practice.
Cooper et al.~\cite{Cooper2010} also showed that the phase shifts obtained using the 
complex Kohn variational method can be obtained exactly from the real-valued 
generalized Kohn method in \cite{eq:uGenKohn}, but we do not use this method.

There is also much less variability in the results of the different complex 
Kohn-type variational methods, and even for different values of $\tau$ in
\cref{eq:uGenTKohn,eq:uGenSKohn}, the phase shifts generally agree to great 
precision (greater than the accuracy that we quote for the phase shifts). Due 
to the stability and agreement of the complex Kohn variational methods, the results quoted 
throughout this paper are for the $S$-matrix complex Kohn unless noted 
otherwise.


%\todoi{I think one of the Humberston group theses has some links/discussion on Schwartz singularities. Definitely mention Cooper, Lucchese, and McCurdy.}


\section{Resonances}
\label{sec:Resonances}
Resonances are where the phase shifts rapidly change by $\pi$. We find that 
there are resonances in each singlet partial wave for Ps-H scattering. Some 
papers refer to these resonances as Breit-Wigner resonances
\cite{Tennyson1984,Stibbe1998}, but according to Bransden and Joachain
\cite[p.596]{Bransden2003}, Breit-Wigner resonances are a special case of Fano
resonances with $q = \pm\infty$.

The resonance positions and widths can be calculated to high 
accuracy by fitting the phase shift data to the following curve \cite{VanReeth2004}:
\beq
\label{eq:ResonanceCurve}
\delta(E) = A + BE + CE^2 + \arctan\left[ \frac{^1\Gamma}{2\left(^1E_R-E\right)} \right]
  + \arctan\left[ \frac{^2\Gamma}{2\left(^2E_R-E\right)} \right].
\eeq
The polynomial part of the above equation corresponds to hard sphere 
scattering. The arctangent parts correspond to the first and second Fano 
resonances \cite{Fano1961,Macek1970,Hazi1979}, with $^1E_R$ and $^2E_R$ as 
the positions of the resonances and $^1\Gamma$ and $^2\Gamma$ as the widths 
of the respective resonances. The $^1$S and $^1$P partial waves use this 
fitting, and the $^1$D and $^1$F partial waves omit the second term, since we 
consider only a single resonance. See
\cref{sec:SWaveResonances,sec:PWaveResonances,sec:DWaveResonance,sec:FWaveResonance}
for further discussion of the resonances for each partial wave.

%From Refs.~\cite{Ray2006,Ray2007} and Ref.~\cite[p.599]{Bransden2003}, resonances
%are metastable states with lifetimes of
%\beq
%\tau \approx \hbar / \Gamma.
%\eeq
%Ray also points out that this is longer than a typical collision time.

\begin{figure}
	\centering
	\includegraphics[width=5.25in]{WaltersEventLine}
	\caption[Event line for singlet Ps-H scattering]{Event line for singlet Ps-H scattering
from Ref.~\cite{Walters2004}. Reprinted with permission from Elsevier Limited.}
	\label{fig:WaltersEventLine}
\end{figure}

As Blackwood et al.\ \cite{Blackwood2002} mention, there are an infinite 
number of Rydberg resonances in each partial wave converging on the e$^+$+H$^-$
threshold at \SI{6.05}{eV}. Walters et al.\ \cite{Walters2004} gives this as a
a more accurate value of \SI{6.0477}{eV}, as shown in \cref{fig:WaltersEventLine}
from their paper. They also note that the Ps(n=3)
threshold is at \SI{6.0470}{eV}, making these two thresholds nearly the same
energy. This work only considers the single channel problem up to the Ps(n=2)
threshold at \SI{5.102}{eV}.

For Ps-H scattering, Drachman predicted that these resonances correspond to 
the metastable state of e$^+$ with the H$^-$ ion \cite{Drachman1979}. A set 
of close coupling papers \cite{Blackwood2002,Blackwood2002b,Walters2004} 
confirms that the H$^-$ channel is important for the resonances, indicating 
that Drachman's prediction was correct. Biswas \cite{Biswas2001} also showed 
that H$^-$ formation is important for describing this system.

The first $^1$S resonance is associated with the $2s$ state \cite{DiRienzi2002b} 
and was first calculated by Hazi and Taylor using a stabilization method
\cite{Hazi1970}. The first $^1$P Rydberg resonance is associated with the $3p$ 
state, not the $2p$ state \cite{DiRienzi2002b}, while the $^1$D resonance 
corresponds with the $3d$ state \cite{DiRienzi2002a}. No such analysis exists
in the current literature for higher partial waves or resonances for the
S-, P- and D-waves other than the first resonance of each.

%\todoi{Should mention how we calculate the resonance parameters for each of the Kohn methods and report the $S$-matrix separately}

\biblio
\end{document}