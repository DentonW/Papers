% -*- root: Dissertation.tex -*-
\documentclass[Dissertation.tex]{subfiles} 
\begin{document}


\chapter{D-Wave}
\label{chp:DWave}

%\todoi{Variation of $\mu$ to get 0.7. Latest variation of other nonlinear parameters} 

\section{Wavefunction}
\label{sec:DWaveFn}

Similar to the discussions for the S-wave (\cref{chp:SWave}) and the P-wave
(\cref{chp:PWave}), here we present the D-wave wavefunction for the Kohn
variational method in \cref{eq:DWaveTrial}. The wavefunction for each of the 
variants of the Kohn can be easily constructed from this by using the general
formalism in \cref{sec:GeneralWave,sec:KohnApplied}.
\begin{equation}
\Psi_t^\pm = \widetilde{S}_2 + L_2^{\pm,t} \, \widetilde{C}_2 + \sum_{i=1}^{N(\omega)} c_i \bar{\phi}_{1i} + \sum_{j=1}^{N(\omega)} d_j \bar{\phi}_{2j} + \sum_{j=1}^{N(\omega)} f_k \bar{\phi}_{12k}
\label{eq:DWaveTrial}
\end{equation}

\noindent Including exchange, $\bar{S}_2$ and $\bar{C}_2$ are
\begin{subequations}
\label{eq:DWaveSandCBar}
\begin{align}
\bar{S}_2 &= \frac{Y_{20}(\theta_\rho,\varphi_\rho)S_{22} \pm Y_{20}(\theta_{\rho^\prime},\varphi_{\rho'})S_{23} }{\sqrt{2}} \label{eq:DWaveSBar} \\
\bar{C}_2 &= \frac{Y_{20}(\theta_\rho,\varphi_\rho)C_{22} \pm Y_{20}(\theta_{\rho^\prime},\varphi_{\rho'})C_{23} }{\sqrt{2}} \label{eq:DWaveCBar} 
\end{align}
\end{subequations}
with
\begin{subequations}
\label{eq:DWaveSandC}
\begin{alignat}{2}
S_{22} &={}&\Phi_{Ps}\left(r_{12}\right) \Phi_H\left(r_3\right) &\sqrt{2\kappa} \,j_2\!\left(\kappa\rho\right) \label{eq:DWaveS22Def} \\
S_{23} &={}&\Phi_{Ps}\left(r_{13}\right) \Phi_H\left(r_2\right) &\sqrt{2\kappa} \,j_2\!\left(\kappa\rho^\prime\right) \label{eq:DWaveS23Def} \\
C_{22} &={}-&\Phi_{Ps}\left(r_{12}\right) \Phi_H\left(r_3\right) &\sqrt{2\kappa} \,n_2\!\left(\kappa\rho\right) f_{2}(\rho) \label{eq:DWaveC22Def} \\
C_{23} &={}-&\Phi_{Ps}\left(r_{13}\right) \Phi_H\left(r_2\right) &\sqrt{2\kappa} \,n_2\!\left(\kappa\rho^\prime\right) f_{2}(\rho^\prime). \label{eq:DWaveC23Def}
\end{alignat}
\end{subequations}

\noindent The shielding factor, $f_{2}(\rho)$, is given by
\beq
f_{2}(\rho) = \left[1 - \ee^{-\mu \rho} \left(1+\frac{\mu}{2}\rho\right)\right]^5.
\label{eq:DWaveShielding}
\eeq
The first and second derivatives of this shielding factor are needed in the
derivations of the matrix elements, and it is worthwhile to work them out separately.
\begin{subequations}
\label{eq:DWaveShieldingDer}
\begin{align}
f_{2}^\prime(\rho) &= \frac{5}{32} \ee^{-5 \mu \rho} \mu (1 + \mu \rho) (2 - 2 \ee^{\mu \rho} + \mu \rho)^4 \\
f_{2}^{\prime\prime}(\rho) &= -\frac{5}{32} \ee^{-5 \mu \rho} \mu^2 (2 - 2 \ee^{\mu \rho} + \mu \rho)^3 \left[4 + 2\mu\rho (5 - \ee^{\mu \rho}) + 5 \mu^2 \rho^2\right]
\end{align}
\end{subequations}

\noindent The $j_2(\kappa\rho)$ and $n_2(\kappa\rho)$ are the spherical Bessel
and Neumann functions given by \cite[p. 729]{Arfken2005}
\begin{subequations}
\label{eq:DWaveBessel}
\begin{align}
j_2(x) & = \left(\frac{3}{x^3}-\frac{1}{x}\right)\sin x - \frac{3}{x^2}\cos x \label{eq:Bessel2} \\
n_2(x) & = -\left(\frac{3}{x^3}-\frac{1}{x}\right)\cos x - \frac{3}{x^2}\sin x. \label{eq:Neumann2}
\end{align}
\end{subequations}

\noindent The $Y_{20}(\theta)$ are an abbreviated form of the spherical harmonics,
since there is azimuthal symmetry. For the D-wave, this is
\beq
\label{eq:DWaveSpherHarm}
Y_{20}(\theta) = Y_{20}(\theta,\varphi) = \sqrt{\frac{5}{16\pi}} (3\cos^2\theta - 1).
\eeq

As noted in \cref{sec:GeneralWave}, the full wavefunction has $(\ell+1) = 3$
short-range symmetries. The short-range terms are given by
\begin{subequations}
\label{eq:DWavePhiBar}
\begin{align}
\bar{\phi}_{1i} &= \left(1 \pm P_{23}\right) Y_{20}(\theta_1) r_1^2 \phi_i \label{eq:DWavePhi1i}\\
\bar{\phi}_{2j} &= \left(1 \pm P_{23}\right) Y_{20}(\theta_2) r_2^2 \phi_j \label{eq:DWavePhi2j}\\
\bar{\phi}_{12k} &= \left(1 \pm P_{23}\right) \psi_{(1,1,2,0)}(\theta_1,\theta_2) r_1 r_2 \phi_k, \label{eq:DWavePhi12k}
\end{align}
\end{subequations}
where $\phi_i$, $\phi_j$ and $\phi_k$ are given by \cref{eq:PhiDef}. The
$\psi_{(1,1,2,0)}(\theta_1,\theta_2)$ is described in \cref{sec:MixedDerivation}.
We also use the shortcuts
\begin{subequations}
\label{eq:DWavePhi}
\begin{align}
\phi_{1i} &= r_1^2 \phi_i \\
\phi_{2j} &= r_2^2 \phi_j \\
\phi_{12k} &= r_1 r_2 \phi_k.
\end{align}
\end{subequations}

To calculate $\mathcal{L} C_2$, we again use the code in \cref{fig:LCMath} to get
\begin{align}
\label{eq:LCMathD}
\nonumber \frac{1}{2} \left(\Laplacian_\rho + \kappa^2\right) \SphericalHarmonicY{2}{0}{\theta_\rho}{\varphi_\rho} & n_2(\kappa\rho) f_2(\rho) = \\
\nonumber -\frac{1}{2 \kappa ^3 \rho ^4} &\left\{\rho  f^{\prime\prime}(\rho ) \left[\left(3-\kappa ^2 \rho ^2\right) \cos (\kappa  \rho )+3 \kappa  \rho  \sin (\kappa  \rho )\right] \right. \\
& \left.+2 f^\prime(\rho ) \left[\kappa  \rho  \left(\kappa ^2 \rho ^2-6\right) \sin (\kappa  \rho )+3 \left(\kappa ^2 \rho ^2-2\right) \cos (\kappa  \rho )\right] \right\}.
\end{align}



\section{Short-Range -- Short-Range Integrals}
\label{sec:DWaveShortShort}

The D-wave short-short integrals are generally more complicated than those 
for the S-wave and P-wave.
\Cref{eq:DWavePhi1Phi1,eq:DWavePhi2Phi2,eq:DWavePhi1Phi2,eq:DWavePhi2Phi1}
give the short-short integrals needed evaluate the 
matrix in \cref{eq:GeneralKohnMatrix}. Full derivations for each of these are 
given in separate notes available on the Wiki \cite{Wiki}.

\begin{align}
\label{eq:DWavePhi1Phi1}
\left(\bar{\phi}_{1i},\mathcal{L} \bar{\phi}_{1j}\right) = &2 \cdot 2\pi \int{ \Bigg\{ \sum_{k=1}^3 \left[ \boldsymbol{\nabla}_{\!\mathbf{r}_k} \nonumber \phi_{1i} \boldsymbol{\cdot} \boldsymbol{\nabla}_{\!\mathbf{r}_k} \phi_{1j} \pm \boldsymbol{\nabla}_{\!\mathbf{r}_k} \phi_{1i} \boldsymbol{\cdot} \boldsymbol{\nabla}_{\!\mathbf{r}_k} \phi_{1j}^\prime \right] } \\
\nonumber  &+ \left. \left[\frac{2}{r_1} - \frac{2}{r_2} - \frac{2}{r_3} - \frac{2}{r_{12}} - \frac{2}{r_{13}} + \frac{2}{r_{23}} - 2 E_H - 2 E_{Ps} - \frac{1}{2}\kappa^2 + \frac{6}{r_1^2} \right] \right. \\
 &\;\;\;\;\; \times \left(\phi_{1i} \phi_{1j} \pm \phi_{1i} \phi_{1j}^\prime \right) \Bigg\} d\tau_{int}
\end{align}

\begin{align}
\label{eq:DWavePhi2Phi2}
\left(\bar{\phi}_{2i},\mathcal{L} \bar{\phi}_{2j}\right) = 2 & \cdot 2\pi \int \Bigg\{ \sum_{k=1}^3 \left[ \boldsymbol{\nabla}_{\!\mathbf{r}_k} \nonumber \phi_{2i} \boldsymbol{\cdot} \boldsymbol{\nabla}_{\!\mathbf{r}_k} \phi_{2j} \pm \left(1-\tfrac{3}{2}\sin^2\theta_{23}\right) \boldsymbol{\nabla}_{\!\mathbf{r}_k} \phi_{2i} \boldsymbol{\cdot} \boldsymbol{\nabla}_{\!\mathbf{r}_k} \phi_{2j}^\prime \right]  + \frac{6}{r_2^2}\phi_{2i}\phi_{2j} \\
 \nonumber &\mp 3 \phi_{2i} \phi_{2j}^\prime \cos\theta_{23} \left[p_i \frac{r_1}{r_3 r_{13}^2} (\cos\theta_{12} - \cos\theta_{23} \cos\theta_{13}) + m_j^\prime \frac{r_1}{r_2 r_{12}^2}(\cos\theta_{13} - \cos\theta_{23} \cos\theta_{12})\right.\\
 \nonumber & \left. \;\;\;\;\;  + \sin^2\theta_{23} \left(q_i \frac{r_2}{r_3 r_{23}^2} + q_j^\prime \frac{r_3}{r_2 r_{23}^2} \right) \right] \\
 \nonumber &+ \left. \left[\frac{2}{r_1} - \frac{2}{r_2} - \frac{2}{r_3} - \frac{2}{r_{12}} - \frac{2}{r_{13}} + \frac{2}{r_{23}} - 2 E_H - 2 E_{Ps} - \frac{1}{2}\kappa^2 \right] \right. \\
 &\;\;\;\;\; \times \left[\phi_{2i} \phi_{2j} \pm \left(1-\tfrac{3}{2}\sin^2\theta_{23}\right) \phi_{2i} \phi_{2j}^\prime \right] \Bigg\} d\tau_{int}
\end{align}

\begin{align}
\label{eq:DWavePhi1Phi2}
\left(\bar{\phi}_{1i},\mathcal{L} \bar{\phi}_{2j}\right) = 2 & \cdot 2\pi \int \Bigg\{ \sum_{k=1}^3 \left[ \left(1-\tfrac{3}{2}\sin^2\theta_{12}\right) \boldsymbol{\nabla}_{\!\mathbf{r}_k} \nonumber \phi_{1i} \boldsymbol{\cdot} \boldsymbol{\nabla}_{\!\mathbf{r}_k} \phi_{2j} \pm \left(1-\tfrac{3}{2}\sin^2\theta_{13}\right) \boldsymbol{\nabla}_{\!\mathbf{r}_k} \phi_{1i} \boldsymbol{\cdot} \boldsymbol{\nabla}_{\!\mathbf{r}_k} \phi_{2j}^\prime \right] \\
 \nonumber &\mp 3 \phi_{1i} \phi_{2j} \cos\theta_{12} \left[q_i \frac{r_3}{r_2 r_{23}^2} (\cos\theta_{13} - \cos\theta_{12} \cos\theta_{23}) + p_j \frac{r_3}{r_1 r_{13}^2}(\cos\theta_{23} - \cos\theta_{12} \cos\theta_{13})\right.\\
 \nonumber & \left. \;\;\;\;\;  + \sin^2\theta_{12} \left(m_i \frac{r_1}{r_2 r_{12}^2} + m_j \frac{r_2}{r_1 r_{12}^2} \right) \right] \\
 \nonumber &\mp 3 \phi_{1i} \phi_{2j}^\prime \cos\theta_{13} \left[q_i \frac{r_2}{r_3 r_{23}^2} (\cos\theta_{12} - \cos\theta_{13} \cos\theta_{23}) + m_j^\prime \frac{r_2}{r_1 r_{12}^2}(\cos\theta_{23} - \cos\theta_{12} \cos\theta_{13})\right.\\
 \nonumber & \left. \;\;\;\;\;  + \sin^2\theta_{13} \left(p_i \frac{r_1}{r_3 r_{13}^2} + p_j^\prime \frac{r_3}{r_1 r_{13}^2} \right) \right] \\
 \nonumber &+ \left. \left[\frac{2}{r_1} - \frac{2}{r_2} - \frac{2}{r_3} - \frac{2}{r_{12}} - \frac{2}{r_{13}} + \frac{2}{r_{23}} - 2 E_H - 2 E_{Ps} - \frac{1}{2}\kappa^2 \right] \right. \\
 &\;\;\;\;\; \times \left[\left(1-\tfrac{3}{2}\sin^2\theta_{12}\right) \phi_{1i} \phi_{2j} \pm \left(1-\tfrac{3}{2}\sin^2\theta_{13}\right) \phi_{1i} \phi_{2j}^\prime \right] \Bigg\} d\tau_{int}
\end{align}

\begin{align}
\label{eq:DWavePhi2Phi1}
\left(\bar{\phi}_{2i},\mathcal{L} \bar{\phi}_{1j}\right) = 2 & \cdot 2\pi \int \Bigg\{ \sum_{k=1}^3 \left(1-\tfrac{3}{2}\sin^2\theta_{12}\right) \left[ \boldsymbol{\nabla}_{\!\mathbf{r}_k} \nonumber \phi_{2i} \boldsymbol{\cdot} \boldsymbol{\nabla}_{\!\mathbf{r}_k} \phi_{1j} \pm \boldsymbol{\nabla}_{\!\mathbf{r}_k} \phi_{2i} \boldsymbol{\cdot} \boldsymbol{\nabla}_{\!\mathbf{r}_k} \phi_{1j}^\prime \right] \\
 \nonumber &\mp 3 \phi_{2i} \phi_{1j} \cos\theta_{12} \left[p_i \frac{r_3}{r_1 {r_{13}}^2} (\cos\theta_{23} - \cos\theta_{12} \cos\theta_{13}) + q_j \frac{r_3}{r_2 {r_{23}}^2}(\cos\theta_{13} - \cos\theta_{12} \cos\theta_{23})\right.\\
 \nonumber & \left. \;\;\;\;\;  + \sin^2\theta_{12} \left(m_i \frac{r_2}{r_1 {r_{12}}^2} + m_j \frac{r_1}{r_2 {r_{12}}^2} \right) \right] \\
 \nonumber &\mp 3 \phi_{2i} \phi_{1j}^\prime \cos\theta_{12} \left[p_i \frac{r_3}{r_1 r_{13}^2} (\cos\theta_{23} - \cos\theta_{12} \cos\theta_{13}) + q_j^\prime \frac{r_3}{r_2 {r_{23}}^2}(\cos\theta_{13} - \cos\theta_{12} \cos\theta_{23})\right.\\
 \nonumber & \left. \;\;\;\;\;  + \sin^2\theta_{12} \left(m_i \frac{r_2}{r_1 {r_{12}}^2} + m_j^\prime \frac{r_1}{r_2 {r_{12}}^2} \right) \right] \\
 \nonumber &+ \left. \left[\frac{2}{r_1} - \frac{2}{r_2} - \frac{2}{r_3} - \frac{2}{r_{12}} - \frac{2}{r_{13}} + \frac{2}{r_{23}} - 2 E_H - 2 E_{Ps} - \frac{1}{2}\kappa^2 \right] \right. \\
 &\;\;\;\;\; \times \left(1-\tfrac{3}{2}\sin^2\theta_{12}\right) \left( \phi_{2i} \phi_{12j} \pm \phi_{2i} \phi_{1j}^\prime \right) \Bigg\} d\tau_{int}
\end{align}




\section{Short-Range -- Long-Range Integrals}
\label{sec:DWaveShortLong}

To simplify these equations, define
\begin{align}
\nonumber \mathscr{L} S_\ell = &\frac{\mathcal{L} S_\ell}{Y_{20}(\theta_\rho)} = \left(\frac{2}{r_1} - \frac{2}{r_2} - \frac{2}{r_{13}} + \frac{2}{r_{23}} \right) \Phi_{Ps}(r_{12}) \Phi_H(r_3) \sqrt{2\kappa} \, j_2(\kappa\rho) \\
\nonumber \mathscr{L} C_\ell = &\frac{\mathcal{L} C_\ell}{Y_{20}(\theta_\rho)} = - \left(\frac{2}{r_1} - \frac{2}{r_2} - \frac{2}{r_{13}} + \frac{2}{r_{23}} \right) \Phi_{Ps}(r_{12}) \Phi_H(r_3) \sqrt{2\kappa} \, n_2(\kappa\rho) f_{2}(\rho) \\
& - \Phi_{Ps}(r_{12}) \Phi_H(r_3) \sqrt{2\kappa} \frac{1}{2\rho} \left\{ \left[4 n_2(\kappa\rho) - 2 \kappa\rho \, n_1(\kappa\rho) \right] f_{2}^\prime(\rho) - \rho \, n_2(\kappa\rho) f_{2}^{\prime\prime}(\rho) \right\}.
\end{align}
Equivalently,
\begin{subequations}
\begin{align}
\mathscr{L} S_\ell^\prime &= \frac{\mathcal{L} S^\prime}{Y_{20}(\theta_{\rho^\prime})} \\
\mathscr{L} C_\ell^\prime &= \frac{\mathcal{L} C^\prime}{Y_{20}(\theta_{\rho^\prime})}.
\end{align}
\end{subequations}

Each of the following equations has two forms that can be used. These 
equations are mainly straightforward applications of the external angular
integrations in \cref{sec:AngularInt} to the matrix elements of
\cref{eq:GeneralKohnMatrix}.

%\begin{align}
%\label{eq:DWavePhi1SBar}
%\nonumber \left(\bar{\phi}_{1i},L \bar{S}\right) = \sqrt{2} \cdot 2\pi & \int \left[ \left(1 - \frac{3 r_2^2 \sin^2\theta_{12}}{8 \rho^2} \right) \left(\phi_{1i} \pm \phi_{1i}^\prime \right) \left(\frac{2}{r_1} - \frac{2}{r_2} - \frac{2}{r_{13}} + \frac{2}{r_{23}} \right) S_{22} \right] d\tau_{int} \\
%\nonumber = \sqrt{2} \cdot 2\pi & \int \phi_{1i} \left[ \left(1 - \frac{3 r_2^2 \sin^2\theta_{12}}{8 \rho^2} \right) \left( \frac{2}{r_1} - \frac{2}{r_2} - \frac{2}{r_{13}} + \frac{2}{r_{23}} \right) S_{22} \right. \\
%& \pm \left. \left(1 - \frac{3 r_3^2 \sin^2\theta_{13}}{8 {\rho^\prime}^2} \right) \left( \frac{2}{r_1} - \frac{2}{r_3} - \frac{2}{r_{12}} + \frac{2}{r_{23}} \right) S_{23} \right] d\tau_{int}
%\end{align}

\begin{align}
\label{eq:DWavePhi1SBar}
\nonumber \left(\bar{\phi}_{1i},\mathcal{L} \bar{S}_\ell\right) = & \sqrt{2} \cdot 2\pi \int \left(1 - \frac{3 r_2^2 \sin^2\theta_{12}}{8 \rho^2} \right) \left(\phi_{1i} \pm \phi_{1i}^\prime \right) \mathscr{L}S_\ell \, d\tau_{int} \\
=& \sqrt{2} \cdot 2\pi \int \phi_{1i} \left[ \left(1 - \frac{3 r_2^2 \sin^2\theta_{12}}{8 \rho^2} \right) \mathscr{L}S_\ell \pm \left(1 - \frac{3 r_3^2 \sin^2\theta_{13}}{8 {\rho^\prime}^2} \right) \mathscr{L}S_\ell^\prime \right] d\tau_{int}
\end{align}

\begin{align}
\label{eq:DWavePhi1CBar}
\nonumber \left(\bar{\phi}_{1i},\mathcal{L} \bar{C}_\ell\right) = & \sqrt{2} \cdot 2\pi \int \left(1 - \frac{3 r_2^2 \sin^2\theta_{12}}{8 \rho^2} \right) \left(\phi_{1i} \pm \phi_{1i}^\prime \right) \mathscr{L}C_\ell \, d\tau_{int} \\
=& \sqrt{2} \cdot 2\pi \int \phi_{1i} \left[ \left(1 - \frac{3 r_2^2 \sin^2\theta_{12}}{8 \rho^2} \right) \mathscr{L}C_\ell \pm \left(1 - \frac{3 r_3^2 \sin^2\theta_{13}}{8 {\rho^\prime}^2} \right) \mathscr{L}C_\ell^\prime \right] d\tau_{int}
\end{align}

%\begin{align}
%\label{eq:DWavePhi2SBar}
%\nonumber \left(\bar{\phi}_{2j},L \bar{S}\right) = \sqrt{2} \cdot 2\pi & \int \left\{ \left[ \left(1 - \frac{3 r_1^2 \sin^2\theta_{12}}{8 \rho^2} \right) \phi_{2j} \pm \left( \frac{3(r_1 \cos\theta_{13} + r_2 \cos\theta_{23})^2}{8 \rho^2} - \frac{1}{2} \right) \phi_{2j}^\prime \right] \left(\frac{2}{r_1} - \frac{2}{r_2} - \frac{2}{r_{13}} + \frac{2}{r_{23}} \right) S_{22} \right\} d\tau_{int} \\
%\nonumber = \sqrt{2} \cdot 2\pi & \int \phi_{2j} \left[ \left(1 - \frac{3 r_1^2 \sin^2\theta_{12}}{8 \rho^2} \right) \left( \frac{2}{r_1} - \frac{2}{r_2} - \frac{2}{r_{13}} + \frac{2}{r_{23}} \right) S_{22} \right. \\
%& \pm \left. \left( \frac{3(r_1 \cos\theta_{12} + r_3 \cos\theta_{23})^2}{8 {\rho^\prime}^2} - \frac{1}{2} \right) \left( \frac{2}{r_1} - \frac{2}{r_3} - \frac{2}{r_{12}} + \frac{2}{r_{23}} \right) S_{23} \right]  d\tau_{int}
%\end{align}

\begin{align}
\label{eq:DWavePhi2SBar}
\nonumber \left(\bar{\phi}_{2j},\mathcal{L} \bar{S}_\ell\right) = & \sqrt{2} \cdot 2\pi \int \left[ \left( \frac{3(r_1 \cos\theta_{12} + r_2)^2}{8 \rho^2} - \frac{1}{2} \right) \phi_{2j} \pm \left( \frac{3(r_1 \cos\theta_{13} + r_2 \cos\theta_{23})^2}{8 \rho^2} - \frac{1}{2} \right) \phi_{2j}^\prime \right] \mathscr{L}S_\ell \, d\tau_{int} \\
=& \sqrt{2} \cdot 2\pi \int \phi_{2j} \left[ \left( \frac{3(r_1 \cos\theta_{12} + r_2)^2}{8 \rho^2} - \frac{1}{2} \right) \mathscr{L}S_\ell \pm \left( \frac{3(r_1 \cos\theta_{12} + r_3 \cos\theta_{23})^2}{8 {\rho^\prime}^2} - \frac{1}{2} \right) \mathscr{L}S_\ell^\prime \right] d\tau_{int}
\end{align}

\begin{align}
\label{eq:DWavePhi2CBar}
\nonumber \left(\bar{\phi}_{2j},\mathcal{L} \bar{C}_\ell\right) = & \sqrt{2} \cdot 2\pi \int \left[ \left( \frac{3(r_1 \cos\theta_{12} + r_2)^2}{8 \rho^2} - \frac{1}{2} \right) \phi_{2j} \pm \left( \frac{3(r_1 \cos\theta_{13} + r_2 \cos\theta_{23})^2}{8 \rho^2} - \frac{1}{2} \right) \phi_{2j}^\prime \right] \mathscr{L}C_\ell \, d\tau_{int} \\
=& \sqrt{2} \cdot 2\pi \int \phi_{2j} \left[ \left( \frac{3(r_1 \cos\theta_{12} + r_2)^2}{8 \rho^2} - \frac{1}{2} \right) \mathscr{L}C_\ell \pm \left( \frac{3(r_1 \cos\theta_{12} + r_3 \cos\theta_{23})^2}{8 {\rho^\prime}^2} - \frac{1}{2} \right) \mathscr{L}C_\ell^\prime \right] d\tau_{int}
\end{align}

\begin{align}
\label{eq:DWavePhi3SBar}
\nonumber \left(\bar{\phi}_{12k},\mathcal{L} \bar{S}_\ell\right) = \sqrt{2} \cdot & \frac{1}{4} \sqrt{\frac{3\pi}{5}} \int \mathscr{L}S_\ell \left\{ \left[ 8 \cos\theta_{12} + \frac{3 r_1 r_2 \sin^2\theta_{12}}{\rho^2} \right] \phi_{12k} \right. \\
\nonumber & \left. \pm \frac{1}{\sqrt{2}} \left[ 3 \frac{\cos\theta_{13}(r_1^2 + r_1 r_2 \cos\theta_{12}) + \cos\theta_{23}(r_1 r_2 + r_2^2 \cos\theta_{12})}{\rho^2} - 4 \cos\theta_{13} \right] \phi_{12k}^\prime  \right\} d\tau_{int} \\
\nonumber = \sqrt{2} \cdot & \frac{1}{4} \sqrt{\frac{3\pi}{5}} \int \phi_{12k} \left\{ \left[ 8 \cos\theta_{12} + \frac{3 r_1 r_2 \sin^2\theta_{12}}{\rho^2} \right] \mathscr{L}S_\ell \right. \\
& \pm \left. \frac{1}{\sqrt{2}} \left[ 3 \frac{\cos\theta_{12}(r_1^2 + r_1 r_3 \cos\theta_{13}) + \cos\theta_{23}(r_1 r_3 + r_3^2 \cos\theta_{13})}{{\rho^\prime}^2} - 4 \cos\theta_{12} \right] \mathscr{L}S_\ell^\prime \right\} d\tau_{int}
\end{align}

\begin{align}
\label{eq:DWavePhi3CBar}
\nonumber \left(\bar{\phi}_{12k},\mathcal{L} \bar{C}_\ell\right) = \sqrt{2} \cdot & \frac{1}{4} \sqrt{\frac{3\pi}{5}} \int \mathscr{L}C_\ell \left\{ \left[ 8 \cos\theta_{12} + \frac{3 r_1 r_2 \sin^2\theta_{12}}{\rho^2} \right] \phi_{12k} \right. \\
\nonumber & \left. \pm \frac{1}{\sqrt{2}} \left[ 3 \frac{\cos\theta_{13}(r_1^2 + r_1 r_2 \cos\theta_{12}) + \cos\theta_{23}(r_1 r_2 + r_2^2 \cos\theta_{12})}{\rho^2} - 4 \cos\theta_{13} \right] \phi_{12k}^\prime  \right\} d\tau_{int} \\
\nonumber = \sqrt{2} \cdot & \frac{1}{4} \sqrt{\frac{3\pi}{5}} \int \phi_{12k} \left\{ \left[ 8 \cos\theta_{12} + \frac{3 r_1 r_2 \sin^2\theta_{12}}{\rho^2} \right] \mathscr{L}C_\ell \right. \\
& \pm \left. \frac{1}{\sqrt{2}} \left[ 3 \frac{\cos\theta_{12}(r_1^2 + r_1 r_3 \cos\theta_{13}) + \cos\theta_{23}(r_1 r_3 + r_3^2 \cos\theta_{13})}{{\rho^\prime}^2} - 4 \cos\theta_{12} \right] \mathscr{L}C_\ell^\prime \right\} d\tau_{int}
\end{align}


\section{Long-Range -- Long-Range Integrals}
\label{sec:DWaveLongLong}

Similar to the short-short terms, the derivations for the long-long terms are 
available on the Wiki \cite{Wiki}.

\begin{align}
\label{eq:DWaveSBarSBar}
\left(\bar{S}_\ell,\mathcal{L}\bar{S}_\ell\right) = \pm 2\pi \int \left\{ S_{22} S_{23} \left(\frac{2}{r_1} - \frac{2}{r_2} - \frac{2}{r_{13}} + \frac{2}{r_{23}} \right) \left[ \frac{3}{8} \frac{(4\rho^2 + 4 {\rho^\prime}^2 - r_{23}^2)^2}{16 \rho^2 {\rho^\prime}^2} - \frac{1}{2} \right] \right\} d\tau_{int}
\end{align}

\begin{align}
\label{eq:DWaveCBarSBar}
\left(\bar{C}_\ell,\mathcal{L}\bar{S}_\ell\right) = \pm 2\pi \int \left\{ S_{22} C_{23} \left(\frac{2}{r_1} - \frac{2}{r_2} - \frac{2}{r_{13}} + \frac{2}{r_{23}} \right) \left[ \frac{3}{8} \frac{(4\rho^2 + 4 {\rho^\prime}^2 - r_{23}^2)^2}{16 \rho^2 {\rho^\prime}^2} - \frac{1}{2} \right] \right\} d\tau_{int}
\end{align}

\begin{align}
\label{eq:DWaveSBarCBar}
\nonumber \left(\bar{S}_\ell,\mathcal{L}\bar{C}_\ell\right) = 2\pi \int \Bigg\{ \pm & S_{23} C_{22} \left(\frac{2}{r_1} - \frac{2}{r_2} - \frac{2}{r_{13}} + \frac{2}{r_{23}} \right) \left[ \frac{3}{8} \frac{(4\rho^2 + 4 {\rho^\prime}^2 - r_{23}^2)^2}{16 \rho^2 {\rho^\prime}^2} - \frac{1}{2} \right] \\
\nonumber - & \left[ S_{22} \pm \left( \frac{3}{8} \frac{(4\rho^2 + 4 {\rho^\prime}^2 - r_{23}^2)^2}{16 \rho^2 {\rho^\prime}^2} - \frac{1}{2} \right) S_{23} \right] \sqrt{2\kappa} \, \Phi_{Ps}\left(r_{12}\right) \Phi_H\left(r_3\right) \\
& \times \frac{1}{2\rho} \left( \left[ 4 n_2(\kappa\rho) - 2 \kappa\rho \, n_1(\kappa\rho) \right] f_2^\prime(\rho) - \rho \, n_2(\kappa\rho) f_2^{\prime\prime}(\rho) \right) \Bigg\} d\tau_{int}
\end{align}

\begin{align}
\label{eq:DWaveCBarCBar}
\nonumber \left(\bar{C}_\ell,\mathcal{L}\bar{C}_\ell\right) = 2\pi \int \Bigg\{ \pm & C_{23} C_{22} \left(\frac{2}{r_1} - \frac{2}{r_2} - \frac{2}{r_{13}} + \frac{2}{r_{23}} \right) \left[ \frac{3}{8} \frac{(4\rho^2 + 4 {\rho^\prime}^2 - r_{23}^2)^2}{16 \rho^2 {\rho^\prime}^2} - \frac{1}{2} \right] \\
\nonumber - & \left[ C_{22} \pm \left( \frac{3}{8} \frac{(4\rho^2 + 4 {\rho^\prime}^2 - r_{23}^2)^2}{16 \rho^2 {\rho^\prime}^2} - \frac{1}{2} \right) C_{23} \right] \sqrt{2\kappa} \, \Phi_{Ps}\left(r_{12}\right) \Phi_H\left(r_3\right) \\
& \times \frac{1}{2\rho} \left( \left[ 4 n_2(\kappa\rho) - 2 \kappa\rho \, n_1(\kappa\rho) \right] f_2^\prime(\rho) - \rho \, n_2(\kappa\rho) f_2^{\prime\prime}(\rho) \right) \Bigg\} d\tau_{int}
\end{align}

\section{Mixed Terms}
\label{sec:MixedTerms}

According to Schwartz \cite{Schwartz1961a}, to have a complete description, 
each partial wave needs $\ell+1$ symmetries. As shown in
\cref{eq:SWaveTrial,eq:PWaveTrial}, we use the full sets of symmetries for the
S-wave and P-wave. There is a third symmetry for the D-wave, which we refer to
as the mixed symmetry terms, or mixed terms, given by
\begin{align}
\label{eq:MixedAng}
\psi(\ell_1,\ell_2,L,M) &= \psi_{(1,1,2,0)} = \sum_{m=-1}^{+1} Y_{1,m}(\theta_1,\varphi_1) Y_{1,m}(\theta_2,\varphi_2) \left< 1,m; 1,-m,0 | 2,0 \right> \nonumber \\
	&= Y_{1,-1}(\theta_1,\varphi_1) Y_{1,+1}(\theta_2,\varphi_2)
    \left< 1,-1,1,+1 | 2,0 \right> \nonumber \\
& \hspace{0.5cm}  + Y_{1,0}(\theta_1,\varphi_1) Y_{1,0}(\theta_2,\varphi_2)
    \left< 1,0,1,0 | 2,0 \right> \nonumber \\
& \hspace{0.5cm} + Y_{1,+1}(\theta_1,\varphi_1) Y_{1,-1}(\theta_2,\varphi_2)
   \left< 1,+1,1,-1 | 2,0 \right>,
\end{align}
where $\ell_1$ and $\ell_2$ are the angular momenta on the particles in Ps,
and $L$ and $M$ give the angular momentum of the Ps.
These three terms can be combined into a single set as
\begin{equation}
\label{eq:MixedAngSimple}
\psi_{(1,1,2,0)}(\theta_1,\theta_2) = \frac{3}{4\uppi} \frac{1}{\sqrt{6}} \left(3 \cos\theta_1 \cos\theta_2 - \cos\theta_{12} \right).
\end{equation}
This avoids the issue of dealing with complex terms in the $m = -1$ and $m = 1$ cases.
Refer to \cref{sec:MixedDerivation} for the derivation of \cref{eq:MixedAngSimple}
from \cref{eq:MixedAng}.

The short-long matrix elements that include the mixed terms are not
much more difficult to deal with than the first and second symmetry terms.
The evaluation of the short-short matrix elements involving the mixed terms is
much more difficult than those with just the first and second symmetry terms. 
I spent some time trying to derive the expressions for the short-short
matrix elements but ran into difficulty doing so. There has been some
preliminary progress on this front, as described in \cref{chp:Unfinished}.

For three-body problems, the mixed terms are substantially easier to deal
with. For e$^+$-H, Refs.~\cite{Brown1985a,BrownThesis,WattsThesis,Humberston1997,VanReeth1997}
used the mixed terms. Dunn et al. \cite{Dunn2000,DunnThesis} also treated
e$^+$-He as a three-body problem, using one-electron models of He and
including the mixed terms.

Prior work \cite{VanReeth1997,VanReethThesis} on e$^+$-He scattering (also a
four-body problem) neglected the mixed terms. The justification they used was
that for e$^+$-H scattering, adding the third symmetry changed the $K$-matrix
elements less than $1.5\%$. The e$^+$-H scattering problem also has mixed
terms, but these are much easier to deal with analytically, since it is a
three-body problem. However, this conclusion is now believed to be in error, as
described in Ref.~\cite{Woods2015}. A corrected code in a preliminary
investigation by Van Reeth and Humberston \cite{VanReeth2015} for e$^+$-H
scattering found that these mixed terms can change the $K$-matrix elements by
about $10\%$ near the Ps formation threshold. Since they believed that the
mixed terms did not contribute much, they were neglected for e$^+$-He scattering
and virtual Ps terms were added in to improve the convergence, getting the final
results to within 1 to 2\% of the corrected code from this preliminary
investigation. It is clear that these virtual Ps terms are no longer needed if
the mixed terms are included.

The same preliminary investigation \cite{VanReeth2015} also looked at e$^-$-H
scattering. This investigation found that the mixed terms change the $^1$D and
$^3$D phase shifts less than 1\%, but the $^1$D phase shifts are affected more
by the inclusion of the mixed terms. Both the $^1$D and $^3$D phase shifts are
affected less at low $\kappa$ and more at higher $\kappa$, though the
contributions are still small.
% \todoi{Peter's email on February 8, 2015 - check values}

Due to the difficulty we had evaluating the analytical portion (the external
angular integrals) for the mixed terms, we only included the first two
symmetries in the D-wave calculations in this work. The preliminary investigation
\cite{VanReeth2015} took place only recently, after we had attempted other
methods (see \cite{chp:Unfinished}) to accelerate the convergence
of the D-wave phase shifts. As we discovered in \cref{sec:DWaveResults}, this
turns out to be an acceptable approximation.


\section{Second Formalism}
\label{sec:DSecondForm}
For the second formalism, the angular momentum needs to be placed on the Ps and
H, so we would need
\begin{subequations}
\label{eq:DWave2ndPhiBar}
\begin{align}
\bar{\phi}_{\rho i} &= \left(1 \pm P_{23}\right) Y_{20}(\theta_\rho) \rho^2 \phi_i \label{eq:DWave2ndPhi1i}\\
\bar{\phi}_{3j} &= \left(1 \pm P_{23}\right) Y_{20}(\theta_3) r_3^2 \phi_j \label{eq:DWave2ndPhi2j}.
\end{align}
\end{subequations}
It will be shown that these are sufficient to describe this system without 
having a third set of terms as in \cref{eq:DWaveTrial}.

To use these in the short-range code, we need an expression similar to
\cref{eq:P2rhoY10}. We cannot use that exact expression, as we have $\rho^2$. We
follow the same procedure as that of obtaining \cref{eq:P2rhoY10}, but the
spherical harmonic is more complicated. Full details of the derivation are
found in the ``Second Formalism P-wave and D-wave.nb'' notebook.
Substituting \cref{eq:CosRho} into \cref{eq:DWaveSpherHarm}, multiplying by
$\rho^2$, and then using \cref{eq:RhoRDef}, we obtain
\begin{equation}
\label{eq:rhoident2}
\rho^2 Y_{20}(\theta_\rho) = \frac{1}{4} \left[r_1^2 Y_{20}(\theta_1) + r_2^2 Y_{20}(\theta_2) \right] + \frac{1}{2} \sqrt{\frac{5}{16 \pi}} 
   r_1 r_2 \left[3 \cos\theta_1 \cos\theta_2 - \cos\theta_{12} \right].
\end{equation}
The first set of brackets is obviously similar to \cref{eq:P2rhoY10} but with
$1 \to 2$ coming from the P-wave to the D-wave. More interesting though is the
second set of brackets, which is clearly the mixed terms in
\cref{eq:DWavePhi12k,eq:MixedAngSimple}.

As mentioned in \cref{sec:MixedTerms}, we do not use the mixed terms in the
D-wave calculation due to their complexity. Because of this, the second
formalism has not been implemented for the D-wave and could be a source of
future work.




\section{Results}
\label{sec:DWaveResults}

This section gives the results of the $^{1,3}$D-wave calculations using the 
nonlinear parameters determined in \cref{sec:DWaveNonlinear}.

\subsection{Phase Shifts}
\label{sec:DWavePhase}


\begin{table}
\centering
\setlength{\tabcolsep}{-2pt}
\footnotesize
\begin{tabular}{@{\hskip 0.1cm}l . . . . . . .}
\toprule
Method & \multicolumn{1}{c}{\phantom{1}0.1} & \multicolumn{1}{c}{\phantom{1}0.2} & \multicolumn{1}{c}{\phantom{1}0.3} & \multicolumn{1}{c}{\phantom{1}0.4} & \multicolumn{1}{c}{\phantom{1}0.5} & \multicolumn{1}{c}{\phantom{1}0.6} & \multicolumn{1}{c}{\phantom{1}0.7} \\
\midrule
This work $(\omega = 6)$ $\delta_2^+$ 				& 1.36^{-4}  & 2.99^{-3}  & 1.60^{-2}  & 4.98^{-2}  & 1.13^{-1}  & 2.06^{-1}  & 3.28^{-1} \\
This work $(\omega \to \infty)$ $\delta_2^+$ 		& 4.24^{-4}  & 3.18^{-3}  & 1.62^{-2}  & 5.05^{-2}  & 1.14^{-1}  & 2.09^{-1}  & 3.33^{-1} \\
\% Diff$^+$											& 103.0\%    & 6.27\%     & 1.54\%     & 1.33\%     & 1.52\%     & 1.67\%     & 1.67\% \\
\% Diff$^+$	CC										& 39.1\%     & 15.4\%     & 7.81\%     & 4.71\%     & 2.62\%     & 0.97\%     & 1.23\% \\
\arrayrulecolor[RGB]{220,220,220}\midrule\arrayrulecolor{black}
CC 14Ps14H+H$^-$ \cite{Walters2004} $\delta_2^+$	& 2.02^{-4}  & 3.49^{-3}  & 1.73^{-2}  & 5.22^{-2}  & 1.16^{-1}  & 2.08^{-1}  & 3.24^{-1} \\
CC 14Ps14H \cite{Blackwood2002} $\delta_2^+$		& 1.46^{-4}  & 3.15^{-3}  & 1.65^{-2}  & 4.95^{-2}  & 1.08^{-1}  & 1.94^{-1}  & 3.02^{-1} \\
3-state CC \cite{Sinha1997} $\delta_2^-$			& 3.22^{-5}  & 9.29^{-4}  & 5.96^{-3}  & 2.01^{-2}  & 4.63^{-2}  & 8.29^{-2}  & 1.23^{-1} \\
SE \cite{Ray1997} $\delta_2^+$ 						& 3.18^{-5}  & 9.17^{-4}  & 5.87^{-3}  & 1.97^{-2}  & 4.54^{-2}  & 8.09^{-2}  & 1.19^{-1} \\
5-state CC \cite{Adhikari1999} $\delta_2^+$			& 1.8^{-5}   & 5.3^{-4}   & 3.5^{-3}   & 1.2^{-2}   & 2.9^{-2}   & 5.5^{-2}   & 8.8^{-2} \\
SE \cite{Hara1975} $\delta_2^+$						& 0.0        & 0.0009     & 0.0058     & 0.0195     & 0.0453     & 0.0810     & 0.1194 \\
\midrule
This work $(\omega = 6)$ $\delta_2^-$ 				& 5.81^{-5}  & 7.12^{-4}  & 1.07^{-3}  & -2.00^{-3} & -1.12^{-2} & -2.65^{-2} & -4.45^{-2} \\
This work $(\omega \to \infty)$ $\delta_2^-$ 		& 3.13^{-4}  & 8.67^{-4}  & 1.41^{-3}  & -1.20^{-3} & -9.34^{-2} & -2.32^{-2} & -4.02^{-2} \\
\% Diff$^-$											& 137.4\%    & 19.6\%     & 24.4\%     & 39.8\%     & 13.5\%     & 9.10\%     & 6.28\% \\
\arrayrulecolor[RGB]{220,220,220}\midrule\arrayrulecolor{black}
CC 14Ps14H \cite{Blackwood2002} $\delta_2^-$		& 8.48^{-5}  & 1.15^{-3}  & 2.84^{-3}  & 2.37^{-3}  & -4.66^{-3} & -1.85^{-2} & -3.27^{-2} \\
3-state CC \cite{Sinha1997} $\delta_2^-$			& -2.74^{-5} & -7.77^{-4} & -4.83^{-3} & -1.55^{-2} & -3.41^{-2} & -5.83^{-2} & -8.25^{-2} \\
SE \cite{Ray1997} $\delta_2^+$ 						& -3.00^{-5} & -8.56^{-4} & -5.37^{-3} & -1.76^{-2} & -3.95^{-2} & -7.03^{-2} & -1.06^{-1} \\
5-state CC \cite{Adhikari1999} $\delta_2^+$			& -1.4^{-5}  & -4.0^{-4}  & -2.6^{-3}  & -8.6^{-3}  & -2.0^{-2}  & -3.6^{-2}  & -5.5^{-2} \\
SE \cite{Hara1975} $\delta_2^+$						& 0.0        & -8.0^{-4}  & -5.3^{-3}  & -1.74^{-2} & -3.95^{-2} & -7.04^{-2} & -1.062^{-1} \\
\bottomrule
\end{tabular}
\caption[$^{1,3}$D results and comparisons]{$^{1,3}$D results and 
comparisons. Values in the header are $\kappa$ in au. \% Diff$^\pm$ is the 
percent difference between the current complex Kohn $\omega = 6$ and 
extrapolated values. \% Diff $^+$ is the percent difference between the 
complex Kohn $\omega = 6$ and CC 14Ps14H+H$^-$ \cite{Walters2004} phase
shifts. Exponents denote
powers of 10.}
\label{tab:DWaveComparisons}
\end{table}

\Cref{tab:DWaveComparisons,fig:DWaveComparisons} show our final results of 
the phase shifts using the $S$-matrix complex Kohn with the nonlinear 
parameters given in \cref{sec:DWaveNonlinear} and compares to multiple other 
calculations by other groups. One of the most easily noticeable problems is
our poor convergence for $\kappa = 0.1$ for both $^1$D and $^3$D. As mentioned
in \cref{sec:Extrapolations} on page~\pageref{sec:Extrapolations}, our code
does not seem to handle very small phase shifts particularly well, and we
likely would have to increase the number of integration points further. Also, the 
lack of mixed terms likely affects the convergence. Both of these effects
likely combine to give a bad extrapolation of over 100\%. However, note that
our phase shifts for $\kappa = 0.1$ are not extremely far from the CC of
Refs.~\cite{Walters2004,Blackwood2002}. We do not show this extrapolation for
$^1$D in Ref.~\cite{Woods2015}, as it is obviously not good.

For $\kappa \geq 0.2$, the extrapolations look more reasonable, and for $^1$D,
our phase shifts look relatively well converged, especially for $\kappa \geq 0.3$.
We believe that with the exception of $\kappa < 0.2$, our $^1$D phase shifts are
reasonably accurate, even with the omission of the mixed terms. The $^1$D phase
shifts for $\omega = 5$ and $\omega = 6$ differ by less than 10\%.

For $^3$D, the extrapolations are worse when the phase shift curve crosses the
x-axis, which happens around $\kappa = 0.35$, as can be seen in
\cref{fig:DWaveComparisons}(b). In Ref.~\cite{Blackwood2002}, they note \label{DWaveSwitch} that
this change in sign of the phase shifts indicates a switch from being repulsive
at low $\kappa$ and attractive at high $\kappa$. In the higher
$\kappa$ region ($\kappa \geq 0.5$),
the $^3$D phase shift convergence is much better, getting down to $\sim\!6\%$.
Due to the worse extrapolations for $^3$D than for the $^{1,3}$S, $^{1,3}$P,
and $^1$D partial waves, we do not report any $^3$D extrapolations in
Ref.~\cite{Woods2015}.

We see from \Cref{tab:DWaveComparisons,fig:DWaveComparisons} that the CC results
\cite{Blackwood2002,Walters2004} are generally above ours. The exception is the
$\kappa = 0.7$ phase shift for $^1$D and the extrapolated phase shifts for
$\kappa = 0.6$ and $0.7$, where the complex Kohn results are slightly higher.
For $^1$D, the complex Kohn and CC \cite{Walters2004} results overall agree
very well, as seen in the fourth line of \cref{tab:DWaveComparisons}, with
percent differences $\sim\!1\%$ at higher $\kappa$.

There is much more discrepancy between the $^3$D complex Kohn and CC phase
shifts, as can be seen easily in \cref{fig:DWaveComparisons} and the inset of
\cref{fig:DWavePhase}. This discrepancy
plus the poorer $^3$D extrapolations show that the $^3$D phase shifts are not
fully converged. It should be noted however that the CC phase shifts could be
overestimates, considering that the CVM \cite{Zhang2012} $^3$S phase shifts
agree extremely well with the complex Kohn, but the CC $^3$S phase shifts are
slightly higher than both. The CC also differs significantly from the complex
Kohn at low $\kappa$ for the $^3$P-wave, where we do not have a neglected
symmetry, and the phase shifts are well converged.

Unfortunately, it is not possible to know at the moment how far the complex 
Kohn phase shifts differ from the true phase shifts, and further 
investigation into the mixed terms could help resolve this discrepancy
(see \cref{chp:Unfinished}). Thankfully, the contribution to the cross sections
in \cref{chp:CrossSections} from the $^3D$ is small, even the differential
cross section. The $^1$D contribution to the cross sections is very small
in the low $\kappa$ region where the phase shifts are less converged. In the
resonance region, where the $^1$D resonance contributes significantly, the
phase shifts are well converged.


\begin{figure}
	\centering
	\includegraphics[width=\textwidth]{dwave-phases}
	\caption[$^{1,3}$D phase shifts]{$^{1,3}$D complex Kohn phase shifts. The $^1S$ CC phase shifts
\cite{Walters2004} are given by \mbox{\textcolor{blue}{$\times$}}, and the
$^3S$ CC phase shifts \cite{Blackwood2002} are given by
\mbox{\textcolor{red}{\textbf{+}}}. Vertical dashed lines denote the complex rotation resonance
positions \cite{Ho1998}.}
	\label{fig:DWavePhase}
\end{figure}

\begin{figure}
	\centering
	\includegraphics[width=5.25in]{dwave-comparisons}
	\caption[Comparison of D-wave phase shifts]{Comparison of $^1$D (a) and $^3$D (b) phase shifts with results from other groups. Results are ordered according to year of publication. This work -- solid curves; \mbox{\textcolor{blue}{$\times$} -- CC \cite{Walters2004};} \mbox{$\CIRCLE$ -- Kohn \cite{VanReeth2003};} \mbox{\textcolor{red}{\textbf{+}} -- CC \cite{Blackwood2002};} \mbox{\textcolor{red}{$\vartriangle$} -- 6-state CC \cite{Sinha2000};} \mbox{$\blacksquare$ -- 5-state CC \cite{Adhikari1999};} \mbox{$\vartriangle$ -- 3-state CC \cite{Sinha1997};} \mbox{\textcolor[RGB]{0,127,0}{$\bigstar$} -- CC \cite{Ray1997};} \mbox{\textcolor{blue}{$\lozenge$} -- Static-exchange \cite{Hara1975}.}}
	\label{fig:DWaveComparisons}
\end{figure}


\subsection{Resonance Parameters}
\label{sec:DWaveResonance}

The $^1$D-wave has a single resonance before the Ps(n=2) threshold, unlike 
the $^1$S- and $^1$P-waves, which have two. We use \cref{eq:ResonanceCurve}
to fit this curve but without the second $\arctan$ term.
As mentioned in the previous section, the $^1$D phase shifts are
well converged in the higher $\kappa$ region, where the resonance
is located.

From the discussion in \cref{sec:DWaveNonlinear}, the complex Kohn resonance 
parameters presented in \cref{tab:DWaveResonancesOther} are likely more 
sensitive to the nonlinear parameters than the two lower partial waves. Our 
resonance parameters agree very well with the complex rotation \cite{Ho1998}.
The 9Ps9H CC results \cite{Blackwood2002} are brought closer to the complex
Kohn and complex rotation results by the inclusion of the H$^-$ channel for
both the position and width. The stabilization \cite{Yan2003} results are
also similar to the complex rotation results, both of which are carried out by
Yan and Ho.

This $^1$D-wave resonance has a very important contribution to the cross
sections in \cref{chp:CrossSections}. Due to their $(2\ell+1)$ and $(\ell+1)$
dependence, the $^1$D resonance has a larger contribution to these cross
sections than the $^1$S- and $^1$P-waves.

\todoi{Mention Drachman as in the computation chapter?}


\setlength{\abovecaptionskip}{6pt}   % 0.5cm as an example
\setlength{\belowcaptionskip}{6pt}   % 0.5cm as an example
\begin{table}
\centering
\begin{tabular}{l l l}
\toprule
Method & $^1E_R \text{ (eV)}$ & $^1\Gamma \text{ (eV)}$ \\
\midrule
Current work: Average $\pm$ standard deviation & $4.720 \pm 0.001$ & $0.0908 \pm 0.0010$ \\
Current work: $S$-matrix complex Kohn & $4.720$ & $0.0909$ \\
CC (9Ps9H + H$^-$) \cite{Walters2004} & $4.899$ & $0.0872$ \\
Stabilization \cite{Yan2003} & $4.714$ & $0.0969$ \\
CC (9Ps9H) \cite{Blackwood2002} & $5.16$ & $0.15$ \\
CC (22Ps1H + H$^-$) \cite{Blackwood2002b} & $4.814$ & $0.065$ \\
Optical potential \cite{DiRienzi2002a} & $4.729$ & $0.327$ \\
Complex rotation \cite{Ho1998} & $4.710 \pm 0.0027$ & $0.0925 \pm 0.0054$  \\
Coupled-pseudostate \cite{Campbell1998} & $5.28$ & $0.47$ \\
\bottomrule
\end{tabular}
\caption{$^1$D-wave resonance parameters}
\label{tab:DWaveResonancesOther}
\end{table}


\section{Summary}
\label{sec:SummaryD}




\biblio
\end{document}