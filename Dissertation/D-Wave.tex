% -*- root: Dissertation.tex -*-
\documentclass[Dissertation.tex]{subfiles} 
\begin{document}


\chapter{D-Wave}
\label{chp:DWave}

\todoi{Variation of $\mu$ to get 0.7. Latest variation of other nonlinear parameters} 

\section{D-Wave Wavefunction}
\label{sec:DWaveFn}

\begin{equation}
\Psi_t^\pm = \bar{S}_2 + L_2^{\pm,t} \, \bar{C}_2 + \sum_{i=1}^N c_i \bar{\phi}_{1i} + \sum_{j=1}^N d_j \bar{\phi}_{2j} + \sum_{j=1}^N f_k \bar{\phi}_{12k}
\label{eq:DWaveTrial}
\end{equation}


\beq
\label{eq:DWaveSpherHarm}
\SphericalHarmonicY{2}{0}{\theta_\rho}{\varphi_\rho} = \sqrt{\frac{5}{16\pi}} \left(3 \cos^2\theta_\rho - 1 \right)
\eeq

\beq
j_2(\kappa \rho) = \left(\frac{3}{(\kappa\rho)^2} - 1 \right) \frac{\sin(\kappa\rho)}{\kappa\rho} - \frac{3 \cos(\kappa\rho)}{(\kappa\rho)^2}
\eeq

\beq
n_2(\kappa \rho) = \left(-\frac{3}{(\kappa\rho)^2} + 1 \right) \frac{\cos(\kappa\rho)}{\kappa\rho} - \frac{3 \sin(\kappa\rho)}{(\kappa\rho)^2}
\eeq



\noindent Including exchange, $\bar{S}$ and $\bar{C}$ are
\begin{subequations}
\label{eq:DWaveSandCBar}
\begin{align}
\bar{S} &= \frac{Y_{20}(\theta_\rho,\varphi_\rho)S_{22} \pm Y_{20}(\theta_{\rho^\prime},\varphi_{\rho'})S_{23} }{\sqrt{2}} \label{eq:DWaveSBar} \\
\bar{C} &= \frac{Y_{20}(\theta_\rho,\varphi_\rho)C_{22} \pm Y_{20}(\theta_{\rho^\prime},\varphi_{\rho'})C_{23} }{\sqrt{2}} \label{eq:DWaveCBar} 
\end{align}
\end{subequations}
with
\begin{subequations}
\label{eq:DWaveSandC}
\begin{alignat}{2}
S_{22} &={}&\Phi_{Ps}\left(r_{12}\right) \Phi_H\left(r_3\right) &\sqrt{2\kappa} \,j_2\!\left(\kappa\rho\right) \label{eq:DWaveS22Def} \\
S_{23} &={}&\Phi_{Ps}\left(r_{13}\right) \Phi_H\left(r_2\right) &\sqrt{2\kappa} \,j_2\!\left(\kappa\rho^\prime\right) \label{eq:DWaveS23Def} \\
C_{22} &={}-&\Phi_{Ps}\left(r_{12}\right) \Phi_H\left(r_3\right) &\sqrt{2\kappa} \,n_2\!\left(\kappa\rho\right) f_{sh}(\rho) \label{eq:DWaveC22Def} \\
C_{23} &={}-&\Phi_{Ps}\left(r_{13}\right) \Phi_H\left(r_2\right) &\sqrt{2\kappa} \,n_2\!\left(\kappa\rho^\prime\right) f_{sh}(\rho^\prime). \label{eq:DWaveC23Def}
\end{alignat}
\end{subequations}

\noindent The shielding factor, $f_{sh}(\rho)$, is given by
\beq
f_{sh}(\rho) = \left[1 - \ee^{-\mu \rho} \left(1+\frac{\mu}{2}\rho\right)\right]^5.
\label{eq:DWaveShielding}
\eeq
The first and second derivatives of this shielding factor are needed in the derivations of the matrix elements, and it is worthwhile to work them out separately.
\begin{subequations}
\label{eq:DWaveShieldingDer}
\begin{align}
f_{sh}^\prime(\rho) &= \frac{5}{32} \ee^{-5 \mu \rho} \mu (1 + \mu \rho) (2 - 2 \ee^{\mu \rho} + \mu \rho)^4 \\
f_{sh}^{\prime\prime}(\rho) &= -\frac{5}{32} \ee^{-5 \mu \rho} \mu^2 (2 - 2 \ee^{\mu \rho} + \mu \rho)^3 \left[4 + 2\mu\rho (5 - \ee^{\mu \rho}) + 5 \mu^2 \rho^2\right]
\end{align}
\end{subequations}

\noindent The short-range terms are given by
\begin{subequations}
\label{eq:DWavePhiBar}
\begin{align}
\bar{\phi}_{1i} &= \left(1 \pm P_{23}\right) Y_{20}(\theta_1) r_1^2 \phi_i \label{eq:DWavePhi1i}\\
\bar{\phi}_{2j} &= \left(1 \pm P_{23}\right) Y_{20}(\theta_2) r_2^2 \phi_j \label{eq:DWavePhi2j}\\
\bar{\phi}_{12k} &= \left(1 \pm P_{23}\right) \psi_{(1,1,2,0)}(\theta_1,\theta_2) r_1 r_2 \phi_k, \label{eq:DWavePhi12k}
\end{align}
\end{subequations}

\noindent where $\phi_i$, $\phi_j$ and $\phi_k$ are given by (\ref{eq:PhiDef}).  We also use the shortcuts
\begin{subequations}
\label{eq:DWavePhi}
\begin{align}
\phi_{1i} &= r_1^2 \phi_i \\
\phi_{2j} &= r_2^2 \phi_j \\
\phi_{2j} &= r_1 r_2 \phi_k.
\end{align}
\end{subequations}


\noindent The $j_2(\kappa\rho)$ and $n_2(\kappa\rho)$ are the spherical Bessel and Neumann functions given by \cite[p. 729]{Arfken2005}
\begin{subequations}
\label{eq:DWaveBessel}
\begin{align}
j_2(x) & = \left(\frac{3}{x^3}-\frac{1}{x}\right)\sin x - \frac{3}{x^2}\cos x \label{eq:Bessel2} \\
n_2(x) & = -\left(\frac{3}{x^3}-\frac{1}{x}\right)\cos x - \frac{3}{x^2}\sin x. \label{eq:Neumann2}
\end{align}
\end{subequations}

\noindent The $Y_{20}(\theta)$ are an abbreviated form of the spherical harmonics, since there is azimuthal symmetry.  For the D-wave, these are
\beq
\label{eq:DWaveSpherHarm}
Y_{20}(\theta) = Y_{20}(\theta,\varphi) = \sqrt{\frac{5}{16\pi}} (3\cos^2\theta - 1).
\eeq



\section{D-Wave Short-Range -- Short-Range Integrals}
\label{sec:DWaveShortShort}

\begin{align}
\label{eq:DWavePhi1Phi1}
\left(\bar{\phi}_{1i},L \bar{\phi}_{1j}\right) = &2 \cdot 2\pi \int{ \Bigg\{ \sum_{k=1}^3 \left[ \boldsymbol{\nabla}_{\!\mathbf{r}_k} \nonumber \phi_{1i} \boldsymbol{\cdot} \boldsymbol{\nabla}_{\!\mathbf{r}_k} \phi_{1j} \pm \boldsymbol{\nabla}_{\!\mathbf{r}_k} \phi_{1i} \boldsymbol{\cdot} \boldsymbol{\nabla}_{\!\mathbf{r}_k} \phi_{1j}^\prime \right] } \\
\nonumber  &+ \left. \left[\frac{2}{r_1} - \frac{2}{r_2} - \frac{2}{r_3} - \frac{2}{r_{12}} - \frac{2}{r_{13}} + \frac{2}{r_{23}} - 2 E_H - 2 E_{Ps} - \frac{1}{2}\kappa^2 + \frac{6}{r_1^2} \right] \right. \\
 &\;\;\;\;\; \times \left(\phi_{1i} \phi_{1j} \pm \phi_{1i} \phi_{1j}^\prime \right) \Bigg\} d\tau_{int}
\end{align}

\begin{align}
\label{eq:DWavePhi2Phi2}
\left(\bar{\phi}_{2i},L \bar{\phi}_{2j}\right) = 2 & \cdot 2\pi \int \Bigg\{ \sum_{k=1}^3 \left[ \boldsymbol{\nabla}_{\!\mathbf{r}_k} \nonumber \phi_{2i} \boldsymbol{\cdot} \boldsymbol{\nabla}_{\!\mathbf{r}_k} \phi_{2j} \pm \left(1-\tfrac{3}{2}\sin^2\theta_{23}\right) \boldsymbol{\nabla}_{\!\mathbf{r}_k} \phi_{2i} \boldsymbol{\cdot} \boldsymbol{\nabla}_{\!\mathbf{r}_k} \phi_{2j}^\prime \right]  + \frac{6}{r_2^2}\phi_{2i}\phi_{2j} \\
 \nonumber &\mp 3 \phi_{2i} \phi_{2j}^\prime \cos\theta_{23} \left[p_i \frac{r_1}{r_3 r_{13}^2} (\cos\theta_{12} - \cos\theta_{23} \cos\theta_{13}) + m_j^\prime \frac{r_1}{r_2 r_{12}^2}(\cos\theta_{13} - \cos\theta_{23} \cos\theta_{12})\right.\\
 \nonumber & \left. \;\;\;\;\;  + \sin^2\theta_{23} \left(q_i \frac{r_2}{r_3 r_{23}^2} + q_j^\prime \frac{r_3}{r_2 r_{23}^2} \right) \right] \\
 \nonumber &+ \left. \left[\frac{2}{r_1} - \frac{2}{r_2} - \frac{2}{r_3} - \frac{2}{r_{12}} - \frac{2}{r_{13}} + \frac{2}{r_{23}} - 2 E_H - 2 E_{Ps} - \frac{1}{2}\kappa^2 \right] \right. \\
 &\;\;\;\;\; \times \left[\phi_{2i} \phi_{2j} \pm \left(1-\tfrac{3}{2}\sin^2\theta_{23}\right) \phi_{2i} \phi_{2j}^\prime \right] \Bigg\} d\tau_{int}
\end{align}

\begin{align}
\label{eq:DWavePhi1Phi2}
\left(\bar{\phi}_{1i},L \bar{\phi}_{2j}\right) = 2 & \cdot 2\pi \int \Bigg\{ \sum_{k=1}^3 \left[ \left(1-\tfrac{3}{2}\sin^2\theta_{12}\right) \boldsymbol{\nabla}_{\!\mathbf{r}_k} \nonumber \phi_{1i} \boldsymbol{\cdot} \boldsymbol{\nabla}_{\!\mathbf{r}_k} \phi_{2j} \pm \left(1-\tfrac{3}{2}\sin^2\theta_{13}\right) \boldsymbol{\nabla}_{\!\mathbf{r}_k} \phi_{1i} \boldsymbol{\cdot} \boldsymbol{\nabla}_{\!\mathbf{r}_k} \phi_{2j}^\prime \right] \\
 \nonumber &\mp 3 \phi_{1i} \phi_{2j} \cos\theta_{12} \left[q_i \frac{r_3}{r_2 r_{23}^2} (\cos\theta_{13} - \cos\theta_{12} \cos\theta_{23}) + p_j \frac{r_3}{r_1 r_{13}^2}(\cos\theta_{23} - \cos\theta_{12} \cos\theta_{13})\right.\\
 \nonumber & \left. \;\;\;\;\;  + \sin^2\theta_{12} \left(m_i \frac{r_1}{r_2 r_{12}^2} + m_j \frac{r_2}{r_1 r_{12}^2} \right) \right] \\
 \nonumber &\mp 3 \phi_{1i} \phi_{2j}^\prime \cos\theta_{13} \left[q_i \frac{r_2}{r_3 r_{23}^2} (\cos\theta_{12} - \cos\theta_{13} \cos\theta_{23}) + m_j^\prime \frac{r_2}{r_1 r_{12}^2}(\cos\theta_{23} - \cos\theta_{12} \cos\theta_{13})\right.\\
 \nonumber & \left. \;\;\;\;\;  + \sin^2\theta_{13} \left(p_i \frac{r_1}{r_3 r_{13}^2} + p_j^\prime \frac{r_3}{r_1 r_{13}^2} \right) \right] \\
 \nonumber &+ \left. \left[\frac{2}{r_1} - \frac{2}{r_2} - \frac{2}{r_3} - \frac{2}{r_{12}} - \frac{2}{r_{13}} + \frac{2}{r_{23}} - 2 E_H - 2 E_{Ps} - \frac{1}{2}\kappa^2 \right] \right. \\
 &\;\;\;\;\; \times \left[\left(1-\tfrac{3}{2}\sin^2\theta_{12}\right) \phi_{1i} \phi_{2j} \pm \left(1-\tfrac{3}{2}\sin^2\theta_{13}\right) \phi_{1i} \phi_{2j}^\prime \right] \Bigg\} d\tau_{int}
\end{align}

\begin{align}
\label{eq:DWavePhi2Phi1}
\left(\bar{\phi}_{2i},L \bar{\phi}_{1j}\right) = 2 & \cdot 2\pi \int \Bigg\{ \sum_{k=1}^3 \left(1-\tfrac{3}{2}\sin^2\theta_{12}\right) \left[ \boldsymbol{\nabla}_{\!\mathbf{r}_k} \nonumber \phi_{2i} \boldsymbol{\cdot} \boldsymbol{\nabla}_{\!\mathbf{r}_k} \phi_{1j} \pm \boldsymbol{\nabla}_{\!\mathbf{r}_k} \phi_{2i} \boldsymbol{\cdot} \boldsymbol{\nabla}_{\!\mathbf{r}_k} \phi_{1j}^\prime \right] \\
 \nonumber &\mp 3 \phi_{2i} \phi_{1j} \cos\theta_{12} \left[p_i \frac{r_3}{r_1 {r_{13}}^2} (\cos\theta_{23} - \cos\theta_{12} \cos\theta_{13}) + q_j \frac{r_3}{r_2 {r_{23}}^2}(\cos\theta_{13} - \cos\theta_{12} \cos\theta_{23})\right.\\
 \nonumber & \left. \;\;\;\;\;  + \sin^2\theta_{12} \left(m_i \frac{r_2}{r_1 {r_{12}}^2} + m_j \frac{r_1}{r_2 {r_{12}}^2} \right) \right] \\
 \nonumber &\mp 3 \phi_{2i} \phi_{1j}^\prime \cos\theta_{12} \left[p_i \frac{r_3}{r_1 r_{13}^2} (\cos\theta_{23} - \cos\theta_{12} \cos\theta_{13}) + q_j^\prime \frac{r_3}{r_2 {r_{23}}^2}(\cos\theta_{13} - \cos\theta_{12} \cos\theta_{23})\right.\\
 \nonumber & \left. \;\;\;\;\;  + \sin^2\theta_{12} \left(m_i \frac{r_2}{r_1 {r_{12}}^2} + m_j^\prime \frac{r_1}{r_2 {r_{12}}^2} \right) \right] \\
 \nonumber &+ \left. \left[\frac{2}{r_1} - \frac{2}{r_2} - \frac{2}{r_3} - \frac{2}{r_{12}} - \frac{2}{r_{13}} + \frac{2}{r_{23}} - 2 E_H - 2 E_{Ps} - \frac{1}{2}\kappa^2 \right] \right. \\
 &\;\;\;\;\; \times \left(1-\tfrac{3}{2}\sin^2\theta_{12}\right) \left( \phi_{2i} \phi_{12j} \pm \phi_{2i} \phi_{1j}^\prime \right) \Bigg\} d\tau_{int}
\end{align}




\section{D-Wave Short-Range -- Long-Range Integrals}
\label{sec:DWaveShortLong}

To simplify these equations, define
\begin{align}
\nonumber \mathscr{L} S = \frac{L S}{Y_{20}(\theta_\rho)} = & \left(\frac{2}{r_1} - \frac{2}{r_2} - \frac{2}{r_{13}} + \frac{2}{r_{23}} \right) \Phi_{Ps}(r_{12}) \Phi_H(r_3) \sqrt{2\kappa} \, j_2(\kappa\rho) \\
\nonumber \mathscr{L} C = \frac{L C}{Y_{20}(\theta_\rho)} = & - \left(\frac{2}{r_1} - \frac{2}{r_2} - \frac{2}{r_{13}} + \frac{2}{r_{23}} \right) \Phi_{Ps}(r_{12}) \Phi_H(r_3) \sqrt{2\kappa} \, n_2(\kappa\rho) f_{sh}(\rho) \\
& - \Phi_{Ps}(r_{12}) \Phi_H(r_3) \sqrt{2\kappa} \frac{1}{2\rho} \left\{ \left[4 n_2(\kappa\rho) - 2 \kappa\rho \, n_1(\kappa\rho) \right] f_{sh}^\prime(\rho) - \rho \, n_2(\kappa\rho) f_{sh}^{\prime\prime}(\rho) \right\}.
\end{align}
Equivalently,
\begin{align}
\mathscr{L} S^\prime &= \frac{L S^\prime}{Y_{20}(\theta_{\rho^\prime})} \\
\mathscr{L} C^\prime &= \frac{L C^\prime}{Y_{20}(\theta_{\rho^\prime})}.
\end{align}

Each of the following equations has two forms that can be used.

%\begin{align}
%\label{eq:DWavePhi1SBar}
%\nonumber \left(\bar{\phi}_{1i},L \bar{S}\right) = \sqrt{2} \cdot 2\pi & \int \left[ \left(1 - \frac{3 r_2^2 \sin^2\theta_{12}}{8 \rho^2} \right) \left(\phi_{1i} \pm \phi_{1i}^\prime \right) \left(\frac{2}{r_1} - \frac{2}{r_2} - \frac{2}{r_{13}} + \frac{2}{r_{23}} \right) S_{22} \right] d\tau_{int} \\
%\nonumber = \sqrt{2} \cdot 2\pi & \int \phi_{1i} \left[ \left(1 - \frac{3 r_2^2 \sin^2\theta_{12}}{8 \rho^2} \right) \left( \frac{2}{r_1} - \frac{2}{r_2} - \frac{2}{r_{13}} + \frac{2}{r_{23}} \right) S_{22} \right. \\
%& \pm \left. \left(1 - \frac{3 r_3^2 \sin^2\theta_{13}}{8 {\rho^\prime}^2} \right) \left( \frac{2}{r_1} - \frac{2}{r_3} - \frac{2}{r_{12}} + \frac{2}{r_{23}} \right) S_{23} \right] d\tau_{int}
%\end{align}

\begin{align}
\label{eq:DWavePhi1SBar}
\nonumber \left(\bar{\phi}_{1i},L \bar{S}\right) = & \sqrt{2} \cdot 2\pi \int \left(1 - \frac{3 r_2^2 \sin^2\theta_{12}}{8 \rho^2} \right) \left(\phi_{1i} \pm \phi_{1i}^\prime \right) \mathscr{L}S \, d\tau_{int} \\
=& \sqrt{2} \cdot 2\pi \int \phi_{1i} \left[ \left(1 - \frac{3 r_2^2 \sin^2\theta_{12}}{8 \rho^2} \right) \mathscr{L}S \pm \left(1 - \frac{3 r_3^2 \sin^2\theta_{13}}{8 {\rho^\prime}^2} \right) \mathscr{L}S^\prime \right] d\tau_{int}
\end{align}

\begin{align}
\label{eq:DWavePhi1CBar}
\nonumber \left(\bar{\phi}_{1i},L \bar{C}\right) = & \sqrt{2} \cdot 2\pi \int \left(1 - \frac{3 r_2^2 \sin^2\theta_{12}}{8 \rho^2} \right) \left(\phi_{1i} \pm \phi_{1i}^\prime \right) \mathscr{L}C \, d\tau_{int} \\
=& \sqrt{2} \cdot 2\pi \int \phi_{1i} \left[ \left(1 - \frac{3 r_2^2 \sin^2\theta_{12}}{8 \rho^2} \right) \mathscr{L}C \pm \left(1 - \frac{3 r_3^2 \sin^2\theta_{13}}{8 {\rho^\prime}^2} \right) \mathscr{L}C^\prime \right] d\tau_{int}
\end{align}

%\begin{align}
%\label{eq:DWavePhi2SBar}
%\nonumber \left(\bar{\phi}_{2j},L \bar{S}\right) = \sqrt{2} \cdot 2\pi & \int \left\{ \left[ \left(1 - \frac{3 r_1^2 \sin^2\theta_{12}}{8 \rho^2} \right) \phi_{2j} \pm \left( \frac{3(r_1 \cos\theta_{13} + r_2 \cos\theta_{23})^2}{8 \rho^2} - \frac{1}{2} \right) \phi_{2j}^\prime \right] \left(\frac{2}{r_1} - \frac{2}{r_2} - \frac{2}{r_{13}} + \frac{2}{r_{23}} \right) S_{22} \right\} d\tau_{int} \\
%\nonumber = \sqrt{2} \cdot 2\pi & \int \phi_{2j} \left[ \left(1 - \frac{3 r_1^2 \sin^2\theta_{12}}{8 \rho^2} \right) \left( \frac{2}{r_1} - \frac{2}{r_2} - \frac{2}{r_{13}} + \frac{2}{r_{23}} \right) S_{22} \right. \\
%& \pm \left. \left( \frac{3(r_1 \cos\theta_{12} + r_3 \cos\theta_{23})^2}{8 {\rho^\prime}^2} - \frac{1}{2} \right) \left( \frac{2}{r_1} - \frac{2}{r_3} - \frac{2}{r_{12}} + \frac{2}{r_{23}} \right) S_{23} \right]  d\tau_{int}
%\end{align}

\begin{align}
\label{eq:DWavePhi2SBar}
\nonumber \left(\bar{\phi}_{2j},L \bar{S}\right) = & \sqrt{2} \cdot 2\pi \int \left[ \left( \frac{3(r_1 \cos\theta_{12} + r_2)^2}{8 \rho^2} - \frac{1}{2} \right) \phi_{2j} \pm \left( \frac{3(r_1 \cos\theta_{13} + r_2 \cos\theta_{23})^2}{8 \rho^2} - \frac{1}{2} \right) \phi_{2j}^\prime \right] \mathscr{L}S \, d\tau_{int} \\
=& \sqrt{2} \cdot 2\pi \int \phi_{2j} \left[ \left( \frac{3(r_1 \cos\theta_{12} + r_2)^2}{8 \rho^2} - \frac{1}{2} \right) \mathscr{L}S \pm \left( \frac{3(r_1 \cos\theta_{12} + r_3 \cos\theta_{23})^2}{8 {\rho^\prime}^2} - \frac{1}{2} \right) \mathscr{L}S^\prime \right] d\tau_{int}
\end{align}

\begin{align}
\label{eq:DWavePhi2CBar}
\nonumber \left(\bar{\phi}_{2j},L \bar{C}\right) = & \sqrt{2} \cdot 2\pi \int \left[ \left( \frac{3(r_1 \cos\theta_{12} + r_2)^2}{8 \rho^2} - \frac{1}{2} \right) \phi_{2j} \pm \left( \frac{3(r_1 \cos\theta_{13} + r_2 \cos\theta_{23})^2}{8 \rho^2} - \frac{1}{2} \right) \phi_{2j}^\prime \right] \mathscr{L}C \, d\tau_{int} \\
=& \sqrt{2} \cdot 2\pi \int \phi_{2j} \left[ \left( \frac{3(r_1 \cos\theta_{12} + r_2)^2}{8 \rho^2} - \frac{1}{2} \right) \mathscr{L}C \pm \left( \frac{3(r_1 \cos\theta_{12} + r_3 \cos\theta_{23})^2}{8 {\rho^\prime}^2} - \frac{1}{2} \right) \mathscr{L}C^\prime \right] d\tau_{int}
\end{align}

\begin{align}
\label{eq:DWavePhi3SBar}
\nonumber \left(\bar{\phi}_{12k},L \bar{S}\right) = \sqrt{2} \cdot & \frac{1}{4} \sqrt{\frac{3\pi}{5}} \int \mathscr{L}S \left\{ \left[ 8 \cos\theta_{12} + \frac{3 r_1 r_2 \sin^2\theta_{12}}{\rho^2} \right] \phi_{12k} \right. \\
\nonumber & \left. \pm \frac{1}{\sqrt{2}} \left[ 3 \frac{\cos\theta_{13}(r_1^2 + r_1 r_2 \cos\theta_{12}) + \cos\theta_{23}(r_1 r_2 + r_2^2 \cos\theta_{12})}{\rho^2} - 4 \cos\theta_{13} \right] \phi_{12k}^\prime  \right\} d\tau_{int} \\
\nonumber = \sqrt{2} \cdot & \frac{1}{4} \sqrt{\frac{3\pi}{5}} \int \phi_{12k} \left\{ \left[ 8 \cos\theta_{12} + \frac{3 r_1 r_2 \sin^2\theta_{12}}{\rho^2} \right] \mathscr{L}S \right. \\
& \pm \left. \frac{1}{\sqrt{2}} \left[ 3 \frac{\cos\theta_{12}(r_1^2 + r_1 r_3 \cos\theta_{13}) + \cos\theta_{23}(r_1 r_3 + r_3^2 \cos\theta_{13})}{{\rho^\prime}^2} - 4 \cos\theta_{12} \right] \mathscr{L}S^\prime \right\} d\tau_{int}
\end{align}

\begin{align}
\label{eq:DWavePhi3CBar}
\nonumber \left(\bar{\phi}_{12k},L \bar{C}\right) = \sqrt{2} \cdot & \frac{1}{4} \sqrt{\frac{3\pi}{5}} \int \mathscr{L}C \left\{ \left[ 8 \cos\theta_{12} + \frac{3 r_1 r_2 \sin^2\theta_{12}}{\rho^2} \right] \phi_{12k} \right. \\
\nonumber & \left. \pm \frac{1}{\sqrt{2}} \left[ 3 \frac{\cos\theta_{13}(r_1^2 + r_1 r_2 \cos\theta_{12}) + \cos\theta_{23}(r_1 r_2 + r_2^2 \cos\theta_{12})}{\rho^2} - 4 \cos\theta_{13} \right] \phi_{12k}^\prime  \right\} d\tau_{int} \\
\nonumber = \sqrt{2} \cdot & \frac{1}{4} \sqrt{\frac{3\pi}{5}} \int \phi_{12k} \left\{ \left[ 8 \cos\theta_{12} + \frac{3 r_1 r_2 \sin^2\theta_{12}}{\rho^2} \right] \mathscr{L}C \right. \\
& \pm \left. \frac{1}{\sqrt{2}} \left[ 3 \frac{\cos\theta_{12}(r_1^2 + r_1 r_3 \cos\theta_{13}) + \cos\theta_{23}(r_1 r_3 + r_3^2 \cos\theta_{13})}{{\rho^\prime}^2} - 4 \cos\theta_{12} \right] \mathscr{L}C^\prime \right\} d\tau_{int}
\end{align}


\section{D-Wave Long-Range -- Long-Range Integrals}
\label{sec:DWaveLongLong}

\begin{align}
\label{eq:DWaveSBarSBar}
\left(\bar{S},L\bar{S}\right) = \pm 2\pi \int \left\{ S_{22} S_{23} \left(\frac{2}{r_1} - \frac{2}{r_2} - \frac{2}{r_{13}} + \frac{2}{r_{23}} \right) \left[ \frac{3}{8} \frac{(4\rho^2 + 4 {\rho^\prime}^2 - r_{23}^2)^2}{16 \rho^2 {\rho^\prime}^2} - \frac{1}{2} \right] \right\} d\tau_{int}
\end{align}

\begin{align}
\label{eq:DWaveCBarSBar}
\left(\bar{C},L\bar{S}\right) = \pm 2\pi \int \left\{ S_{22} C_{23} \left(\frac{2}{r_1} - \frac{2}{r_2} - \frac{2}{r_{13}} + \frac{2}{r_{23}} \right) \left[ \frac{3}{8} \frac{(4\rho^2 + 4 {\rho^\prime}^2 - r_{23}^2)^2}{16 \rho^2 {\rho^\prime}^2} - \frac{1}{2} \right] \right\} d\tau_{int}
\end{align}

\begin{align}
\label{eq:DWaveSBarCBar}
\nonumber \left(\bar{S},L\bar{C}\right) = 2\pi \int \Bigg\{ \pm & S_{23} C_{22} \left(\frac{2}{r_1} - \frac{2}{r_2} - \frac{2}{r_{13}} + \frac{2}{r_{23}} \right) \left[ \frac{3}{8} \frac{(4\rho^2 + 4 {\rho^\prime}^2 - r_{23}^2)^2}{16 \rho^2 {\rho^\prime}^2} - \frac{1}{2} \right] \\
\nonumber - & \left[ S_{22} \pm \left( \frac{3}{8} \frac{(4\rho^2 + 4 {\rho^\prime}^2 - r_{23}^2)^2}{16 \rho^2 {\rho^\prime}^2} - \frac{1}{2} \right) S_{23} \right] \sqrt{2\kappa} \, \Phi_{Ps}\left(r_{12}\right) \Phi_H\left(r_3\right) \\
& \times \frac{1}{2\rho} \left( \left[ 4 n_2(\kappa\rho) - 2 \kappa\rho \, n_1(\kappa\rho) \right] f_{sh}^\prime(\rho) - \rho \, n_2(\kappa\rho) f_{sh}^{\prime\prime}(\rho) \right) \Bigg\} d\tau_{int}
\end{align}

\begin{align}
\label{eq:DWaveCBarCBar}
\nonumber \left(\bar{C},L\bar{C}\right) = 2\pi \int \Bigg\{ \pm & C_{23} C_{22} \left(\frac{2}{r_1} - \frac{2}{r_2} - \frac{2}{r_{13}} + \frac{2}{r_{23}} \right) \left[ \frac{3}{8} \frac{(4\rho^2 + 4 {\rho^\prime}^2 - r_{23}^2)^2}{16 \rho^2 {\rho^\prime}^2} - \frac{1}{2} \right] \\
\nonumber - & \left[ C_{22} \pm \left( \frac{3}{8} \frac{(4\rho^2 + 4 {\rho^\prime}^2 - r_{23}^2)^2}{16 \rho^2 {\rho^\prime}^2} - \frac{1}{2} \right) C_{23} \right] \sqrt{2\kappa} \, \Phi_{Ps}\left(r_{12}\right) \Phi_H\left(r_3\right) \\
& \times \frac{1}{2\rho} \left( \left[ 4 n_2(\kappa\rho) - 2 \kappa\rho \, n_1(\kappa\rho) \right] f_{sh}^\prime(\rho) - \rho \, n_2(\kappa\rho) f_{sh}^{\prime\prime}(\rho) \right) \Bigg\} d\tau_{int}
\end{align}

\section{D-Wave Mixed Terms}
\label{sec:MixedTerms}

According to Schwartz \cite{Schwartz1961a}, to have a complete description, each partial wave needs $\ell+1$ symmetries. As shown in \cref{eq:SWaveTrial,eq:PWaveTrial}, we use the full sets of symmetries for the S-wave and P-wave.

\begin{align}
\label{eq:MixedAng}
\psi(\ell_1,\ell_2,L,M) &= \psi_{(1,1,2,0)} = \sum_{m=-1}^{+1} Y_{1,m}(\theta_1,\varphi_1) Y_{1,m}(\theta_2,\varphi_2) \left< 1,m; 1,-m,0 | 2,0 \right> \nonumber \\
	&= Y_{1,-1}(\theta_1,\varphi_1) Y_{1,+1}(\theta_2,\varphi_2)
    \left< 1,-1,1,+1 | 2,0 \right> \nonumber \\
& \  + Y_{1,0}(\theta_1,\varphi_1) Y_{1,0}(\theta_2,\varphi_2)
    \left< 1,0,1,0 | 2,0 \right> \nonumber \\
& \ + Y_{1,+1}(\theta_1,\varphi_1) Y_{1,-1}(\theta_2,\varphi_2)
   \left< 1,+1,1,-1 | 2,0 \right>,
\end{align}
where $\ell_1$ and $\ell_2$ are the angular momenta on the particles in Ps, and $L$ and $M$ give the angular momentum of the Ps.
These can be combined into a single set as
\begin{equation}
\label{eq:MixedAngSimple}
\psi_{(1,1,2,0)}(\theta_1,\theta_2) = \frac{3}{4\uppi} \frac{1}{\sqrt{6}} \left(3 \cos\theta_1 \cos\theta_2 - \cos\theta_{12} \right).
\end{equation}
This avoids the issue of dealing with complex terms in the $m = -1$ and $m = 1$ cases. Refer to \cref{sec:MixedDerivation} for this derivation.

\todoi{Add latest mixed term work by Peter}

\textbf{Papers of Humberston et al. that use the mixed symmetry terms}
\begin{itemize}
	\item e$^+$-H: \cite{Brown1985a,BrownThesis,WattsThesis,Humberston1997,VanReeth1997}
	\item e$^+$-He (treated with single electron and model potential): \cite{Dunn2000,DunnThesis}
\end{itemize}

\textbf{Papers of Humberston et al. that do not use the mixed symmetry terms}
\begin{itemize}
	\item e$^+$-He: \cite{VanReeth1997,VanReethThesis}
\end{itemize}

\todoi{Mention Peter's current work on the mixed terms \cite{VanReeth2015}.}


\section{D-Wave Second Formalism}
\label{sec:DSecondForm}
For the second formalism, the angular momentum needs to be placed on the Ps and H, so we would need
\begin{subequations}
\label{eq:DWave2ndPhiBar}
\begin{align}
\bar{\phi}_{\rho i} &= \left(1 \pm P_{23}\right) Y_{20}(\theta_\rho) \rho^2 \phi_i \label{eq:DWave2ndPhi1i}\\
\bar{\phi}_{3j} &= \left(1 \pm P_{23}\right) Y_{20}(\theta_3) r_3^2 \phi_j \label{eq:DWave2ndPhi2j}.
\end{align}
\end{subequations}
It will be shown that these are sufficient to describe this system without 
having a third set of terms as in \cref{eq:DWaveTrial}.

To use these in the short-range code, we need an expression similar to
\cref{eq:P2rhoY10}. We cannot use that expression, as we have $\rho^2$. We
follow the same procedure as that of obtaining \cref{eq:P2rhoY10}, but the
spherical harmonic is more complicated. Full details of the derivation are
found in the ``Second Formalism P-wave and D-wave.nb'' notebook.
Substituting \cref{eq:CosRho} into \cref{eq:DWaveSpherHarm}, multiplying by
$\rho^2$, and then using \cref{eq:RhoRDef}, we obtain
\begin{equation}
\label{eq:rhoident2}
\rho^2 Y_{20}(\theta_\rho) = \frac{1}{4} \left[r_1^2 Y_{20}(\theta_1) + r_2^2 Y_{20}(\theta_2) \right] + \frac{1}{2} \sqrt{\frac{5}{16 \pi}} 
   r_1 r_2 \left[3 \cos\theta_1 \cos\theta_2 - \cos\theta_{12} \right].
\end{equation}
The first set of brackets is obviously similar to \cref{eq:P2rhoY10} but with
$1 \to 2$ coming from the P-wave to the D-wave. More interesting though is the
second set of brackets, which is clearly the mixed terms in
\cref{eq:DWavePhi12k,eq:MixedAngSimple}.

As mentioned in \cref{sec:MixedTerms}, we do not use the mixed terms in the
D-wave calculation due to their complexity. Because of this, the second
formalism has not been implemented for the D-wave and could be a source of
future work.


\section{Determination of Nonlinear Parameters}
\label{sec:DWaveNonlinear}

\todoi{Do we want to just move that table here and to the P-wave chapter?}
Using the simplex method described in \cref{sec:Simplex}, we obtained a set of
nonlinear parameters for $^1$D and $^3$D in \cref{tab:NonlinearOptimizedPD}.
We realized when calculating the phase shifts however that these were more
sensitive to the values of the nonlinear parameters than the S-wave and P-wave,
especially for $^3$D. This is likely due to the short-range terms trying to
make up for the missing mixed symmetry terms. We performed some manual
optimization of the nonlinear parameters for two $\kappa$ values to try to
improve the phase shifts.



\cref{eq:ConvRatio}

\begin{figure}[H]
	\centering
	\includegraphics[width=7in]{dwave-singlet-alpha-k01-variation}
	\caption{$^{1}$D $\kappa = 0.1$}
	\label{fig:dwave-singlet-alpha-k01-variation}
\end{figure}

\begin{figure}[H]
	\centering
	\includegraphics[width=7in]{dwave-singlet-alpha-k06-variation}
	\caption{$^1$D $\kappa = 0.6$}
	\label{fig:dwave-singlet-alpha-k06-variation}
\end{figure}


\begin{figure}[H]
	\centering
	\includegraphics[width=7in]{dwave-singlet-beta-k01-variation}
	\caption{$^{1}$D $\kappa = 0.1$}
	\label{fig:dwave-singlet-beta-k01-variation}
\end{figure}

\begin{figure}[H]
	\centering
	\includegraphics[width=7in]{dwave-singlet-beta-k06-variation}
	\caption{$^1$D $\kappa = 0.6$}
	\label{fig:dwave-singlet-beta-k06-variation}
\end{figure}

$\beta = 0.8$


\begin{figure}[H]
	\centering
	\includegraphics[width=7in]{dwave-triplet-alpha-k01-variation}
	\caption{$^3$D $\kappa = 0.1$}
	\label{fig:dwave-triplet-alpha-k01-variation}
\end{figure}

\begin{figure}[H]
	\centering
	\includegraphics[width=7in]{dwave-triplet-alpha-k06-variation}
	\caption{$^3$D $\kappa = 0.6$}
	\label{fig:dwave-triplet-alpha-k06-variation}
\end{figure}


%\setlength{\abovecaptionskip}{6pt}   % 0.5cm as an example
%\setlength{\belowcaptionskip}{6pt}   % 0.5cm as an example
%\begin{table}[H]
%\centering
%\begin{tabular}{l l l}
%\toprule
%Method & $^1E_R \text{ (eV)}$ & $^1\Gamma \text{ (eV)}$ \\
%\midrule
%This work & $4.7189 \pm 0.0002$ & $0.0860 \pm 0.0005$ \\
%CC (9Ps9H + H$^-$) \cite{Walters2004} & $4.899$ & $0.0872$ \\
%Stabilization \cite{Yan2003} & $4.714$ & $0.0969$ \\
%Coupled-pseudostate \cite{Campbell1998} & $5.28$ & $0.47$ \\
%CC (9Ps9H) \cite{Blackwood2002}) & $5.16$ & $0.15$ \\
%CC (22Ps1H + H$^-$) \cite{Blackwood2002b} & $4.814$ & $0.065$ \\
%Optical potential \cite{DiRienzi2002a} & $4.729$ & $0.327$ \\
%Complex rotation \cite{Ho1998} & $4.710 \pm 0.0027$ & $0.0925 \pm 0.0054$  \\
%\bottomrule
%\end{tabular}
%\caption{D-Wave Resonance Parameters}
%\label{tab:DWaveResonancesOther}
%\end{table}


% Please add the following required packages to your document preamble:
% \usepackage{booktabs}
\begin{table}[h]
\centering
\begin{tabular}{cccccc}
\toprule
$\kappa$ & $\mathcal{R}(2)$ & $\mathcal{R}(3)$ & $\mathcal{R}(4)$ & $\mathcal{R}(5)$ & $\mathcal{R}(6)$ \\
\midrule
0.3 & 1.523 & 0.795 & 0.572 & 0.491 & 0.377 \\
0.4 & 1.286 & 0.680 & 0.475 & 0.412 & 0.429 \\
0.5 & 1.048 & 0.603 & 0.464 & 0.442 & 0.466 \\
0.6 & 0.826 & 0.534 & 0.575 & 0.435 & 0.481 \\
0.7 & 0.471 & 0.874 & 0.472 & 0.521 & 0.465 \\
\bottomrule
\end{tabular}
\caption{My caption}
\label{my-label}
\end{table}


\begin{table}[h]
\centering
\begin{tabular}{cccccc}
\toprule
$\kappa$ & $\mathcal{R}(2)$ & $\mathcal{R}(3)$ & $\mathcal{R}(4)$ & $\mathcal{R}(5)$ & $\mathcal{R}(6)$ \\
\midrule
0.3 & 2.475 & 0.352 & 1.611 & 0.316 & 1.628 \\
0.4 & 2.157 & 0.291 & 1.372 & 0.209 & 1.902 \\
0.5 & 1.835 & 0.254 & 1.133 & 0.225 & 1.343 \\
0.6 & 1.539 & 0.237 & 0.979 & 0.299 & 0.667 \\
0.7 & 1.206 & 0.306 & 0.696 & 0.360 & 0.441 \\
\bottomrule
\end{tabular}
\caption{My caption}
\label{my-label}
\end{table}

\begin{table}[h]
\centering
\begin{tabular}{cccccc}
\toprule
$\kappa$ & $\mathcal{R}(2)$ & $\mathcal{R}(3)$ & $\mathcal{R}(4)$ & $\mathcal{R}(5)$ & $\mathcal{R}(6)$ \\
\midrule
0.3 & 3.950 & 0.932 & 0.564 & 0.512 & 0.432 \\
0.4 & 3.399 & 0.811 & 0.446 & 0.404 & 0.469 \\
0.5 & 2.812 & 0.740 & 0.417 & 0.411 & 0.519 \\
0.6 & 2.261 & 0.730 & 0.466 & 0.400 & 0.527 \\
0.7 & 1.714 & 0.860 & 0.490 & 0.433 & 0.484 \\
\bottomrule
\end{tabular}
\caption{My caption}
\label{my-label}
\end{table}

\begin{table}[h]
\centering
\begin{tabular}{cccccc}
\toprule
$\kappa$ & $\mathcal{R}(2)$ & $\mathcal{R}(3)$ & $\mathcal{R}(4)$ & $\mathcal{R}(5)$ & $\mathcal{R}(6)$ \\
\midrule
0.3 & 7.392 & 3.424 & 0.505 & 0.426 & 1.768 \\
0.4 & 6.978 & 3.035 & 0.390 & 0.265 & 2.421 \\
0.5 & 6.435 & 2.625 & 0.345 & 0.229 & 2.235 \\
0.6 & 5.739 & 2.241 & 0.393 & 0.242 & 1.251 \\
0.7 & 5.082 & 2.063 & 0.504 & 0.244 & 0.723 \\
\bottomrule
\end{tabular}
\caption{My caption}
\label{my-label}
\end{table}



\section{D-Wave Results}
\label{sec:DWaveResults}

\subsection{Phase Shifts}
\label{sec:DWavePhase}


\begin{table}[H]
\centering
\setlength{\tabcolsep}{-2pt}
\footnotesize
\begin{tabular}{@{\hskip 0.1cm}l . . . . . . .}
\toprule
Method & \multicolumn{1}{c}{\phantom{1}0.1} & \multicolumn{1}{c}{\phantom{1}0.2} & \multicolumn{1}{c}{\phantom{1}0.3} & \multicolumn{1}{c}{\phantom{1}0.4} & \multicolumn{1}{c}{\phantom{1}0.5} & \multicolumn{1}{c}{\phantom{1}0.6} & \multicolumn{1}{c}{\phantom{1}0.7} \\
\midrule
This work $(\omega = 6)$ $\delta_2^+$ 				& 1.358^{-4} & 2.987^{-3} & 1.592^{-2} & 4.933^{-2} & 1.113^{-1} & 2.027^{-1} & 3.215^{-1} \\
CC 14Ps14H+H$^-$ \cite{Walters2004} $\delta_2^+$	& 2.02^{-4}  & 3.49^{-3}  & 1.73^{-2}  & 5.22^{-2}  & 1.16^{-1}  & 2.08^{-1}  & 3.24^{-1} \\
CC 14Ps14H \cite{Blackwood2002} $\delta_2^+$		& 1.46^{-4}  & 3.15^{-3}  & 1.65^{-2}  & 4.95^{-2}  & 1.08^{-1}  & 1.94^{-1}  & 3.02^{-1} \\
3-state CC \cite{Sinha1997} $\delta_2^-$			& 3.22^{-5}  & 9.29^{-4}  & 5.96^{-3}  & 2.01^{-2}  & 4.63^{-2}  & 8.29^{-2}  & 1.23^{-1} \\
SE \cite{Ray1997} $\delta_2^+$ 						& 3.18^{-5}  & 9.17^{-4}  & 5.87^{-3}  & 1.97^{-2}  & 4.54^{-2}  & 8.09^{-2}  & 1.19^{-1} \\
5-state CC \cite{Adhikari1999} $\delta_2^+$			& 1.8^{-5}   & 5.3^{-4}   & 3.5^{-3}   & 1.2^{-2}   & 2.9^{-2}   & 5.5^{-2}   & 8.8^{-2} \\
SE \cite{Hara1975} $\delta_2^+$						& 0.0        & 0.0009     & 0.0058     & 0.0195     & 0.0453     & 0.0810     & 0.1194 \\
\midrule
This work $(\omega = 6)$ $\delta_2^-$ 				& 5.808^{-5}  & 7.120^{-4}  & 1.065^{-3}  & -2.002^{-3} & -1.122^{-2} & -2.647^{-2} & -4.451^{-2} \\
CC 14Ps14H \cite{Blackwood2002} $\delta_2^-$		& 8.48^{-5}   & 1.15^{-3}   & 2.84^{-3}   & 2.37^{-3}   & -4.66^{-3}  & -1.85^{-2}  & -3.27^{-2} \\
3-state CC \cite{Sinha1997} $\delta_2^-$			& -2.74^{-5}  & -7.77^{-4}  & -4.83^{-3}  & -1.55^{-2}  & -3.41^{-2}  & -5.83^{-2}  & -8.25^{-2} \\
SE \cite{Ray1997} $\delta_2^+$ 						& -3.00^{-5}  & -8.56^{-4}  & -5.37^{-3}  & -1.76^{-2}  & -3.95^{-2}  & -7.03^{-2}  & -1.06^{-1} \\
5-state CC \cite{Adhikari1999} $\delta_2^+$			& -1.4^{-5}   & -4.0^{-4}   & -2.6^{-3}   & -8.6^{-3}   & -2.0^{-2}   & -3.6^{-2}   & -5.5^{-2} \\
SE \cite{Hara1975} $\delta_2^+$						& 0.0         & -8.0^{-4}   & -5.3^{-3}   & -1.74^{-2}  & -3.95^{-2}  & -7.04^{-2}  & -1.062^{-1} \\
\bottomrule
\end{tabular}
\caption[$^{1,3}$D comparisons]{$^{1,3}$D comparisons. Values in the header are $\kappa$ in au. Exponents denote powers of 10.}
\label{tab:DWaveComparisons}
\end{table}

\todoi{Maybe use dcolumn}
\todoi{Comparisons and discussion similar to what's in PRA paper}
\todoi{Note how repulsive and attractive for $^3$D}


\begin{figure}[H]
	\centering
	\includegraphics[width=7in]{dwave-phases}
	\caption{$^{1,3}$D phase shifts}
	\label{fig:DWavePhase}
\end{figure}


\begin{figure}[H]
	\centering
	\includegraphics[width=5.25in]{dwave-comparisons}
	\caption[Comparison of D-wave phase shifts]{Comparison of $^1$D (a) and $^3$D (b) phase shifts with results from other groups. Results are ordered according to year of publication. This work -- solid curves; \mbox{\textcolor{blue}{$\times$} -- CC \cite{Walters2004};} \mbox{$\CIRCLE$ -- Kohn \cite{VanReeth2003};} \mbox{\textcolor{red}{\textbf{+}} -- CC \cite{Blackwood2002};} \mbox{$\triangledown$ -- SVM 2002 \cite{Ivanov2002};} \mbox{\textcolor{red}{$\vartriangle$} -- 6-state CC \cite{Sinha2000};} \mbox{$\blacksquare$ -- 5-state CC \cite{Adhikari1999};} \mbox{$\vartriangle$ -- 3-state CC \cite{Sinha1997};} \mbox{\textcolor[RGB]{0,127,0}{$\bigstar$} -- CC \cite{Ray1997};} \mbox{\textcolor{blue}{$\lozenge$} -- Static-exchange \cite{Hara1975}.}}
	\label{fig:DWaveComparisons}
\end{figure}


\subsection{Extrapolations}
\label{sec:DWaveExtrap}


\begin{table}[H]
\centering
\begin{tabular}{c | c c c}
\toprule
$\kappa$ & $\delta^+ (\omega \rightarrow \infty) (4-6)$ & $\delta^- (\omega \rightarrow \infty) (4-6)$  \\
\midrule
0.1 & $1.516^{-4}$ & N/A \\
0.2 & $3.051^{-3}$ & N/A \\
0.3 & $1.606^{-2}$ & N/A \\
0.4 & $4.972^{-2}$ & N/A \\
0.5 & $1.122^{-1}$ & N/A \\
0.6 & $2.044^{-1}$ & N/A \\
0.7 & $1.324^{-1}$ & N/A \\
\bottomrule
\end{tabular}
\caption{D-Wave Extrapolation Results}
\label{tab:DWaveExtrap}
\end{table}

\todoi{Update these}


\subsection{Resonance Parameters}
\label{sec:DWaveResonance}



\setlength{\abovecaptionskip}{6pt}   % 0.5cm as an example
\setlength{\belowcaptionskip}{6pt}   % 0.5cm as an example
\begin{table}[H]
\centering
\begin{tabular}{l l l}
\toprule
Method & $^1E_R \text{ (eV)}$ & $^1\Gamma \text{ (eV)}$ \\
\midrule
This work & $4.7189 \pm 0.0002$ & $0.0860 \pm 0.0005$ \\
CC (9Ps9H + H$^-$) \cite{Walters2004} & $4.899$ & $0.0872$ \\
Stabilization \cite{Yan2003} & $4.714$ & $0.0969$ \\
Coupled-pseudostate \cite{Campbell1998} & $5.28$ & $0.47$ \\
CC (9Ps9H) \cite{Blackwood2002}) & $5.16$ & $0.15$ \\
CC (22Ps1H + H$^-$) \cite{Blackwood2002b} & $4.814$ & $0.065$ \\
Optical potential \cite{DiRienzi2002a} & $4.729$ & $0.327$ \\
Complex rotation \cite{Ho1998} & $4.710 \pm 0.0027$ & $0.0925 \pm 0.0054$  \\
\bottomrule
\end{tabular}
\caption{D-Wave Resonance Parameters}
\label{tab:DWaveResonancesOther}
\end{table}



\biblio
\end{document}