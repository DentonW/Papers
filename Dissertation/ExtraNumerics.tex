% -*- root: Dissertation.tex -*-
\documentclass[Dissertation.tex]{subfiles} 
\begin{document}

\chapter{Extra Numerics}
\label{chp:ExtraNumerics}

\iftoggle{UNT}{This}{\lettrine{\textcolor{startcolor}{T}}{his}}
Appendix gives more
details than what is provided in \cref{chp:Computation} on
computations.

\section{Short-Range Code Nonlinear Parameter Optimization}
\label{sec:BoundOptimization}

A powerful property of the Rayleigh-Ritz variational method is the 
ability to systematically improve the wavefunction to lower the upper bound 
on the energy. By either adding terms to the expansion in
\cref{eq:BoundWavefn_psi} or changing the nonlinear parameters $\alpha$,
$\beta$ and $\gamma$, the energy can be reduced and a possible minimum
found. This is a three-dimensional optimization problem, and we tried 
multiple methods for the nonlinear parameter optimization.

%\subsection{Newton's Method}
%\label{sec:Newton}
%From Ref. \cite{Yan1999} (and easily derived), if the wavefunction $\Psi$
%is properly normalized, the energy can be minimized by setting the
%derivative with respect to each of the nonlinear parameters equal to 0 by
%\beq
%\frac{\partial E}{\partial \alpha} = 2 \left< \Psi \Big| H \Big| \frac{\partial\Psi}{\partial \alpha} \right> - 2 \left< \Psi \Big| \frac{\partial\Psi}{\partial \alpha} \right>.
%\label{eq:EnergyDerivativeNorm}
%\eeq
%With this, the wavefunction is minimized using the 1-D Newton's method for 
%each parameter separately \cite{Sauer2006}. Newton's method was unstable for 
%anything other than a small basis set ($\omega \leq 4$), so I evaluated other 
%methods. Newton's method for n-dimensions can be used as well, though it 
%requires computation of the Jacobian.

\subsection{Broyden's Method}
\label{sec:Broyden}

%\todoi{``Bad''}

%The next method I tried was Broyden's method \cite{Sauer2006}, which can 
We used Broyden's method \cite{Sauer2006}, which can 
solve for all three nonlinear parameters simultaneously. This was more stable 
than the 1-D Newton method. The second Broyden's method, sometimes referred 
to as the ``bad Broyden's method'' was used here. As Kvaalen points out, this 
method is perfectly usable and can be faster than the first Broyden's method 
\cite{Kvaalen1991}. \Cref{tab:NonlinearOptimized3SBroyden,tab:BroydenPWaveSingOpt}
show the nonlinear parameters used for $^3$S and $^1$P. All other partial waves
used the simplex method described in \cref{sec:Simplex}. This is because I
already had a number of results for $^3$S and $^1$P, so running everything
again for simplex-optimized nonlinear parameters was unnecessary. For $^3$S,
we were able to use even more terms than we could use for $^1$S \cref{tab:Nonlinear},
and we were able to use the same number of terms for $^1$P and $^3$P.

\setlength{\abovecaptionskip}{6pt}   % 0.5cm as an example
\setlength{\belowcaptionskip}{6pt}   % 0.5cm as an example
\begin{table}
\centering
\begin{tabular}{c c c c}
\toprule
$\omega$ & $\alpha$ & $\beta$ & $\gamma$ \\ [0.5ex]
\midrule
1 & 0.264440 & 0.831645 & 0.498871 \\
2 & 0.356175 & 0.452426 & 0.829591 \\
3 & 0.347611 & 0.467298 & 0.814971 \\
4 & 0.323300 & 0.333783 & 0.974653 \\
\bottomrule
\end{tabular}
\caption{Broyden optimized $^3S$ nonlinear parameters}
\label{tab:NonlinearOptimized3SBroyden}
\end{table}


\begin{table}
\centering
\begin{tabular}{c c c c}
\toprule
$\omega$ & $\alpha$ & $\beta$ & $\gamma$ \\ [0.5ex]
\midrule
1 & 0.47767 & 0.50273 & 0.97498 \\
2 & 0.48253 & 0.49342 & 0.96874 \\
3 & 0.42803 & 0.43099 & 0.98993 \\
4 & 0.39740 & 0.37617 & 0.96205 \\
\bottomrule
\end{tabular}
\caption{Broyden optimized $^1$P nonlinear parameters}
\label{tab:BroydenPWaveSingOpt}
\end{table}



\subsection{Simplex Method}
\label{sec:Simplex}

%\setlength{\abovecaptionskip}{6pt}   % 0.5cm as an example
%\setlength{\belowcaptionskip}{6pt}   % 0.5cm as an example
\begin{table}
\centering
\begin{tabular}{c c c c}
\toprule
$\omega$ & $\alpha$ & $\beta$ & $\gamma$ \\ [0.5ex]
\midrule
0 & 0.30226 & 0.45479 & 1.07962 \\
1 & 0.53592 & 0.59453 & 1.02206 \\
2 & 0.57450 & 0.65222 & 0.98020 \\
3 & 0.58966 & 0.63150 & 0.97397 \\
4 & 0.58493 & 0.60995 & 0.98610 \\
5 & 0.58691 & 0.58045 & 1.03321 \\
\bottomrule
\end{tabular}
\caption{Simplex optimized $^1$S nonlinear parameters}
\label{tab:NonlinearOptimized1SSimplex}
\end{table}

Broyden's method was more stable than Newton's method for this work, but I 
also tried the \texttt{gsl\_multimin\_fminimizer\_nmsimplex} routine from the 
GNU Scientific Library, which is an implementation of the simplex method
\cite{GSL,GSLsimplex}. This was the most stable of the three methods tried to 
optimize $\alpha$, $\beta$, and $\gamma$ simultaneously. I normally stopped 
at $\omega = 5$ for the optimization, and the S-wave singlet runs are shown 
in \cref{tab:NonlinearOptimized1SSimplex}. \Cref{tab:NonlinearOptimizedPD} 
has the optimized nonlinear parameters used for the P-wave and D-wave. Due to 
the slowness of the general short-range code (see \cref{sec:GeneralShort}), 
the F-wave through G-wave just use the parameters $\alpha = 0.5$, $\beta = 0.6$,
and $\gamma = 1.1$.

\begin{table}
\small
\centering
\begin{tabular}{c c c c}
\toprule
Partial Wave & $\alpha$ & $\beta$ & $\gamma$ \\
\midrule
$^3$P & 0.310 & 0.311 & 0.995 \\
$^1$D & 0.359 & 0.368 & 0.976 \\
$^3$D & 0.356 & 0.365 & 0.976 \\
\bottomrule
\end{tabular}
\caption{Simplex optimized nonlinear parameters for the P-wave and D-wave}
\label{tab:NonlinearOptimizedPD}
\end{table}

With the work on the second formalism for the P-wave \cref{sec:PWaveOpt}, it 
was also possible to use the simplex method to optimize all 6 nonlinear 
parameters simultaneously instead of having to optimize each symmetry 
separately. %If the sector-based method of Drake and Yan \cite{Yan1995} is 
%ever implemented for this problem, it would be possible to optimize all 15 
%parameters simultaneously, but it may be much more computationally feasible 
%to optimize each sector separately.


\subsection{P-Wave Nonlinear Parameter Optimization}
\label{sec:PWaveOpt}

Using the simplex method described in \cref{sec:Simplex}, we optimized the nonlinear parameters for both the first and second formalisms. \Cref{tab:SimplexPWaveSingOpt,tab:SimplexPWaveTripOpt} show the results of these optimizations in the second and third columns. We also let each symmetry have its own set of nonlinear parameters for each symmetry in the fourth and fifth columns. This was inspired by the work of Yan and Ho \cite{Yan1999}, where they used 5 different sets of nonlinear parameters to calculate the PsH ground state energy.

\begin{table}
\footnotesize
\centering
\begin{tabular}{c c c c c}
\toprule
\toprule
$\omega$ & 1st formalism / 1 set & 2nd formalism / 1 set & 1st formalism / 2 sets & 2nd formalism / 2 sets \\
\midrule
\midrule
 &  &  & 0.5341, 0.4536, 1.0139 & 0.5668, 0.4686, 0.9787 \\
1 & 0.4776, 0.5027, 0.9749 & 0.4571, 0.5700, 0.9266 & 0.3792, 0.5455, 0.9816 & 0.6777, 0.8587, 0.4813 \\
 & $\textbf{-0.666819968640}$ & $\textbf{-0.663226610680}$ & $\textbf{-0.670015702237}$ & $\textbf{-0.665355147531}$ \\
\midrule
 &  &  & 0.5270, 0.4394, 1.0036 & 0.4497, 0.5039, 0.9459 \\
2 & 0.4825, 0.4934, 0.9687 & 0.4734, 0.5162, 0.9584 & 0.4087, 0.5213, 0.9704 & 0.3963, 1.0233, 0.4327 \\
 & $\textbf{-0.700681070987}$ & $\textbf{-0.699190666285}$ & $\textbf{-0.701936448530}$ & $\textbf{-0.700245066225}$ \\
\midrule
 &  &  & 0.4632, 0.3918, 1.001 & 0.4653, 0.4512, 0.9905 \\
3 & 0.4297, 0.4337, 0.9808 & 0.4317, 0.4564, 0.9621 & 0.3844, 0.4620, 0.9740 & 0.8745, 0.9796, 0.4957 \\
 & $\textbf{-0.718093418924}$ & $\textbf{-0.717282613790}$ & $\textbf{-0.718548496648}$ & $\textbf{-0.717931026880}$ \\
\midrule
 &  &  & 0.3954, 0.3505, 0.9997 & 0.3744, 0.3746, 0.9537 \\
4 & 0.3740, 0.3744, 0.9898 & 0.3803, 0.3951, 0.9648 & 0.3478, 0.39493, 0.9798 & 0.4078, 0.9010, 0.3351 \\
 & $\textbf{-0.727918723553}$ & $\textbf{-0.727281885394}$ & $\textbf{-0.728067443345}$ & $\textbf{-0.727981667586}$ \\
\midrule
 &  &  & 0.3401, 0.3181, 0.9983 & 0.3373, 0.3390, 0.9635 \\
5 & 0.3293, 0.3289, 0.9939 & 0.3371, 0.3468, 0.9680 & 0.3174, 0.3397, 0.9880 & 0.4299, 0.9452, 0.3023 \\
 & $\textbf{-0.734233160953}$ & $\textbf{-0.733680013812}$ & $\textbf{-0.734264232997}$ & $\textbf{-0.734219573964}$ \\
\bottomrule
\bottomrule
\end{tabular}
\caption{Simplex $^1$P-Wave Short-Range Optimization}
\label{tab:SimplexPWaveSingOpt}
\end{table}


\begin{table}
\footnotesize
\centering
\begin{tabular}{c c c c c}
\toprule
\toprule
$\omega$ & 1st formalism / 1 set & 2nd formalism / 1 set & 1st formalism / 2 sets & 2nd formalism / 2 sets \\
\midrule
\midrule
 &  &  & 0.3832, 0.2911, 0.9894 & 0.3367, 0.3436, 0.9517 \\
1 & 0.3316, 0.3535, 0.9956 & 0.3302, 0.3540, 0.9909 & 0.2854, 0.3959, 1.0112 & 0.7606, 0.6593, 0.2304 \\
 & $\textbf{-0.624190839847}$ & $\textbf{-0.622936676609}$ & $\textbf{-0.629448013148}$ & $\textbf{-0.626442778437}$ \\
\midrule
 &  &  & 0.4056, 0.3294, 0.9799 & 0.3623, 0.3952, 0.9716 \\
2 & 0.3664, 0.3750, 0.9900 & 0.3679, 0.3896, 0.9753 & 0.3226, 0.4097, 1.0078 & 0.1961, 0.7724, 0.7692 \\
 & $\textbf{-0.674819577647}$ & $\textbf{-0.672880514312}$ & $\textbf{-0.676837948726}$ & $\textbf{-0.673438894110}$ \\
\midrule
 &  &  & 0.3874, 0.3314, 0.9823 & 0.3557, 0.3794, 0.9593 \\
3 & 0.3585, 0.3613, 0.9911 & 0.3639, 0.3810, 0.9690 & 0.3257, 0.3881, 1.0036 & 0.2080, 0.5994, 0.7711 \\
 & $\textbf{-0.704764841366}$ & $\textbf{-0.703191879865}$ & $\textbf{-0.705411707053}$ & $\textbf{-0.703452729525}$ \\
\midrule
 &  &  & 0.3539, 0.3192, 0.9879 & 0.3392, 0.3502, 0.9789 \\
4 & 0.3357, 0.3366, 0.9936 & 0.3427, 0.3554, 0.9680 & 0.3162, 0.3539, 0.9989 & 0.1624, 0.6564, 0.9512 \\
 & $\textbf{-0.721696022987}$ & $\textbf{-0.720513417195}$ & $\textbf{-0.721859567659}$ & $\textbf{-0.720596381145}$ \\
\midrule
 &  &  & 0.3184, 0.3038, 0.9935 & 0.3177, 0.3271, 0.9769 \\
5 & 0.3106, 0.3109, 0.9953 & 0.3182, 0.3275, 0.9682 & 0.3025, 0.3186, 0.9966 & 0.1142, 0.4274, 1.1088 \\
 & $\textbf{-0.731463030326}$ & $\textbf{-0.730592482596}$ & $\textbf{-0.731486455300}$ & $\textbf{-0.730615490923}$ \\
\bottomrule
\bottomrule
\end{tabular}
\caption{Simplex $^3$P Short-Range Optimization}
\label{tab:SimplexPWaveTripOpt}
\end{table}

From \cref{tab:SimplexPWaveSingOpt,tab:SimplexPWaveTripOpt}, we see that 
lower energy eigenvalues are always obtained with the first formalism with 1 
set versus the second formalism with 1 set. Likewise, the first formalism has 
lower energy eigenvalues than the second formalism when both have 2 sets. For 
$\omega \geq 2$, the first formalism with 1 set even has lower energy 
eigenvalues than the second symmetry with 2 sets. We had trouble obtaining 
phase shifts with two different sets of nonlinear parameters due to increased 
linear dependence, but for higher $\omega$, we also see that the energy does 
not change much. Using more than one set of nonlinear parameters could be 
explored further in future work, and we have done preliminary investigation 
into this for the S-wave.


\subsection{D-Wave Nonlinear Parameter Optimization}
\label{sec:DWaveNonlinear}

%\todoi{Do we want to just move that table here and to the P-wave chapter?}
The comparison to the CC results \cite{Walters2004,Blackwood2002}
in \cref{tab:DWaveComparisons,fig:DWavePhase} for the $^1$D-wave is
reasonable, with the CC results below the complex Kohn results at $\kappa = 0.7$.
For $^3$D, the CC results are much higher, as can be seen in the inset
in \cref{fig:DWavePhase}. The $^1$D phase shifts are small, so their overall
contribution to the integrated cross section is small. 
This lead us to investigate whether the phase shifts could be improved by a
better selection of the short-range nonlinear parameters. If the phase shifts
were fully converged, varying the nonlinear parameters should have little
effect on their values. Both Van Reeth \cite{VanReethPrivate} and I
investigated this.

Using the simplex method described in \cref{sec:Simplex}, we obtained a set of
nonlinear parameters for $^1$D and $^3$D in \cref{tab:NonlinearOptimizedPD}.
We realized when calculating the phase shifts however that these were more
sensitive to the values of the nonlinear parameters than the S-wave and P-wave,
especially for $^3$D. This is likely due to the short-range terms trying to
make up for the missing mixed symmetry terms. We performed some manual
optimization of the nonlinear parameters for two $\kappa$ values to try to
improve the phase shifts.

For this investigation, we chose $\kappa$ values in two different regions:
one at lower $\kappa$ and another at higher $\kappa$. After optimization with
these, we also checked convergence ratios in the more sensitive resonance
region. The $\kappa = 0.1$ choice
was made, since this is the lowest value that we report. We chose $\kappa = 0.6$
for the higher $\kappa$ region, as this gets closer to the Ps(n=2) threshold
but is far enough away from the $^1$D resonance to avoid sensitivity of the
nonlinear parameters due to the resonance.

For these variations, we kept $\gamma$ constant and used $\omega = 4$. From 
\cref{tab:Nonlinear}, the value of $\gamma$ was found to be near $1$ using 
the simplex method (\cref{sec:Simplex}) for every partial wave for both the 
singlet and triplet. An explanation is that the $r_3$ coordinate represents 
the electron in H, and $\gamma = 1$ gives the short-range terms multiplied by 
the H wavefunction given in \cref{eq:HWave} \cite{VanReethPrivate}. We used the
original nonlinear parameters found by the simplex method and given in
\cref{tab:NonlinearOptimizedPD} as a starting set. For
$^1$D, these are $\alpha = 0.359$, $\beta = 0.368$, and $\gamma = 0.976$. For
$^3$D, these are $\alpha = 0.356$, $\beta = 0.365$, and $\gamma = 0.976$.

For the first variation, we investigated the $\alpha$ nonlinear parameter at
$\kappa = 0.1$. We varied $\alpha$ to see its effect on the phase shifts.
\Cref{fig:dwave-singlet-alpha-k01-variation} shows the results of this
variation. There is a maximum in the phase shift in 
\cref{fig:dwave-singlet-alpha-k01-variation}(a) and a large difference between
the phase shifts at low and high $\alpha$. If we decrease
$\alpha$ from its value of 0.359, however, from
\cref{fig:dwave-singlet-alpha-k01-variation}(b),
the convergence ratio $R'(4)$, given by \cref{eq:ConvRatio},
increases drastically. From this analysis, we had hoped for higher phase shifts,
but we have a trade-off between this and reasonable convergence ratios. Due to
this, we have kept the nonlinear parameter $\alpha$ at 0.359, which seems to be
a reasonable compromise between higher phase shifts and better convergence
ratios. This is likely an indication of the amount of numerical instability
we have with small phase shifts.

%\todoi{Move $R'(4)$ to the left or move y-axis numbers to the right. Also narrow these figures to fit within the margins.}

\begin{figure}
	\centering
	\includegraphics[width=\textwidth]{dwave-singlet-alpha-k01-variation}
	\caption[Variation of the nonlinear parameter $\alpha$ for $^{1}$D at $\kappa = 0.1$]{Phase shifts (a) and convergence ratios (b) for variation of the nonlinear parameter $\alpha$ for $^{1}$D at $\kappa = 0.1$}
	\label{fig:dwave-singlet-alpha-k01-variation}
\end{figure}

At the higher $\kappa$ of 0.6, the variation looks very different, as seen in
\cref{fig:dwave-singlet-alpha-k06-variation}. The maximum is at about
$\alpha = 0.6$, and $R'(4)$ is much less than 1. Interestingly,
$R'(4)$ decreases monotonically as $\alpha$ is increased. For
$\kappa = 0.6$, it is clear that choosing $\alpha = 0.6$ is much better than
the original 0.359.

\begin{figure}
	\centering
	\includegraphics[width=\textwidth]{dwave-singlet-alpha-k06-variation}
	\caption[Variation of the nonlinear parameter $\alpha$ for $^{1}$D at $\kappa = 0.6$]{Phase shifts (a) and convergence ratios (b) for variation of the nonlinear parameter $\alpha$ for $^{1}$D at $\kappa = 0.6$}
	\label{fig:dwave-singlet-alpha-k06-variation}
\end{figure}

Starting from the original nonlinear parameters, we also varied $\beta$. The
$\beta$ variation looks very similar to the $\alpha$ derivation, but there is
a surprising breakdown of the phase shifts when $\beta > 0.6$. At $\beta = 0.7$,
$\delta_2^+$ increases significantly, and $R'(4) > 5$. For $\beta = 0.8$,
$\delta_2^+ = -2.0694^{-3}$, and $R'(4) = -128.3$, so the phase shifts
for large $\beta$ are obviously not reliable. When $\beta$ is smaller, we see
very similar behavior to that of the $\alpha$ variation in
\cref{fig:dwave-singlet-alpha-k01-variation}. The convergence ratio increases as
$\beta$ gets too low, and the maximum is around $\beta = 0.3$. Similar to the
$\alpha$ variation, we kept the original $\beta = 0.368$ value.

\begin{figure}
	\centering
	\includegraphics[width=\textwidth]{dwave-singlet-beta-k01-variation}
	\caption[Variation of the nonlinear parameter $\beta$ for $^{1}$D at $\kappa = 0.1$]{Phase shifts (a) and convergence ratios (b) for variation of the nonlinear parameter $\beta$ for $^{1}$D at $\kappa = 0.1$}
	\label{fig:dwave-singlet-beta-k01-variation}
\end{figure}

For $\kappa = 0.6$, we see a breakdown when $\beta$ is large as well. The plot
in \cref{fig:dwave-singlet-beta-k06-variation} shows that we have not hit the maximum
$\delta_2^+$ before the phase shifts exhibit breakdown. The $\alpha$ variation
appears to be more stable than the $\beta$ variation from these runs.

\begin{figure}
	\centering
	\includegraphics[width=\textwidth]{dwave-singlet-beta-k06-variation}
	\caption[Variation of the nonlinear parameter $\beta$ for $^{1}$D at $\kappa = 0.6$]{Phase shifts (a) and convergence ratios (b) for variation of the nonlinear parameter $\beta$ for $^{1}$D at $\kappa = 0.6$}
	\label{fig:dwave-singlet-beta-k06-variation}
\end{figure}

The variations for $^3$D look similar to that of $^1$D. There are less points
in \cref{fig:dwave-triplet-alpha-k01-variation}, but we can see that as
$\alpha$ is lowered, $R'(4)$ increases, and there is a maximum around
$\alpha = 0.3$. Again, we kept the original $\alpha$ of 0.356.

\begin{figure}
	\centering
	\includegraphics[width=\textwidth]{dwave-triplet-alpha-k01-variation}
	\caption[Variation of the nonlinear parameter $\alpha$ for $^{3}$D at $\kappa = 0.1$]{Phase shifts (a) and convergence ratios (b) for variation of the nonlinear parameter $\alpha$ for $^{3}$D at $\kappa = 0.1$}
	\label{fig:dwave-triplet-alpha-k01-variation}
\end{figure}



\begin{figure}
	\centering
	\includegraphics[width=\textwidth]{dwave-triplet-alpha-k06-variation}
	\caption[Variation of the nonlinear parameter $\alpha$ for $^{3}$D at $\kappa = 0.6$]{Phase shifts (a) and convergence ratios (b) for variation of the nonlinear parameter $\alpha$ for $^{3}$D at $\kappa = 0.6$}
	\label{fig:dwave-triplet-alpha-k06-variation}
\end{figure}

Due to the problems we had with the $\beta$ variation for $^1$D
and the original $\alpha$, we did not pursue this for $^3$D.
With these new choices of $^{1,3}$D nonlinear parameters for higher $\kappa$
that use a higher $\alpha$ of 0.6, we also investigated varying $\beta$, as
shown in \Cref{fig:dwave-singlet-alphabeta-k06-variation} for $^1$D.
Again, the phase shifts are unreliable as $\beta$ is increased much.
The equivalent plots for $^3$D are in \cref{fig:dwave-triplet-alphabeta-k06-variation}.
For this, the phase shifts break down even earlier, starting after $\beta = 0.5$.

\begin{figure}
	\centering
	\includegraphics[width=\textwidth]{dwave-singlet-alphabeta-k06-variation}
	\caption[Variation of the nonlinear parameter $\beta$ for $^{1}$D at $\kappa = 0.1$ and $\alpha = 0.6$]{Phase shifts (a) and convergence ratios (b) for variation of the nonlinear parameter $\beta$ for $^{1}$D at $\kappa = 0.1$ and $\alpha = 0.6$}
	\label{fig:dwave-singlet-alphabeta-k06-variation}
\end{figure}

\begin{figure}
	\centering
	\includegraphics[width=\textwidth]{dwave-triplet-alphabeta-k06-variation}
	\caption[Variation of the nonlinear parameter $\beta$ for $^{3}$D at $\kappa = 0.1$ and $\alpha = 0.6$]{Phase shifts (a) and convergence ratios (b) for variation of the nonlinear parameter $\beta$ for $^{3}$D at $\kappa = 0.1$ and $\alpha = 0.6$}
	\label{fig:dwave-triplet-alphabeta-k06-variation}
\end{figure}

We had previously noticed for multiple partial waves that if $\alpha$ and
$\beta$ are equal, linear dependence becomes a large issue. Based on this and
the graphs in \cref{fig:dwave-singlet-alphabeta-k06-variation,fig:dwave-triplet-alphabeta-k06-variation},
we tried the set of nonlinear parameters $\alpha = 0.6$, $\beta = 0.5$,
and $\gamma = 0.976$ for $^{1,3}$D for a full run of $\omega = 6$.
Using the Todd procedure in \cref{sec:ToddBound}, this $^1$D set only uses 844
terms, and the $^3$D set uses 854 terms.

\Cref{tab:D1WaveBetaVar} compares the sets with $\beta = 0.368$ and $0.5$ (using
$\alpha = 0.6$ and $\gamma = 0.976$). We do not use the restricted set described
in \cref{sec:Restricted} for our final calculations, but a comparison with it
can give an idea of how stable the phase shifts are. Comparing the full and
restricted sets for $\beta = 0.368$, the phase shifts do not change much, even
though we are reducing the basis set from 913 to 720 terms. However, comparison
of the full and restricted sets for $\beta = 0.5$ shows a much larger change in
the phase shifts, despite only changing from 844 to 720 terms. This seems to
indicate that the set with $\beta = 0.5$ is not as stable and could potentially
suffer from linear dependence.

\begin{table}[h]
\centering
\begin{tabular}{c....}
\toprule
$\kappa$ & \multicolumn{1} {c}{$\delta_2^-$ full $\beta = 0.368$} & \multicolumn{1} {c}{$\delta_2^-$ restricted $\beta = 0.368$}  & \multicolumn{1} {c}{$\delta_2^-$ full $\beta = 0.5$}  & \multicolumn{1} {c}{$\delta_2^-$ restricted $\beta = 0.5$} \\
\midrule
0.3 & 1.599^{-2} & 1.595^{-2} & 1.693^{-2} & 1.600^{-2} \\
0.4 & 4.978^{-2} & 4.965^{-2} & 5.176^{-2} & 4.987^{-2} \\
0.5 & 1.126^{-1} & 1.124^{-1} & 1.152^{-1} & 1.130^{-1} \\
0.6 & 2.058^{-1} & 2.053^{-1} & 2.083^{-1} & 2.066^{-1} \\
0.7 & 3.275^{-1} & 3.269^{-1} & 3.302^{-1} & 3.293^{-1} \\
\bottomrule
\end{tabular}
\caption[$^1$D phase shifts for varying $\beta$]{$^1$D phase shifts for sets of nonlinear parameters with $\alpha = 0.6$ and $\gamma = 0.976$. The full set for $\beta = 0.368$ has 913 terms, and the full set for $\beta = 0.5$ has 844 terms. The restricted sets have 720 terms.}
\label{tab:D1WaveBetaVar}
\end{table}

To see whether the $\beta = 0.5$ set has linear dependence issues, we compared
the convergence ratios. From \cref{tab:D1Beta368VarConv}, we see that the
convergence ratios are less than 0.5 for $\omega = 6$, which indicates good
convergence. \Cref{tab:D1Beta5VarConv} however shows a problem with linear
dependence for the $\beta = 0.5$ set. Based on this analysis, we have chosen
the set $\alpha = 0.359$, $\beta = 0.368$, and $\gamma = 0.976$ for higher
$\kappa$ for $^1$D.

\begin{table}
\centering
\begin{tabular}{cccccc}
\toprule
$\kappa$ & $R'(2)$ & $R'(3)$ & $R'(4)$ & $R'(5)$ & $R'(6)$ \\
\midrule
0.3 & 1.523 & 0.795 & 0.572 & 0.491 & 0.377 \\
0.4 & 1.286 & 0.680 & 0.475 & 0.412 & 0.429 \\
0.5 & 1.048 & 0.603 & 0.464 & 0.442 & 0.466 \\
0.6 & 0.826 & 0.534 & 0.575 & 0.435 & 0.481 \\
0.7 & 0.471 & 0.874 & 0.472 & 0.521 & 0.465 \\
\bottomrule
\end{tabular}
\caption{Convergence ratios for $^1D$ at multiple $\omega$ values for the full $\beta = 0.368$ set}
\label{tab:D1Beta368VarConv}
\end{table}


\begin{table}
\centering
\begin{tabular}{cccccc}
\toprule
$\kappa$ & $R'(2)$ & $R'(3)$ & $R'(4)$ & $R'(5)$ & $R'(6)$ \\
\midrule
0.3 & 2.475 & 0.352 & 1.611 & 0.316 & 1.628 \\
0.4 & 2.157 & 0.291 & 1.372 & 0.209 & 1.902 \\
0.5 & 1.835 & 0.254 & 1.133 & 0.225 & 1.343 \\
0.6 & 1.539 & 0.237 & 0.979 & 0.299 & 0.667 \\
0.7 & 1.206 & 0.306 & 0.696 & 0.360 & 0.441 \\
\bottomrule
\end{tabular}
\caption{Convergence ratios for $^1D$ at multiple $\omega$ values for the full $\beta = 0.5$ set}
\label{tab:D1Beta5VarConv}
\end{table}


\begin{table}
\centering
\begin{tabular}{c....}
\toprule
$\kappa$ & \multicolumn{1} {c}{$\delta_2^-$ full $\beta = 0.368$} & \multicolumn{1} {c}{$\delta_2^-$ restricted $\beta = 0.368$}  & \multicolumn{1} {c}{$\delta_2^-$ full $\beta = 0.5$}  & \multicolumn{1} {c}{$\delta_2^-$ restricted $\beta = 0.5$} \\
\midrule
0.3  & 1.100^{-3}  & 1.058^{-3}  & 1.977^{-3}  & 1.079^{-3}   \\
0.4  & -1.796^{-3} & -1.898^{-3} & 8.016^{-6}  & -1.751^{-3}  \\
0.5  & -1.070^{-2} & -1.087^{-2} & -8.432^{-3} & -1.052^{-2}  \\
0.6  & -2.544^{-2} & -2.568^{-2} & -2.331^{-2} & -2.501^{-2}  \\
0.7  & -4.281^{-2} & -4.314^{-2} & -4.085^{-2} & -4.191^{-2}  \\
\bottomrule
\end{tabular}
\caption[$^3$D phase shifts for varying $\beta$]{$^3$D phase shifts for sets of nonlinear parameters with $\alpha = 0.6$ and $\gamma = 0.976$. The full set for $\beta = 0.365$ has 913 terms, and the full set for $\beta = 0.5$ has 854 terms. The restricted sets have 720 terms.}
\label{tab:D3WaveBetaVar}
\end{table}

We also looked at the convergence ratios for $^3$D, as given in
\cref{tab:D3Beta365VarConv,tab:D3Beta5VarConv}. Again, the set of nonlinear
parameters with $\beta = 0.5$ is problematic, giving $R'(6) > 1$ for
most $\kappa$ values. From this analysis, we decided on the $^3$D nonlinear
parameters of $\alpha = 0.356$, $\beta = 0.365$, and $\gamma = 0.976$.

\begin{table}
\centering
\begin{tabular}{cccccc}
\toprule
$\kappa$ & $R'(2)$ & $R'(3)$ & $R'(4)$ & $R'(5)$ & $R'(6)$ \\
\midrule
0.3 & 3.950 & 0.932 & 0.564 & 0.512 & 0.432 \\
0.4 & 3.399 & 0.811 & 0.446 & 0.404 & 0.469 \\
0.5 & 2.812 & 0.740 & 0.417 & 0.411 & 0.519 \\
0.6 & 2.261 & 0.730 & 0.466 & 0.400 & 0.527 \\
0.7 & 1.714 & 0.860 & 0.490 & 0.433 & 0.484 \\
\bottomrule
\end{tabular}
\caption{Convergence ratios for $^3D$ at multiple $\omega$ values for the full $\beta = 0.365$ set}
\label{tab:D3Beta365VarConv}
\end{table}

\begin{table}
\centering
\begin{tabular}{cccccc}
\toprule
$\kappa$ & $R'(2)$ & $R'(3)$ & $R'(4)$ & $R'(5)$ & $R'(6)$ \\
\midrule
0.3 & 7.392 & 3.424 & 0.505 & 0.426 & 1.768 \\
0.4 & 6.978 & 3.035 & 0.390 & 0.265 & 2.421 \\
0.5 & 6.435 & 2.625 & 0.345 & 0.229 & 2.235 \\
0.6 & 5.739 & 2.241 & 0.393 & 0.242 & 1.251 \\
0.7 & 5.082 & 2.063 & 0.504 & 0.244 & 0.723 \\
\bottomrule
\end{tabular}
\caption{Convergence ratios for $^3D$ at multiple $\omega$ values for the full $\beta = 0.5$ set}
\label{tab:D3Beta5VarConv}
\end{table}


\section{Nonlinear Parameters and Terms Used}
\label{sec:NonlinParam}

\begin{table}
  \centering
  \begin{tabular}{cclccccc}
	\toprule
	Partial wave & $\omega$ & $N^\prime(\omega)$ & $\alpha$ & $\beta$ & $\gamma$ & $\mu$ & $m_\ell$ \\
	\midrule
	$^1$S                      & 7 & 1505        & 0.568 & 0.580 & 1.093 & 0.9 & 1 \\
	$^3$S                      & 7 & 1633        & 0.323 & 0.334 & 0.975 & 0.9 & 1 \\
	$^1$P                      & 7 & 1000        & 0.397 & 0.376 & 0.962 & 0.9 & 3 \\
	$^3$P                      & 7 & 1000        & 0.310 & 0.311 & 0.995 & 0.9 & 3 \\
	$^1$D $(\kappa < 0.3)$     & 6 & 916         & 0.359 & 0.368 & 0.976 & 0.7 & 7 \\
	$^1$D $(\kappa \geq 0.3)$  & 6 & 913         & 0.600 & 0.368 & 0.976 & 0.7 & 7 \\
	$^3$D $(\kappa < 0.3)$     & 6 & 919         & 0.356 & 0.365 & 0.976 & 0.7 & 7 \\
	$^3$D $(\kappa \geq 0.3)$  & 6 & 913         & 0.600 & 0.365 & 0.976 & 0.7 & 7 \\
	$^1$F $(\kappa < 0.4)$     & 5 & $385^\star$ & 0.359 & 0.368 & 0.976 & 0.7 & 7 \\
	$^1$F $(\kappa \geq 0.4)$  & 5 & 462         & 0.500 & 0.600 & 1.100 & 0.7 & 7 \\
	$^3$F $(\kappa < 0.4)$     & 5 & $385^\star$ & 0.356 & 0.365 & 0.976 & 0.7 & 7 \\
    $^3$F $(\kappa \geq 0.4)$  & 5 & 462         & 0.600 & 0.365 & 0.976 & 0.7 & 7 \\
	$^1$G $(\kappa < 0.45)$    & 5 & 462         & 0.359 & 0.368 & 0.976 & 0.7 & 9 \\
    $^1$G $(\kappa \geq 0.45)$ & 5 & 462         & 0.500 & 0.600 & 1.100 & 0.7 & 9 \\
	$^3$G $(\kappa < 0.45)$    & 5 & 462         & 0.356 & 0.365 & 0.976 & 0.7 & 9 \\
    $^3$G $(\kappa \geq 0.45)$ & 5 & 462         & 0.600 & 0.365 & 0.976 & 0.7 & 9 \\
	$^1$H $(\kappa < 0.5)$     & 5 & 462         & 0.359 & 0.368 & 0.976 & 0.7 & 11 \\
	$^1$H $(\kappa \geq 0.5)$  & 5 & 462         & 0.500 & 0.600 & 1.100 & 0.7 & 11 \\
	$^3$H $(\kappa < 0.45)$    & 5 & 462         & 0.356 & 0.365 & 0.976 & 0.7 & 11 \\
    $^3$H $(\kappa \geq 0.45)$ & 5 & 462         & 0.600 & 0.365 & 0.976 & 0.7 & 11 \\
	\bottomrule
  \end{tabular}
  \caption[Parameters for each partial wave]{Parameters for each partial wave. Numbers marked with a star indicate the
restriction in the $r_3$ power described in \cref{sec:Restricted}.}
  \label{tab:Nonlinear}
\end{table}

\Cref{tab:Nonlinear} gives the nonlinear parameters and total short-range
terms used for each partial wave. This table is the same as that in our
Ref.~\cite{Woods2015}.


%\section{L\"uchow and Kleindienst Algorithm}
%\label{sec:LuchowBound}
%The algorithm that L\"uchow and Kleindienst \cite{Luchow1992} propose is 
%similar to Todd's method \cite{sec:ToddBound}. They use all terms through $\omega = 5$, since 
%none of these have linear dependence within this set. This fact could be 
%applied to Allan Todd's algorithm to speed up the calculation. The rest of 
%the terms are partitioned up into blocks of 300-400 terms, except the last 
%block, which will have modulus(N - 462, block size) terms. A key difference 
%from Allan Todd's algorithm is that the terms should be ordered in terms of 
%increasing $\omega$. In Allan Todd's algorithm, the initial ordering does not 
%matter. The terms in the first block are added to the basis set for
%$\omega = 5$, which has 462 terms. If we use a block with 300 terms, then 300 matrices 
%of size $463 \times 463$ are created, and their eigenvalues are determined. Whichever 
%term gives the lowest energy when added to this set is saved to the basis 
%set. This continues with ever-increasing matrix sizes, until all terms in the 
%block are used, or a linear dependence is noticed. L\"uchow and Kleindienst 
%propose that linear dependences can be prevented by omitting terms that do 
%not contribute significantly to the energy. The energy from the previous step 
%is saved, and the energy from the current step is compared to this. If the 
%difference in energies is small, e.g. $\Delta E < 10^{-10} E_h$ in their 
%paper, then that term is omitted. Since the lowest energy is chosen at each 
%step, when this $\Delta E$ is too small, the rest of the block can be omitted.
%
%Two parameters that are adjustable in this algorithm are the block size and 
%cutoff value for the energy. To determine the best values for each, we
%set $\omega = 7$, and one of the parameters was held constant. To 
%determine the optimal block size, an energy cutoff of $5\e{-9}$ was used.
%\begin{center}
%\begin{tabular}{|c|c|c|c|}
%\hline
%Block size & Used terms & Energy & Time (min)\\
%\hline
%200 & 1041 & -0.789 181 890 & 470 \\
%250 & 1024 & -0.789 184 500 & 572 \\
%300 & 960 &  -0.789 180 520 & 749 \\
%350 & 999 &  -0.789 181 942 & unknown \\
%400 & 1012 & -0.789 185 588 & 852 \\
%500 & 985 &  -0.789 184 116 & 764 \\
%\hline
%\end{tabular}
%\end{center}
%
%For most trials, increasing the block size also increased the time it took to 
%complete the computation. It can be seen that a block size of N-462 gives a 
%modified version of Todd's algorithm, since we are testing all terms at each 
%step. A block size of 1 would correspond to no reordering of terms, leading 
%to including problematic terms. The cutoff value for $\Delta E$,
%$E_{\text{cutoff}}$, is $10^{-10} E_h$ in L\"uchow and Kleindienst's paper
%\cite{Luchow1992}, but a value of $10^{-10}$ in our calculations led to LAPACK 
%errors due to linear dependences. The block size was set to 400, and it was
%determined that an energy cutoff of $10^{-9}$ was the largest value that caused linear 
%dependences.
%
%\begin{center}
%\begin{tabular}{|c|c|c|}
%\hline
%Energy cutoff & Used terms & Energy\\
%\hline
%2\e{-9} & 1116 & -0.789 185 858 \\
%3\e{-9} & 1053 & -0.789 186 346 \\
%4\e{-9} & 1043 & -0.789 185 173 \\
%5\e{-9} & 1012 & -0.789 185 588 \\
%1\e{-8} & 963 &  -0.789 188 024 \\
%\hline
%\end{tabular}
%\end{center}
%
%Unfortunately, the intuition that a larger block size and smaller energy 
%cutoff will yield a better final energy is not necessarily correct. The best 
%value of the energy was actually obtained for the largest tested energy 
%cutoff of $10^{-8}$. A block size of 500 also gave a worse energy than did a 
%block size of 400. More tests need to be performed to determine the optimal 
%value of both parameters, and these could possibly change for higher values 
%of $\omega$ or for a different system.
%
%We decided to use the Todd algorithm (\cref{sec:ToddBound}) due to the better 
%energy obtained and the larger set of terms returned. The Todd algorithm is 
%also more consistent between runs. The trade-off is the increased time to 
%perform a calculation, especially for large $\omega$.
%
%\begin{figure}
	%\centering
	%{\includegraphics[height=2in]{Luchow}}
	%\caption{Diagram of L\"uchow and Kleindienst procedure}
	%\label{fig:Luchow}
%\end{figure}


\section{S-Wave Maximum \texorpdfstring{$\mu$}{mu}}
\label{sec:MaxMu}

\begin{figure}
	\centering
	\includegraphics[width=5in]{swave-max-mu}
	\caption[Maximum $\mu$ for $^1$S-wave]{Value of $\mu$ in the shielding function, \cref{eq:PartialWaveShielding}, that gives the largest phase shift versus the number of terms for the $^1$S-wave with $\kappa = 0.1$}
	\label{fig:swave-max-mu}
\end{figure}

For the first three partial waves, we truncated the short-range basis set of
size $N(\omega)$ using the method in \cref{sec:CompPhase}. For $^1$S however,
we found that we could more easily determine the cutoff by performing phase 
shift runs for multiple $\mu$ values and various $\kappa$ to see what the 
optimal value of $\mu$ was that gave the highest phase shift. As shown in
\cref{fig:swave-max-mu}, this value of $\mu$ was fairly stable through most of
the $N$ range, but it suddenly spikes up after 1505 terms and fluctuates
greatly. For some $\kappa$ values, we also saw that when this happened, the
value of $\mu$ that gave the highest phase shift was sometimes unbounded,
definitely indicating linear dependence. Interestingly, this behavior was
not noticed for the $^3$S-wave. We did not try this for other partial waves,
but it gave a good cutoff point for the $^1$S-wave.


\section{Gaussian Quadratures}
\label{sec:GaussQuad}

%\todoi{How hardcoded Gaussian abscissae and weights}
%\todoi{Mention looking into \cite{Bailey2004,Ma1996,Mori2001,Odrzywolek2011,Fukuda2005}}

Gaussian quadratures are used to integrate many classes of integrals. In their
most general form, these quadratures are given by \cite[p.887]{Abramowitz1965}
\beq
\label{eq:GeneralQuadratures}
\int_a^b W(x) f(x) dx \approx \sum_{i=1}^n w_i f(x_i).
\eeq
Gaussian quadratures are particularly attractive, since they give exact 
results for polynomials up to degree $2n-1$. The weight function $W(x)$ can 
be chosen for certain classes of integrals. Three main types of weight 
functions are used in this work. For a discussion on the number of quadrature 
points used, refer to \cref{sec:QuadraturePoints}.

For most integrals over finite intervals, we use the Gauss-Legendre quadrature.
The exception is when we integrate over an interior angle ($\varphi_{ij}$),
where we use the Chebyshev-Gauss quadrature. For semi-infinite integrations,
we use Gauss-Laguerre quadratures. Each of these can be seen in
Ref.~\cite[p.887-890]{Abramowitz1965}.

The Gauss-Laguerre quadrature is typically over the range $(-1,1)$ but can be
modified to arbitrary finite intervals using a standard formula given in
Ref.~\cite[p.887]{Abramowitz1965}. Discussion over adapting the Gauss-Laguerre
quadrature to use an arbitrary lower limit and an extra constant multiplying
the variable in the exponential is found in \cref{sec:GaussLag}. Discussion over
adapting the Chebyshev-Gauss quadrature to the $\varphi_{ij}$ innermost
integrations is given in \cref{sec:ChebyshevGauss}.


%\subsection{Gauss-Legendre Quadrature}
%\label{sec:GaussLegendre}
%If the weight function is chosen as $W(x)=1$, and the integration interval is $(-1,1)$, this is known as Gauss-Legendre quadrature (or sometimes known simply as Gaussian quadrature). The orthogonal polynomials used are the Legendre polynomials, $P_n(x)$. \Cref{eq:GeneralQuadratures} becomes
%\beq
%\label{eq:GaussLeg}
%\int_{-1}^1 f(x) dx \approx \sum_{i=1}^n w_i f(x_i),
%\eeq
%where the $x_i$ abscissas are the $i^{th}$ zeros of $L_n(x)$, and the weights are given by
%\beq
%\label{eq:GaussLegWeights}
%w_i = \frac{2}{(1-x_i^2)[P^\prime_n(x_i)]^2}.
%\eeq
%The limits of integration must be from $-1$ to $1$, but this is generalized by using the transformation \cite{Abramowitz1965}
%\begin{align}
%\label{eq:GaussLegGen}
%\int_a^b f(x) dx &= \frac{b-a}{2} \int_{-1}^1 f \left(\frac{b-a}{2} x + \frac{a+b}{2}\right) dx \\
%&\approx \frac{b-a}{2} \sum_{i=1}^n w_i f \left(\frac{b-a}{2} x_i + \frac{a+b}{2}\right).
%\end{align}


\subsection{Gauss-Laguerre Quadrature}
\label{sec:GaussLag}
%The Gauss-Legendre quadrature cannot be used on semi-infinite intervals, so we use the Gauss-Laguerre quadrature in these cases. The orthogonal polynomials in this case are the Laguerre polynomials, $L_n(x)$, and the weight function is $W(x) = e^{-x}$. \Cref{eq:GeneralQuadratures} becomes
%\beq
%\label{eq:GaussLag}
%\int_0^\infty e^{-x} f(x) dx \approx \sum_{i=1}^n w_i f(x_i),
%\eeq
%where the $x_i$ abscissas are the $i^{th}$ zeros of $L_n(x)$, and the weights are given by
%\beq
%\label{eq:GaussLagWeights}
%w_i = \frac{x_i}{(n+1)^2 [L_{n+1}(x_i)]^2}.
%\eeq

The Gauss-Legendre quadrature cannot be used on semi-infinite intervals, so 
we use the Gauss-Laguerre quadrature in these cases. The orthogonal 
polynomials in this case are the Laguerre polynomials, $L_n(x)$, and the 
weight function is $W(x) = e^{-x}$. The specific form of
\cref{eq:GeneralQuadratures} is
\beq
\label{eq:GaussLag}
\int_0^\infty e^{-x} f(x) dx \approx \sum_{i=1}^n w_i f(x_i).
\eeq
When the integration is over the interval $(a,\infty)$ instead, \cref{eq:GaussLag} is transformed by
\beq
\label{eq:GaussLagGen1}
\int_a^\infty e^{-x} f(x) dx = \int_0^\infty e^{-(x+a)} f(x+a) dx = e^{-a} \int_0^\infty e^{-x} f(x+a) dx \approx e^{-a} \sum_{i=1}^n w_i f(x_i+a).
\eeq

\noindent A more general form of this is obtained by using a coefficient in the exponential, i.e.
\beq
\label{eq:GaussLagGen2}
\int_a^\infty e^{-m x} f(x) dx = \frac{1}{m} \int_a^\infty e^{-y} f\left(\frac{y}{m}\right) dy,
\eeq
where we have defined $y = m x$.  This allows for \cref{eq:GaussLagGen1} to be generalized to
\begin{align}
\label{eq:GaussLagGen}
\nonumber \int_a^\infty e^{-m x} f(x) dx &= \frac{1}{m} \int_{ma}^\infty e^{-y} f\left(\frac{y}{m}\right) dy = \frac{1}{m} \int_0^\infty e^{-(y+ma)} f\left(\frac{y}{m}+a\right) dy \\
& = \frac{e^{-ma}}{m} \int_0^\infty e^{-y} f\left(\frac{y}{m}+a\right) dy \approx \frac{e^{-ma}}{m} \sum_{i=1}^n w_i f\left(\frac{y_i}{m}+a\right).
\end{align}
The $y_i$ abscissas and $w_i$ weights are the same as %the less general case in
%\cref{eq:GaussLag,eq:GaussLagWeights}.
the standard Gauss-Laguerre quadrature. This more general form
quadrature is what we use for semi-infinite integrations.


%\subsection{Chebyshev--Gauss Quadrature}
%\label{sec:ChebyshevGauss1}
%If the weight function is chosen as $W(x)=\frac{1}{\sqrt{1-x^2}}$, and the
%integration interval is $(-1,1)$, this is known as Chebyshev-Gauss quadrature.
%The orthogonal polynomials used are the Chebyshev polynomials of the first
%kind, $T_n(x)$. \Cref{eq:GeneralQuadratures} becomes
%\beq
%\label{eq:GaussCheb}
%\int_{-1}^1 \frac{f(x)}{\sqrt{1-x^2}} dx \approx \sum_{i=1}^n w_i f(x_i),
%\eeq
%where
%\beq
%\label{eq:GaussChebAbsWeights}
%x_i = \cos\left(\frac{2i-1}{n}\pi\right) \text{ and } w_i = \frac{\pi}{n}.
%\eeq
%This quadrature is used for the internal angular integrations. A discussion of
%how to use this for these integrations is found in \cref{sec:ChebyshevGauss}.


\subsection{Chebyshev-Gauss Quadrature}
\label{sec:ChebyshevGauss}

To integrate over the $\varphi_{ij}$ variable, we can use \cite[p.79]{VanReethThesis}
\begin{equation}
\Int{ D(\cos \varphi_{23})}{\varphi_{23},0,2\pi} \approx \frac{2\pi}{n}\sum_{i=1}^n D\left[\cos\left(\frac{2i-1}{2n}\pi\right)\right].
\label{eq:PVRAng}
\end{equation}
To prove this, we start with
\begin{equation}
\Int{ D(\cos \varphi_{23}) }{\varphi_{23},0,\pi} \approx \frac{\pi}{n}\sum_{i=1}^n D\left[\cos\left(\frac{2i-1}{2n}\pi\right)\right].
\label{eq:PVRAngNew}
\end{equation}
Both of these equations are variations on Gaussian quadratures. The
Chebyshev-Gauss quadrature is given by
\cite{Abramowitz1965,MathworldChebyshevGauss}
\begin{equation}
\Int{ \frac{f\left( x \right)}{\sqrt{1 - x^2}} }{x,-1,1} \approx \sum_{i = 1}^n {w_i}f\left( {{x_i}} \right),
\label{eq:ChebyshevGauss}
\end{equation}
with $w_i = \frac{\pi}{n}$ and $x_i = \cos\left(\frac{2i-1}{2n}\pi\right)$.

Starting from the left side of \cref{eq:PVRAngNew}, we have
\begin{align}
\Int{ D\left( {\cos \varphi } \right) }{\varphi,0,\pi} &= \Int{ \frac{1}{{\sqrt {1 - {{\cos }^2}\varphi } }}\sqrt {1 - {{\cos }^2}\varphi } \: D\!\left( {\cos \varphi } \right) }{\varphi,0,\pi} \\
& = \Int{ \frac{1}{{\sqrt {1 - {{\cos }^2}\varphi } }} D\!\left( {\cos \varphi } \right)\sin \varphi \: }{\varphi,0,\pi}.
\label{eq:PVRAngNew2}
\end{align}

Using the substitution $x = \cos \varphi$ and $dx = -\sin \varphi \:d\varphi$
changes the limits to $x_{min} \!\!=\! \cos 0 \! =\! 1$ and
$x_{max} = \cos \pi = -1$. \Cref{eq:PVRAngNew2} becomes
\begin{equation}
\int_0^\pi  D\left( {\cos \varphi } \right){\textrm{d}}\varphi = - \int_1^{ - 1} \frac{1}{{\sqrt {1 - {x^2}} }}D\left( x \right){\textrm{d}}x = \int_{ - 1}^1 \frac{1}{{\sqrt {1 - {x^2}} }}D\left( x \right){\textrm{d}}x.
\label{eq:PVRAngNew3}
\end{equation}
This is the exact form needed for Gauss-Chebyshev quadrature (with f = D):
\begin{equation}
\int_0^\pi  D\left( {\cos \varphi } \right){\textrm{d}}\varphi \approx \frac{\pi }{n} \sum_{i = 1}^n D\!\left[ {\cos \left( {\frac{{2i - 1}}{{2n}}\pi } \right)} \right]
\label{eq:AngQuadrature}
\end{equation}

This proves only form \cref{eq:PVRAngNew}. To prove \cref{eq:PVRAng}, we split up the integration into two parts:
\begin{equation}
\int_0^{2\pi } D\left( {\cos {\varphi_{23}}} \right){\textrm{d}}{\varphi_{23}} = \int_0^\pi  D\left( {\cos {\varphi_{23}}} \right){\textrm{d}}{\varphi_{23}} + \int _\pi ^{2\pi } D\left( {\cos {\varphi _{23}}} \right){\textrm{d}}{\varphi _{23}}.
\label{eq:AngSplit}
\end{equation}
The first integration is just \cref{eq:PVRAngNew}. The only difference between the first and second integration is the limits. Defining $y = \varphi - \pi$ gives
\begin{equation}
\int _\pi ^{2\pi } D\left( {\cos {\varphi}} \right){\textrm{d}}{\varphi} = \int_0^\pi D\left[\cos(\varphi)\right] dy = \int_0^\pi D\left[\cos(y+\pi)\right] dy.
\end{equation}
If we also define $z = \cos \varphi = \cos(y+\pi)$ and $dz = \sin y \,dy$, we get an expression the same as \cref{eq:PVRAngNew3}:
\begin{equation}
\int_0^\pi D\left(\cos\varphi\right) dy = \int_0^\pi \frac{1}{\sqrt{1-z^2}} D(z)\, sin\varphi\, dy = \int_{-1}^1 \frac{1}{\sqrt{1-z^2}} D(z) dz
\end{equation}
Since this is the same as \cref{eq:PVRAngNew3}, we just combine this with \cref{eq:AngSplit,eq:AngQuadrature} to get \cref{eq:PVRAng}, proving the first form.



\subsection{Selection of Quadrature Points}
\label{sec:SelQuadPoints}

%\todoi{Show how compare to Peter's set - including information from the Mathematica notebook}

The number of quadrature points for each coordinate in the 6-dimensional 
integrations is critical to have fully converged results. In our testing, the 
$r_1$ coordinate (the coordinate of $e^+$) required more integration points 
than any other. The $r_2$ and $r_3$ coordinates required less points, and the 
interparticle terms ($r_{12}$, $r_{13}$ and $r_{23}$) required the least. 
These results only apply to the long-long and long-short terms, as the short-
short terms are integrated using the asymptotic expansion method (see
\cref{sec:MatrixShort}).

To determine this, we held the number of integration points fixed, except for 
one coordinate, which we increased in steps. The difference in the output 
between steps was used to analyze how important each coordinate was. Also, 
some terms are more sensitive to the number of integration points than other. 
For instance, the $(\bar{S}_\ell,\mathcal{L} \bar{S}_\ell)$ term converges 
relatively quickly, while the $(\bar{C}_\ell,\mathcal{L} \bar{C}_\ell)$ 
term requires more integration points.

We also used a sample input file from Van Reeth \cite{VanReethPrivate} for
the number of quadrature points he used in his code as a starting point.
This is shown in \cref{tab:BaseEffectiveCoords}.
%Using the comparison program described in \cref{sec:CompareProg}, we
Using a comparison program with the Developer's Image Library \cite{DevIL}, we
determined a set of quadrature points that yielded good convergence of
the matrix elements with a reasonable runtime. The final set of quadrature
points that we use for all partial waves is shown in
\cref{tab:OptimalEffectiveCoords}. This set yielded good convergence for
partial waves through the H-wave.

%We also used a sample input file from Van Reeth \cite{VanReethPrivate} for the number of quadrature 
%points he used in his code. In \cref{fig:OriginalPhaseShifts-pvr}, this 
%set of points was used. After approximately 525 short-range terms, it is 
%immediately apparent that the phase shifts diverge, with a particularly large 
%deviation in the inverse Kohn output. The Kohn and complex Kohn methods 
%converge again shortly before 600 terms and stay converged up to 1000 terms.
%
%Adding 5 points to all integrations yields a much more well-behaved graph in 
%\cref{fig:OriginalPhaseShifts-pvrplus5}. There are slight variations, 
%one of which can be seen in the inset graph at about 540 terms, but the 
%results are overall converged up to 1100 terms.
%
%Increasing the number of points can have a huge impact on performance. Adding 
%5 points to an $\omega = 0$ calculation takes 3.5 hours, while adding 10 
%points makes it take 6.1 hours, or approximately twice the time. So we have 
%to make a trade-off between accuracy and performance.
%
%To determine an optimal set of points, 10 points were added to Van Reeth's 
%set in all coordinates, and a run was done with this set. Then 5 points were 
%subtracted from one coordinate from this set for each run, once for each 
%coordinate. The difference in the matrix elements and phase shifts was small 
%for the 2nd, 3rd and 4th coordinates. A final run using 5 points extra from 
%Van Reeth's set for the 2nd, 3rd and 4th coordinates, along with 10 points 
%extra in all other coordinates, showed small differences from the first set 
%that had 10 points extra in all coordinates. Reducing the points of any of 
%the other coordinates resulted in larger differences from the set with 10 
%extra points in all coordinates. This final set, referred to as the ``optimal 
%set'' in \cref{tab:OptimalEffectiveCoords}, is used in all of the current
%runs. Increasing the number of 
%integration points further than could potentially lead to better accuracy, 
%though numerical instabilities start appearing with 15 extra points over the 
%base set.
%
%
%\begin{figure}
	%\centering
	%\includegraphics[width=\textwidth]{OriginalPhaseShifts-pvr}
	%\caption{Phase shifts with original ordering $(\omega = 7)$}
	%\label{fig:OriginalPhaseShifts-pvr}
%\end{figure}
%
%\begin{figure}
	%\centering
	%\includegraphics[width=\textwidth]{OriginalPhaseShifts-pvrplus5}
	%\caption{Phase shifts with original ordering and 5 extra points $(\omega = 7)$}
	%\label{fig:OriginalPhaseShifts-pvrplus5}
%\end{figure}
%
%

%\subsection{Comparison Program}
%\label{sec:CompareProg}
%\begin{figure}
	%\resizebox{1.0\textwidth}{!}{\includegraphics{QuadPoints/ColorKey.pdf}}
	%\caption{Color key for differences}
	%\label{fig:ColorKey}
%\end{figure}
%I wrote the program \emph{Comparison} \cite{GitHub} to visually compare two runs with different numbers of integration points. This program calculates the relative difference between similar matrix elements of the two input files. The relative difference is given by \cref{eq:PercentDiff}.
%Since the values of matrix elements can range over many orders of magnitude, the relative difference is an appropriate measure of the change caused by varying the number of integration points.
%%The other option would have been to use the absolute difference, which is given by 
%%\beq
%%\text{diff}_{abs} = \left| elem_1 - elem_2 \right| \times 100\%.
%%\eeq
%%However, this is a poor measure of the differences, and the relative differences give much more useful information for our purposes.
%
%The comparison program calculates these relative differences for each matrix element between the two input files and creates an image using the colors in \cref{fig:ColorKey} by using the Developer's Image Library \cite{DevIL}. So if a column of pixels in the output image is yellow, then the relative difference for that matrix element is on the order of $10^{-6}$. This is an example image from running this program:
%\begin{figure}
	%\centering
	%\resizebox{0.8\textwidth}{!}{\includegraphics{QuadPoints/Example.png}}
	%\caption{Example matrix element comparison image}
	%\label{fig:QuadExample}
%\end{figure}
%This only looks at the output from the long-range programs, and this is the first row (or column) of the $A$ matrix. The \emph{Comparison} program outputs a vertical line for each matrix element to make them more visible than outputting a single pixel. More examples and descriptions of the command line syntax are available on the \htmladdnormallink{Wiki}{http://129.120.30.66/wiki/index.php/Long-range_graphical_comparison_program} and GitHub \cite{Wiki,GitHub}.

%\subsection{Application to the P-Wave}
%
%For the P-wave, we noticed that the phase shifts for smaller $\kappa$ values were more sensitive to the number of integration points than higher $\kappa$ values. In general, the smaller the value of $\kappa$, the smaller the phase shift will be. The smaller phase shifts for the P-wave also caused it to be more sensitive than the S-wave.
%
%Using the comparison program, we found that the $1/r_{23}$ term in the potential needed more integration points when $q_i = 0$. The integrations for the other three potential terms already produced reasonably converged results. All runs in this section are performed with $\kappa = 0.1$.
%
%If we took the set of integration points that we used in the S-wave problem as the base set, then compared this with the same set but an extra 5 points in the $1^{st}$ coordinate, the resulting difference image for the $A$ matrix looks like this:
%\begin{figure}
	%\centering
	%\resizebox{1.0\textwidth}{!}{\includegraphics{QuadPoints/BasevsBaseplus5.png}}
	%\caption{Base set versus base set plus 5 in $r_1$}
	%\label{fig:BasevsBaseplus5}
%\end{figure}
%\noindent Notice that there is little blue and plenty of yellows, oranges and reds. The goal is to get as much black and blue as possible, though it is impossible to get entirely black and blue with limited precision. It is evident that the matrix elements are not well-converged when we compare the phase shifts from these two runs. For the first run with the base set, the phase shift is 0.02346176. The corresponding phase shift for the second run is 0.02294583. We desire three significant figures in the phase shifts, so this is certainly not good enough.
%
%Doing a similar run with the base set and another with the base set plus 10 in $r_1$, the difference image is given by \cref{fig:BasevsBaseplus10}.
%\begin{figure}
	%\centering
	%\resizebox{1.0\textwidth}{!}{\includegraphics{QuadPoints/BasevsBaseplus10.png}}
	%\caption{Base set versus base set plus 10 in $r_1$}
	%\label{fig:BasevsBaseplus10}
%\end{figure}
%\noindent This image is not much different from the previous image. Even when we compare the ``plus 5'' run with the ``plus 10'' run, the image does not change much. The phase shift for the ``plus 10'' run is 0.02257283, which is still significantly different from the ``plus 5'' run.
%
%We tried adding up to 25 extra points to the $1^{st}$ coordinate of the base set, but the matrix elements (and consequently, the phase shifts) still did not converge well enough. To perform the Gauss-Laguerre and Gauss-Legendre quadratures described in \cref{sec:GaussQuad}, we have to compute the abscissae and weights for the Laguerre and Legendre polynomials. The weights depend on the abscissae, and the abscissae depend only on $n$, the number of quadrature points to use. This allows us to hardcode values for the abscissae and weights to speed up computations marginally. Previously, the long-range code used hardcoded values for some multiples of 5 but not all that were used. This code also did not have values hardcoded when the value of $n$ was not a multiple of 5. The integration points were then changed to use values that were multiples of 5, and all were hardcoded. The hardcoded values were also done in extended precision by using \emph{Mathematica}.
%
%\Cref{fig:Base5vsBase5hardcoderound} shows the changes going from a previous ``plus 5'' run to a ``plus 5'' run with all hardcoded abscissae and weights. There is a relatively large difference in the matrix elements by making this change.
%\begin{figure}
	%\centering
	%\resizebox{1.0\textwidth}{!}{\includegraphics{QuadPoints/Base5vsBase5hardcoderound.png}}
	%\caption{Changes going to hardcoded abscissae and weights}
	%\label{fig:Base5vsBase5hardcoderound}
%\end{figure}
%
%The results were much better for the base set versus the ``plus 5'' set when hardcoded values for the abscissae and weight were used. \Cref{fig:BasehardcodevsBase5hardcode} shows this comparison, which has the large sections of black. All black and dark blue values are for terms where $q_i > 0$. This integration is done separately, as discussed in \cref{sec:Swaveqigt0} for the S-wave.
%\begin{figure}
	%\centering
	%\resizebox{1.0\textwidth}{!}{\includegraphics{QuadPoints/BasehardcodevsBase5hardcode.png}}
	%\caption[Base set with hardcoded values vs. base set plus 5 with hardcoded values]{Base set with hardcoded values versus base set plus 5 with hardcoded values}
	%\label{fig:BasehardcodevsBase5hardcode}
%\end{figure}
%
%We then completed a series of runs where we increased the number of integration points for each coordinate for $\omega = 2$. \Cref{fig:PlusCoord1-d} shows that the $1/r_{23}$ integration needs an extra 15 points in the first coordinate. The runs for the other coordinates were all performed with an extra 15 points in the first coordinate as well. From Figures \ref{fig:PlusCoord2} through \ref{fig:PlusCoord8}, we see that the $5^{th}$ through $8^{th}$ coordinates possibly need more integration points.
%
%\begin{figure}
%\centering
%\subfloat[Part 1][Base vs. plus 5]{\includegraphics[height=1.4in]{QuadPoints/R23-BasevsBaseplus5-1st.png} \label{fig:PlusCoord1-a}}
%\subfloat[Part 2][Plus 5 vs. plus 10]{\includegraphics[height=1.4in]{QuadPoints/R23-Baseplus5vsBaseplus10-1st.png} \label{fig:PlusCoord1-b}}
%\subfloat[Part 3][Plus 10 vs. plus 15]{\includegraphics[height=1.4in]{QuadPoints/R23-Baseplus10vsBaseplus15-1st.png} \label{fig:PlusCoord1-c}}
%\subfloat[Part 4][Plus 15 vs. plus 20]{\includegraphics[height=1.4in]{QuadPoints/R23-Baseplus15vsBaseplus20-1st.png} \label{fig:PlusCoord1-d}}
%\caption{Comparison of extra points in $1^{st}$ coordinate after hardcoded abscissae}
%\label{fig:PlusCoord1}
%\end{figure}
%
%\begin{figure}
%\centering
%\subfloat[Part 1][Plus 5 vs. plus 10]{\includegraphics[height=0.8in]{QuadPoints/r23-2nd-plus5vsplus10.png} \label{fig:PlusCoord2-a}}
%\subfloat[Part 2][Plus 10 vs. plus 15]{\includegraphics[height=0.8in]{QuadPoints/r23-2nd-plus5vsplus10.png} \label{fig:PlusCoord2-b}}
%\subfloat[Part 3][Plus 15 vs. plus 20]{\includegraphics[height=0.8in]{QuadPoints/r23-2nd-plus5vsplus10.png} \label{fig:PlusCoord2-c}}
%\caption{Comparison of extra points in $2^{nd}$ coordinate after hardcoded abscissae}
%\label{fig:PlusCoord2}
%\end{figure}
%
%\begin{figure}
%\centering
%\subfloat[Part 1][Plus 5 vs. plus 10]{\includegraphics[height=0.8in]{QuadPoints/r23-3rd-plus5vsplus10.png} \label{fig:PlusCoord3-a}}
%\subfloat[Part 2][Plus 10 vs. plus 15]{\includegraphics[height=0.8in]{QuadPoints/r23-3rd-plus5vsplus10.png} \label{fig:PlusCoord3-b}}
%\subfloat[Part 3][Plus 15 vs. plus 20]{\includegraphics[height=0.8in]{QuadPoints/r23-3rd-plus5vsplus10.png} \label{fig:PlusCoord3-c}}
%\caption{Comparison of extra points in $3^{rd}$ coordinate after hardcoded abscissae}
%\label{fig:PlusCoord3}
%\end{figure}
%
%\begin{figure}
%\centering
%\subfloat[Part 1][Plus 5 vs. plus 10]{\includegraphics[height=0.8in]{QuadPoints/r23-4th-plus5vsplus10.png} \label{fig:PlusCoord4-a}}
%\caption{Comparison of extra points in $4^{th}$ coordinate after hardcoded abscissae}
%\label{fig:PlusCoord4}
%\end{figure}
%
%\begin{figure}
%\centering
%\subfloat[Part 1][Plus 5 vs. plus 10]{\includegraphics[height=0.8in]{QuadPoints/r23-5th-plus5vsplus10.png} \label{fig:PlusCoord5-a}}
%\subfloat[Part 2][Plus 10 vs. plus 15]{\includegraphics[height=0.8in]{QuadPoints/r23-5th-plus5vsplus10.png} \label{fig:PlusCoord5-b}}
%\subfloat[Part 3][Plus 15 vs. plus 20]{\includegraphics[height=0.8in]{QuadPoints/r23-5th-plus5vsplus10.png} \label{fig:PlusCoord5-c}}
%\caption{Comparison of extra points in $5^{th}$ coordinate after hardcoded abscissae}
%\label{fig:PlusCoord5}
%\end{figure}
%
%\begin{figure}
%\centering
%\subfloat[Part 1][Plus 5 vs. plus 10]{\includegraphics[height=0.8in]{QuadPoints/r23-6th-plus5vsplus10.png} \label{fig:PlusCoord6-a}}
%\subfloat[Part 2][Plus 10 vs. plus 15]{\includegraphics[height=0.8in]{QuadPoints/r23-6th-plus5vsplus10.png} \label{fig:PlusCoord6-b}}
%\subfloat[Part 3][Plus 15 vs. plus 20]{\includegraphics[height=0.8in]{QuadPoints/r23-6th-plus5vsplus10.png} \label{fig:PlusCoord6-c}}
%\caption{Comparison of extra points in $6^{th}$ coordinate after hardcoded abscissae}
%\label{fig:PlusCoord6}
%\end{figure}
%
%\begin{figure}
%\centering
%\subfloat[Part 1][Plus 5 vs. plus 10]{\includegraphics[height=0.8in]{QuadPoints/r23-7th-plus5vsplus10.png} \label{fig:PlusCoord7-a}}
%\subfloat[Part 2][Plus 10 vs. plus 15]{\includegraphics[height=0.8in]{QuadPoints/r23-7th-plus5vsplus10.png} \label{fig:PlusCoord7-b}}
%\subfloat[Part 3][Plus 15 vs. plus 20]{\includegraphics[height=0.8in]{QuadPoints/r23-7th-plus5vsplus10.png} \label{fig:PlusCoord7-c}}
%\caption{Comparison of extra points in $7^{th}$ coordinate after hardcoded abscissae}
%\label{fig:PlusCoord7}
%\end{figure}
%
%\begin{figure}
%\centering
%\subfloat[Part 1][Plus 5 vs. plus 10]{\includegraphics[height=0.8in]{QuadPoints/r23-8th-plus5vsplus10.png} \label{fig:PlusCoord8-a}}
%\subfloat[Part 2][Plus 10 vs. plus 15]{\includegraphics[height=0.8in]{QuadPoints/r23-8th-plus5vsplus10.png} \label{fig:PlusCoord8-b}}
%\subfloat[Part 3][Plus 15 vs. plus 20]{\includegraphics[height=0.8in]{QuadPoints/r23-8th-plus5vsplus10.png} \label{fig:PlusCoord8-c}}
%\caption{Comparison of extra points in $8^{th}$ coordinate after hardcoded abscissae}
%\label{fig:PlusCoord8}
%\end{figure}
%
%These tests give us an idea of what coordinates need more integration points. The phase shifts, not the matrix elements, are what we are concerned with, so we ran a number of tests for the phase shifts for $\omega = 6$. \Cref{tab:5thcoordExtraPoints} shows how the phase shift stabilizes when the $5^{th}$ coordinate has an extra 10-15 points. For \cref{tab:678thcoordExtraPoints}, we have 15 extra points in the $1^{st}$ and $5^{th}$ coordinates, then add points to all of the $6^{th}$, $7^{th}$ and $8^{th}$ coordinates simultaneously. To the required precision (3 significant figures), the phase shift does not change with this last set of changes. From these runs, we finally determined that the $1^{st}$ and $5^{th}$ coordinates needed 15 extra integration points each, and we can save adding extra points to the other coordinates.
%
%\begin{table}
%\centering
%\begin{tabular}{c c}
%\toprule
%Extra points & Phase shift \\
%\midrule
%+5 & 0.02273585 \\
%+10 & 0.02256803 \\
%+15 & 0.02257100 \\
%+20 & 0.02258546 \\
%+25 & 0.02257720 \\
%+30 & 0.02257869 \\
%+35 & 0.02257844 \\
%\bottomrule
%\end{tabular}
%\caption{$5^{th}$ Coordinate Comparisons}
%\label{tab:5thcoordExtraPoints}
%\end{table}
%
%
%\begin{table}
%\centering
%\begin{tabular}{c c}
%\toprule
%Extra points & Phase shift \\
%\midrule
%+5 & 0.02255816 \\
%+10 & 0.02255351 \\
%+15 & 0.02255181 \\
%\bottomrule
%\end{tabular}
%\caption{Extra points in the $6^{th}$, $7^{th}$ and $8^{th}$ coordinates}
%\label{tab:678thcoordExtraPoints}
%\end{table}


\subsection{Quadrature Points}
\label{sec:QuadraturePoints}

%\todoi{Put in information from ``Integration Point Numbers.nb''}

Describing the number of points used in integrating the different coordinates 
in \cref{sec:LongLongInt,sec:ShortLongInt} can be confusing, so we have taken 
to grouping the sets of points as in
\Cref{tab:EffectiveCoords1,tab:EffectiveCoords2}. Each column of these tables 
is referred to as an ``effective coordinate'', as the $r_2$ and $r_3$ 
integrations are split into two parts, as described in \cref{sec:Cusps}.
The ``Lag'' entries use the Gauss-Laguerre quadrature, ``Leg'' uses the
Gauss-Legendre quadrature, and ``Che'' uses Chebyshev-Gauss quadrature. Note
that the only difference between \cref{tab:EffectiveCoords1,tab:EffectiveCoords2}
is that the S-wave long-long matrix elements are integrated using perimetric
coordinates when the $r_{23}^{-1}$ is not present, while the other partial
waves use the same coordinates as the long-short terms.

\begin{table}
\centering
\footnotesize
\begin{tabular}{c c c c c c c c c}
\toprule
 & Coord & Coord & Coord & Coord & Coord & Coord & Coord & Coord \\
Integral & 1 & 2 & 3 & 4 & 5 & 6 & 7 & 8 \\
\midrule
 Long-long, no $r_{23}^{-1}$ & $x$ Lag & $y$ Lag & $z$ Lag & $r_3$ Leg & $r_3$ Leg & $r_{13}$ Leg & & \\
 Long-long, $r_{23}^{-1}$ & $r_1$ Lag & $r_2$ Leg & $r_2$ Lag & $r_3$ Leg & $r_3$ Lag & $\varphi_{12}$ Che & $r_{13}$ Leg & $r_{23}$ Leg \\
\midrule
 Long-short $q_i = 0$, no $r_{23}^{-1}$ & $r_1$ Lag & $r_2$ Leg & $r_2$ Lag & $r_3$ Leg & $r_3$ Lag & $r_{12}$ Leg & $r_{13}$ Leg & \\
 Long-short $q_i = 0$, $r_{23}^{-1}$ & $r_1$ Lag & $r_2$ Leg & $r_2$ Lag & $r_3$ Leg & $r_3$ Lag & $r_{12}$ Leg & $\varphi_{13}$ Che & $r_{23}$ Leg \\
 Long-short $q_i > 0$ & $r_1$ Lag & $r_2$ Leg & $r_2$ Lag & $r_3$ Leg & $r_3$ Lag & $r_{12}$ Leg & $r_{13}$ Leg & $\varphi_{23}$ Che \\
\bottomrule
\end{tabular}
\caption{Description of the values in \cref{tab:BaseEffectiveCoords,tab:OptimalEffectiveCoords} for the S-wave. Refer to the text for explanation of this table.}
\label{tab:EffectiveCoords1}
\end{table}

\begin{table}
\centering
\footnotesize
\begin{tabular}{c c c c c c c c c}
\toprule
 & Coord & Coord & Coord & Coord & Coord & Coord & Coord & Coord \\
Integral & 1 & 2 & 3 & 4 & 5 & 6 & 7 & 8 \\
\midrule
 Long-long, no $r_{23}^{-1}$ & $r_1$ Lag & $r_2$ Leg & $r_2$ Lag & $r_3$ Leg & $r_3$ Lag & $r_{12}$ Leg & $r_{13}$ Leg & \\
 Long-long, $r_{23}^{-1}$ & $r_1$ Lag & $r_2$ Leg & $r_2$ Lag & $r_3$ Leg & $r_3$ Lag & $\varphi_{12}$ Che & $r_{13}$ Leg & $r_{23}$ Leg \\
\midrule
 Long-short $q_i = 0$, no $r_{23}^{-1}$ & $r_1$ Lag & $r_2$ Leg & $r_2$ Lag & $r_3$ Leg & $r_3$ Lag & $r_{12}$ Leg & $r_{13}$ Leg & \\
 Long-short $q_i = 0$, $r_{23}^{-1}$ & $r_1$ Lag & $r_2$ Leg & $r_2$ Lag & $r_3$ Leg & $r_3$ Lag & $r_{12}$ Leg & $\varphi_{13}$ Che & $r_{23}$ Leg \\
 Long-short $q_i > 0$ & $r_1$ Lag & $r_2$ Leg & $r_2$ Lag & $r_3$ Leg & $r_3$ Lag & $r_{12}$ Leg & $r_{13}$ Leg & $\varphi_{23}$ Che \\
\bottomrule
\end{tabular}
\caption{Description of the values in \cref{tab:BaseEffectiveCoords,tab:OptimalEffectiveCoords} for $\ell > 0$. Refer to the text for explanation of this table.}
\label{tab:EffectiveCoords2}
\end{table}

\Cref{tab:BaseEffectiveCoords} shows the set of integration points used by Van
Reeth \cite{VanReethPrivate}. This was the starting point for trying to obtain
better convergence of the matrix element integrations, as described in
\cref{sec:SelQuadPoints}. Using a comparison program to investigate this
convergence, we determined a more extensive set of integration points given
in \cref{tab:OptimalEffectiveCoords} that we use for all partial waves.
%Interestingly, for the S-wave, this did
%not yield appreciably different phase shifts from those given in
%Ref.~\cite{VanReeth2003}, as shown in \cref{sec:SWavePhase}.

\begin{table}
\centering
\footnotesize
\begin{tabular}{c c c c c c c c c}
\toprule
 & Coord & Coord & Coord & Coord & Coord & Coord & Coord & Coord\\
Integral & 1 & 2 & 3 & 4 & 5 & 6 & 7 & 8 \\
\midrule
 Long-long, no $r_{23}^{-1}$ & 45 & 35 & 35 & 35 & 28 & 15 & & \\
 Long-long, $r_{23}^{-1}$ & 65 & 35 & 28 & 35 & 28 & 12 & 15 & 15 \\
\midrule
 Long-short $q_i = 0$, no $r_{23}^{-1}$ & 90 & 57 & 34 & 57 & 34 & 30 & 30 & \\
 Long-short $q_i = 0$, $r_{23}^{-1}$ & 90 & 58 & 30 & 55 & 35 & 33 & 33 & 33 \\
 Long-short $q_i > 0$ & 90 & 57 & 34 & 57 & 34 & 30 & 30 & 30 \\
\bottomrule
\end{tabular}
\caption{Base set of effective coordinates for integrations from \cref{VanReethPrivate}}
\label{tab:BaseEffectiveCoords}
\end{table}

\begin{table}
\centering
\footnotesize
\begin{tabular}{c c c c c c c c c}
\toprule
 & Coord & Coord & Coord & Coord & Coord & Coord & Coord & Coord\\
Integral & 1 & 2 & 3 & 4 & 5 & 6 & 7 & 8 \\
\midrule
 Long-long, no $r_{23}^{-1}$			&  75 & 40 & 40 & 40 & 40 & 25 & & \\
 Long-long, $r_{23}^{-1}$				&  75 & 40 & 40 & 40 & 40 & 25 & 25 & 25 \\
\midrule
 Long-short $q_i = 0$, no $r_{23}^{-1}$	& 100 & 65 & 45 & 65 & 45 & 45 & 45 & \\
 Long-short $q_i = 0$, $r_{23}^{-1}$	& 115 & 65 & 45 & 65 & 45 & 45 & 45 & 45 \\
 Long-short $q_i > 0$					& 100 & 65 & 45 & 65 & 45 & 45 & 45 & 45 \\
\bottomrule
\end{tabular}
\caption{Final set of effective coordinates for integrations}
\label{tab:OptimalEffectiveCoords}
\end{table}



%\section{Extra Schwartz Singularity Discussion}
%\label{sec:ExtraSchwartz}
%
%Interestingly, adding terms will change which Kohn method has a Schwartz 
%singularity. \Cref{fig:PhaseTau-Omega} shows how increasing $\omega$ (adding 
%terms) will change the $\tau$ in the generalized Kohn where a Schwartz 
%singularity occurs. \Cref{fig:PhaseTau-Kappa} also shows how for the 
%same set of short-range terms, varying $\kappa$ will change the $\tau$ value 
%where these singularities occur.
%
%\begin{figure}
	%\centering
	%\includegraphics[width=\textwidth]{PhaseTau-Omega}
	%\caption{Phase shifts versus $\tau$ for $\kappa = 0.1$ and varying $\omega$ values}
	%\label{fig:PhaseTau-Omega}
%\end{figure}
%
%\begin{figure}
	%\centering
	%\includegraphics[width=\textwidth]{PhaseTau-Kappa}
	%\caption{Phase shifts versus $\tau$ for $\omega = 7$ and varying $\kappa$ values}
	%\label{fig:PhaseTau-Kappa}
%\end{figure}
%
%\begin{figure}
	%\centering
	%\includegraphics[width=4in]{Det-Omega=6,Kappa=01}
	%\caption[Determinant of $\textbf{\emph{A}}$ versus $\tau$]{Determinant of $\textbf{\emph{A}}$ versus $\tau$ for $\omega = 6$ with 875 terms of Todd reordering}
	%\label{fig:Det-Omega=6}
%\end{figure}
%
%Cooper et al. \cite{Cooper2009} showed that if $\Det{\textbf{\emph{A}}}$ is 
%plotted with respect to $\tau$, there are actually two $\tau$ values where
%$\Det{\textbf{\emph{A}}} \approx 0$, but only one of them corresponds to a 
%Schwartz singularity. They refer to the second root as an anomaly-free 
%singularity. \Cref{fig:Det-Omega=6} shows an equivalent plot for our 
%data. The roots of $\Det{\textbf{\emph{A}}}$ are at approximately
%$\tau = 0.125$ and $\tau = 1.144$. The first root corresponds well with the
%singularity in the $\omega = 6$ graph in \cref{fig:PhaseTau-Omega}. The second
%root is referred to as an ``anomaly-free singularity'' in
%Cooper et al.\ \cite{Cooper2009} and does not correspond to a Schwartz
%singularity. The roots are easily found by fitting the data to equation (20)
%and solving equation (21) in Cooper et al.\ \cite{Cooper2009}.


\section{Resonance Fitting}
\label{sec:ResonanceFit}

To find the resonance parameters (positions and widths), the phase shifts for 
multiple energy values are fitted to \cref{eq:ResonanceCurve}. There have 
been multiple programs described in the literature
\cite{Tennyson1984, Stibbe1998, Sochi2013} to do these fittings. %, which I
%was not aware of when I 
%originally wrote code to do these.
Some of the difficulty with this type of 
fitting is choosing appropriate guesses for the resonance parameters.
Ref.~\cite{Sochi2013} describes methods that some groups use to try to identify 
resonance parameters.

%This nonlinear fitting can be difficult for general programs. I first tried 
%the \texttt{Fit} and \texttt{FindFit} functions of
%Mathematica\textsuperscript{\textregistered} 8.0 \cite{Mathematica}, but these were unable to fit our 
%data to this complicated nonlinear function. I next tried the
%\texttt{curve\_fit} function of SciPy \cite{SciPy} in Python\textsuperscript{\textregistered}
%2.6 \cite{Python}, which was better able to fit the nonresonant (polynomial) 
%part but had trouble with the $\arctan$ terms.

With the help of Bosca \cite{BoscaPrivate}, I wrote a MATLAB\textsuperscript{\textregistered}
\cite{matlab} script that uses the \texttt{nlinfit} routine of MATLAB. 
\texttt{nlinfit} is specifically designed for fitting to nonlinear functions, 
and its robust option allows for a variety of weighting functions to be used. 
This script does the fitting for all eight of the possible weightings: 
Bisquare, Andrews, Cauchy, Fair, Huber, Logistic, Talwar and Welsch. This 
fitting routine is also not as sensitive to the initial guesses as the 
Mathematica and SciPy routines.

Later on, I adapted this resonance fitting code to be called from IPython 
\cite{ipython} using the mlabwrap \cite{mlabwrap} Python to MATLAB wrapper. 
This allows fits and graphs of the fittings to be in the same IPython 
notebook, including the flexibility of querying a MySQL database for the 
phase shift data for any partial wave at any number of terms and for
any Kohn-type variational method. The mlabwrap package is difficult to install properly, 
requiring a compilation against MATLAB. The mlabwrap-purepy package
\cite{mlabwrappurepy} has been created to simplify this, but I have not tried it 
yet.

The results of fitting the phase shifts from the $S$ matrix are shown in
\cref{fig:swave-resonance-uncorrected-fits1,fig:swave-resonance-uncorrected-fits2} 
for each of the weighting functions, and the resonance parameters are given 
in each subfigure. There is good agreement between the fits performed with 
each of the weighting functions, and the Fair is the furthest from the 
others. In our testing, the Cauchy is consistently one of the best choices 
for fitting. In addition, \cref{fig:swave-resonance-uncorrected-res} has 
plots of the residuals (the absolute value of the difference between the 
fitted curve and the actual phase shifts). Each graph also has the residual 
sum of squares (RSS) calculated. The RSS gives an easy way to compare the 
performance of the weightings with each other. The Fair has a notably larger 
RSS than all of the other fittings.

\begin{figure}
	\centering
	\includegraphics[width=\textwidth]{swave-resonance-uncorrected-fits1}
	\caption{First set of uncorrected resonance fitting graphs for $^1$S at $\omega = 7$}
	\label{fig:swave-resonance-uncorrected-fits1}
\end{figure}

\begin{figure}
	\centering
	\includegraphics[width=\textwidth]{swave-resonance-uncorrected-fits2}
	\caption{Second set of uncorrected resonance fitting graphs for $^1$S at $\omega = 7$}
	\label{fig:swave-resonance-uncorrected-fits2}
\end{figure}

\begin{figure}
	\centering
	\includegraphics[width=\textwidth]{swave-resonance-uncorrected-res}
	\caption{Residuals for uncorrected resonance fitting graphs for $^1$S at $\omega = 7$}
	\label{fig:swave-resonance-uncorrected-res}
\end{figure}

One important thing to notice with the fits in
\cref{fig:swave-resonance-uncorrected-fits1,fig:swave-resonance-uncorrected-fits2} is that the
$S$-matrix complex Kohn phase shifts extend above the fitted curve at the resonances before 
matching up again on the right side of the resonance below $-3.0$. The red 
curves still fit to the data well, but the fits can be improved by correcting 
for these. 

In \cref{eq:ResonanceCurve}, the first three polynomial terms form the 
background, and the two $\arctan$ terms represent the resonances. The range 
of $\arctan$ is $(-\frac{\pi}{2},\frac{\pi}{2})$, so the $\arctan$ parts of 
this model cannot bring the phase shift from the background (near 2.0) all 
the way to 0.0, as it can only add up to $\frac{\pi}{2}$ to the background. 
It is important to realize that the Kohn-type variational methods will only return phase 
shifts in a certain range. From \cref{eq:GenKohnL}, we are not finding the 
phase shifts directly but are rather calculating $\tan \delta_\ell$. The 
phase shifts found this way sometimes have to have $\pi$ added or subtracted 
if they are outside this range.

\begin{figure}
	\centering
	\includegraphics[width=5in]{swave-phases-reson-pi}
	\caption[$^1$S resonance data showing correction]{$^1$S resonance data showing raw data from $S$-matrix complex Kohn and the corrected version. The solid gray line shows the polynomial background, and the vertical dashed lines give the calculated resonance positions.}
	\label{fig:swave-phases-reson-pi}
\end{figure}

\Cref{fig:swave-phases-reson-pi} shows the results of subtracting $\pi$ from 
phase shifts that are more than $\frac{\pi}{2}$ above the background. The 
original and the corrected data are both shown on this plot, and the 
background is given as a gray line. Note that the slope of the original data 
changes when crossing over the vertical dashed lines, which gives the 
resonance positions. When corrected, these upper points are moved down to 
their appropriate place, shown as x's on the graph. These match up better 
with the fitting curve. This fitting is an iterative process, because the 
background polynomial has to be determined with \cref{eq:ResonanceCurve} 
before we can do this correction. Then the phase shifts are fitted again 
after performing the correction.

\begin{figure}
	\centering
	\includegraphics[width=\textwidth]{swave-resonance-corrected-fits1}
	\caption{First set of corrected resonance fitting graphs for $^1$S at $\omega = 7$}
	\label{fig:swave-resonance-corrected-fits1}
\end{figure}

\begin{figure}
	\centering
	\includegraphics[width=\textwidth]{swave-resonance-corrected-fits2}
	\caption{Second set of corrected resonance fitting graphs for $^1$S at $\omega = 7$}
	\label{fig:swave-resonance-corrected-fits2}
\end{figure}

\begin{figure}
	\centering
	\includegraphics[width=\textwidth]{swave-resonance-corrected-res}
	\caption{Residuals for corrected resonance fitting graphs for $^1$S at $\omega = 7$}
	\label{fig:swave-resonance-corrected-res}
\end{figure}

Finally, this corrected and re-fitted data is shown in
\cref{fig:swave-resonance-corrected-fits1,fig:swave-resonance-corrected-fits2}. The different 
weighting methods agree extremely well now, with the only notable differences 
being in the $^2\Gamma$ width of the second resonance for the Bisquare, 
Andrews, and Talwar fits. This is still a very small difference, and the 
other resonance parameters are almost all identical.
\Cref{fig:swave-resonance-corrected-res} gives the residuals and RSS for the corrected phase
shifts, similar to \cref{fig:swave-resonance-uncorrected-res}. The Bisquare, Andrews, 
and Talwar weightings have larger RSS values than the other weightings, 
indicating that their fits are not quite as accurate, but they are still 
relatively accurate.

The differences between the different weighting methods gives us one way to 
determine errors for the resonance parameters. Additionally, comparing the 
different Kohn-type variational methods methods gives us an idea of the error. The real-valued 
generalized Kohn variational methods give more disagreement, so we compute the resonance 
parameters for each of these. We also have to take into account 
Schwartz singularities. In \cref{fig:schwartz-singularity}(a) on
\pageref{fig:schwartz-singularity}, the second resonance parameters are not as 
accurate. The fitting routine described here chooses fitting parameters of
$^2E_R = \SI{5.0295}{eV}$ and $^2\Gamma = \SI{0.06011}{eV}$.
In \cref{fig:schwartz-singularity}(b), where there are no Schwartz 
singularities, $^2E_R = \SI{5.0278}{eV}$ and $^2\Gamma = \SI{0.06075}{eV}$. 
This highlights the importance of using multiple Kohn-type variational methods and trying to 
detect Schwartz singularities. After removing obvious Schwartz singularities, 
the mean value of the different Kohn-type variational methods for all weightings is taken for 
each resonance parameter, and these are the results listed for each of the 
partial waves in
\cref{sec:SWaveResonances,sec:PWaveResonances,sec:DWaveResonance,sec:FWaveResonance}.
The errors given in these tables are simply the standard deviation with all 
generalized Kohn variational methods and weightings used for each resonance parameter. For this, we 
do not use the generalized $S$-matrix or $T$-matrix complex Kohn methods,
because using these would decrease the error, as they agree to a significant
precision.



\biblio
\end{document}
