\documentclass[Dissertation.tex]{subfiles} 
\begin{document}

\newpage
\chapter{Trial Wavefunction and Kohn Method Application}

\section{Wavefunction}

Our trial wavefunction for scattering of positronium from hydrogen is
\begin{equation}
\Psi_t^\pm = \bar{S} + \tan \eta_t \, \bar{C} + \sum_{i=1}^N c_i \bar{\phi}_i^t
\label{eq:TrialSimple}
\end{equation}

\noindent where

\begin{subequations}\label{SCphiBarDef}
\begin{align}
\bar{S} &= \frac{\left( 1 \pm P_{23} \right) S}{\sqrt{2}} \label{SBarDef} \\
\bar{C} &= \frac{\left( 1 \pm P_{23} \right) C}{\sqrt{2}} \label{CBarDef} \\
\bar{\phi}_i^t &= \left( 1 \pm P_{23} \right) \phi_i \label{PhiBarDef}
\end{align}
\end{subequations}

\noindent and

\begin{subequations}\label{eq:SCPhiDef}
\begin{align}
S &= Y_{0,0}\left( \theta_\rho, \phi_\rho \right) \Phi_{Ps}\left(r_{12}\right) \Phi_H\left(r_3\right) \sqrt{2\kappa} \,j_0\!\left(\kappa\rho\right) \label{eq:SDef} \\
C &= -Y_{0,0}\left( \theta_\rho, \phi_\rho \right) \Phi_{Ps}\left(r_{12}\right) \Phi_H\left(r_3\right) \sqrt{2\kappa} \,n_0\!\left(\kappa\rho\right) \left[1 - \exp(-\mu \rho) (1+\frac{\mu}{2}\rho)\right] \label{eq:CDef} \\
\phi_i &= e^{-\left(\alpha r_1 + \beta r_2 + \gamma r_3 \right)} r_1^{k_i} r_2^{l_i} r_{12}^{m_i} r_3^{n_i} r_{13}^{p_i} r_{23}^{q_i}. \label{eq:PhiDef}
\end{align}
\end{subequations}

\noindent The $\frac{1}{\sqrt{2}}$ and $Y_{0,0}\left( \theta_\rho, \phi_\rho \right) = \frac{1}{\sqrt{4\pi}}$ for $\bar{\phi}_i^t$ are included in the $c_i$ coefficients in (\ref{eq:TrialSimple}).

$\rho$ and $\rho'$ are defined as
\begin{subequations}
\begin{align}
\vec{\rho} &= \frac{1}{2}\left(\vec{r_1} + \vec{r_2}\right) \label{eq:RhoDef}\\
\vec{\rho}^\prime &= \frac{1}{2}\left(\vec{r_1} + \vec{r_3}\right) \label{eq:RhopDef}.
\end{align}
\end{subequations}

\noindent As a shortcut, we can also write
\begin{equation}
S^\prime = P_{23} S \text{ and } C^\prime = P_{23} C.
\label{eq:SCprime}
\end{equation}

The Hamiltonian for the fundamental Coulombic system is
\begin{align}
H = -\frac{1}{2} \nabla_{r_1}^2 - \frac{1}{2} \nabla_{r_2}^2 - \frac{1}{2} \nabla_{r_3}^2 + \frac {1}{r_1}-\frac {1}{r_2}-\frac {1}{r_3}-\frac {1}{r_{12}}-\frac {1}{r_{13}}+\frac {1}{r_{23}}	\label{Hamiltonian1}
\end{align}

\noindent The Hamiltonian can also be expressed in terms of other variables as
\begin{align}
H = -\frac{1}{4} \nabla_{\rho}^2 - \frac{1}{2} \nabla_{r_3}^2 - \nabla_{r_{12}}^2 + \frac {1}{r_1}-\frac {1}{r_2}-\frac {1}{r_3}-\frac {1}{r_{12}}-\frac {1}{r_{13}}+\frac {1}{r_{23}}
	\label{Hamiltonian2}
\end{align}

\begin{align}
H = -\frac{1}{4} \nabla_{\rho^\prime}^2 - \frac{1}{2} \nabla_{r_2}^2 - \nabla_{r_{13}}^2 + \frac {1}{r_1}-\frac {1}{r_2}-\frac {1}{r_3}-\frac {1}{r_{12}}-\frac {1}{r_{13}}+\frac {1}{r_{23}}
	\label{Hamiltonian3}
\end{align}


\section{Using the Kohn Method}

We use the Kohn method (\ref{KMatrixKohn}) with our trial function (\ref{eq:TrialSimple}) to get
\beq
\lambda_v = \tan\delta_l^v = \tan\delta_l^t - (\psi_t,L \psi_t) = \lambda_t - \Big(\bar{S} + \lambda_t \bar{C} + \sum_i c_i \bar{\phi}_i, L (\bar{S} + \lambda_t \bar{C} + \sum_j c_j \bar{\phi}_j )\Big).
\eeq

The property of the Kohn functional that it is stationary with respect to variations in the linear parameters (\ref{Partials1}) can be written specifically in our case as
\beq
\frac{\partial \lambda_v}{\partial \lambda_t} = 0  \text{ and } \frac{\partial \lambda_v}{\partial c_i} = 0 \text{ where $i = 1,\ldots,N$}.
\label{eq:KohnStationary}
\eeq

Performing the first variation gives
\beq
0 = \frac{\partial \lambda_v}{\partial \lambda_t} = 1 - \Big[(\bar{S},L\bar{C}) + (\bar{C},L\bar{S}) + \frac{\partial}{\partial \lambda_t}(\lambda_t \bar{C},L \lambda_t \bar{C}) + (\bar{C}, L \sum_i c_i \bar{\phi}_i) + (\sum_i c_i \bar{\phi}_i, L \bar{C}) \Big].
\label{eq:PdLambda1}
\eeq

\noindent The third term in brackets becomes
\beq
\frac{\partial}{\partial \lambda_t} (\lambda_t \bar{C},L \lambda_t \bar{C}) = (\bar{C},L \bar{C}) \frac{\partial}{\partial \lambda_t} \lambda_t^2 = 2(\bar{C},L\bar{C}) \lambda_t.
\eeq

\noindent The last two terms of (\ref{eq:PdLambda1}) are equal to each other, and we can use (\ref{eq:SLCandCLSBar}) to rewrite this.
\beq
0 = -(\bar{C},L\bar{S}) - (\bar{C},L\bar{S}) - 2 \lambda_t (\bar{C},L\bar{C}) - 2 \sum_i c_i (\bar{C},L\bar{\phi}_i)
\eeq

\noindent Rearranging gives
\beq
-(\bar{C},L\bar{S}) = \lambda_t (\bar{C},L\bar{C}) + \sum_i c_i (\bar{C},L\bar{\phi}_i)
\label{eq:PdLambda}
\eeq

Now we perform the variation with respect to a general $c_k$ as in (\ref{eq:KohnStationary}).
\beq
0 = \frac{\partial \lambda_v}{\partial c_k} = -\Big[ (\bar{S},L \bar{\phi}_k) + \lambda_t (\bar{C},L \bar{\phi}_k) + (\bar{\phi}_k,L \bar{S}) + \lambda_t (\bar{\phi}_k,L \bar{C}) + \frac{\partial}{\partial c_k} (\sum_i c_i \bar{\phi}_i, L \sum_j c_j \bar{\phi}_j) \Big]
\label{eq:PdCk1}
\eeq

If $c_i \ne c_j$,
\begin{subequations}
\begin{align}
\frac{\partial}{\partial c_i} (c_i \bar{\phi}_i, L \sum_{j \ne i} c_j \bar{\phi}_j) &= \sum_{j \ne i} c_j (\bar{\phi}_i, L \bar{\phi}_j) \text{ and} \\
\frac{\partial}{\partial c_j} (\sum_{i \ne j} c_i \bar{\phi}_i, L c_j \bar{\phi}_j) &= \sum_{i \ne j} c_i (\bar{\phi}_i, L \bar{\phi}_j).
\end{align}
\end{subequations}

\noindent These two equations are equivalent, since $\left( \bar{\phi}_i, L \bar{\phi}_j \right) = \left( \bar{\phi}_j, L \bar{\phi}_i \right)$ by (\ref{PhiLPhiPerm}).

If $c_i = c_j$,
\beq
\frac{\partial}{\partial c_i} \left( c_i \bar{\phi}_i, L c_j \bar{\phi}_j \right) = \frac{\partial}{\partial c_i} \left( c_i \bar{\phi}_i, L c_i \bar{\phi}_i \right) = \frac{\partial}{\partial c_i} c_i^2 \left( \bar{\phi}_i, L \bar{\phi}_j \right) = 2 \, c_i \left( \bar{\phi}_i, L \bar{\phi}_j \right).
\eeq

\noindent We can also use (\ref{eq:PhiLSPerm}) and (\ref{eq:PhiLCPerm}) to reduce (\ref{eq:PdCk1}) to
\beq
0 = -\Big[ 2 (\bar{\phi}_k, L \bar{S}) + 2 \lambda_t (\bar{\phi}_k, L \bar{C}) + 2 \sum_i (\bar{\phi}_k, L c_i \bar{\phi}_i) \Big].
\eeq

\noindent
Rearranging gives
\beq
-\left( \bar{\phi}_k, L \bar{S} \right) = \lambda_t \left( \bar{\phi}_k, L \bar{C} \right) + \sum_i \left( \bar{\phi}_k, L c_i \bar{\phi}_i \right).
\label{eq:PdCk}
\eeq

The set of linear equations in (\ref{eq:PdLambda}) and (\ref{PdCk}) can be written in matrix form as
\begin{equation}
\label{eq:KohnMatrix}
\begin{bmatrix} 
 (\bar{C},L\bar{C}) & (\bar{C},L\bar{\phi}_1) & \cdots & (\bar{C},L\bar{\phi}_j) & \cdots\\
 (\bar{\phi}_1,L\bar{C}) & (\bar{\phi}_1,L\bar{\phi}_1) & \cdots & (\bar{\phi}_1,L\bar{\phi}_j) & \cdots\\
 \vdots & \vdots & \ddots & \vdots \\
 (\bar{\phi}_i,L\bar{C}) & (\bar{\phi}_i,L\bar{\phi}_1) & \cdots & (\bar{\phi}_i,L\bar{\phi}_j) & \cdots\\
 \vdots & \vdots & & \vdots & \\
\end{bmatrix}
\begin{bmatrix}
\Lambda_t\\
c_1\\
\vdots\\
c_i\\
\vdots
\end{bmatrix}
= -
\begin{bmatrix}
(\bar{C},L\bar{S}) \\
(\bar{\phi}_1,L\bar{S}) \\
\vdots \\
(\bar{\phi}_i,L\bar{S}) \\
\vdots
\end{bmatrix}.
\end{equation}

\noindent This matrix equation can be rewritten as
\beq
\textbf{\emph{AX = -B}}.
\eeq

\noindent Solving this for $\textbf{\emph{X}}$,
\beq
\textbf{\emph{X = $-A^{-1}$B}}.
\eeq

\section{Integrating the Short-Range--Short-Range Terms}
\label{sec:SWaveShortShort}

\section{Integrating the Remaining Matrix Matrix Elements}
We start with the functional
\begin{align}
	F \equiv \left({g,{Lf}}\right)-\left({f,{Lg}}\right).
\end{align}

Only the first three terms of the above functional have to be evaluated, as the other terms go to 0 with the subtraction.
\begin{align}
	F=\left({-g,{\frac {1}{2}{\nabla }_{{{r}_{{1}}}}^{{2}}f}}\right)+\left({f,{\frac {1}{2}{\nabla }_{{{r}_{{1}}}}^{{2}}g}}\right)+
	\left({-g,{\frac {1}{2}{\nabla }_{{{r}_{{2}}}}^{{2}}f}}\right)+\left({f,{\frac {1}{2}{\nabla }_{{{r}_{{2}}}}^{{2}}g}}\right)+
	\left({-g,{\frac {1}{2}{\nabla }_{{{r}_{{3}}}}^{{2}}f}}\right)+\left({f,{\frac {1}{2}{\nabla }_{{{r}_{{3}}}}^{{2}}g}}\right)
\end{align}

We can use Green's theorem on each pair of terms:

\begin{align}
	{\int _{{V}_{{2}}}{{\int _{{V}_{{1}}}{\left[{u{\nabla }_{{1}}^{{2}}v-v{\nabla }_{{1}}^{{2}}u}\right]{d{{\tau }_{{1}}}}}}{d{{\tau }_{{2}}}}}}
	= \int_{V_2} \int_{S_1} \left[u \vec{\nabla}_1 v - v \vec{\nabla}_1 u \right] \cdot d\vec{\sigma}_1 d\tau_2
\end{align}

\begin{subequations}\label{elem_symm}
\begin{align}
\left({{{\overline\phi }}_{{i}},{L{{\overline\phi }}_{{j}}}}\right) &= \left({{{\overline\phi }}_{{j}},{L{{\overline\phi }}_{{i}}}}\right) \\
\left({{{\overline\phi }}_{{i}},{L\overline{S}}}\right) &= \left({\overline{S},{L{{\overline\phi }}_{{i}}}}\right) \\
\left({{{\overline\phi }}_{{i}},{L\overline{C}}}\right) &= \left({\overline{C},{L{{\overline\phi }}_{{i}}}}\right)
\end{align}
\end{subequations}

We just need to consider the long-range--long-range matrix elements.  Examining the functional
 $F = \left(\overline{S}, L\overline{C}\right) - \left(\overline{C}, L\overline{S}\right)$:
\begin{equation*}
F = \left(\frac{1}{\sqrt{2}} \left[S \pm S^\prime \right], L \frac{1}{\sqrt{2}}\left[C \pm C^\prime \right] \right) -
    \left(\frac{1}{\sqrt{2}} \left[C \pm C^\prime \right], L \frac{1}{\sqrt{2}}\left[S \pm S^\prime \right] \right)
\end{equation*}
\begin{equation}
= \frac{1}{2}\left[(S,LC) \pm (S,LC^\prime) \pm (S^\prime,LC) + (S^\prime,LC^\prime) - (C,LS) \mp (C,LS^\prime) \mp (C^\prime,LS) - (C^\prime,LS^\prime)\right]
\end{equation}

From (\ref{}), $(S,LC) = (S^\prime,LC^\prime)$, $(S^\prime,LC) = (S,LC^\prime)$, $(C,LS) = (S^\prime,LC^\prime)$ and $(C^\prime,LS) = (C,LS^\prime)$, causing the above to reduce to
\beqs
F = \left[ (S,LC) - (C,LS)\right] \pm \left[ (S,LC^\prime) - (C^\prime,LS)\right].
\eeqs
\beq
\equiv G \pm G^\prime
\label{GBarDef}
\eeq

By Green's theorem,
\begin{align}
\nonumber
G = &\int\limits_{V_3} \int\limits_{V_{12}} \int\limits_{S_\rho} \left[ S \frac{\nabla_\rho}{2} C - C \frac{\nabla}{2} S\right] \cdot d\vec{\sigma}_\rho d\tau_{12} d\tau_3
  + \int\limits_{V_\rho} \int\limits_{V_{12}} \int\limits_{S_3} \left[ S \nabla_{r_3} C - C \nabla_{r_3} S\right] \cdot d\vec{\sigma}_3 d\tau_{12} d\tau_{\rho} \\
  + &\int\limits_{V_\rho} \int\limits_{V_{13}}\int\limits_{S_{12}} \left[ S \: 2 \nabla_{r_{12}} C - C \: 2 \nabla_{r_{12}} S\right] \cdot d\vec{\sigma}_{12} d\tau_{13} d\tau_\rho
  \label{GDef}
\end{align}
The $\frac{1}{2}$ in the first term, the 1 in the second and the 2 in the third are from the appropriate Hamiltonian (\ref{Hamiltonian2}).
The positronium and hydrogen functions have an exponential dependence on $r_3$ and $r_{12}$, respectively.  So the second term goes to 0 due to the $r_3$ dependence, and so does the third term ($r_{12}$ dependence).

The surface elements under consideration are normal to $\hat{\rho}$, so we can ignore the angular dependence in $\vec{\nabla}_\rho$.  Then (\ref{Gdef}) becomes
\beq
G = \frac{1}{2} \int\limits_{V_3} \int\limits_{V_{12}} \left[\int\limits_{S_\rho} \left(C \frac{\partial S}{\partial \rho} - S \frac{\partial C}{\partial \rho} \right) \rho^2 \sin \theta_\rho d\theta_\rho d\varphi_\rho \right] d\tau_{12} d\tau_3
\label{GDef2}
\eeq

Using (\ref{eq:SCPhiDef}), we have
\begin{align}
\nonumber\frac{\partial S}{\partial \rho} & \asymplim{\rho} Y_{0,0} \left(\theta_\rho,\varphi_\rho \right) \Phi_{Ps}(r_{12}) \Phi_{H}(r_{3}) \sqrt{2\kappa} \frac{\partial}{\partial\rho} \frac{\sin(\kappa\rho)}{\kappa\rho} \\
\nonumber & \;\: = \frac{1}{\sqrt{4\pi}} \Phi_{Ps}(r_{12}) \Phi_{H}(r_{3}) \sqrt{2\kappa} \left(\frac{\cos(\kappa\rho)}{\rho} - \frac{\sin(\kappa\rho)}{\kappa\rho^2}\right) \\
& \;\: =\frac{1}{\sqrt{4\pi}} \Phi_{Ps}(r_{12}) \Phi_{H}(r_{3}) \sqrt{2\kappa} \left(\frac{\kappa\rho\cos(\kappa\rho) - \sin(\kappa\rho)}{\kappa\rho^2} \right)
\label{SPartial}
\end{align}

For $\displaystyle \frac{\partial C}{\partial \rho}$, notice that the shielding factor $(1-e^{-\mu\rho})(1+\frac{\mu}{2}\rho)$ goes to 1 as $\rho\to\infty$.
\begin{align}
\nonumber\frac{\partial S}{\partial \rho} & \asymplim{\rho} Y_{0,0} \left(\theta_\rho,\varphi_\rho \right) \Phi_{Ps}(r_{12}) \Phi_{H}(r_{3}) \sqrt{2\kappa} \frac{\partial}{\partial\rho} \frac{\cos(\kappa\rho)}{\kappa\rho} \\
\nonumber & \;\: = \frac{1}{\sqrt{4\pi}} \Phi_{Ps}(r_{12}) \Phi_{H}(r_{3}) \sqrt{2\kappa} \left(\frac{\sin(\kappa\rho)}{\rho} + \frac{\cos(\kappa\rho)}{\kappa\rho^2}\right) \\
& \;\: =\frac{1}{\sqrt{4\pi}} \Phi_{Ps}(r_{12}) \Phi_{H}(r_{3}) \sqrt{2\kappa} \left(\frac{\kappa\rho\sin(\kappa\rho) + \cos(\kappa\rho)}{\kappa\rho^2} \right)
\label{CPartial}
\end{align}

Substituting (\ref{SPartial}) and (\ref{CPartial}) into the parentheses of (\ref{GDef2}) yields
\begin{align}
\nonumber & \left(C \frac{\partial S}{\partial \rho} - S \frac{\partial C}{\partial \rho} \right) \\
\nonumber & \asymplim{\rho} \frac{1}{4\pi} \Phi_{Ps}(r_{12})^2 \Phi_{H}(r_{3})^2 \:2 \kappa
\left[\frac{\cos(\kappa\rho)}{\kappa\rho}\left(\frac{\kappa\rho\cos(\kappa\rho) - \sin(\kappa\rho)}{\kappa\rho^2}\right)
+ \frac{\sin(\kappa\rho)}{\kappa\rho}\left(\frac{\kappa\rho\sin(\kappa\rho) + \cos(\kappa\rho)}{\kappa\rho^2}\right) \right] \\
\nonumber & \;\: = \Phi_{Ps}(r_{12})^2 \Phi_{H}(r_{3})^2 \left[\frac{\kappa \rho \cos^2(\kappa\rho) - \cos(\kappa\rho)\sin(\kappa\rho) + \kappa\rho \sin^2(\kappa\rho) + \sin(\kappa\rho)\cos(\kappa\rho)}{2 \pi \kappa \rho^3} \right] \\
& \;\: = \frac{\Phi_{Ps}(r_{12})^2 \Phi_{H}(r_{3})^2}{2\pi\rho^2}
\end{align}

Substituting this into equation (\ref{GDef2}) gives
\begin{align}
\nonumber G &= \frac{1}{4\pi} \int\limits_{V_3} \int\limits_{V_{12}} \left[ \int\limits_{S_\rho} \Phi_{Ps}(r_{12})^2 \Phi_{H}(r_{3})^2 \rho^2 \sin \theta_\rho d\theta_\rho d\varphi_\rho \right] d\tau_{12} d\tau_3 \\
&= \frac{1}{4\pi} \int\limits_{V_3} \int\limits_{V_{12}} \Phi_{Ps}(r_{12})^2 \Phi_{H}(r_{3})^2 \left[ \int\limits_{S_\rho} \rho^2 \sin \theta_\rho d\theta_\rho d\varphi_\rho \right] d\tau_{12} d\tau_3
\end{align}

Performing the integration in brackets simply yields $4\pi$, so
\beq
G = \int\limits_{V_3} \int\limits_{V_{12}} \Phi_{Ps}(r_{12})^2 \Phi_{H}(r_{3})^2 d\tau_{12} d\tau_3
\eeq

Since $\Phi_{Ps}(r_{12})$ is only dependent on $r_{12}$ and $\Phi_{H}(r_{3})$ is only dependent on $r_3$, plus the fact that they are both orthonormal, the integration gives 1.  So we finally have
\beq
G = 1.
\eeq

From (\ref{GBarDef}), $G = (S,LC) - (C,LS)$, so
\beq
(S,LC) = (C,LS) + 1.
\label{eq:SLCandCLS}
\eeq

%Cite
From (\cite{}), the wavenumber is defined as
\beq
k = \frac{\sqrt{2 m E}}{\hbar}
\label{eq:Wavenumber}
\eeq

Back to (\ref{eq:Wavenumber}), since positronium has twice the mass of a positron, $m \to 2m$, the kinetic energy of the positronium atom is $T = \frac{1}{4} \kappa^2$, where $\kappa$ is the wavenumber of Ps.  Including the binding energy of H and Ps to get the total energy gives
\beq
E_T = E_H + E_{Ps} + \frac{1}{4} \kappa^2.
\eeq

For the ground states of H and Ps,
\beq
E_T = -\frac{1}{2} - \frac{1}{4} + \frac{1}{4} \kappa^2 = -\frac{3}{4} + \frac{1}{4} \kappa^2 \:\: \text{(in a.u.)}.
\label{eq:EnergyTotal}
\eeq

Separately, the H and Ps equations are (large values of $\rho$) respectively
\beq
\left(-\frac{1}{2}\nabla_{r_3}^2 - \frac{1}{r_3}\right) \Phi_H(r_3) = E_H \Phi_H(r_3)
\label{eq:HEqn}
\eeq
\beq
\left(-\nabla_{r_{12}}^2 - \frac{1}{r_{12}}\right) \Phi_{Ps}(r_{12}) = E_{Ps} \Phi_{Ps}(r_{12})
\label{eq:PsEqn}
\eeq


\section{Matrix Elements}
\begin{align}
\nonumber LS &= \left(-\frac{1}{2}\nabla_\rho^2 - \nabla_{r_3}^2 - 2\nabla_{r_{12}}^2 + \frac{2}{r_1} - \frac{2}{r_2} - \frac{2}{r_3} - \frac{2}{r_{12}} - \frac{2}{r_{13}} + \frac{2}{r_{23}} - 2 E_T\right) \Phi_{Ps}(r_{12}) \Phi_H(r_3) \frac{\sin(\kappa\rho)}{\kappa\rho} \sqrt{\frac{2\kappa}{4\pi}} \\
&= \left(-\frac{1}{2}\nabla_\rho^2 - \nabla_{r_3}^2 - 2\nabla_{r_{12}}^2 + \frac{2}{r_1} - \frac{2}{r_2} - \frac{2}{r_3} - \frac{2}{r_{12}} - \frac{2}{r_{13}} + \frac{2}{r_{23}} - 2 E_H - 2 E_{Ps} - \frac{1}{2}\kappa^2 \right) \Phi_{Ps}(r_{12}) \Phi_H(r_3) \frac{\sin(\kappa\rho)}{\kappa\rho} \sqrt{\frac{2\kappa}{4\pi}}.
\end{align}

Since $\Phi_H$ and $\frac{\sin(\kappa\rho)}{\kappa\rho}$ are independent of $r_3$, we can use (\ref{eq:HEqn}) to replace the $\left(-\frac{1}{2}\nabla_{r_3}^2 - \frac{1}{r_3}\right)$ with $E_H$, which gives a cancellation with the $-2 E_H$:
\beq
LS = \left(-\frac{1}{2}\nabla_\rho^2 - 2\nabla_{r_{12}}^2 + \frac{2}{r_1} - \frac{2}{r_2} - \frac{2}{r_{12}} - \frac{2}{r_{13}} + \frac{2}{r_{23}} - 2 E_{Ps} - \frac{1}{2}\kappa^2 \right) \Phi_{Ps}(r_{12}) \Phi_H(r_3) \frac{\sin(\kappa\rho)}{\kappa\rho} \sqrt{\frac{2\kappa}{4\pi}}
\eeq

\noindent We can do the same with (\ref{eq:PsEqn}):
\beq
LS = \left(-\frac{1}{2}\nabla_\rho^2 + \frac{2}{r_1} - \frac{2}{r_2} - \frac{2}{r_{13}} + \frac{2}{r_{23}} - \frac{1}{2}\kappa^2 \right) \Phi_{Ps}(r_{12}) \Phi_H(r_3) \frac{\sin(\kappa\rho)}{\kappa\rho} \sqrt{\frac{2\kappa}{4\pi}}
\label{LS1}
\eeq

The only part that explicitly depends on $\rho$ is $j_0(\kappa\rho)$.  We can see that $j_0(\kappa\rho)$ is an eigenfunction of $\nabla_\rho^2$:
From the file 'PVR Thesis.nb',
\beq
\nabla_\rho^2 \: j_0(\kappa\rho) = \frac{1}{\rho^2 } \frac{\partial}{\partial\rho} \left[ \rho^2 \frac{\partial}{\partial\rho} \left( \frac{\sin(\kappa\rho)}{\kappa\rho} \right)\right] = -\kappa^2 \, \frac{\sin(\kappa\rho)}{\kappa\rho} = -\kappa^2 j_0(\kappa\rho)
\label{eigenj0}
\eeq

\noindent Similarly,
\beq
\nabla_\rho^2 \: n_0(\kappa\rho) = -\frac{1}{\rho^2 } \frac{\partial}{\partial\rho} \left[ \rho^2 \frac{\partial}{\partial\rho} \left( \frac{\cos(\kappa\rho)}{\kappa\rho} \right)\right] = \kappa^2 \, \frac{\cos(\kappa\rho)}{\kappa\rho} = -\kappa^2 n_0(\kappa\rho)
\label{eigenn0}
\eeq

\noindent So $j_0(\kr)$ is an eigenfunction of $-\nabla_\rho^2$ with eigenvalue $-\kappa^2$.  Now using (\ref{eigenj0}) in (\ref{LS1}) gives
\beq
LS = \left( \frac{2}{r_1} - \frac{2}{r_2} - \frac{2}{r_{13}} + \frac{2}{r_{23}} \right) \Phi_{Ps}(r_{12}) \Phi_H(r_3) \frac{\sin(\kr)}{\kr} \sqrt{\frac{2\kappa}{4\pi}}
\label{eq:LS2}
\eeq

\beq
(S,LS) = \left( \left[ \frac{2}{r_1} - \frac{2}{r_2} - \frac{2}{r_{13}} + \frac{2}{r_{23}} \right] S^2 \right)
\label{eq:SLS}
\eeq

The potential energy terms in brackets are antisymmetric, i.e. replacing $1 \leftrightarrow 2$.  Also notice from (\ref{eq:SCPhiDef}) that S and C are symmetric.  So under the $1 \leftrightarrow 2$ permutation, $(S,LS)$ and $(C,LS)$ are antisymmetric.  Thus, we have
\beq
(S,LS) = 0 \text{ and } (C,LS) = 0.
\eeq

From (\ref{SCphiBarDef}),
\beq
(\bar{S},L\bar{S}) = \frac{1}{2} \left((S \pm S^\prime),L(S \pm S^\prime)\right) = \frac{1}{2} \left[(S,LS) \pm (S,LS^\prime) \pm (S^\prime,LS) + (S^\prime,L S^\prime) \right]
\eeq

\noindent From the symmetry of these elements (as in (\ref{})),
\beq
(S,LS) = (S^\prime,LS^\prime) = 0.
\eeq

\noindent Also,
\beq
(S,LS^\prime) = (S^\prime,LS).
\eeq

\beq
(\bar{S},L\bar{S}) = \pm \left(S^\prime,LS\right) = \pm \left(S^\prime, \left[ \frac{2}{r_1} - \frac{2}{r_2} - \frac{2}{r_{13}} + \frac{2}{r_{23}} \right] S\right)
\label{eq:SbarLSbar}
\eeq

The same computation applies for $\bar{C}$:
\beq
(\bar{C},L\bar{S}) = \pm \left(C^\prime,LS\right) = \pm \left(C^\prime, \left[ \frac{2}{r_1} - \frac{2}{r_2} - \frac{2}{r_{13}} + \frac{2}{r_{23}} \right] S\right)
\label{eq:CbarLSbar}
\eeq

We can use (\ref{eq:SLCandCLS}) to find $(\bar{S},L\bar{C})$ from this:
\beq
\left(\bar{S},L\bar{C}\right) = \left(\bar{C},L\bar{S}\right) + 1
\label{eq:SLCandCLSBar}
\eeq

Now we need to do the same analysis for $L\bar{C}$ as we just did for $L\bar{S}$.
\begin{align}
LC = & \left(-\frac{1}{2}\nabla_\rho^2 - \nabla_{r_3}^2 - 2\nabla_{r_{12}}^2 + \frac{2}{r_1} - \frac{2}{r_2} - \frac{2}{r_3} - \frac{2}{r_{12}} - \frac{2}{r_{13}} + \frac{2}{r_{23}} - 2 E_H - 2 E_{Ps} - \frac{1}{2}\kappa^2 \right) \\
 & \times \Phi_{Ps}(r_{12}) \Phi_H(r_3) \frac{\sin(\kappa\rho)}{\kappa\rho} \sqrt{\frac{2\kappa}{4\pi}} \left[1 + e^{-\mu\rho} \left(1 + \frac{\mu}{2} \rho \right) \right]
\label{LC1}
\end{align}

\noindent Again, we use (\ref{eq:HEqn}) and (\ref{eq:PsEqn}) to simplify this expression.
\beq
LC = \left(-\frac{1}{2}\nabla_\rho^2 + \frac{2}{r_1} - \frac{2}{r_2} - \frac{2}{r_3} - \frac{2}{r_{12}} - \frac{2}{r_{13}} + \frac{2}{r_{23}}  - \frac{1}{2}\kappa^2\right) 
 \Phi_{Ps}(r_{12}) \Phi_H(r_3) \frac{\sin(\kappa\rho)}{\kappa\rho} \sqrt{\frac{2\kappa}{4\pi}} \left[1 + e^{-\mu\rho} \left(1 + \frac{\mu}{2} \rho \right) \right]
\label{LC2}
\eeq

From (\ref{eigenn0}), $n_0(\kr)$ is an eigenfunction of $\nabla_\rho^2$ with eigenvalue $-\kappa^2$.  Then $\displaystyle -\frac{1}{2} \nabla_\rho^2 \frac{\cos(\kr)}{\kr}$ could be replaced by $\displaystyle \frac{1}{2}\kappa^2\frac{\cos(\kr)}{\kr}$ if the shielding term could be ignored.  However, it also depends on $\rho$, so we have to calculate this separately.

\begin{align}
\nonumber -\frac{1}{2} \nabla_\rho^2 & \left\lbrace \frac{\cos(\kr)}{\kr} \left[1 - e^{-\mr} \left(1 + \frac{\mu}{2} \rho \right)\right]\right\rbrace \\
\nonumber &= \frac{e^{-\mr}}{2\kappa\rho} \left[ \frac{\mu^3 \rho}{2} \cos(\kr) + \kappa^2 \left(-1 + e^{\mr} -\frac{\mr}{2} \right) \cos(\kr) + \kappa\mu (1+\mr) \sin(\kr) \right] \\
\nonumber &= \frac{e^{-\mr} \mu^3\rho}{4\kr} \cos(\kr) + \frac{\kappa^2}{2} \left[-e^{-\mr} + 1 - \frac{\mr}{2} e^{-\mr}\right] \frac{\cos(\kr)}{\kr} + \frac{e^{-\mr}}{2\kr} \kappa\mu (1 + \mr)\sin(\kr) \\
 &= \frac{e^{-\mr} \mu^3\rho}{4\kr} \cos(\kr) + \frac{\kappa^2}{2} \left[1 - e^{-\mr}\left(1 + \frac{\mu}{2} \rho \right) \right] \frac{\cos(\kr)}{\kr} + \frac{e^{-\mr}}{2\kr} \kappa\mu (1 + \mr)\sin(\kr)
\label{laplacianshield}
\end{align}

\noindent The second term here cancels the $\displaystyle \frac{1}{2}\kappa^2$ in parentheses in (\ref{LC2}).  Now we have
\begin{align}
LC = \: & \Phi_{Ps}(r_{12}) \Phi_H(r_3) \sqrt{\frac{2\kappa}{4\pi}} \left\{ \left(\frac{2}{r_1} - \frac{2}{r_2} - \frac{2}{r_{13}} + \frac{2}{r_{23}} \right) \frac{\cos(\kr)}{\kr} \left[1 - e^{-\mr} \left(1 + \frac{\mu}{2}\rho \right) \right] \right. \nonumber\\
& + \left. \frac{e^{-\mr} \mu^3 \rho}{4} \frac{\cos(\kr)}{\kr} + \frac{e^{-\mr}}{2} \kappa \mu (1+\mu\rho) \frac{\sin(\kr)}{\kr} \right\}
\label{eq:LC}
\end{align}

To find $L\bar{C}$, we also have to analyze $LC^\prime$.  From (\ref{eq:SCprime}) and (\ref{eq:SCPhiDef}),
\beq
C^\prime = P_{23} C = -Y_{0,0}(\theta_{\rho^\prime}, \varphi_{\rho^\prime}) \Phi_{Ps}(r_{13}) \Phi_H(r_2) \sqrt{2\kappa} n_0(\kr) \left[1 - e^{-\mu\rho^\prime}\left(1 + \frac{\mu}{2} \rho^\prime \right) \right].
\eeq

\noindent The calculation is exactly the same as that for $LC$, except $\rho\to\rho^\prime$ and $2\leftrightarrow3$.

\begin{align}
LC^\prime = & \left(-\frac{1}{2}\nabla_{\rhop}^2 - \nabla_{r_2}^2 - 2\nabla_{r_{13}}^2 + \frac{2}{r_1} - \frac{2}{r_2} - \frac{2}{r_3} - \frac{2}{r_{12}} - \frac{2}{r_{13}} + \frac{2}{r_{23}} - 2 E_H - 2 E_{Ps} - \frac{1}{2}\kappa^2 \right) \nonumber \\
 & \times \Phi_{Ps}(r_{13}) \Phi_H(r_2) \frac{\sin(\kappa\rhop)}{\kappa\rhop} \sqrt{\frac{2\kappa}{4\pi}} \left[1 + e^{-\mu\rhop} \left(1 + \frac{\mu}{2} \rhop \right) \right]
\label{LCP1}
\end{align}

We can modify (\ref{eq:HEqn}) and (\ref{eq:PsEqn}) to simplify this:
\begin{align}
&\left(-\frac{1}{2}\nabla_{r_2}^2 - \frac{1}{r_2}\right) \Phi_H(r_2) = E_H \Phi_H(r_2) \text{ and} \\
&\left(-\nabla_{r_{13}}^2 - \frac{1}{r_{13}}\right) \Phi_{Ps}(r_{13}) = E_{Ps} \Phi_{Ps}(r_{13}).
\end{align}

\beq
LC^\prime = \left(-\frac{1}{2}\nabla_{\rhop}^2 + \frac{2}{r_1} - \frac{2}{r_3} - \frac{2}{r_{12}}  + \frac{2}{r_{23}} - \frac{1}{2}\kappa^2 \right) \Phi_{Ps}(r_{13}) \Phi_H(r_2) \frac{\sin(\kappa\rhop)}{\kappa\rhop} \sqrt{\frac{2\kappa}{4\pi}} \left[1 + e^{-\mu\rhop} \left(1 + \frac{\mu}{2} \rhop \right) \right]
\label{eq:LCP2}
\eeq

Modifying the result from (\ref{laplacianshield}) to use $\rhop$ instead of $\rho$ yields
\begin{align}
\nonumber -\frac{1}{2} \nabla_\rhop^2 & \left\lbrace \frac{\cos(\krp)}{\krp} \left[1 - e^{-\mrp} \left(1 + \frac{\mu}{2} \rhop \right)\right]\right\rbrace \\
 &= \frac{e^{-\mrp} \mu^3\rhop}{4\kr} \cos(\krp) + \frac{\kappa^2}{2} \left[1 - e^{-\mrp}\left(1 + \frac{\mu}{2} \rhop \right) \right] \frac{\cos(\krp)}{\krp} + \frac{e^{-\mrp}}{2\krp} \kappa\mu (1 + \mrp)\sin(\krp)
\end{align}

\noindent Similar to (\ref{eq:LC}), (\ref{eq:LCP2}) becomes
\begin{align}
LC^\prime = \: & \Phi_{Ps}(r_{13}) \Phi_H(r_2) \sqrt{\frac{2\kappa}{4\pi}} \left\{ \left(\frac{2}{r_1} - \frac{2}{r_3} - \frac{2}{r_{12}} + \frac{2}{r_{23}} \right) \frac{\cos(\krp)}{\krp} \left[1 - e^{-\mrp} \left(1 + \frac{\mu}{2}\rhop \right) \right] \right. \nonumber\\
& + \left. \frac{e^{-\mrp} \mu^3 \rhop}{4} \frac{\cos(\krp)}{\krp} + \frac{e^{-\mrp}}{2} \kappa \mu (1+\mu\rhop) \frac{\sin(\krp)}{\krp} \right\}
\label{eq:LCP}
\end{align}

\noindent Combining (\ref{eq:LC}) and (\ref{eq:LCP}) gives
\begin{align}
L\bar{C} = \:\: &\frac{1}{\sqrt{2}} L(C \pm C^\prime) = \frac{1}{\sqrt{2}} (LC \pm LC^\prime) \nonumber \\
= \:\: &\frac{1}{\sqrt{8\pi}} \Phi_{Ps}(r_{12}) \Phi_H(r_3) \sqrt{2\kappa} \nonumber  \\
&\times \left\{ \frac{\kappa\mu}{2} e^{-\mr} (1+\mr) \frac{\sin(\kr)}{\kr} + \frac{\mu^3 \rho}{4} e^{-\mr} \frac{\cos(\kr)}{\kr} \right. \nonumber \\
&+ \left. \left(\frac{2}{r_1} - \frac{2}{r_2} - \frac{2}{r_{13}} + \frac{2}{r_{23}}\right) \frac{\cos(\kr)}{\kr} \left[1 - e^{-\mr} \left(1 + \frac{\mu}{2}\rho\right)\right] \right\} \nonumber \\
\pm &\frac{1}{\sqrt{8\pi}} \Phi_{Ps}(r_{13}) \Phi_H(r_2) \sqrt{2\kappa} \nonumber  \\
&\times \left\{ \frac{\kappa\mu}{2} e^{-\mrp} (1+\mrp) \frac{\sin(\krp)}{\krp} + \frac{\mu^3 \rhop}{4} e^{-\mrp} \frac{\cos(\krp)}{\krp} \right. \nonumber \\
&+ \left. \left(\frac{2}{r_1} - \frac{2}{r_3} - \frac{2}{r_{12}} + \frac{2}{r_{23}}\right) \frac{\cos(\krp)}{\krp} \left[1 - e^{-\mrp} \left(1 + \frac{\mu}{2}\rhop\right)\right] \right\}
\label{LCBar}
\end{align}

The properties of the permutation operator give that
\beq
(C,LC) = (C',LC') \text{ and } (C,LC') = (C',LC)
\eeq

Using these two relations gives that
\begin{align}
\left(\bar{C},L\bar{C}\right) &= \frac{1}{2}\left(\left(C \pm C'\right),L(C \pm C')\right) \nonumber\\
 &= \frac{1}{2}\left[(C,LC) \pm (C,LC') \pm (C',LC) + (C',LC')\right] = \frac{1}{2}\left[2(C,LC) \pm 2(C',LC)\right] \nonumber\\
 &= (C,LC) \pm (C',LC).
 \label{CLC1}
\end{align}

Substitute (\ref{eq:CDef}) and (\ref{eq:LC}) in (\ref{CLC1}) to get
\begin{align}
\left(\bar{C},L\bar{C}\right) = &(C,LC) \pm (C',LC) = \left((C \pm C'),LC\right) \nonumber\\
 = &\left((C \pm C') \Phi_{Ps}(r_{12}) \Phi_H(r_3) \sqrt{\frac{2\kappa}{4\pi}} \left\{ \left(\frac{2}{r_1} - \frac{2}{r_2} - \frac{2}{r_{13}} + \frac{2}{r_{23}} \right) \frac{\cos(\kr)}{\kr} \left[1 - e^{-\mr}\left(1 + \frac{\mu}{2}\rho \right) \right] \right.\right. \\
   &+ \left.\left. \frac{e^{-\mr} \mu^3 \rho}{4} \frac{\cos(\kr)}{\kr} + \frac{e^-{\mr}}{2} \kappa\mu (1+\mr) \frac{\sin(\kr)}{\kr} \right\} \right).
 \label{CBarLCBar1}
\end{align}

These can be split up to simplify:
\begin{align}
\left(\bar{C},L\bar{C}\right) = &\sqrt{\frac{2\kappa}{4\pi}} \frac{\kappa\mu}{2} \left((C \pm C') e^{-\mr}(1+\mr) \frac{\sin(\kr)}{\kr} \Phi_{Ps}(r_{12}) \Phi_H(r_3) \right)  \nonumber\\
 &+ \sqrt{\frac{2\kappa}{4\pi}} \frac{\mu^3}{4} \left((C \pm C') e^{-\mr} \frac{\cos(\kr)}{\kr} \Phi_{Ps}(r_{12}) \Phi_H(r_3) \right) \nonumber\\
 &+ \sqrt{\frac{2\kappa}{4\pi}} \left((C \pm C') \left(\frac{2}{r_1} - \frac{2}{r_3} - \frac{2}{r_{12}} + \frac{2}{r_{23}}\right) \frac{\cos(\kr)}{\kr} \left[1 - e^{-\mr}\left(1 + \frac{\mu}{2}\rho \right)\right] \Phi_{Ps}(r_{12}) \Phi_H(r_3) \right) \nonumber\\
= &\frac{\kappa\mu}{2} \left((C \pm C') e^{-\mr}(1+\mr) S\right) \nonumber\\
 &+ \sqrt{\frac{2\kappa}{4\pi}} \frac{\mu^3}{4} \left((C \pm C') e^{-\mr} \frac{\cos(\kr)}{\kr} \Phi_{Ps}(r_{12}) \Phi_H(r_3) \right) \nonumber\\
 &+ \left((C \pm C') \left(\frac{2}{r_1} - \frac{2}{r_3} - \frac{2}{r_{12}} + \frac{2}{r_{23}}\right) C \right).
 \label{CBarLCBar2}
\end{align}

Looking at just the last term:
\beq
\left((C \pm C') \left(\frac{2}{r_1} - \frac{2}{r_3} - \frac{2}{r_{12}} + \frac{2}{r_{23}}\right) C \right) = \left( \left(\frac{2}{r_1} - \frac{2}{r_3} - \frac{2}{r_{12}} + \frac{2}{r_{23}}\right) C^2 \right) \pm \left( \left(\frac{2}{r_1} - \frac{2}{r_3} - \frac{2}{r_{12}} + \frac{2}{r_{23}}\right) C'C \right)
\eeq

\noindent The first set of parentheses has the same form as (\ref{SLS}).  These terms in parentheses are antisymmetric with respect to the $1 \leftrightarrow 2$ permutation.  Also, $C$ is symmetric with respect to this permutation.  So the first set of parentheses is 0.  Thus,
\beq
\left((C \pm C') \left(\frac{2}{r_1} - \frac{2}{r_3} - \frac{2}{r_{12}} + \frac{2}{r_{23}}\right) C \right) = \pm \left(C' \left(\frac{2}{r_1} - \frac{2}{r_3} - \frac{2}{r_{12}} + \frac{2}{r_{23}}\right) C \right)
\eeq

\noindent We finally have that
\begin{align}
\left(\bar{C},L\bar{C}\right) = \: &\frac{\kappa\mu}{2} \left((C \pm C') e^{-\mr}(1+\mr) S\right) \nonumber\\
 &+ \sqrt{\frac{2\kappa}{4\pi}} \frac{\mu^3}{4} \left((C \pm C') e^{-\mr} \frac{\cos(\kr)}{\kr} \Phi_{Ps}(r_{12}) \Phi_H(r_3) \right) \nonumber\\
 &\pm \left(C' \left(\frac{2}{r_1} - \frac{2}{r_3} - \frac{2}{r_{12}} + \frac{2}{r_{23}}\right) C \right).
 \label{CBarLCBar}
\end{align}

\textbf{@TODO:} Put this elsewhere.
We chose to absorb both the $\frac{1}{\sqrt{2}}$ and $Y_{0,0}(\theta_\rho,\varphi_\rho) = \frac{1}{\sqrt{4\pi}}$ into the $c_i$ constants of the short-range terms (\ref{eq:PhiDef}).

$LS'$ is constructed by performing the $2 \leftrightarrow 3$ permutation of (\ref{eq:LS2}):
\beq
\label{eq:LSP2}
LS' = \left( \frac{2}{r_1} - \frac{2}{r_3} - \frac{2}{r_{12}} + \frac{2}{r_{23}} \right) S'.
\eeq

So $(\bar{\phi}_i, L\bar{S})$ is a straightforward calculation.
\beq
\label{eq:PhiBarLSBar1}
(\bar{\phi}_i, L\bar{S}) = \left( (\phi_i \pm \phi_i') L \frac{(S \pm S')}{\sqrt{2}}\right)
= \frac{1}{\sqrt{2}} \left(\phi_i L S \pm \phi_i L S' \pm \phi_i' L S + \phi_i' L S'\right)
\eeq

Again, from the properties of the $P_{23}$ permutation operator,
\beq
(\phi_i,LS) = (\phi_i',LS') \text{ and } (\phi_i,LS') = (\phi_i',LS).
\eeq

Equation (\ref{eq:PhiBarLSBar1}) becomes
\begin{subequations}
\label{eq:PhiBarLSBar2}
\begin{align}
(\bar{\phi}_i, L\bar{S}) = \frac{1}{\sqrt{2}} \left[2(\phi_i,LS) \pm 2(\phi_i',LS)\right] &= \frac{2}{\sqrt{2}} \left[(\phi_i,LS) \pm (\phi_i',LS)\right] \label{eq:PhiBarLSBar2a} \\
 &= \frac{2}{\sqrt{2}} \left[(\phi_i,LS) \pm (\phi_i,LS')\right]  \label{eq:PhiBarLSBar2b}
\end{align}
\end{subequations}

Notice that (\ref{eq:PhiBarLSBar2a}) and (\ref{eq:PhiBarLSBar2b}) are equivalent ways of writing this expression.  Either could be used, depending on the form desired for the computation.
From (\ref{eq:LS2}) and (\ref{eq:LSP2}),
\begin{align}
(\bar{\phi}_i, L\bar{S}) &= \frac{2}{\sqrt{2}} \left[\left( \phi_i \left( \frac{2}{r_1} - \frac{2}{r_2} - \frac{2}{r_{13}} + \frac{2}{r_{23}} \right) S\right) \pm \left( \phi_i' \left( \frac{2}{r_1} - \frac{2}{r_2} - \frac{2}{r_{13}} + \frac{2}{r_{23}} \right) S\right)\right] \\
 &= \frac{2}{\sqrt{2}} \left[\left( \phi_i \left( \frac{2}{r_1} - \frac{2}{r_2} - \frac{2}{r_{13}} + \frac{2}{r_{23}} \right) S\right) \pm \left( \phi_i \left( \frac{2}{r_1} - \frac{2}{r_3} - \frac{2}{r_{12}} + \frac{2}{r_{23}} \right) S'\right)\right]
\end{align}

\noindent We also have that
\beq
(\bar{\phi}_i, L\bar{S}) = (\bar{S}, L\bar{\phi}_i).
\label{eq:PhiLSPerm}
\eeq

The analysis for $(\bar{\phi}_i, L\bar{C})$ and $(\bar{C}, L\bar{\phi}_i)$ is similar to the above analysis.
\begin{subequations}
\begin{align}
(\bar{\phi}_i, L\bar{C}) &= \frac{2}{\sqrt{2}} \left[(\phi_i,LC) \pm (\phi_i',LC)\right] \label{PhiBarLCBar2a} \\
 &= \frac{2}{\sqrt{2}} \left[(\phi_i,LC) \pm (\phi_i,LC')\right]  \label{PhiBarLCBar2b}
\end{align}
\end{subequations}

Using (\ref{eq:LC}) and (\ref{eq:LCP}) in the above gives our final result for $(\bar{\phi}_i, L\bar{C})$.
\begin{subequations}
\begin{align}
(\bar{\phi}_i, L\bar{C}) = \sqrt{\frac{\kappa}{\pi}} &\left( (\phi_i \pm \phi_i') \Phi_{Ps}(r_{12}) \Phi_H(r_3) \left\{ \left(\frac{2}{r_1} - \frac{2}{r_2} - \frac{2}{r_{13}} + \frac{2}{r_{23}}\right) \frac{\cos(\kr)}{\kr} \left[1 - e^{-\mr}\left(1 + \frac{\mu}{2}\rho\right)\right] \right.\right. \nonumber\\
&+ \left.\left.\frac{e^{-\mr}\mu^3\rho}{4} \frac{\cos(\kr)}{\kr} + \frac{e^{-\mr}}{2} \kappa\mu (1+\mr) \frac{\sin(\kr)}{\kr}  \right\}\right) \\
= \sqrt{\frac{\kappa}{\pi}} &\left( \phi_i \Phi_{Ps}(r_{12}) \Phi_H(r_3) \left\{ \left(\frac{2}{r_1} - \frac{2}{r_2} - \frac{2}{r_{13}} + \frac{2}{r_{23}}\right) \frac{\cos(\kr)}{\kr} \left[1 - e^{-\mr}\left(1 + \frac{\mu}{2}\rho\right)\right] \right.\right. \nonumber\\
&+ \left.\left.\frac{e^{-\mr}\mu^3\rho}{4} \frac{\cos(\kr)}{\kr} + \frac{e^{-\mr}}{2} \kappa\mu (1+\mr) \frac{\sin(\kr)}{\kr}  \right\}\right) \nonumber\\
\pm \sqrt{\frac{\kappa}{\pi}} &\left( \phi_i \Phi_{Ps}(r_{12}) \Phi_H(r_3) \left\{ \left(\frac{2}{r_1} - \frac{2}{r_3} - \frac{2}{r_{12}} + \frac{2}{r_{23}}\right) \frac{\cos(\krp)}{\krp} \left[1 - e^{-\mrp}\left(1 + \frac{\mu}{2}\rhop\right)\right] \right.\right. \nonumber\\
&+ \left.\left.\frac{e^{-\mr}\mu^3\rhop}{4} \frac{\cos(\krp)}{\krp} + \frac{e^{-\mrp}}{2} \kappa\mu (1+\mrp) \frac{\sin(\krp)}{\krp}  \right\}\right)
\end{align}
\end{subequations}

Similar to (\ref{eq:PhiLSPerm}), we have
\beq
(\bar{\phi}_i, L\bar{C}) = (\bar{C}, L\bar{\phi}_i).
\label{eq:PhiLCPerm}
\eeq

The short-range -- short-range interactions are calculated in a separate document ``3.24 Using Spherical Coordinates''.

Lastly, we need to prove that $\left(\bar{\phi}_i, L \bar{\phi}_j\right) = \left(\bar{\phi}_j, L \bar{\phi}_i\right)$.  Consider the functional
\beq
F = (\phi_i, L \phi_j) - (\phi_j L \phi_i).
\eeq

Using (\ref{Hamiltonian1}),
\beq
	F=\left({-\phi_i,{\frac {1}{2}{\nabla }_{{r}_{1}}^{2}\phi_j}}\right)+\left({\phi_j,{\frac {1}{2}{\nabla }_{{r}_{1}}^{2}\phi_i}}\right)+
	\left({-\phi_i,{\frac {1}{2}{\nabla }_{{r}_{2}}^{2}\phi_j}}\right)+\left({\phi_j,{\frac {1}{2}{\nabla }_{{r}_{2}}^{2}\phi_i}}\right)+
	\left({-\phi_i,{\frac {1}{2}{\nabla }_{{r}_{3}}^{2}\phi_j}}\right)+\left({\phi_j,{\frac {1}{2}{\nabla }_{{r}_{3}}^{2}\phi_i}}\right)
\eeq

From (\ref{eq:PhiDef}), the $\bar{\phi}_i$ (and $\bar{\phi}_j$) have a decaying exponential dependence in $r_1$, $r_2$ and $r_3$, so the surface terms in the integrations will vanish.  Combining this fact with (\ref{PhiBarDef}) gives that $F = 0$.  Then
\beq
\left(\bar{\phi}_i, L \bar{\phi}_j\right) = \left(\bar{\phi}_j, L \bar{\phi}_i\right)
\label{PhiLPhiPerm}
\eeq

The claims in (\ref{eq:PhiLSPerm}) and (\ref{eq:PhiLCPerm}) are proven using the same method.  Both (\ref{eq:PhiLSPerm}) and (\ref{eq:PhiLCPerm}) include $\bar{\phi}_i$, so these terms do have a decaying exponential dependence, making the claims true.

%(\bar{\phi}_i, L\bar{C}) = \frac{\sqrt{2\kappa}}{4\pi} &\left( (\phi_i \pm \phi_i') \Phi_{Ps}(r_{12}) \Phi_H(r_3) \left\{ \left(\frac{2}{r_1} - \frac{2}{r_2} - \frac{2}{r_{13}} + \frac{2}{r_{23}}\right) \frac{\cos(\kr)}{\kr} \left[1 - e^{-\mr}\left(1 + \frac{\mu}{2}\rho\right)\right] \right.\right. \nonumber\\
%&+ \frac{e^{-\mr}\mu^3\rho}{4} \frac{\cos(\kr)}{\kr} + \frac{e^{-\mr}}{2} \kappa\mu (1+\mr) \frac{\sin(\kr)} \left.\left. \right\}\right)



\section{Complex Kohn}

The complex Kohn method provides anomaly-free phase shifts, except in exceptional circumstances \cite{Lucchese1989}.  In this section, we will derive the complex Kohn variational principle and solve for the phase shift.

Similar to the treatment in \cite{Cooper2010}, the complex-valued trial wavefunction is
\beq
\breve{\Psi}_t^\pm = \bar{S} + T_t \, \bar{W} + \sum_{i=1}^N c_i' \bar{\phi_i}^t,
\label{eq:TrialComplex}
\eeq

where
\beq
\bar{W} = \bar{S} + \ii \bar{C}.
\label{eq:WDef}
\eeq

$\bar{S}$, $\bar{C}$ and $\bar{\phi_i}$ are the same as in (\ref{SCphiBarDef}).  Note that this is different than Cooper et al. \cite{Cooper2010}, in that we do not use a variable $\tau$, and our $\bar{W}$ is their $\bar{T}$ but with $\bar{C}$ and $\bar{S}$ swapped, which gives a wavefunction of the correct form (an outgoing wave).

The version of (\ref{eq:IlDefU}) for the full wavefunction is
\beq
I_l[\Psi_l^t] \equiv \left<\Psi_l^t | L | \Psi_l^t \right> = \int \Psi_l^t(\vec{r}) L_l \Psi_l^t(\vec{r}) \,d\vec{r}.
\label{eq:IlDefPsi}
\eeq

\noindent Similar to (\ref{eq:UlTrialRelation}),
\beq
\Psi_l^t(r) = \Psi_l(r) + \delta \Psi_l(r).
\label{eq:PsilTrialRelation}
\eeq

\noindent The variation of $I_l$ is
\begin{align}
\nonumber \delta I_l &= I_l[\Psi_l^t] - I[\Psi_l] \\
\nonumber &= I[\Psi_l + \delta \Psi_l] - I[\Psi_l] \\
\nonumber &= \left<\Psi_l^t | L_l | \Psi_l^t\right> - \left<\Psi_l | L_l | \Psi_l\right> \\
&= \left<\Psi_l | L_l | \Psi_l\right> + \left<\Psi_l | L_l | \delta\Psi_l\right> + \left<\delta\Psi_l | L_l | \Psi_l\right> + \left<\delta\Psi_l | L_l | \delta\Psi_l\right> - \left<\Psi | L_l | \Psi_l \right>.
\label{eq:IlPsiVariation1}
\end{align}

\noindent The first and last terms are equal to 0, by virtue of (\ref{eq:ExactIl}).  The $\left<\delta\Psi_l | L | \delta\Psi_l\right>$ term is of second order in $\delta\Psi_l$, so this is neglected.  We also drop the $l$ subscript from this point.

Since $L \Psi = 0$,
\beq
\left<\delta\Psi_l | L | \Psi_l\right> = -\left<\delta\Psi_l | L | \Psi_l\right>,
\eeq

\noindent which combined with the definition of $L$ from (\ref{LDef}), allows us to write the above equation as
\beq
\delta I_l = 2 \left<\Psi_l | H\!-\!E | \delta\Psi_l\right> - 2 \left<\delta\Psi_l | H\!-\!E | \Psi_l\right>
\label{eq:IlPsiVariation2}
\eeq

As in (\ref{GDef}) and (\ref{GDef2}), Green's theorem gives
\begin{align}
\nonumber \left<\delta\Psi_l | \nabla_\rho^2 | \Psi_l\right> - \left<\delta\Psi_l | \nabla_\rho^2 | \Psi_l\right> 
&= \int\limits_{V_3} \int\limits_{V_{12}} \int\limits_{S_\rho} \left[ \Psi \nabla_\rho^2 \delta\Psi - \delta\Psi \nabla_\rho^2 \Psi \right] d\tau_\rho \, d\tau_{12} d\tau_3 \\
&= \int\limits_{V_3} \int\limits_{V_{12}} \int\limits_{S_\rho} \left[ \Psi \vec{\nabla}_\rho \delta\Psi - \delta\Psi \vec{\nabla}_\rho \Psi \right] \cdot d\vec{\sigma}_\rho d\tau_{12} d\tau_3
\label{eq:ComplexGreensThm}
\end{align}

We define $\delta I'$ as
\begin{align}
\nonumber \delta I' \equiv &\int\limits_{V_3} \int\limits_{V_{12}} \int\limits_{S_\rho} \left[\Psi \vec{\nabla}_\rho \delta\Psi - \delta\Psi \vec{\nabla}_\rho \Psi \right] \cdot d\vec{\sigma}_\rho d\tau_{12} d\tau_3
  + \int\limits_{V_\rho} \int\limits_{V_{12}} \int\limits_{S_3} \left[\Psi \vec{\nabla}_{r_3} \delta\Psi - \delta\Psi \vec{\nabla}_{r_3} \Psi \right] \cdot d\vec{\sigma}_3 d\tau_{12} d\tau_{\rho} \\
  + &\int\limits_{V_\rho} \int\limits_{V_{13}}\int\limits_{S_{12}} \left[\Psi \vec{\nabla}_{r_{12}} \delta\Psi - \delta\Psi \vec{\nabla}_{r_{12}} \Psi \right] \cdot d\vec{\sigma}_{12} d\tau_{13} d\tau_\rho.
\end{align}

\noindent Due to the exponential fall-off in (\ref{eq:HWave}) and (\ref{eq:PsWave}), the last two terms cancel, leaving
\beq
\nonumber \delta I' \equiv \int\limits_{V_3} \int\limits_{V_{12}} \int\limits_{S_\rho} \left[\Psi \vec{\nabla}_\rho \delta\Psi - \delta\Psi \vec{\nabla}_\rho \Psi \right] \cdot d\vec{\sigma}_\rho d\tau_{12} d\tau_3.
\eeq

We can drop the dot product, since the surface elements are in the same direction as $\nabla_\rho$.
Using the above equation in (\ref{eq:IlPsiVariation2}), along with (\ref{Hamiltonian2}), (\ref{eq:HEqn}), (\ref{eq:PsEqn}) and (\ref{eq:EnergyTotal}) gives a cancellation in the terms without Laplacian operators, yielding
\begin{align}
\nonumber \delta I = &-\frac{1}{2} \int\limits_{V_3} \int\limits_{V_{12}} \int\limits_{S_\rho} \left[ \Psi \nabla_\rho \delta\Psi - \delta\Psi \nabla_\rho \Psi \right] d\sigma_\rho d\tau_{12} d\tau_3 \\
\nonumber & + 2 \int \Psi\left(H_H + H_{Ps} + \frac{1}{r_1} - \frac{1}{r_2} - \frac{1}{r_{13}} + \frac{1}{r_{23}} - E_H - E_{Ps} + \frac{1}{4} \kappa^2\right) \delta\Psi \, d\sigma_\rho d\tau_{12} d\tau_3 \\
\nonumber & + 2 \int \delta\Psi\left(H_H + H_{Ps} + \frac{1}{r_1} - \frac{1}{r_2} - \frac{1}{r_{13}} + \frac{1}{r_{23}} - E_H - E_{Ps} + \frac{1}{4} \kappa^2\right) \Psi \, d\sigma_\rho d\tau_{12} d\tau_3 \\
= &-\frac{1}{2} \int\limits_{V_3} \int\limits_{V_{12}} \int\limits_{S_\rho} \left[ \Psi \nabla_\rho \delta\Psi - \delta\Psi \nabla_\rho \Psi \right] d\sigma_\rho d\tau_{12} d\tau_3
\label{eq:ComplexIl1}
\end{align}

The variation of $\Psi$ is
\beq
\delta\Psi = \Psi^t - \Psi = (\bar{S} + T_t \bar{W}) - (\bar{S} + T \bar{W}) = (T_t - T) \bar{W} = \delta T \,\bar{W}
\eeq

\noindent Using this in (\ref{eq:ComplexIl1}) gives
\beq
\delta I = -\frac{1}{2} \int\limits_{V_3} \int\limits_{V_{12}} \int\limits_{S_\rho} \left\{ \left[(\bar{S} + T\bar{W})\nabla_\rho(T_t - T)\bar{W}\right] - \left[(T_t - T)\bar{W} \, \nabla_\rho (\bar{S} + T \bar{W})\right] \right\} d\sigma_\rho d\tau_{12} d\tau_3
\eeq

Now evaluating the $\nabla_\rho$ on $j_0$ and $-n_0$,
\begin{align}
\nonumber
\nabla_\rho \frac{\sin{\kr}}{\kr} &= \kappa \frac{\cos{\kr}}{\kr} - \frac{\sin{\kr}}{\kappa\rho^2} \\
\nabla_\rho \frac{\cos{\kr}}{\kr} &= -\kappa \frac{\sin{\kr}}{\kr} - \frac{\cos{\kr}}{\kappa\rho^2},
\end{align}

\noindent which, if we only take the leading terms, gives
\begin{alignat}{2}
\nonumber\nabla_\rho \bar{S} &={}& &\kappa \bar{C} \\
\nonumber\nabla_\rho \bar{C} &={}& -&\kappa \bar{S} \\
\nabla_\rho \bar{W} &={}& -&\kappa \bar{S} + \ii \kappa \bar{C} = \ii\kappa(\ii \bar{S} + \bar{C}) = \ii\kappa\bar{W}.
\label{eq:NablaOnSCW}
\end{alignat}

Substituting these in $\delta I$ gives
\begin{align}
\nonumber \delta I &= -\frac{1}{2} \int\limits_{V_3} \int\limits_{V_{12}} \int\limits_{S_\rho} \left\{ \left[(\bar{S} + T\bar{W})(T_t - T)\ii\kappa\bar{W}\right] - \left[(T_t - T)\bar{W} (\kappa\bar{S} + T \ii\kappa\bar{W})\right] \right\} d\sigma_\rho d\tau_{12} d\tau_3 \\
\nonumber &= -\frac{\kappa}{2} (T_t - T)\int\limits_{V_3} \int\limits_{V_{12}} \int\limits_{S_\rho} \left[ \left(\ii \bar{S}\bar{W} + \ii T \bar{W}^2 - \bar{W}\bar{C} - \ii T \bar{W}\right) \right] d\sigma_\rho d\tau_{12} d\tau_3 \\
\nonumber &= -\frac{\kappa}{2} \int\limits_{V_3} \int\limits_{V_{12}} \int\limits_{S_\rho} \left[ \ii T_t \bar{S}\bar{W} + \ii T_t T\bar{W}^2 - \ii T \bar{S}\bar{W} - \ii T^2\bar{W}^2 - T_t\bar{W}\bar{C} - \ii T_t T \bar{W} + T \bar{W}\bar{C} + \ii T^2 \bar{W} \right] d\sigma_\rho d\tau_{12} d\tau_3
\end{align}

\noindent Neglecting quadratic terms of $T^2$ or $T_t T$, we have
\begin{alignat}{2}
\nonumber \delta I &={}& -&\frac{\kappa}{2} (T_t - T) \int\limits_{V_3} \int\limits_{V_{12}} \int\limits_{S_\rho} \left[\ii \bar{S}\bar{W} - \bar{W}\bar{C}\right] d\sigma_\rho d\tau_{12} d\tau_3 \\
\nonumber &={}& -&\frac{\kappa}{2} (T_t - T) \int\limits_{V_3} \int\limits_{V_{12}} \int\limits_{S_\rho} \left[\ii\bar{S}\left(\bar{C} + \ii\bar{S}\right) - \left(\bar{C} + \ii\bar{S}\right)\bar{C}\right] d\sigma_\rho d\tau_{12} d\tau_3 \\
&={}& &\frac{\kappa}{2} (T_t - T) \int\limits_{V_3} \int\limits_{V_{12}} \int\limits_{S_\rho} \left(\bar{S}^2 + \bar{C}^2\right) d\sigma_\rho d\tau_{12} d\tau_3.
\end{alignat}

For $l = 0$,
\begin{align}
\nonumber \delta I &= \frac{\kappa}{2} (T_t - T) \int\limits_{V_3} \int\limits_{V_{12}} \int\limits_{S_\rho} Y_{0,0}\left( \theta_\rho, \phi_\rho \right)^2 \Phi_{Ps}(r_{12})^2 \Phi_H(r_3)^2 \left[ \frac{\sin^2(\kappa\rho)}{\rho^2} + \frac{\cos^2(\kappa\rho)}{\rho^2}\right] \frac{2}{\kappa} d\sigma_\rho d\tau_{12} d\tau_3 \\
&= (T_t - T) \int\limits_{V_3} \int\limits_{V_{12}} \int\limits_{S_\rho} Y_{0,0}\left( \theta_\rho, \phi_\rho \right)^2 \Phi_{Ps}(r_{12})^2 \Phi_H(r_3)^2 \frac{1}{\rho^2} \rho^2 \sin\theta_\rho d\theta_\rho d\phi_\rho d\tau_{12} d\tau_3
\end{align}

Since the Ps and H functions are normalized,
\beq
\int\limits_{V_3}\! \Phi_H(r_3) d\tau_3 = 1 \text{ and } \int\limits_{V_{12}}\! \Phi_{Ps}(r_{12}) d\tau_{12} = 1.
\label{eq:PsHNormalization}
\eeq

This leaves us with
\beq
\delta I = (T_t - T) \int\limits_{S_\rho} Y_{0,0}\left( \theta_\rho, \phi_\rho \right)^2 \sin\theta_\rho d\theta_\rho d\phi_\rho
\eeq

Upon performing the angular integrations, we finally obtain our equation for the variational method of
\beq
\delta I = (T_t - T)
\eeq
or
\beq
I[\Psi_t] - I[\Psi] = I[\Psi_t] = T_t - T.
\eeq

\noindent Writing this as a variation gives
\beq
T_v = T_t - I[\Psi_t].
\label{eq:ComplexKohnVariation}
\eeq

Substituting our trial wavefunction (\ref{eq:TrialComplex}) into (\ref{IlDefPsi}) gives
\begin{align}
\nonumber I[\Psi_t] = \Big< &\bar{S} + T_t \, \bar{W} + \sum_{i} c_i' \bar{\phi_i}^t \;\Big| L \Big|\; \bar{S} + T_t \, \bar{W} + \sum_{i} c_i' \bar{\phi_i}^t \Big> \\
\nonumber = \Big( &\bar{S} L \bar{S} + T_t \bar{S} L \bar{W} + \bar{S} L \sum_{i} c_i \bar{\phi_i} + T_t \bar{W} L \bar{S} + T_t^2 \bar{W}L\bar{W} + T_t \bar{W} L \sum_{i} c_i \bar{\phi_i} \Big. \\
& \Big. \sum_{i} c_i \bar{\phi_i} L \bar{S} + T_t \sum_{i} c_i \bar{\phi_i} L \bar{W} + \sum_{i} \sum_{j} c_i c_j \bar{\phi_i} L \bar{\phi_j} \Big) 
\end{align}

\noindent Using the bra-ket notation, the bra should be conjugated, but as \cite{} points out, the conjugation should not be performed in these calculations.  All further calculations in this section implicitly omit the complex conjugation of the bra.

We can use this now in the stationary property of the complex Kohn functional, i.e.
\beq
\frac{\partial T_v}{\partial T_t} = 0  \text{ and } \frac{\partial T_v}{\partial c_i} = 0 \text{ where $i = 1,\ldots,N$}.
\label{eq:ComplexKohnStationary}
\eeq

Using (\ref{eq:WDef}) and (\ref{eq:SLCandCLSBar}),
\begin{align}
\nonumber (\bar{S},L \bar{W}) - (\bar{W},L \bar{S}) &= (\bar{S},L\bar{C}) + \ii(\bar{S},L\bar{S}) - (\bar{C},L\bar{S}) - \ii (\bar{S},L\bar{S}) \\
& = (\bar{S},L\bar{C}) - (\bar{C},L\bar{S}) = 1.
\label{eq:WLCandCLWBar}
\end{align}

\noindent Using (\ref{eq:PhiLSPerm}) and (\ref{eq:PhiLCPerm}),
\beq
(\bar{\phi}_i, L\bar{W}) = (\bar{W}, L\bar{\phi}_i).
\label{eq:PhiLWPerm}
\eeq

Performing the first variation and using (\ref{eq:PhiLWPerm}) and (\ref{eq:WLCandCLWBar}) gives
\begin{align}
\nonumber 0 &= 1 - \Big[(\bar{S},L \bar{W}) + (\bar{W},L \bar{S}) + 2\, T_t (\bar{W},L\bar{W}) + \sum_i c_i (\bar{W},L \bar{\phi_i}) + \sum_i c_i (\bar{\phi_i},L \bar{W}) \Big]. \\
&= 2\, (\bar{W},L \bar{S}) + 2\, T_t (\bar{W},L \bar{W}) + 2 \sum_i c_i (\bar{W},L \bar{\phi_i})
\label{eq:Complex1stVar1}
\end{align}

\noindent Rearranging, we have
\beq
-(\bar{W},L \bar{S}) = T_t (\bar{W},L \bar{W}) + \sum_i c_i (\bar{W},L \bar{\phi_i}).
\label{eq:Complex1stVar2}
\eeq

Performing the second variation with respect to an arbitrary $c_k$ yields
\beq
0 = (\bar{S},L \bar{\phi_k}) + T_t (\bar{W},L\bar{\phi_k}) + (\bar{\phi_k},L\bar{S}) + T_t (\bar{\phi_k},L\bar{W}) + \frac{\partial}{\partial c_k} \sum_i \sum_j c_i c_j (\bar{\phi_i},L\bar{\phi_j}).
\label{eq:Complex2ndVar1}
\eeq

\noindent With (\ref{eq:PhiLWPerm}), the last term evalutes to
\begin{align}
\nonumber \frac{\partial}{\partial c_k} \sum_i \sum_j c_i c_j (\bar{\phi_i},L\bar{\phi_j}) &= \sum_{j\neq k} c_j (\bar{\phi_k},L\bar{\phi_j}) + \sum_{i\neq k} c_i  (\bar{\phi_i},L\bar{\phi_k}) + 2\, c_k (\bar{\phi_k},L\bar{\phi_k}) \\
&= 2 \sum_{j\neq k} c_j (\bar{\phi_k},L\bar{\phi_j}) + 2 c_k (\bar{\phi_k},L\bar{\phi_k}) = 2 \sum_j c_j (\bar{\phi_k},L\bar{\phi_j}).
\end{align}

\noindent Substituting this back into (\ref{eq:Complex2ndVar1}) along with (\ref{eq:PhiLSPerm}) and rearranging gives our other linear equations of
\beq
-(\bar{\phi_k},L\bar{S}) = T_t (\bar{\phi_k},L\bar{W}) + \sum_j c_j (\bar{\phi_k},L\bar{\phi_j}).
\label{eq:Complex2ndVar2}
\eeq

Equations (\ref{eq:Complex1stVar2}) and (\ref{eq:Complex2ndVar2}) are then written in matrix form as

\begin{equation}
\label{eq:ComplexKohnMatrix}
\begin{bmatrix} 
 (\bar{W},L\bar{W}) & (\bar{W},L\bar{\phi}_1) & \cdots & (\bar{W},L\bar{\phi}_j) & \cdots\\
 (\bar{\phi}_1,L\bar{W}) & (\bar{\phi}_1,L\bar{\phi}_1) & \cdots & (\bar{\phi}_1,L\bar{\phi}_j) & \cdots\\
 \vdots & \vdots & \ddots & \vdots \\
 (\bar{\phi}_i,L\bar{W}) & (\bar{\phi}_i,L\bar{\phi}_1) & \cdots & (\bar{\phi}_i,L\bar{\phi}_j) & \cdots\\
 \vdots & \vdots & & \vdots & \\
\end{bmatrix}
\begin{bmatrix}
\Lambda_t\\
c_1\\
\vdots\\
c_i\\
\vdots
\end{bmatrix}
= -
\begin{bmatrix}
(\bar{W},L\bar{S}) \\
(\bar{\phi}_1,L\bar{S}) \\
\vdots \\
(\bar{\phi}_i,L\bar{S}) \\
\vdots
\end{bmatrix}.
\end{equation}

\noindent This matrix equation can be rewritten as
\beq
\textbf{\emph{AX = -B}}.
\eeq

\noindent Solving this for $\textbf{\emph{X}}$,
\beq
\textbf{\emph{X = $-A^{-1}$B}}.
\eeq


\section{Generalized Kohn}
\label{sec:GenKohn}
Previously for the Kohn and inverse Kohn methods, we chose $\gamma = 0$ and $\gamma = \frac{\pi}{2}$ respectively in equation \ref{eq:AsympExact}.  Cooper et al.\ perform a similar treatment for what is referred to as the generalized Kohn method, and we have adapted it to our particular wavefunction \cite{Cooper2009, Cooper2010}.  The form is

\begin{equation}
\tilde{\Psi}_t^\pm = \tilde{S} + \tilde{\Lambda}_t \tilde{C} + \sum_{i=1}^N c_i \bar{\phi_i}^t ,
\label{eq:TrialSimpleGeneral}
\end{equation}

\noindent with
\begin{equation}
\label{eq:GenKohnDef}
\tilde{\Lambda}_t = \tan(\eta_t-\tau)
\end{equation}

\noindent and

\begin{equation}
\begin{bmatrix}
\tilde{S} \\
\tilde{C}
\end{bmatrix}
=
\begin{bmatrix}
\cos(\tau) & \sin(\tau) \\
-\sin(\tau) & \cos(\tau)
\end{bmatrix}
\begin{bmatrix}
\bar{S} \\
\bar{C}
\end{bmatrix},
\end{equation}


\noindent where $\bar{S}$, $\bar{C}$ and $\bar{\phi}_i$ are defined by (\ref{SCphiBarDef}).  The derivation to determine the phase shifts is very similar to that of the Kohn method, but the $\tau$ parameter is taken into account.  The matrix equation for the generalized Kohn method is similar in form to the Kohn method given by equation (\ref{eq:KohnMatrix}).

\begin{equation}
\label{eq:GenKohnMatrix}
\begin{bmatrix} 
 (\tilde{C},L\tilde{C}) & (\tilde{C},L\tilde{\phi}_1) & \cdots & (\tilde{C},L\bar{\phi}_j) & \cdots\\
 (\bar{\phi}_1,L\tilde{C}) & (\bar{\phi}_1,L\bar{\phi}_1) & \cdots & (\bar{\phi}_1,L\bar{\phi}_j) & \cdots\\
 \vdots & \vdots & \ddots & \vdots \\
 (\bar{\phi}_i,L\tilde{C}) & (\bar{\phi}_i,L\bar{\phi}_1) & \cdots & (\bar{\phi}_i,L\bar{\phi}_j) & \cdots\\
 \vdots & \vdots & & \vdots & \\
\end{bmatrix}
\begin{bmatrix}
\tilde{\Lambda}_t\\
\tilde{c}_1\\
\vdots\\
\tilde{c}_i\\
\vdots
\end{bmatrix}
= -
\begin{bmatrix}
(\tilde{C},L\tilde{S}) \\
(\bar{\phi}_1,L\tilde{S}) \\
\vdots \\
(\bar{\phi}_i,L\tilde{S}) \\
\vdots
\end{bmatrix}.
\end{equation}

Discussion on the use of this method is provided in the later section \ref{sec:CompGenKohn}.


\end{document}
