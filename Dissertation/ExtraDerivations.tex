\documentclass[main.tex]{subfiles} 
\begin{document}

\newpage
\chapter{Extra Derivations}
\label{chp:ExtraDer}

\section{\texorpdfstring{$_2F_1$}{Hypergeometric} Identity}
\label{sec:Hypergeometric}

The backwards recursion relation for the hypergeometric function is given in references \cite{Drake1995} and \cite{Frolov2003} as
\beq
\label{eq:HyperIdentity}
_2F_1(1,a;c;z) = 1 + \left( \frac{a}{c} \right) z \,\,_2F_1(1,a+1;c+1;z).
\eeq

From Abramowitz and Stegun \cite{Abramowitz1965}, the definition of the hypergeometric function is given by
\beq
\label{eq:HyperDef}
_2F_1(\alpha,\beta;\gamma;z) = 1 + \sum_{n=1}^{\infty} \frac{(\alpha)_n \cdot (\beta)_n}{(\gamma)_n} \frac{z^n}{n!},
\eeq
where $(x)_n$ is the Pochhammer symbol given by \cite{Abramowitz1965}
\beq
\label{eq:PochhammerDef}
(x)_n \equiv \frac{\Gamma(x+n)}{\Gamma(x)} = x(x+1) \cdots  (x+n-1)
\eeq
with $(x)_0 = 1$.  A special case is $(x)_1 = n!$, where $n!$ is the factorial.

Using the above definition of the Pochhammer symbol, we can easily see that
\beq
\label{eq:PochPlus1}
(x+1)_n = \frac{(x+n)}{(x)} \cdot (x)_n = \frac{(x)_{n+1}}{x}.
\eeq

From \ref{eq:HyperDef} and \ref{eq:PochPlus1}, 
\begin{align}
\nonumber _2F_1(1,a+1;c+1;z) &= 1 + \sum_{n=1}^{\infty} \frac{(1)_n \cdot (a+1)_n}{(c+1)_n} \frac{z^n}{n!} = 1 + \sum_{n=1}^{\infty} \frac{(a+1)_n}{(c+1)_n} z^n \\
&= 1 + \left(\frac{c}{a}\right) \sum_{n=1}^{\infty} \frac{(a)_{n+1}}{(c)_{n+1}} z^n = 1 + \left(\frac{c}{a}\right) \sum_{n=2}^{\infty} \frac{(a)_n}{(c)_n} z^{n-1}.
\end{align}
Multiplying by $\left(\frac{c}{a}\right) z$, we now have
\beq
\label{eq:Hyper1}
\left(\frac{c}{a}\right) z _2F_1(1,a+1;c+1;z) = \left(\frac{c}{a}\right) z + \sum_{n=2}^{\infty} \frac{(a)_n}{(c)_n} z^n = \sum_{n=1}^{\infty} \frac{(a)_n}{(c)_n} z^n.
\eeq

From the definition in equation \ref{eq:HyperDef},
\beq
\label{eq:Hyper2}
_2F_1(1,a;c;z) = 1 + \sum_{n=1}^{\infty} \frac{(a)_n}{(c)_n} z^n,
\eeq
which, when combined with equation \ref{eq:Hyper1}, gives the final result of
\beq
_2F_1(1,a;c;z) = 1 + \left( \frac{a}{c} \right) z \,\,_2F_1(1,a+1;c+1;z).
\eeq

\end{document}