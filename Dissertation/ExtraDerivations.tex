% -*- root: Dissertation.tex -*-
\documentclass[Dissertation.tex]{subfiles} 
\begin{document}

\chapter{Extra Derivations}
\label{chp:ExtraDer}

\iftoggle{UNT}{This}{\lettrine{\textcolor{startcolor}{T}}{his}}
Appendix mainly contains short derivations and equations that are not 
critical to understanding the main results but are nonetheless needed for 
this work.

\section{\texorpdfstring{$\rho$ and $\rho'$} {rho and rho'} Definitions}
\label{sec:RhoDef}
From equation 2.1 of Peter Van Reeth's thesis \cite{VanReethThesis} and Armour and Humberston's paper \cite{Armour1991},

\beq
\bm{\rho} = \frac{1}{2} \left( \bm{r}_1 + \bm{r}_2 \right).
\label{eq:RhoDef1}
\eeq
By switching coordinates 2 and 3, we have
\beq
\bm{\rho}' = \frac{1}{2} \left( \bm{r}_1 + \bm{r}_3 \right).
\label{eq:RhoDef2}
\eeq
In the original coordinate system (\cref{fig:CoordinateSystemOriginal}),
\begin{align}
\nonumber \bm{r}_1 &= \left( r_1 \sin \theta_1 \cos \varphi_1, r_1 \sin \theta_1 \sin \varphi_1, r_1 \cos \theta_1 \right) \\
\bm{r}_3 &= \left( r_3 \sin \theta_3 \cos \varphi_3, r_3 \sin \theta_3 \sin \varphi_3, r_3 \cos \theta_3 \right)
\end{align}
In the rotated coordinate system (\cref{fig:CoordinateSystemRotated}),
\begin{align}
\nonumber \bm{r}_1 &= (0, 0, r_1) \\
\bm{r}_3 &= \left( r_3 \sin \theta_{13} \cos \varphi_{13}, r_3 \sin \theta_{13} \sin \varphi_{13}, r_3 \cos \theta_{13} \right)
\end{align}
\begin{align}
\nonumber \left| r_1 + r_3 \right|^2 &= r_3^2 \sin^2 \theta_{13} \cos^2 \varphi_{13} + r_3^2 \sin^2 \theta_{13} \sin^2 \varphi_{13} + (r_1 + r_3 \cos \theta_{13})^2\\
\nonumber &= r_3^2 \sin^2 \theta_{13} + r_1^2 + r_3^2 \cos^2 \theta_{13} + 2 r_1 r_3 \cos \theta_{13} \\
&= r_1^2 + r_3^2 + 2 r_1 r_3 \cos \theta_{13}.
\label{eq:RhoDef3}
\end{align}

Also, using the law of cosines,
\begin{equation}
r_{13}^2 = r_1^2 + r_3^2 - 2 r_1 r_3 \cos \theta_{13}
\label{eq:CosLaw}
\end{equation}
Substituting \cref{eq:CosLaw} into \cref{eq:RhoDef3} gives
\begin{equation}
\left| r_1 + r_3 \right|^2 = 2 \left( r_1^2 + r_3^2 \right) - r_{13}^2.
\label{eq:RhoDef4}
\end{equation}
From \cref{eq:RhoDef2,eq:RhoDef4},
\begin{equation}
\label{eq:RhopRDef}
\rho' = \frac{1}{2} \left[ 2 \left(r_1^2 + r_3^2 \right) - r_{13}^2 \right] ^ \frac{1}{2}.
\end{equation}
Similarly,
\begin{equation}
\label{eq:RhoRDef}
\rho = \frac{1}{2} \left[ 2 \left(r_1^2 + r_2^2 \right) - r_{12}^2 \right] ^ \frac{1}{2}.
\end{equation}


\section{Perimetric Coordinates}
\label{sec:PerimetricCoords}

Perimetric coordinates are used for the long-long integrations in the S-wave
code. If perimetric coordinates are used for $r_1$, $r_2$ and $r_{12}$, then
these are defined by \cite{Armour1991}

\begin{align}
\label{eq:PerimetricCoords1}
\nonumber x &= r_1 + r_2 - r_{12} \\
\nonumber y &= r_2 + r_{12} - r_1 \\
z &= r_{12} + r_1 - r_2.
\end{align}
These can alternately be written as
\begin{align}
\label{eq:PerimetricCoords2}
\nonumber r_1 &= \frac{x+z}{2} \\
\nonumber r_2 &= \frac{x+y}{2} \\
r_{12} &= \frac{y+z}{2}.
\end{align}

From \cref{eq:dTauS23}, the volume element after integration over the external angles is
\beq
d\tau = 8\pi^2 dr_1 r_2 dr_2 r_3 dr_3 r_{12} dr_{12} r_{13} dr_{13} d\varphi_{23}.
\eeq
We need to perform a change of variables to use perimetric coordinates for $r_1$, $r_2$ and $r_{12}$. The Jacobian is
\beq
\label{eq:PerimetricJacobian}
J(x,y,z) = 
\left| {\begin{array}{ccc}
 \frac{\partial r_1}{\partial x} & \frac{\partial r_1}{\partial y} & \frac{\partial r_1}{\partial z}  \\
 \frac{\partial r_2}{\partial x} & \frac{\partial r_2}{\partial y} & \frac{\partial r_2}{\partial z}  \\
 \frac{\partial r_{12}}{\partial x} & \frac{\partial r_{12}}{\partial y} & \frac{\partial r_{12}}{\partial z}  \\
 \end{array} } \right|
=
\left| {\begin{array}{ccc}
 \frac{1}{2} & 0 & \frac{1}{2} \\
 \frac{1}{2} & \frac{1}{2} & 0 \\
 0 & \frac{1}{2} & \frac{1}{2}
 \end{array} } \right|
=
\frac{1}{4}.
\eeq

\noindent This gives a transformed volume element of
\beq
\label{eq:PerimetricVolEl}
d\tau = 2\pi^2 r_2 r_3 r_{12} r_{13} dx\, dy\, dz\, dr_3\, dr_{13}\, d\varphi_{23}.
\eeq

\noindent The limits for each of the perimetric coordinates are 0 to $\infty$.


\section{Spherical Bessel Derivatives}
\label{sec:SphBess}

%\begin{figure}
	%\centering
	%\includegraphics[width=3.5in]{Untitled-2}
	%\caption{Mathematica}
	%\label{fig:Untitled-2}
%\end{figure}

\subsection{First Derivative}
\label{sec:SphBess1}

From Abramowitz and Stegun \cite[p.437]{Abramowitz1965},
\begin{subequations}
\begin{align}
j_\ell(z) &= z^{-1} \left[ P(\ell+\tfrac{1}{2}, z ) \sin(z-\tfrac{1}{2}\ell\pi) + Q(\ell+\tfrac{1}{2},z) \cos(z-\tfrac{1}{2}\ell\pi) \right] \\
n_\ell(z) &= (-1)^{\ell+1} z^{-1} \left[ P(\ell+\tfrac{1}{2}, z ) \cos(z+\tfrac{1}{2}\ell\pi) - Q(\ell+\tfrac{1}{2},z) \sin(z+\tfrac{1}{2}\ell\pi) \right].
\end{align}
\end{subequations}
where
\begin{subequations}
\begin{align}
P(\ell+\tfrac{1}{2}, z) &= 1 - \frac{\Factorial{\ell+2}}{\Factorial{2} \GammaFunc{\ell-1}} (2z)^{-2} + \ldots \\
Q(\ell+\tfrac{1}{2}, z) &= \frac{\Factorial{\ell+1}}{\Factorial{1} \GammaFunc{\ell}} (2z)^{-1} - \frac{\Factorial{n+3}}{\Factorial{3} \GammaFunc{\ell-2}} (2z)^{-3} + \ldots
\end{align}
\end{subequations}
Using \emph{Mathematica} with these expansions, we get
\begin{subequations}
\begin{align}
\label{eq:SphBesExpan}
j_\ell(z) &= \frac{\sin(z - \tfrac{1}{2}\ell\pi)}{z} + \ldots \\
n_\ell(z) &= \frac{(-1)^{\ell+1} \cos(z + \tfrac{1}{2}\ell\pi)}{z} + \ldots
\end{align}
\end{subequations}
and
\begin{subequations}
\begin{align}
\label{eq:SphBesExpanDer}
{j_\ell}^\prime(z) &= \frac{\cos(z - \tfrac{1}{2}\ell\pi)}{z} + \ldots \\
{n_\ell}^\prime(z) &= \frac{(-1)^{\ell+2} \sin(z + \tfrac{1}{2}\ell\pi)}{z} + \ldots
\end{align}
\end{subequations}

In a more general format than that of Abramowitz and Stegun \cite[p.73]{Abramowitz1965},
\beq
\cos \left(z-\frac{\ell \pi}{2} \right) = \begin{cases} (-1)^{\ell/2} \cos z & \mbox{if } \ell \mbox{ is even} \\ 
(-1)^{(\ell-1)/2} \sin z & \mbox{if } \ell \mbox{ is odd} \end{cases} 
\eeq
and
\beq
\sin \left(z-\frac{\ell \pi}{2} \right) = \begin{cases} (-1)^{\ell/2} \sin z & \mbox{if } \ell \mbox{ is even} \\ 
(-1)^{(\ell+1)/2} \cos z & \mbox{if } \ell \mbox{ is odd} \end{cases} .
\eeq
Also,
\beq
\cos \left(z+\frac{\ell \pi}{2} \right) = \begin{cases} (-1)^{\ell/2} \cos z & \mbox{if } \ell \mbox{ is even} \\ 
(-1)^{(\ell+1)/2} \sin z & \mbox{if } \ell \mbox{ is odd} \end{cases} 
\eeq
and
\beq
\sin \left(z+\frac{\ell \pi}{2} \right) = \begin{cases} (-1)^{\ell/2} \sin z & \mbox{if } \ell \mbox{ is even} \\ 
(-1)^{(\ell-1)/2} \cos z & \mbox{if } \ell \mbox{ is odd} \end{cases} .
\eeq
Using these with \cref{eq:SphBesExpan,eq:SphBesExpanDer}, we see that to first order, there is a relationship between these functions and their derivatives given by
\begin{subequations}
\begin{align}
\label{eq:SphBesDerRel}
{j_\ell}^\prime(z) &\approx -n_\ell(z) \\
{n_\ell}^\prime(z) &\approx j_\ell(z).
\end{align}
\end{subequations}
This allows us to write the gradient of $\widetilde{S}_\ell$ and
$\widetilde{C}_\ell$ for arbitrary $\ell$ to first order in \cref{eq:GradSC}.


\subsection{Second Derivative}
\label{sec:SphBess2}

The output we get from the \emph{Mathematica} notebook ``First Partial Waves LS General.nb'' \cite{GitHub,Wiki} is
\begin{align}
\label{eq:LSGenLapl1}
\frac{\Laplacian_\rho \left[\SphericalHarmonicY{\ell}{0}{\theta_\rho}{\varphi_\rho} j_\ell(\kappa\rho) \right]}{\SphericalHarmonicY{\ell}{0}{\theta_\rho}{\varphi_\rho} j_\ell(\kappa\rho)} = \frac{(n + n^2 - \kappa^2 \rho^2) \LegendreP{\ell, \cos\theta} + 2 \cot\theta \, \AssocLegendreP{\ell}{1}{\cos\theta} + \AssocLegendreP{\ell}{2}{\cos\theta}} {\rho^2 \LegendreP{\ell, \cos\theta}}.
\end{align}

From Ref.~\cite{WolframPnm}, a recurrence relation for the associated Legendre polynomials is
\beq
\AssocLegendreP{\ell}{\mu+1}{z} - \left[\mu(\mu-1) - \ell(\ell+1)\right] \AssocLegendreP{\ell}{\mu-1}{z} + \frac{2 \mu z}{\sqrt{1-z^2}} \AssocLegendreP{\ell}{\mu}{z} = 0.
\eeq
Note that other books \cite{Abramowitz1965,Zwillinger2003} give slightly different forms of this recurrence relation. If we set $\mu = 1$ and $z = \cos\theta$, this becomes
\beq
\AssocLegendreP{\ell}{2}{\cos\theta} + \left(\ell^2+\ell \right) \AssocLegendreP{\ell}{0}{z} + \frac{2 \cos\theta}{\sin\theta} \AssocLegendreP{\ell}{1}{\cos\theta} = 0.
\eeq
Using the definition of $\cot$ and solving for $\left(\ell^2+\ell \right)\AssocLegendreP{\ell}{0}{z} = \left(\ell^2+\ell \right)\LegendreP{\ell, \cos\theta}$,
\beq
\left(\ell^2+\ell \right)\LegendreP{\ell, \cos\theta} = -\AssocLegendreP{\ell}{2}{\cos\theta} - 2 \cot\theta \, \AssocLegendreP{\ell}{1}{\cos\theta}.
\eeq
Substituting this back into \cref{eq:LSGenLapl1}, we get cancellations with the associated Legendre polynomials and see that $\SphericalHarmonicY{\ell}{0}{\theta_\rho}{\varphi_\rho} j_\ell(\kappa\rho)$ is an eigenfunction of $\Laplacian_\rho$ with eigenvalue $-\kappa^2$:
\begin{align}
\label{eq:LSGenLapl2}
\Laplacian_\rho \left[\SphericalHarmonicY{\ell}{0}{\theta_\rho}{\varphi_\rho} j_\ell(\kappa\rho) \right] = \frac{(-\kappa^2 \rho^2) \LegendreP{\ell, \cos\theta} } {\rho^2 \LegendreP{\ell, \cos\theta}}
= -\kappa^2 \, \SphericalHarmonicY{\ell}{0}{\theta_\rho}{\varphi_\rho} j_\ell(\kappa\rho).
\end{align}



\section{\texorpdfstring{$_2F_1$}{Hypergeometric} Recursion Relation}
\label{sec:Hypergeometric}

The backwards recursion relation for the hypergeometric function is used in the
short-range code (\cref{sec:CompShort}). This is given in Refs.~\cite{Drake1995,Frolov2003} as
\beq
\label{eq:HyperIdentity}
\Hypergeometric{2}{1}{1,a}{c}{z} = 1 + \left( \frac{a}{c} \right) z \,\, \Hypergeometric{2}{1}{1,a+1}{c+1}{z}.
\eeq

From Abramowitz and Stegun \cite{Abramowitz1965}, the definition of the hypergeometric function is given by
\beq
\label{eq:HyperDef}
_2F_1(\alpha,\beta;\gamma;z) = 1 + \sum_{n=1}^{\infty} \frac{(\alpha)_n \cdot (\beta)_n}{(\gamma)_n} \frac{z^n}{n!},
\eeq
where $(x)_n$ is the Pochhammer symbol given by \cite{Abramowitz1965}
\beq
\label{eq:PochhammerDef}
(x)_n \equiv \frac{\Gamma(x+n)}{\Gamma(x)} = x(x+1) \cdots  (x+n-1)
\eeq
with $(x)_0 = 1$.  A special case is $(x)_1 = n!$, where $n!$ is the factorial.

Using the above definition of the Pochhammer symbol, we can easily see that
\beq
\label{eq:PochPlus1}
(x+1)_n = \frac{(x+n)}{(x)} \cdot (x)_n = \frac{(x)_{n+1}}{x}.
\eeq
From \cref{eq:HyperDef,eq:PochPlus1}, 
\begin{align}
\nonumber _2F_1(1,a+1;c+1;z) &= 1 + \sum_{n=1}^{\infty} \frac{(1)_n \cdot (a+1)_n}{(c+1)_n} \frac{z^n}{n!} = 1 + \sum_{n=1}^{\infty} \frac{(a+1)_n}{(c+1)_n} z^n \\
&= 1 + \left(\frac{c}{a}\right) \sum_{n=1}^{\infty} \frac{(a)_{n+1}}{(c)_{n+1}} z^n = 1 + \left(\frac{c}{a}\right) \sum_{n=2}^{\infty} \frac{(a)_n}{(c)_n} z^{n-1}.
\end{align}
Multiplying by $\left(\frac{c}{a}\right) z$, we now have
\beq
\label{eq:Hyper1}
\left(\frac{c}{a}\right) z \,\, _2F_1(1,a+1;c+1;z) = \left(\frac{c}{a}\right) z + \sum_{n=2}^{\infty} \frac{(a)_n}{(c)_n} z^n = \sum_{n=1}^{\infty} \frac{(a)_n}{(c)_n} z^n.
\eeq
From the definition in \cref{eq:HyperDef},
\beq
\label{eq:Hyper2}
_2F_1(1,a;c;z) = 1 + \sum_{n=1}^{\infty} \frac{(a)_n}{(c)_n} z^n,
\eeq
which, when combined with \cref{eq:Hyper1}, gives the final result of
\beq
_2F_1(1,a;c;z) = 1 + \left( \frac{a}{c} \right) z \,\,_2F_1(1,a+1;c+1;z).
\eeq


\section{Shielding Function}
\label{sec:ShieldingFunc}

The spherical Neumann functions, $n_\ell(\kappa\rho)$, in $C_\ell$ go to
$-\infty$ at the origin, and we remove this singularity with the shielding
function. The shielding function given by \cref{eq:PartialWaveShielding},
$f_\ell(\rho) = \left[1 - \ee^{-\mu \rho} \left(1+\frac{\mu}{2}\rho\right)\right]^{m_\ell}$,
is slightly different than earlier work \cite{VanReeth2003,VanReeth2004} on
Ps-H scattering, which used $f(\rho) = (1 - \ee^{-\lambda \rho})^3$.

This work is based on notes from Van Reeth \cite{VanReethPrivate}\
for the S-wave shielding function.
We want to have $C_\ell$ behaving similar to $S_\ell$ at the origin. To
accomplish this, we take a series expansion of both to see what the dominant
terms are. To more easily see the behavior, we only take expansions of the
spherical Bessel and Neumann functions for the S-wave. For $S_0$, the series expansion of
$j_0$ is
\begin{equation}
j_0(\kappa\rho) \sim 1 - \frac{\kappa^2 \rho^2}{6} + \frac{\kappa^4 \rho^4}{120} + \mathcal{O}(\rho^5).
\end{equation}
For $C_0$, the series expansion of $n_0$ is
\begin{equation}
n_0(\kappa\rho) \sim -\frac{1}{\kappa\rho} + \frac{\kappa \rho}{2} - \frac{\kappa^3 \rho^3}{24} + \mathcal{O}(\rho^4).
\end{equation}
The singular nature at the origin is easily seen by the first term. To have
$C_0 \sim S_0$ at the origin, the shielding function needs to change the
leading term to a constant. If we choose a shielding function of the form
$\left[1 - \ee^{-\mu \rho} \left(1+a \rho\right)\right]$, the
expansion at the origin of this multiplied by the spherical Neumann function is
\begin{equation}
n_0(\kappa\rho) \left[1 - \ee^{-\mu \rho} \left(1+a \rho\right)\right] \sim
-\frac{\mu-a}{\kappa} - \frac{\left(a\mu - \frac{\mu^2}{2}\right)\rho}{\kappa}
+ \left(\frac{1}{2}\kappa(\mu-a) - \frac{\mu^2 - 3 a \mu^2}{6\kappa}\right) \rho^2 + \ldots
\end{equation}
This is no longer singular, but it has a $\rho$ term, so if we set
$a = \frac{\mu}{2}$, the second term disappears, leaving us with
\begin{equation}
n_0(\kappa\rho) \left[1 - \ee^{-\mu \rho} \left(1+\frac{\mu}{2}\rho\right)\right] \sim
-\frac{\mu}{2\kappa} + \left(\frac{\kappa\mu}{4} + \frac{\mu^3}{12\kappa}\right) \rho^2 + \ldots \ .
\end{equation}
This shows that with the choice of shielding function in \cref{eq:PartialWaveShielding},
we have $C_0$ behaving similar to $S_0$ at the origin.

The \emph{Mathematica} notebook ``Shielding Factor.nb'' found on the GitHub
page \cite{GitHub} shows these expansions and the expansions for the P-wave
and D-wave. We normally choose $m_\ell = (2\ell+1)$, but for the D-wave, we
used $m_\ell = 7$ when we were trying to improve the convergence, which
ultimately did not give improved results over $m_\ell = 5$. This notebook
also shows the first and second derivatives of the shielding function given
in \cref{eq:Shielding1Der,eq:Shielding2Der}.

This notebook also has interactive graphs that show how the shielding function
with and without the spherical Neumann function behaves with differing $m_\ell$,
$\mu$, and $\ell$ values. \Cref{fig:shielding-func-n} shows that as $m_\ell$
increases for fixed $\mu$ of 0.9, it takes a larger $\rho$ before $C_\ell$
becomes significant. \Cref{fig:shielding-func-mu} keeps $m_\ell$ constant at 7
and varies $\mu$. This figure shows that smaller $\mu$ values give a strong
contribution for $C_\ell$. \Cref{tab:Nonlinear} shows the $\mu$ and $m_\ell$
values we use for each partial wave.
\begin{figure}
	\centering
	\includegraphics[width=5in]{shielding-func-n}
	\caption[Shielding function $f_\ell$ variation with respect to $\rho$ for multiple values of $m_\ell$]{Shielding function $f_\ell$ variation with respect to $\rho$ for multiple values of $m_\ell$ with $\mu = 0.9$}
	\label{fig:shielding-func-n}
\end{figure}
\begin{figure}
	\centering
	\includegraphics[width=5in]{shielding-func-mu}
	\caption[Shielding function $f_\ell$ variation with respect to $\rho$ for multiple values of $\mu$]{Shielding function $f_\ell$ variation with respect to $\rho$ for multiple values of $\mu$ with $m_\ell = 7$}
	\label{fig:shielding-func-mu}
\end{figure}


\section{D-Wave Mixed Symmetry Terms}
\label{sec:MixedDerivation}

This derivation proves \cref{eq:MixedAngSimple}. Using \cref{eq:MixedAng} and
substituting the appropriate spherical harmonics and Clebsch-Gordan coefficients,
\begin{align}
\label{eq:Psi1}
\psi_{(1,1,2,0)} = &\sum_{m=-1}^{+1} Y_{1,m}(\theta_1,\varphi_1) Y_{1,m}(\theta_2,\varphi_2) \left< 1,m; 1,-m,0 | 2,0 \right> \nonumber \\
	= &-\sqrt{\frac{3}{8\uppi}} \sin\theta_1 \ee^{-\ii \varphi_1} \sqrt{\frac{3}{8\uppi}} \sin\theta_2 \ee^{\ii \varphi_1} \frac{1}{\sqrt{6}} \nonumber \\
	& + \sqrt{\frac{3}{4\uppi}} \cos\theta_1 \sqrt{\frac{3}{4\uppi}} \sin\theta_2 \frac{2}{\sqrt{6}} \nonumber \\
	& -\sqrt{\frac{3}{8\uppi}} \sin\theta_1 \ee^{\ii \varphi_1} \sqrt{\frac{3}{8\uppi}} \sin\theta_2 \ee^{-\ii \varphi_1} \frac{1}{\sqrt{6}}.
\end{align}
Using \cite[p.192]{VanReethThesis}
\beq
\cos\theta_{12} = \sin\theta_1 \sin\theta_2 \cos(\varphi_1 - \varphi_2) + \cos\theta_1 \cos\theta_2
\eeq
%and the standard
%\beq
%\cos\theta = \frac{\ee^{\ii \theta} + \ee^{-\ii \theta}}{2}
%\eeq
in \cref{eq:Psi1}, we obtain \cref{eq:MixedAngSimple}:
\begin{align}
\label{eq:Psi2}
\psi_{(1,1,2,0)} &= -\frac{3}{8\uppi} \frac{1}{\sqrt{6}} \sin\theta_1 \sin\theta_2 \cdot 2 \cos(\varphi_1 - \varphi_2) + \frac{3}{4\uppi} \frac{1}{\sqrt{6}} \cdot 2 \cos\theta_1 \cos\theta_2 \nonumber \\
&= \frac{3}{4\uppi} \frac{1}{\sqrt{6}} \left(3 \cos\theta_1 \cos\theta_2 - \cos\theta_{12} \right).
\end{align}


%\section{Fano Resonances}
%\label{sec:FanoResonances}
%As mentioned in \cref{sec:Resonances}, the resonances we encounter in Ps-H
%scattering are Fano resonances. Breit-Wigner resonances are a special case of Fano
%resonances with $q = \pm\infty$.
%Fano resonances correspond to a resonance with a flat cross section 
%background, which we do not have, as seen later in \cref{fig:singlet-cross-sections}.
%Fano's equation for the cross section is \cite[p.596]{Bransden2003}
%\beq
%\label{eq:FanoRes}
%\sigma_\ell = \frac{4\pi}{\kappa^2} (2\ell+1) \sin^2 \xi_\ell \frac{(q+
%\epsilon)^2}{1+\epsilon^2},
%\eeq
%with reduced energy given by
%\beq
%\epsilon = \frac{E - E_r}{\Gamma/2}
%\eeq
%and resonance shape parameter
%\beq
%q = -\cot \xi_\ell.
%\eeq
%The effect of this $q$ parameter is seen in \cref{fig:FanoCross}. This allows 
%the cross section to have an asymmetric shape near a resonance. Notice that 
%the special case of a Breit-Wigner resonance is given as $q = \pm\infty$. The 
%dotted line is the curve $4 \pi (2\ell+1) / \kappa^2$, which gives an 
%envelope to this function. An interactive demonstration of Fano resonances
%(but restricted to $q \geq 0$ is available from the Wolfram Demonstrations
%Project \cite{FanoDemo}. The fitting in \cref{eq:ResonanceCurve} is preferred
%over \cref{eq:FanoRes}, since it is simple to add more resonance terms,
%and there is no additional $q$ parameter to determine.
%
%\begin{figure}
	%\centering
	%\includegraphics[width=5.25in]{fano-cross}
	%\caption[Cross sections of Fano resonances]{Cross sections of Fano 
%resonances with differing values of $q$, shown for S-wave scattering}
	%\label{fig:FanoCross}
%\end{figure}


%\section{D-Wave Second Formalism}
%\label{sec:DSecondForm}
%For the D-wave second formalism, similar to the P-wave second formalism
%(\cref{sec:PWave2Formalism}), the angular momentum is placed on the Ps and
%H, so we would need
%\begin{subequations}
%\label{eq:DWave2ndPhiBar}
%\begin{align}
%\bar{\phi}_{\rho i} &= \left(1 \pm P_{23}\right) Y_{20}(\theta_\rho) \rho^2 \phi_i \label{eq:DWave2ndPhi1i}\\
%\bar{\phi}_{3j} &= \left(1 \pm P_{23}\right) Y_{20}(\theta_3) r_3^2 \phi_j \label{eq:DWave2ndPhi2j}.
%\end{align}
%\end{subequations}
%It will be shown that these are sufficient to describe this system without 
%having a third set of terms as in \cref{eq:DWaveTrial}.
%
%To use these in the short-range code, we need an expression similar to
%\cref{eq:P2rhoY10}. We cannot use that exact expression, as we have $\rho^2$. We
%follow the same procedure as that of obtaining \cref{eq:P2rhoY10}, but the
%spherical harmonic is more complicated. Full details of the derivation are
%found in the ``Second Formalism P-wave and D-wave.nb'' notebook found on
%figshare \cite{figshare}.
%Substituting \cref{eq:CosRho} into \cref{eq:DWaveSpherHarm}, multiplying by
%$\rho^2$, and then using \cref{eq:RhoRDef}, we obtain
%\begin{equation}
%\label{eq:rhoident2}
%\rho^2 Y_{20}(\theta_\rho) = \frac{1}{4} \left[r_1^2 Y_{20}(\theta_1) + r_2^2 Y_{20}(\theta_2) \right] + \frac{1}{2} \sqrt{\frac{5}{16 \pi}} 
   %r_1 r_2 \left[3 \cos\theta_1 \cos\theta_2 - \cos\theta_{12} \right].
%\end{equation}
%The first set of brackets is obviously similar to \cref{eq:P2rhoY10} but with
%$1 \to 2$ coming from the P-wave to the D-wave. More interesting though is the
%second set of brackets, which is clearly the mixed terms in
%\cref{eq:DWavePhi12k,eq:MixedAngSimple}.
%
%As mentioned in \cref{sec:MixedTerms}, we do not use the mixed terms in the
%D-wave calculation due to their complexity. Because of this, the second
%formalism has not been implemented for the D-wave. % and could be a source of
%%future work.


\section{Spherical Functions}
\label{sec:SphericalFunc}
%

This section gives the spherical harmonics, spherical Bessel functions, and
spherical Neumann functions through $\ell = 5$ for easier reference. These
were all obtained using the appropriate functions in \emph{Mathematica}
\cite{Mathematica}.

\todoi{Arfken and Weber? Abramowitz?}

{
\renewcommand{\arraystretch}{1.5}
\begin{table}
\centering
\begin{tabular}{l l}
\toprule\\[-1.2cm]
Partial Wave & $\SphericalHarmonicY{\ell}{0}{\theta}{\phi}$ \\
\midrule
S-Wave & $\frac{1}{\sqrt{4\pi}}$ \\
P-Wave & $\sqrt{\frac{3}{4\pi}} \cos\theta$ \\
D-Wave & $\sqrt{\frac{5}{16\pi}} (3\cos^2\theta - 1)$ \\
F-Wave & $\sqrt{\frac{7}{16\pi}} \left(5 \cos^3\theta - 3 \cos\theta \right)$ \\
G-Wave & $\sqrt{\frac{9}{256\pi}} \left(35 \cos^4\theta - 30 \cos^2\theta + 3 \right)$ \\
H-Wave & $\sqrt{\frac{11}{256\pi}} \left(63 \cos^5\theta - 70 \cos^3\theta + 15 \cos\theta \right)$ \\
\bottomrule
\end{tabular}
\caption{Spherical harmonics for partial waves $\ell = 0$ through 5}
\label{tab:SphHarm}
\end{table}
}

{
\renewcommand{\arraystretch}{1.5}
\begin{table}
\centering
\begin{tabular}{l l}
\toprule\\[-1.2cm]
Partial Wave & $j_\ell(z)$ \\
\midrule
S-Wave & $\frac{\sin(\kappa\rho)}{\kappa\rho}$ \\
P-Wave & $\frac{\sin z}{z^2} - \frac{\cos z}{z}$ \\
D-Wave & $\left(\frac{3}{z^3}-\frac{1}{z}\right)\sin z - \frac{3}{z^2}\cos z$ \\
F-Wave & $\frac{\left(z^2-15\right) \cos z}{z^3}-\frac{3 \left(2 z^2-5\right) \sin z}{z^4}$ \\
G-Wave & $\frac{5 \left(2 z^2-21\right) \cos z}{z^4}+\frac{\left(z^4-45 z^2+105\right) \sin z}{z^5}$ \\
H-Wave & $\frac{15 \left(z^4-28 z^2+63\right) \sin z}{z^6}+\frac{\left(-z^4+105 z^2-945\right) \cos z}{z^5}$ \\
\bottomrule
\end{tabular}
\caption{Spherical Bessel functions for partial waves $\ell = 0$ through 5}
\label{tab:SphBess}
\end{table}
}

{
\renewcommand{\arraystretch}{1.5}
\begin{table}
\centering
\begin{tabular}{l l}
\toprule\\[-1.2cm]
Partial Wave & $n_\ell(z)$ \\
\midrule
S-Wave & $-\frac{\cos z}{z}$ \\
P-Wave & $-\frac{\cos z}{z^2} - \frac{\sin z}{z}$ \\
D-Wave & $-\left(\frac{3}{z^3}-\frac{1}{z}\right)\cos z - \frac{3}{z^2}\sin z$ \\
F-Wave & $\frac{3 \left(2 z^2-5\right) \cos z}{z^4}+\frac{\left(z^2-15\right) \sin z}{z^3}$ \\
G-Wave & $\frac{5 \left(2 z^2-21\right) \sin z}{z^4}+\frac{\left(-z^4+45 z^2-105\right) \cos z}{z^5}$ \\
H-Wave & $\frac{\left(-z^4+105 z^2-945\right) \sin z}{z^5}-\frac{15 \left(z^4-28 z^2+63\right) \cos z}{z^6}$ \\
\bottomrule
\end{tabular}
\caption{Spherical Neumann functions for partial waves $\ell = 0$ through 5}
\label{tab:SphNeum}
\end{table}
}




\section{Miscellaneous}
\label{sec:Misc}

The cosine factors present in many of the matrix element equations are easily 
expressed in terms of $r_i$ and $r_{ij}$ by using the law of cosines
\cite[p.174]{CRC1978}:
\beq
\label{eq:Cosines}
\cos\theta_{12} = \frac{r_1^2 + r_2^2 - r_{12}^2}{2 r_1 r_2}, \ \ \ \ \cos\theta_{13} = \frac{r_1^2 + r_3^2 - r_{13}^2}{2 r_1 r_3} \ \ \ \text{and}  \ \ \cos\theta_{23} = \frac{r_2^2 + r_3^2 - r_{23}^2}{2 r_2 r_3}.
\eeq
This allows us to express all short-short matrix elements in the form needed by
the short-short methods described in \cref{sec:CompShort}.



\biblio
\end{document}