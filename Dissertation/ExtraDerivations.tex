% -*- root: Dissertation.tex -*-
\documentclass[Dissertation.tex]{subfiles} 
\begin{document}

\chapter{Extra Derivations}
\label{chp:ExtraDer}


\section{\texorpdfstring{$\rho$ and $\rho'$} {rho and rho'} Definitions}
\label{sec:RhoDef}
From equation 2.1 of Peter Van Reeth's thesis \cite{VanReethThesis} and Armour and Humberston's paper \cite{Armour1991},

\beq
\vec{\rho} = \frac{1}{2} \left( \vec{r_1} + \vec{r_2} \right).
\label{eq:RhoDef1}
\eeq
By switching coordinates 2 and 3, we have
\beq
\vec{\rho'} = \frac{1}{2} \left( \vec{r_1} + \vec{r_3} \right).
\label{eq:RhoDef2}
\eeq
In the original coordinate system (\cref{fig:CoordinateSystemOriginal}),
\begin{align}
\nonumber \vec{r_1} &= \left( r_1 \sin \theta_1 \cos \varphi_1, r_1 \sin \theta_1 \sin \varphi_1, r_1 \cos \theta_1 \right) \\
\vec{r_3} &= \left( r_3 \sin \theta_3 \cos \varphi_3, r_3 \sin \theta_3 \sin \varphi_3, r_3 \cos \theta_3 \right)
\end{align}
In the rotated coordinate system (\cref{fig:CoordinateSystemRotated}),
\begin{align}
\nonumber \vec{r_1} &= (0, 0, r_1) \\
\vec{r_3} &= \left( r_3 \sin \theta_{13} \cos \varphi_{13}, r_3 \sin \theta_{13} \sin \varphi_{13}, r_3 \cos \theta_{13} \right)
\end{align}
\begin{align}
\nonumber \left| r_1 + r_3 \right|^2 &= r_3^2 \sin^2 \theta_{13} \cos^2 \varphi_{13} + r_3^2 \sin^2 \theta_{13} \sin^2 \varphi_{13} + (r_1 + r_3 \cos \theta_{13})^2\\
\nonumber &= r_3^2 \sin^2 \theta_{13} + r_1^2 + r_3^2 \cos^2 \theta_{13} + 2 r_1 r_3 \cos \theta_{13} \\
&= r_1^2 + r_3^2 + 2 r_1 r_3 \cos \theta_{13}.
\label{eq:RhoDef3}
\end{align}

Also, using the law of cosines,
\begin{equation}
r_{13}^2 = r_1^2 + r_3^2 - 2 r_1 r_3 \cos \theta_{13}
\label{eq:CosLaw}
\end{equation}
Substituting (\ref{eq:CosLaw}) into (\ref{eq:RhoDef3}) gives
\begin{equation}
\left| r_1 + r_3 \right|^2 = 2 \left( r_1^2 + r_3^2 \right) - r_{13}^2.
\label{eq:RhoDef4}
\end{equation}
From (\ref{eq:RhoDef2}) and (\ref{eq:RhoDef4}),
\begin{equation}
\label{eq:RhopRDef}
\rho' = \frac{1}{2} \left[ 2 \left(r_1^2 + r_3^2 \right) - r_{13}^2 \right] ^ \frac{1}{2}.
\end{equation}
Similarly,
\begin{equation}
\label{eq:RhoRDef}
\rho = \frac{1}{2} \left[ 2 \left(r_1^2 + r_2^2 \right) - r_{12}^2 \right] ^ \frac{1}{2}.
\end{equation}


\section{Spherical Bessel Derivatives}
\label{sec:SphBess}

%\begin{figure}[H]
	%\centering
	%\includegraphics[width=3.5in]{Untitled-2}
	%\caption{Mathematica}
	%\label{fig:Untitled-2}
%\end{figure}

\subsection{First Derivative}
\label{sec:SphBess1}

From Abramowitz and Stegun \cite[p.437]{Abramowitz1965},
\begin{subequations}
\begin{align}
j_\ell(z) &= z^{-1} \left[ P(\ell+\tfrac{1}{2}, z ) \sin(z-\tfrac{1}{2}\ell\pi) + Q(\ell+\tfrac{1}{2},z) \cos(z-\tfrac{1}{2}\ell\pi) \right] \\
n_\ell(z) &= (-1)^{\ell+1} z^{-1} \left[ P(\ell+\tfrac{1}{2}, z ) \cos(z+\tfrac{1}{2}\ell\pi) - Q(\ell+\tfrac{1}{2},z) \sin(z+\tfrac{1}{2}\ell\pi) \right].
\end{align}
\end{subequations}
where
\begin{subequations}
\begin{align}
P(\ell+\tfrac{1}{2}, z) &= 1 - \frac{\Factorial{\ell+2}}{\Factorial{2} \GammaFunc{\ell-1}} (2z)^{-2} + \ldots \\
Q(\ell+\tfrac{1}{2}, z) &= \frac{\Factorial{\ell+1}}{\Factorial{1} \GammaFunc{\ell}} (2z)^{-1} - \frac{\Factorial{n+3}}{\Factorial{3} \GammaFunc{\ell-2}} (2z)^{-3} + \ldots
\end{align}
\end{subequations}
Using \emph{Mathematica} with these expansions, we get
\begin{subequations}
\begin{align}
\label{eq:SphBesExpan}
j_\ell(z) &= \frac{\sin(z - \tfrac{1}{2}\ell\pi)}{z} + \ldots \\
n_\ell(z) &= \frac{(-1)^{\ell+1} \cos(z + \tfrac{1}{2}\ell\pi)}{z} + \ldots
\end{align}
\end{subequations}
and
\begin{subequations}
\begin{align}
\label{eq:SphBesExpanDer}
{j_\ell}^\prime(z) &= \frac{\cos(z - \tfrac{1}{2}\ell\pi)}{z} + \ldots \\
{n_\ell}^\prime(z) &= \frac{(-1)^{\ell+2} \sin(z + \tfrac{1}{2}\ell\pi)}{z} + \ldots
\end{align}
\end{subequations}

In a more general format than that of Abramowitz and Stegun \cite[p.73]{Abramowitz1965},
\beq
\cos \left(z-\frac{\ell \pi}{2} \right) = \begin{cases} (-1)^{\ell/2} \cos z & \mbox{if } \ell \mbox{ is even} \\ 
(-1)^{(\ell-1)/2} \sin z & \mbox{if } \ell \mbox{ is odd} \end{cases} 
\eeq
and
\beq
\sin \left(z-\frac{\ell \pi}{2} \right) = \begin{cases} (-1)^{\ell/2} \sin z & \mbox{if } \ell \mbox{ is even} \\ 
(-1)^{(\ell+1)/2} \cos z & \mbox{if } \ell \mbox{ is odd} \end{cases} .
\eeq
Also,
\beq
\cos \left(z+\frac{\ell \pi}{2} \right) = \begin{cases} (-1)^{\ell/2} \cos z & \mbox{if } \ell \mbox{ is even} \\ 
(-1)^{(\ell+1)/2} \sin z & \mbox{if } \ell \mbox{ is odd} \end{cases} 
\eeq
and
\beq
\sin \left(z+\frac{\ell \pi}{2} \right) = \begin{cases} (-1)^{\ell/2} \sin z & \mbox{if } \ell \mbox{ is even} \\ 
(-1)^{(\ell-1)/2} \cos z & \mbox{if } \ell \mbox{ is odd} \end{cases} .
\eeq
Using these with \cref{eq:SphBesExpan,eq:SphBesExpanDer}, we see that to first order, there is a relationship between these functions and their derivatives given by
\begin{subequations}
\begin{align}
\label{eq:SphBesDerRel}
{j_\ell}^\prime(z) &\approx -n_\ell(z) \\
{n_\ell}^\prime(z) &\approx j_\ell(z).
\end{align}
\end{subequations}


\subsection{Second Derivative}
\label{sec:SphBess2}

The output we get from the \emph{Mathematica} notebook ``First Partial Waves LS General.nb'' is
\begin{align}
\label{eq:LSGenLapl1}
\frac{\Laplacian_\rho \left[\SphericalHarmonicY{\ell}{0}{\theta_\rho}{\varphi_\rho} j_\ell(\kappa\rho) \right]}{\SphericalHarmonicY{\ell}{0}{\theta_\rho}{\varphi_\rho} j_\ell(\kappa\rho)} = \frac{(n + n^2 - \kappa^2 \rho^2) \LegendreP{\ell, \cos\theta} + 2 \cot\theta \, \AssocLegendreP{\ell}{1}{\cos\theta} + \AssocLegendreP{\ell}{2}{\cos\theta}} {\rho^2 \LegendreP{\ell, \cos\theta}}.
\end{align}

From Ref.~\cite{WolframPnm}, a recurrence relation for the associated Legendre polynomials is
\beq
\AssocLegendreP{\ell}{\mu+1}{z} - \left[\mu(\mu-1) - \ell(\ell+1)\right] \AssocLegendreP{\ell}{\mu-1}{z} + \frac{2 \mu z}{\sqrt{1-z^2}} \AssocLegendreP{\ell}{\mu}{z} = 0.
\eeq
Note that other books \cite{Abramowitz1965,Zwillinger2003} give slightly different forms of this recurrence relation. If we set $\mu = 1$ and $z = \cos\theta$, this becomes
\beq
\AssocLegendreP{\ell}{2}{\cos\theta} + \left(\ell^2+\ell \right) \AssocLegendreP{\ell}{0}{z} + \frac{2 \cos\theta}{\sin\theta} \AssocLegendreP{\ell}{1}{\cos\theta} = 0.
\eeq
Using the definition of $\cot$ and solving for $\left(\ell^2+\ell \right)\AssocLegendreP{\ell}{0}{z} = \left(\ell^2+\ell \right)\LegendreP{\ell, \cos\theta}$,
\beq
\left(\ell^2+\ell \right)\LegendreP{\ell, \cos\theta} = -\AssocLegendreP{\ell}{2}{\cos\theta} - 2 \cot\theta \, \AssocLegendreP{\ell}{1}{\cos\theta}.
\eeq
Substituting this back into \cref{eq:LSGenLapl1}, we get cancellations with the associated Legendre polynomials and see that $\SphericalHarmonicY{\ell}{0}{\theta_\rho}{\varphi_\rho} j_\ell(\kappa\rho)$ is an eigenfunction of $\Laplacian_\rho$ with eigenvalue $-\kappa^2$:
\begin{align}
\label{eq:LSGenLapl2}
\Laplacian_\rho \left[\SphericalHarmonicY{\ell}{0}{\theta_\rho}{\varphi_\rho} j_\ell(\kappa\rho) \right] = \frac{(-\kappa^2 \rho^2) \LegendreP{\ell, \cos\theta} } {\rho^2 \LegendreP{\ell, \cos\theta}}
= -\kappa^2 \, \SphericalHarmonicY{\ell}{0}{\theta_\rho}{\varphi_\rho} j_\ell(\kappa\rho).
\end{align}



\section{\texorpdfstring{$_2F_1$}{Hypergeometric} Recursion Relation}
\label{sec:Hypergeometric}

The backwards recursion relation for the hypergeometric function is given in Refs. \cite{Drake1995} and \cite{Frolov2003} as
\beq
\label{eq:HyperIdentity}
\Hypergeometric{2}{1}{1,a}{c}{z} = 1 + \left( \frac{a}{c} \right) z \,\, \Hypergeometric{2}{1}{1,a+1}{c+1}{z}.
\eeq

From Abramowitz and Stegun \cite{Abramowitz1965}, the definition of the hypergeometric function is given by
\beq
\label{eq:HyperDef}
_2F_1(\alpha,\beta;\gamma;z) = 1 + \sum_{n=1}^{\infty} \frac{(\alpha)_n \cdot (\beta)_n}{(\gamma)_n} \frac{z^n}{n!},
\eeq
where $(x)_n$ is the Pochhammer symbol given by \cite{Abramowitz1965}
\beq
\label{eq:PochhammerDef}
(x)_n \equiv \frac{\Gamma(x+n)}{\Gamma(x)} = x(x+1) \cdots  (x+n-1)
\eeq
with $(x)_0 = 1$.  A special case is $(x)_1 = n!$, where $n!$ is the factorial.

Using the above definition of the Pochhammer symbol, we can easily see that
\beq
\label{eq:PochPlus1}
(x+1)_n = \frac{(x+n)}{(x)} \cdot (x)_n = \frac{(x)_{n+1}}{x}.
\eeq
From \ref{eq:HyperDef} and \ref{eq:PochPlus1}, 
\begin{align}
\nonumber _2F_1(1,a+1;c+1;z) &= 1 + \sum_{n=1}^{\infty} \frac{(1)_n \cdot (a+1)_n}{(c+1)_n} \frac{z^n}{n!} = 1 + \sum_{n=1}^{\infty} \frac{(a+1)_n}{(c+1)_n} z^n \\
&= 1 + \left(\frac{c}{a}\right) \sum_{n=1}^{\infty} \frac{(a)_{n+1}}{(c)_{n+1}} z^n = 1 + \left(\frac{c}{a}\right) \sum_{n=2}^{\infty} \frac{(a)_n}{(c)_n} z^{n-1}.
\end{align}
Multiplying by $\left(\frac{c}{a}\right) z$, we now have
\beq
\label{eq:Hyper1}
\left(\frac{c}{a}\right) z \,\, _2F_1(1,a+1;c+1;z) = \left(\frac{c}{a}\right) z + \sum_{n=2}^{\infty} \frac{(a)_n}{(c)_n} z^n = \sum_{n=1}^{\infty} \frac{(a)_n}{(c)_n} z^n.
\eeq
From the definition in equation \ref{eq:HyperDef},
\beq
\label{eq:Hyper2}
_2F_1(1,a;c;z) = 1 + \sum_{n=1}^{\infty} \frac{(a)_n}{(c)_n} z^n,
\eeq
which, when combined with equation \ref{eq:Hyper1}, gives the final result of
\beq
_2F_1(1,a;c;z) = 1 + \left( \frac{a}{c} \right) z \,\,_2F_1(1,a+1;c+1;z).
\eeq


\section{Shielding Function}
\label{sec:ShieldingFunc}

\todoi{Give how C acts like S with shielding function}



\section{Mixed Terms}
\label{sec:MixedDerivation}

This derivation proves \cref{eq:MixedAngSimple}. Using \cref{sec:MixedTerms} and substituting the appropriate spherical harmonics and Clebsch-Gordan coefficients,
\begin{align}
\psi_{(1,1,2,0)} = &\sum_{m=-1}^{+1} Y_{1,m}(\theta_1,\varphi_1) Y_{1,m}(\theta_2,\varphi_2) \left< 1,m; 1,-m,0 | 2,0 \right> \nonumber \\
	= &-\sqrt{\frac{3}{8\uppi}} \sin\theta_1 \ee^{-\ii \varphi_1} \sqrt{\frac{3}{8\uppi}} \sin\theta_2 \ee^{\ii \varphi_1} \frac{1}{\sqrt{6}} \nonumber \\
	& + \sqrt{\frac{3}{4\uppi}} \cos\theta_1 \sqrt{\frac{3}{4\uppi}} \sin\theta_2 \frac{2}{\sqrt{6}} \nonumber \\
	& -\sqrt{\frac{3}{8\uppi}} \sin\theta_1 \ee^{\ii \varphi_1} \sqrt{\frac{3}{8\uppi}} \sin\theta_2 \ee^{-\ii \varphi_1} \frac{1}{\sqrt{6}}.
\end{align}
From \cref{},
\beq
\cos\theta_{12} = \sin\theta_1 \sin\theta_2 \cos(\varphi_1 - \varphi_2) + \cos\theta_1 \cos\theta_2.
\eeq
Using this with the standard
\beq
\cos\theta = \frac{\ee^{\ii \theta} + \ee^{-\ii \theta}}{2},
\eeq
we obtain \cref{eq:MixedAngSimple}:
\begin{align}
\psi_{(1,1,2,0)} &= -\frac{3}{8\uppi} \frac{1}{\sqrt{6}} \sin\theta_1 \sin\theta_2 \cdot 2 \cos(\varphi_1 - \varphi_2) + \frac{3}{4\uppi} \frac{1}{\sqrt{6}} \cdot 2 \cos\theta_1 \cos\theta_2 \nonumber \\
&= \frac{3}{4\uppi} \frac{1}{\sqrt{6}} \left(3 \cos\theta_1 \cos\theta_2 - \cos\theta_{12} \right).
\end{align}


\section{Fano Resonances}
\label{sec:FanoResonances}
As mentioned in \cref{sec:Resonances}, the resonances we encounter in Ps-H
scattering are Fano resonances. Breit-Wigner resonances are a special case of Fano
resonances with $q = \pm\infty$.
Fano resonances correspond to a resonance with a flat cross section 
background, which we do not have, as seen later in \cref{fig:singlet-cross-sections}.
Fano's equation for the cross section is \cite[p.596]{Bransden2003}
\beq
\label{eq:FanoRes}
\sigma_\ell = \frac{4\pi}{\kappa^2} (2\ell+1) \sin^2 \xi_\ell \frac{(q+
\epsilon)^2}{1+\epsilon^2},
\eeq
with reduced energy given by
\beq
\epsilon = \frac{E - E_r}{\Gamma/2}
\eeq
and resonance shape parameter
\beq
q = -\cot \xi_\ell.
\eeq
The effect of this $q$ parameter is seen in \cref{fig:FanoCross}. This allows 
the cross section to have an asymmetric shape near a resonance. Notice that 
the special case of a Breit-Wigner resonance is given as $q = \pm\infty$. The 
dotted line is the curve $4 \pi (2\ell+1) / \kappa^2$, which gives an 
envelope to this function. An interactive demonstration of Fano resonances
(but restricted to $q \geq 0$ is available from the Wolfram Demonstrations
Project \cite{FanoDemo}.

\begin{figure}[H]
	\centering
	\includegraphics[width=5.25in]{fano-cross}
	\caption[Cross sections of Fano resonances]{Cross sections of Fano 
resonances with differing values of $q$, shown for S-wave scattering}
	\label{fig:FanoCross}
\end{figure}



\biblio
\end{document}