\documentclass[main.tex]{subfiles} 
\begin{document}


\chapter{Integrations}
\label{sec:Integrations}


\section{Perimetric Coordinates}
\label{sec:PerimetricCoords}

Perimetric coordinates are used for some integrations.  If perimetric coordinates are used for $r_1$, $r_2$ and $r_{12}$, then these are defined by \cite{Armour1991}

\begin{align}
\label{eq:PerimetricCoords1}
\nonumber x &= r_1 + r_2 - r_{12} \\
\nonumber y &= r_2 + r_{12} - r_1 \\
z &= r_{12} + r_1 - r_2.
\end{align}

These can alternately be written as
\begin{align}
\label{eq:PerimetricCoords2}
\nonumber r_1 &= \frac{x+z}{2} \\
\nonumber r_2 &= \frac{x+y}{2} \\
\nonumber r_{12} &= \frac{y+z}{2}.
\end{align}

From equation \ref{}, the volume element after integration over the external angles is


\beq
\label{eq:dtau}
d\tau = 8\pi^2 dr_1 r_2 dr_2 r_3 dr_3 r_{12} dr_{12} r_{13} dr_{13} d\phi_{23}.
\eeq

We need to perform a change of variables to use perimetric coordinates for $r_1$, $r_2$ and $r_{12}$.  The Jacobian is
\beq
\label{eq:PerimetricJacobian}
J(x,y,z) = 
\left| {\begin{array}{ccc}
 \frac{\partial r_1}{\partial x} & \frac{\partial r_1}{\partial y} & \frac{\partial r_1}{\partial z}  \\
 \frac{\partial r_2}{\partial x} & \frac{\partial r_2}{\partial y} & \frac{\partial r_2}{\partial z}  \\
 \frac{\partial r_{12}}{\partial x} & \frac{\partial r_{12}}{\partial y} & \frac{\partial r_{12}}{\partial z}  \\
 \end{array} } \right|
=
\left| {\begin{array}{ccc}
 \frac{1}{2} & 0 & \frac{1}{2} \\
 \frac{1}{2} & \frac{1}{2} & 0 \\
 0 & \frac{1}{2} & \frac{1}{2}
 \end{array} } \right|
=
\frac{1}{4}.
\eeq

\noindent This gives a transformed volume element of
\beq
\label{eq:PerimetricVolEl}
d\tau = 2\pi^2 r_2 r_3 r_{12} r_{13} dx\, dy\, dz\, dr_3\, dr_{13}\, d\phi_{23}.
\eeq

\noindent The limits for each of the perimetric coordinates are 0 to $\infty$.


\section{Gaussian Quadratures}
Gaussian quadratures are used to integrate many classes of integrals.  In their most general form, these quadratures are given by
\beq
\label{eq:GeneralQuadratures}
\int_a^b W(x) f(x) dx \approx \sum_{i=1}^n w_i f(x_i).
\eeq

\noindent Gaussian quadratures are particularly attractive, since they give exact results for polynomials up to degree $2n-1$.  The weight function $W(x)$ can be chosen for certain classes of integrals.  Three main types of weight functions are used in this work.


\subsection{Gauss-Legendre Quadrature}
\label{sec:GaussLegendre}
If the weight function is chosen as $W(x)=1$, and the integration interval is $(-1,1)$, this is known as Gauss-Legendre quadrature (or sometimes known simply as Gaussian quadrature).  The orthogonal polynomials used are the Legendre polynomials, $P_n(x)$.  Equation \ref{eq:GeneralQuadratures} becomes
\beq
\label{eq:GaussLeg}
\int_{-1}^1 f(x) dx \approx \sum_{i=1}^n w_i f(x_i),
\eeq
where the $x_i$ abscissas are the $i^{th}$ zeros of $P_n(x)$ and the weights are given by
\beq
\label{eq:GaussLegWeights}
w_i = \frac{2}{(1-x_i^2)[P^\prime_n(x_i)]^2}.
\eeq

The limits of integration must be from $-1$ to $1$, but this is generalized by using the transformation \cite{Abramowitz1965}
\begin{align}
\label{eq:GaussLegGen}
\int_a^b f(x) dx &= \frac{b-a}{2} \int_{-1}^1 f \left(\frac{b-a}{2} x + \frac{a+b}{2}\right) dx \\
&\approx \frac{b-a}{2} \sum_{i=1}^n w_i f \left(\frac{b-a}{2} x_i + \frac{a+b}{2}\right).
\end{align}


\subsection{Gauss-Laguerre Quadrature}
\label{sec:GaussLag}
The Gauss-Legendre quadrature cannot be used on semi-infinite intervals, so we use the Gauss-Laguerre quadrature in these cases.  The orthogonal polynomials in this case are the Laguerre polynomials, $L_n(x)$, and the weight function is $W(x) = e^{-x}$.  Equation \ref{eq:GeneralQuadratures} becomes
\beq
\label{eq:GaussLag}
\int_0^\infty e^{-x} f(x) dx \approx \sum_{i=1}^n w_i f(x_i),
\eeq
where the $x_i$ abscissas are the $i^{th}$ zeros of $P_n(x)$ and the weights are given by
\beq
\label{eq:GaussLagWeights}
w_i = \frac{x_i}{(n+1)^2 [L_{n+1}(x_i)]^2}.
\eeq

When the integration is over the interval $(a,\infty)$, equation \ref{eq:GaussLag} is easily transformed by
\beq
\label{eq:GaussLagGen1}
\int_a^\infty e^{-x} f(x) dx = \int_0^\infty e^{-(x+a)} f(x+a) dx = e^{-a} \int_0^\infty e^{-x} f(x+a) dx \approx e^{-a} \sum_{i=1}^n w_i f(x_i+a).
\eeq

\noindent A more general form of this is obtained by using a coefficient in the exponential, i.e.
\beq
\label{eq:GaussLagGen2}
\int_a^\infty e^{-m x} f(x) dx = \frac{1}{m} \int_a^\infty e^{-y} f\left(\frac{y}{m}\right) dy,
\eeq
where we have defined $y = m x$.  This allows for equation \ref{eq:GaussLagGen1} to be generalized to
\begin{align}
\label{eq:GaussLagGen}
\nonumber \int_a^\infty e^{-m x} f(x) dx &= \frac{1}{m} \int_a^\infty e^{-y} f\left(\frac{y}{m}\right) dy = \frac{1}{m} \int_0^\infty e^{-(y+a)} f\left(\frac{y}{m}+a\right) dy \\
& = \frac{e^{-a}}{m} \int_0^\infty e^{-y} f\left(\frac{y}{m}+a\right) dy \approx \frac{e^{-a}}{m} \sum_{i=1}^n w_i f\left(\frac{y_i}{m}+a\right).
\end{align}
The $y_i$ abscissas and $w_i$ weights are the same as the less general case in equations \ref{eq:GaussLag} and \ref{eq:GaussLagWeights}.  This general form of Gauss-Laguerre quadrature is what we use for our semi-infinite integrations.


\subsection{Chebyshev--Gauss Quadrature}
\label{sec:ChebyshevGauss1}
If the weight function is chosen as $W(x)=\frac{1}{\sqrt{1-x^2}}$, and the integration interval is $(-1,1)$, this is known as Chebyshev--Gauss quadrature.  The orthogonal polynomials used are the Chebyshev polynomials of the first kind, $T_n(x)$.  Equation \ref{eq:GeneralQuadratures} becomes
\beq
\label{eq:GaussCheb}
\int_{-1}^1 \frac{f(x)}{\sqrt{1-x^2}} dx \approx \sum_{i=1}^n w_i f(x_i),
\eeq
where
\beq
\label{eq:GaussChebAbsWeights}
x_i = \cos\left(\frac{2i-1}{n}\pi\right) \text{ and } w_i = \frac{\pi}{n}.
\eeq

This quadrature is used for the internal angular integrations.  A discussion of how to use this for these integrations is found in appendix \ref{sec:ChebyshevGauss}.

\section{Long-Range -- Long-Range}
\label{sec:LongLongInt}
The scattering program calculates only the short-range -- long-range (short-long) and long-range -- long-range (long-long) matrix elements.  The volume element in equation \ref{eq:dtau} has an internal angle of $\phi_{23}$ to integrate over.  When a term has a a negative power of $r_{23}$, a large number of integration points must be used for reasonable accuracy.  Instead, we split the integration such that one part is missing the the $r_{23}^{-1}$ term and the other contains only the $r_{23}^{-1}$ term.

The first integration excluding the $r_{23}^{-1}$ term has negative powers of $r_i$ and $r_{ij}$ cancelled by the corresponding terms in the volume element given by equation \ref{}.  For the integration over the $r_{23}^{-1}$ term, we use an alternative volume element, namely that given by equation \ref{}.  The $r_{23}^{-1}$ is then cancelled by the $r_{23}$ in this volume element.

\textbf{@TODO:} Add the section about the volume elements in an appendix.

\subsection{Integration without the \texorpdfstring{$r_{23}^{-1}$} {1/r23} term}
\label{sec:LongLongNoR23}
The simplest long-long matrix element to evaluate is $(\bar{S},L\bar{S})$.  From equation \ref{eq:SbarLSbar}, not including its $r_{23}^{-1}$ term, this is
\beq
(\bar{S},L\bar{S})_A = \pm \left(S^\prime,LS\right) = \pm \left(S^\prime, \left[ \frac{2}{r_1} - \frac{2}{r_2} - \frac{2}{r_{13}}\right] S\right).
\eeq

For this type of integration, we use perimetric coordinates as described in section \ref{sec:PerimetricCoords}.
\beq
\label{eq:SBarSBarInt}
(\bar{S},L\bar{S})_A = \pm 2\pi^2 \int_0^\infty \int_0^\infty \int_0^\infty \int_0^\infty \int_{|r_1 - r_3|}^{|r_1 + r_3|} \int_0^{2\pi}  S^\prime S \left[ \frac{2}{r_1} - \frac{2}{r_2} - \frac{2}{r_{13}}\right] r_2 r_3 r_{12} r_{13}\, d\phi_{23}\, dr_{13}\, dr_3\, dz\, dy\, dx
\eeq

The $\phi_{23}$ integration is done analytically.  Since $S$ and $S^\prime$ have no $r_{23}$ dependence and there is no $r_23$ term in the brackets, the integration over $\phi_{23}$ is simply $2\pi$.  The $r_{13}$ integration uses the Gauss-Laguerre quadrature from section \ref{sec:GaussLegendre}.  The $x$, $y$ and $z$ integrations use Gauss-Laguerre quadrature (section \ref{sec:GaussLag}), since they are semi-infinite.

\textbf{@TODO:} Describe the discontinuity and why it exists.

The $r_3$ integration could also be performed using just the Gauss-Laguerre quadrature.  However, the integrand for the $r_3$ integration has a discontinuity in its slope at $r_3=r_1$, creating a cusp, so the accuracy is improved greatly if we split the integration interval into two parts and employ different quadratures for each.  The integration is split to use Gauss-Legendre on the interval $(0,r_1)$ and Gauss-Laguerre on the interval $(r_1,\infty)$.

%\subsection{Integration over the \texorpdfstring{$r_{23}^{-1}$} {1/r23} term}
\subsection[Integration over the 1/r23 term]{Integration over the $r_{23}^{-1}$ term}
\label{sec:LongLongR23}
The other part of the $(\bar{S},L\bar{S})$ integral contains the $r_{23}^{-1}$ term.

\beq
(\bar{S},L\bar{S})_B = \pm \left(S^\prime, \left[ \frac{2}{r_{23}}\right] S\right)
\eeq

\textbf{@TODO:} Why exactly do we use perimetric coordinates?

\noindent The volume element for this integral is $d\tau^\prime$ from equation \ref{}.  The integration also does not need to be converted to perimetric coordinates, so its form is
\beq
(\bar{S},L\bar{S})_B = \pm 8\pi^2 \int_0^\infty \int_0^\infty \int_0^\infty \int_{|r_1 - r_3|}^{|r_1 + r_3|} \int_{|r_2 - r_3|}^{|r_2 + r_3|} \int_0^{2\pi}  S^\prime S \frac{2}{r_{23}} r_1 r_2 r_{13} r_{23}\, d\phi_{12}\, dr_{23}\, dr_{13}\, dr_2\, dr_3\, dr_1.
\eeq

The $r_{13}$ and $r_{23}$ integrals have finite limits, so here we use Gauss-Legendre quadrature.  Again, for the internal angular integration, this time over $\phi_{12}$, we use Chebyshev-Gauss quadrature.  The cusp in the $r_3$ integration is at $r_3 = r_1$, and the cusp in the $r_2$ integration is at $r_2 = r_3$.  Similar to before, we split up these integrations by using Gauss-Legendre before the cusp and Gauss-Laguerre after the cusp.

The $(\bar{C},L\bar{S})$ and $(\bar{C},L\bar{C})$ terms are integrated in the same manner as the $(\bar{S},L\bar{S})$ integral just described.


\section{Short-Range -- Long-Range}
\label{sec:ShortLongInt}
We will consider only the $(\bar{\phi}_i,L\bar{S})$ integrations here, as the $(\bar{\phi}_i,L\bar{C})$ integrals are evaluated in the same manner.  As in the case of long-range -- long-range integrations in section \ref{sec:LongLongInt}, we split up the integration into two parts -- one containing the $r_{23}^{-1}$ term and another containing the rest of the terms.  The short-range terms have the added benefit of the possibility of the polynomial $r_{23}^{\,q_i}$ being present, which cancels the $r_{23}^{-1}$ term or gives it an overall positive power.


From \ref{eq:PhiBarLSBar2b}, \ref{eq:LS2} and \ref{eq:LSP2}, 

\begin{align}
\label{eq:PhiLSBarInt}
\nonumber (\bar{\phi}_i, L\bar{S}) &= \frac{2}{\sqrt{2}} \left(\phi_i,L\bar{S}\right) \\
 &= \frac{2}{\sqrt{2}} \int \phi_i \left[ \left( \frac{2}{r_1} - \frac{2}{r_2} - \frac{2}{r_{13}} + \frac{2}{r_{23}} \right)S \pm \left( \frac{2}{r_1} - \frac{2}{r_3} - \frac{2}{r_{12}} + \frac{2}{r_{23}} \right) S^\prime \right]  d\tau.
\end{align}

\subsection{Case I: $q_i > 0$}
When $q_i > 0$ in $\phi_i$ (equation \ref{eq:PhiDef}), the power of $r_{23}$ is equal to or greater than 0.  Gaussian quadratures can safely integrate this type of term, so we integrate the full expression in equation \ref{eq:PhiLSBarInt}.
\begin{align}
\label{eq:PhiLSBarIntFull}
\nonumber (\bar{\phi}_i, L\bar{S}) =& \, \frac{2}{\sqrt{2}} \cdot 8\pi^2  \int_0^\infty \int_0^\infty \int_0^\infty \int_{|r_1 - r_2|}^{|r_1 + r_2|} \int_{|r_1 - r_3|}^{|r_1 + r_3|} \int_0^{2\pi} \phi_i \\
&\times \left[ \left( \frac{2}{r_1} - \frac{2}{r_2} - \frac{2}{r_{13}} + \frac{2}{r_{23}} \right)S \pm \left( \frac{2}{r_1} - \frac{2}{r_3} - \frac{2}{r_{12}} + \frac{2}{r_{23}} \right) S^\prime \right]  r_2 r_3 r_{12} r_{13}\, d\phi_{23}\, dr_{13}\, dr_{12}\, dr_3\, dr_2\, dr_1
\end{align}

Similar to the long-long integrations from section \ref{sec:LongLongInt}, the $r_1$ integration is performed using the Gauss-Laguerre quadrature.  The $r_2$ integral is broken into two parts at the cusp of $r_2 = r_1$, with the Gauss-Legendre quadrature before the cusp and the Gauss-Laguerre quadrature after the cusp.  In the $r_3$ coordinate, there is a cusp at $r_3 = r_2$, so the integration is also split up into Gauss-Legendre before the cusp and Gauss-Laguerre after the cusp.  The finite intervals for $r_{12}$ and $r_{13}$ ensure that we can use Gauss-Legendre quadratures for these coordinates.  The $\phi_{23}$ integration uses the Chebyshev-Gauss quadrature.

\subsection{Case II: $q_i = 0$}
When $q_i = 0$, the overall power of the $r_{23}^{-1}$ term is $-1$, so we cannot use the Gaussian quadratures in the form of \ref{eq:PhiLSBarIntFull}.  Similar to the long-long integrations, the $r_{23}^{-1}$ term is integrated separately, using the same type of integrations as equation \ref{eq:PhiLSBarIntFull}.  Refer to the previous section for the description of the quadratures used.
\begin{align}
\label{eq:PhiLSBarIntNoR23}
\nonumber (\bar{\phi}_i, L\bar{S}) =& \,\frac{2}{\sqrt{2}} \int \phi_i \left[ \left( \frac{2}{r_1} - \frac{2}{r_2} - \frac{2}{r_{13}} \right)S \pm \left( \frac{2}{r_1} - \frac{2}{r_3} - \frac{2}{r_{12}} \right) S^\prime \right]  d\tau \\
=&\, \frac{2}{\sqrt{2}} \cdot 8\pi^2  \int_0^\infty \int_0^\infty \int_0^\infty \int_{|r_1 - r_2|}^{|r_1 + r_2|} \int_{|r_1 - r_3|}^{|r_1 + r_3|} \int_0^{2\pi} \phi_i \\
&\times \left[ \left( \frac{2}{r_1} - \frac{2}{r_2} - \frac{2}{r_{13}} \right)S \pm \left( \frac{2}{r_1} - \frac{2}{r_3} - \frac{2}{r_{12}} \right) S^\prime \right]  r_2 r_3 r_{12} r_{13}\, d\phi_{23}\, dr_{13}\, dr_{12}\, dr_3\, dr_2\, dr_1
\end{align}

The integration over the $r_{23}^{-1}$ term is done the same way as the second integration of the long-long matrix elements in section \ref{sec:LongLongInt}.  The $r_{23}$ in the $d\tau^\prime$ volume element cancels the $r_{23}^{-1}$ term.  Refer to section \ref{sec:LongLongR23} for a description of the quadratures used here.
\begin{align}
\label{eq:PhiLSBarIntR23}
\nonumber (\bar{\phi}_i, L\bar{S}) =& \,\frac{2}{\sqrt{2}} \int \phi_i \left[ \frac{2}{r_{23}}\left(S \pm S^\prime\right) \right] d\tau^\prime \\
=&\, \frac{2}{\sqrt{2}} \cdot 8\pi^2  \int_0^\infty \int_0^\infty \int_0^\infty \int_{|r_1 - r_3|}^{|r_1 + r_3|} \int_{|r_2 - r_3|}^{|r_2 + r_3|} \int_0^{2\pi} \phi_i \\
&\times \left[ \frac{2}{r_{23}}\left(S \pm S^\prime\right) \right]  r_1 r_2 r_{13} r_{23}\, d\phi_{12}\, dr_{23}\, dr_{13}\, dr_2\, dr_3\, dr_1
\end{align}



\end{document}
