\documentclass[Dissertation.tex]{subfiles} 
\begin{document}


\chapter{Kohn Variational Methods}


\section{General Wavefunction}

The applications of the Kohn, inverse Kohn, complex Kohns and the generalized Kohn to our trial wavefunctions for each of the partial waves through the H-wave are all very similar in form. The trial wavefunctions for the partial waves (\cref{eq:SWaveTrialSimple,eq:PWaveSimple,eq:DWaveSimple}) can be written in a general form as

\beq
\Psi_t^\pm = \widetilde{S} + L_\ell^t \, \widetilde{C} + \sum_{i=1}^N c_i \bar{\phi}_i^t.
\label{eq:GeneralWaveTrial}
\eeq

The short-range $\bar{\phi}_i^t$ terms can represent terms of different symmetries, such as the $\bar{\phi}_{i1}$ and $\bar{\phi}_{j2}$ of the P-wave in \cref{eq:PWaveSimple}. The only requirement in this derivation is that these are Hylleraas-type short-range terms. In addition to letting the $\widetilde{S}$ and $\widetilde{C}$ represent the $\bar{S}$ and $\bar{C}$ for the different partial waves, we can define them in such a way as to use multiple Kohn methods (Kohn, inverse Kohn, etc.). We begin by defining a $2\times 2$ matrix $\textbf{u}$ which satisfies

\beq
\label{eq:GenSCMatrix}
\begin{bmatrix}
\widetilde{S} \\
\widetilde{C}
\end{bmatrix}
=
\textbf{u}
\begin{bmatrix}
\bar{S} \\
\bar{C}
\end{bmatrix}
=
\begin{bmatrix}
u_{00} & u_{01} \\
u_{10} & u_{11}
\end{bmatrix}
\begin{bmatrix}
\bar{S} \\
\bar{C}
\end{bmatrix}.
\eeq

\noindent This notation is similar to that of Lucchese \cite{Lucchese1989} and Cooper, Plummer, and Armour \cite{Cooper2010}. From this, it can easily be seen that
\begin{subequations}
\label{eq:TildeSCDef}
\begin{align}
\widetilde{S} &= u_{00} \bar{S} + u_{01} \bar{C} \\
\widetilde{C} &= u_{10} \bar{S} + u_{11} \bar{C}.
\end{align}
\end{subequations}

Generally,
\beq
\bar{S} = \frac{1}{\sqrt{2}}(S \pm S^\prime) \text{ and } \bar{C} = \frac{1}{\sqrt{2}}(C \pm C^\prime),
\eeq
where $S^\prime = P_{23}S$ and $C^\prime = P_{23}C$.
The general form for the long-range terms $S$ and $C$ is
\begin{subequations}
\label{eq:GenSandC}
\begin{align}
S = \,&\SphericalHarmonicY{\ell}{0}{\theta_\rho}{\phi_\rho} \Phi_{Ps}\left(r_{12}\right) \Phi_H\left(r_3\right) \sqrt{2\kappa} \,j_\ell\!\left(\kappa\rho\right) \label{eq:GenSDef} \\
C = -&\SphericalHarmonicY{\ell}{0}{\theta_\rho}{\phi_\rho} \Phi_{Ps}\left(r_{12}\right) \Phi_H\left(r_3\right) \sqrt{2\kappa} \,n_\ell\!\left(\kappa\rho\right) f_\ell(\rho) \label{eq:GenCDef}.
\end{align}
\end{subequations}
The shielding function, $f_\ell(\rho)$, removes the singularity at the origin due to the Neumann function, $n_\ell$. The form that we have chosen for this is
\begin{equation}
f_\ell(\rho) = \left[1 - \ee^{-\mu \rho} \left(1+\frac{\mu}{2}\rho\right)
\right]^{m_\ell}.
\label{eq:PartialWaveShielding}
\end{equation}
At a minimum, $m_\ell$ is chosen so that $C$ behaves like $S$ as $\rho \to 0$. \todoi{Derivation. Table of values of $m_\ell$.}


\section{General Kohn Principle}
\label{sec:GenKohnPrinciple}


In this work, we only consider the first six partial waves, i.e. $\ell = 0$--$5$
for the S-, P-, D-, F-, G-, and H-waves. There is no general proof of 
this, but for these first six partial waves, a relation between $S$ and $C$ 
is seen by taking the gradient of each with respect to $\rho$. For the S-wave,
$\ell = 0$, the gradient \todo{Not gradient} of the spherical Bessel function is

\beq
\frac{\partial}{\partial \rho} j_0(\kappa\rho) = \frac{\partial}{\partial \rho} \frac{\sin(\kappa\rho)}{\kappa\rho} = \frac{\cos(\kappa\rho)}{\rho} - \frac{\sin(\kappa\rho)}{\kappa \rho^2}.
\eeq
The leading term in $\rho$ for this case is $\frac{\cos(\kappa\rho)}{\rho}$, which is equal to $\kappa n_0(\kappa\rho)$. Using the definitions from \cref{eq:GenSandC} for the first few partial waves considered, and only for the leading terms, it can easily be seen that
\begin{subequations}
\label{eq:GenSCGrad}
\begin{align}
\nabla_\rho S &\approx \kappa C \text{ and}\\
\nabla_\rho C &\approx \kappa S.
\end{align}
\end{subequations}




\section{General Kohn Principle Derivation}
\label{sec:KohnDerivation}

Much of this derivation is similar to that in Peter Van Reeth's thesis \cite{VanReethThesis} but is for single channel scattering and also generalized to a variety of Kohn variational methods. His thesis covers the Kohn and inverse Kohn methods for two channel e$^+$-He scattering. For this derivation, I will use \cref{eq:GeneralWaveTrial} but drop the short-range $\phi_i^t$ terms. The derivation follows through the same with these terms, but it is clearer to ignore them here. %Likewise, we only consider the direct terms here. The final result of this section also applies to the exchanged terms.

The version of \cref{eq:IlDefU} for the full wavefunction is
\begin{equation}
I[\Psi_\ell^t]\equiv \left<\Psi_\ell^{t\star} | \mathcal{L} | \Psi_\ell^t \right> = \left(\Psi_\ell^t, \mathcal{L} \Psi_\ell^t \right) = \int \Psi_\ell^t \mathcal{L}
  \Psi_\ell^t \,d\tau.
\label{eq:IlDefPsi}
\end{equation}
The operator $\mathcal{L}$ is given by
\beq
\label{eq:LDef}
\mathcal{L} = 2(H-E).
\eeq
Note that for the exact wavefunction $\Psi_\ell$,
\beq
\label{eq:Il0}
I[\Psi_\ell] = 0.
\eeq
Normally, the bra in bra-ket notation is conjugated, but as noted by Cooper et al. \cite{Cooper2010}, the bra is not conjugated for the Kohn variational methods.

\noindent Similar to \cref{eq:UlTrialRelation},
\beq
\label{eq:PsilTrialRelation}
\Psi_\ell^t = \Psi_\ell + \delta \Psi_\ell.
\eeq

\noindent The variation of $I_\ell$ is
\begin{align}
\label{eq:IlPsiVariation1}
\nonumber \delta I_\ell &= I_\ell[\Psi_\ell^t] - I_\ell[\Psi_\ell] \\
\nonumber &= I_\ell[\Psi_\ell + \delta \Psi_\ell] - I_\ell[\Psi_\ell] \\
&= (\Psi_\ell, \mathcal{L} \Psi_\ell) + (\Psi_\ell, \mathcal{L} \,\delta\Psi_\ell) + (\delta\Psi_\ell, \mathcal{L} \Psi_\ell) + (\delta\Psi_\ell, \mathcal{L} \,\delta\Psi_\ell) - (\Psi_\ell, \mathcal{L} \Psi_\ell).
\end{align}
The first and last terms are equal to 0, by virtue of \cref{eq:Il0}. %We drop the $\ell$ subscript from this point.

The $(\delta\Psi_\ell, \mathcal{L} \,\delta\Psi_\ell)$ term is of second order in $\delta\Psi_l$, so this can be neglected. Denoting this approximation as $\delta I_\ell^\prime$ gives
\beq
\label{eq:IlPrimeDef}
\delta I_\ell^\prime = \delta I_\ell - (\delta\Psi_\ell, \mathcal{L} \,\delta\Psi_\ell).
\eeq

Since $\mathcal{L}\Psi_\ell = 0$,
\beq
(\delta\Psi_\ell, \mathcal{L} \Psi_\ell) = -(\delta\Psi_\ell, \mathcal{L} \Psi_\ell),
\eeq

\noindent which combined with the definition of $\mathcal{L}$ from \cref{eq:LDef}, allows us to write the above equation as
\beq
\delta I_\ell^\prime = 2 \Braket{\Psi_\ell^\star | H\!-\!E | \delta\Psi_\ell} - 2 \Braket{\delta\Psi_\ell^\star | H\!-\!E | \Psi_\ell }.
\label{eq:IlPsiVariation2}
\eeq

The Hamiltonian for Ps-H scattering with Coulombic potentials is
\begin{align}
H = -\frac{1}{4} \Laplacian_\rho - \frac{1}{2} \Laplacian_{r_3} - \Laplacian_{r_{12}} + \frac{1}{r_1} - \frac{1}{r_2} - \frac{1}{r_3} - \frac{1}{r_{12}} - \frac{1}{r_{13}} + \frac{1}{r_{23}}.
\label{eq:Hamiltonian2}
\end{align}

\noindent When substituted in \cref{eq:IlPsiVariation2} and using the total energy from \cref{eq:TotalEnergy}, this gives
\begin{align}
\label{eq:IlPsiVariation3}
\delta I_\ell^\prime = 2 \int\limits_{V_{12}} \int\limits_{V_3} \int\limits_{V_\rho} \Psi_\ell &\left[-\frac{1}{4} \Laplacian_\rho - \frac{1}{2} \Laplacian_{r_3} - \Laplacian_{r_{12}} + \frac{1}{r_1} - \frac{1}{r_2} - \frac{1}{r_3} \right. \nonumber \\
  &- \left. \frac{1}{r_{12}} - \frac{1}{r_{13}} + \frac{1}{r_{23}} - E_H - E_{Ps} - \frac{1}{4}\kappa^2 \right] \delta \Psi_\ell \,d\tau_\rho d\tau_{r_3} d\tau_{r_{12}} \nonumber \\
 - 2 \int\limits_{V_{12}} \int\limits_{V_3} \int\limits_{V_\rho} \delta \Psi_\ell &\left[-\frac{1}{4} \Laplacian_\rho - \frac{1}{2} \Laplacian_{r_3} - \Laplacian_{r_{12}} + \frac{1}{r_1} - \frac{1}{r_2} - \frac{1}{r_3} \right. \nonumber \\
  &- \left. \frac{1}{r_{12}} - \frac{1}{r_{13}} + \frac{1}{r_{23}} - E_H - E_{Ps} - \frac{1}{4}\kappa^2 \right] \Psi_\ell \,d\tau_\rho d\tau_{r_3} d\tau_{r_{12}}.
\end{align}

Realizing that the Hamiltonians for H and Ps are given by
\begin{subequations}
\label{eq:HPsHamil}
\begin{align}
H_H =& -\frac{1}{2} \Laplacian_{r_3} - \frac{1}{r_3} \label{eq:HHamil} \\
H_{Ps} =& -\frac{1}{4} \Laplacian_{r_{12}} - \frac{1}{r_{12}}, \label{eq:PsHamil}
\end{align}
\end{subequations}
and rearranging terms, \cref{eq:IlPsiVariation3} becomes
\begin{align}
\label{eq:IlPsiVariation4}
\delta I_\ell^\prime = -&\frac{1}{2} \int\limits_{V_{12}} \int\limits_{V_3} \int\limits_{V_\rho} \left[\Psi \Laplacian_\rho \,\delta\Psi - \delta\Psi \Laplacian_\rho \Psi \right] \,d\tau_\rho d\tau_{r_3} d\tau_{r_{12}} \nonumber \\
+ &2 \int\limits_{V_{12}} \int\limits_{V_3} \int\limits_{V_\rho} \Psi \left[H_H + H_{Ps} + \frac{1}{r_1} - \frac{1}{r_2} - \frac{1}{r_{13}} + \frac{1}{r_{23}} - E_H - E_{Ps} - \frac{1}{4}\kappa^2 \right] \delta\Psi \,d\tau_\rho d\tau_{r_3} d\tau_{r_{12}} \nonumber \\
- &2 \int\limits_{V_{12}} \int\limits_{V_3} \int\limits_{V_\rho} \delta\Psi \left[H_H + H_{Ps} + \frac{1}{r_1} - \frac{1}{r_2} - \frac{1}{r_{13}} + \frac{1}{r_{23}} - E_H - E_{Ps} - \frac{1}{4}\kappa^2 \right] \Psi \,d\tau_\rho d\tau_{r_3} d\tau_{r_{12}}.
\end{align}
Due to the exponential form of $\Phi_{Ps}(r_{12})$ and $\Phi_H(r_3)$ given in \cref{}, the last two lines cancel each other.

From Green's theorem,
\begin{align}
\label{eq:GreensThm}
\nonumber \Braket{\Psi_\ell^\star | \Laplacian_\rho | \delta\Psi_\ell } - \Braket{\delta\Psi_\ell^\star | \laplacian_\rho | \Psi_\ell}
&= \int\limits_{V_3} \int\limits_{V_{12}} \int\limits_{V_\rho} \left[ \Psi \laplacian_\rho \,\delta\Psi - \delta\Psi \laplacian_\rho \Psi \right] d\tau_\rho \, d\tau_{12} d\tau_3 \\
&= \int\limits_{V_3} \int\limits_{V_{12}} \int\limits_{S_\rho} \left[ \Psi \grad_\rho \delta\Psi - \delta\Psi \grad_\rho \Psi \right] \cdot d\bm{\sigma}_\rho d\tau_{12} d\tau_3.
\end{align}
$S_\rho$ is the surface at $\rho \rightarrow \infty$. We can drop the dot product, since the surface elements are in the same direction as $\grad_\rho$.

From \cref{eq:PsilTrialRelation} and \cref{eq:GeneralWaveTrial},
\beq
\label{eq:DeltaPsi}
\delta \Psi = \Psi_\ell^t - \Psi_\ell = (\widetilde{S} + L_\ell^t \, \widetilde{C}) - (\widetilde{S} + L_\ell \, \widetilde{C}) = (L_\ell^t - L_\ell) \widetilde{C} \,.
\eeq
Substituting this into \cref{eq:IlPsiVariation4} with \cref{eq:GreensThm},
\beq
\label{eq:IlPsiVariation5}
\delta I_\ell^\prime = -\frac{1}{2} (L_\ell^t - L_\ell) \int\limits_{V_{12}} \int\limits_{V_3} \int\limits_{S_\rho} \left[(\widetilde{S} + L_\ell \, \widetilde{C}) \grad_\rho \widetilde{C} - \widetilde{C} \grad_\rho (\widetilde{S} + L_\ell \, \widetilde{C}) \right] d\sigma_\rho d\tau_{12} d\tau_3.
\eeq


\noindent Generalizing this to $\widetilde{S}$ and $\widetilde{C}$ gives
\begin{subequations}
\label{eq:GenGenSCGrad}
\begin{align}
\grad_{\rho/\rho^\prime} \widetilde{S} &= \frac{1}{\sqrt{2}} \left[u_{00} \left(\grad_\rho S \pm \grad_{\rho^\prime} S^\prime \right) + u_{01} \left(\grad_\rho C \pm \grad_{\rho^\prime} C^\prime \right) \right] \approx \kappa \left[ u_{00} \bar{C} - u_{01} \bar{S} \right]\\
\grad_{\rho/\rho^\prime} \widetilde{C} &= \frac{1}{\sqrt{2}} \left[u_{10} \left(\grad_\rho S \pm \grad_{\rho^\prime} S^\prime \right) + u_{11} \left(\grad_\rho C \pm \grad_{\rho^\prime} C^\prime \right) \right] \approx \kappa \left[ u_{10} \bar{C} - u_{11} \bar{S} \right].
\end{align}
\end{subequations}

The Laplacian acting on $\widetilde{S}$ and $\widetilde{C}$ in \cref{eq:TildeSCDef} gives


\begin{subequations}
\label{eq:GradSC}
\begin{align}
\grad_\rho \widetilde{S} &= \kappa \left[u_{00} \bar{C} - u_{01} \bar{S} \right] \\
\grad_\rho \widetilde{C} &= \kappa \left[u_{10} \bar{C} - u_{11} \bar{S} \right].
\end{align}
\end{subequations}
Substituting this into \cref{eq:IlPsiVariation5},
\begin{align}
\label{eq:IlPsiVariation6}
\delta I_\ell^\prime = -\frac{1}{2} (L_\ell^t - L_\ell) & \int\limits_{V_{12}} \int\limits_{V_3} \int\limits_{S_\rho}  \left\{(\widetilde{S} + L_\ell \, \widetilde{C}) \kappa (u_{10} \bar{C} - u_{11} \bar{S}) \right. \nonumber \\
 &- \left. \widetilde{C} \kappa \left[(u_{00} \bar{C} - u_{01} \bar{S}) + L_\ell (u_{10} \bar{C} - u_{11} \bar{S}) \right] \right\} d\sigma_\rho d\tau_{12} d\tau_3.
\end{align}
Omitting terms quadratic in $L_\ell$ or $L_\ell^t$, including $L_\ell^t L_\ell$,
\begin{align}
\label{eq:IlPsiVariation6}
\delta I_\ell^\prime = -\frac{1}{2} \kappa (L_\ell^t - L_\ell) & \int\limits_{V_{12}} \int\limits_{V_3} \int\limits_{S_\rho} \left[\widetilde{S} (u_{10} \bar{C} - u_{11} \bar{S}) - \widetilde{C} (u_{00} \bar{C} - u_{01} \bar{S}) \right] d\sigma_\rho d\tau_{12} d\tau_3 \nonumber \\
= -\frac{1}{2} \kappa (L_\ell^t - L_\ell) & \int\limits_{V_{12}} \int\limits_{V_3} \int\limits_{S_\rho} \left[\widetilde{S} u_{10} \bar{C} - \widetilde{S} u_{11} \bar{S} - \widetilde{C} u_{00} \bar{C} + \widetilde{C} u_{01} \bar{S} \right] d\sigma_\rho d\tau_{12} d\tau_3 \nonumber \\
= -\frac{1}{2} \kappa (L_\ell^t - L_\ell) & \int\limits_{V_{12}} \int\limits_{V_3} \int\limits_{S_\rho} \left[u_{00} u_{10} \bar{S} \bar{C} + u_{01} u_{10} \bar{C}^2 - u_{00} u_{11} \bar{S}^2 - u_{01} u_{11} \bar{C} \bar{S}\right. \nonumber \\
& \left. - u_{10} u_{00} \bar{S} \bar{C} - u_{11} u_{00} \bar{C}^2 + u_{10} u_{01} \bar{S}^2 + u_{11} u_{01} \bar{C} \bar{S}   \right] d\sigma_\rho d\tau_{12} d\tau_3 \nonumber \\
= -\frac{1}{2} \kappa (L_\ell^t - L_\ell) & \int\limits_{V_{12}} \int\limits_{V_3} \int\limits_{S_\rho} (u_{10} u_{01} - u_{00} u_{11}) \left( \bar{S}^2 + \bar{C}^2 \right) d\sigma_\rho d\tau_{12} d\tau_3 \nonumber \\
= \frac{1}{2} \kappa (L_\ell^t - L_\ell) & \Det{\textbf{u}} \int\limits_{V_{12}} \int\limits_{V_3} \int\limits_{S_\rho} \left( \bar{S}^2 + \bar{C}^2 \right) d\sigma_\rho d\tau_{12} d\tau_3.
\end{align}

The rest of this derivation considers only the direct terms. The final result applies as well when the exchanged terms are included. Since we are considering the surface as $\rho \rightarrow \infty$, $f_\ell(\rho)$ in \cref{eq:GenCDef} becomes 1. Then from \cref{eq:GenSandC},
\beq
S^2 + C^2 = \SphericalHarmonicY{\ell}{0}{\theta_\rho}{\phi_\rho}^2 \Phi_{Ps}\left(r_{12}\right)^2 \Phi_H\left(r_3\right)^2 (2 \kappa) \left[j_\ell\!\left(\kappa\rho\right)^2 + n_\ell\!\left(\kappa\rho\right)^2\right].
\eeq
The asymptotic forms of $j_\ell$ and $n_\ell$ are given by \cite[p.729]{Arfken2005}
\begin{subequations}
\label{eq:AsymSphBes}
\begin{align}
\lim_{\rho \to \infty} j_\ell\!\left(\kappa\rho\right) &= \frac{1}{\kappa\rho} \sin\left(\kappa\rho - \frac{n \pi}{2}\right) \label{eq:AsymJl} \\
\lim_{\rho \to \infty} n_\ell\!\left(\kappa\rho\right) &= \frac{1}{\kappa\rho} \cos\left(\kappa\rho - \frac{n \pi}{2}\right). \label{eq:AsymNl}
\end{align}
\end{subequations}
As $\rho \to \infty$,
\beq
S^2 + C^2 = \SphericalHarmonicY{\ell}{0}{\theta_\rho}{\phi_\rho}^2 \Phi_{Ps}\left(r_{12}\right)^2 \Phi_H\left(r_3\right)^2 (2 \kappa) \frac{1}{\kappa^2 \rho^2}.
\eeq

Substituting this in \cref{eq:IlPsiVariation6} and expanding the $d\sigma_\rho$ differential,
\begin{align}
\label{eq:IlPsiVariation7}
\delta I_\ell^\prime =& \kappa (L_\ell^t - L_\ell) \Det{\textbf{u}} \int\limits_{V_{12}} \int\limits_{V_3} \int\limits_{S_\rho} \SphericalHarmonicY{\ell}{0}{\theta_\rho}{\phi_\rho}^2 \Phi_{Ps}\left(r_{12}\right)^2 \Phi_H\left(r_3\right)^2 \frac{1}{\kappa \rho^2} \rho^2 \sin\theta_\rho d\theta_\rho d\phi_\rho d\tau_{12} d\tau_3 \nonumber \\
=& (L_\ell^t - L_\ell) \Det{\textbf{u}} \int\limits_{V_{12}} \int\limits_{V_3} \int\limits_{S_\rho} \SphericalHarmonicY{\ell}{0}{\theta_\rho}{\phi_\rho}^2 \Phi_{Ps}\left(r_{12}\right)^2 \Phi_H\left(r_3\right)^2 \sin\theta_\rho d\theta_\rho d\phi_\rho d\tau_{12} d\tau_3.
\end{align}
Since the Ps and H eigenfunctions are normalized, i.e.
\beq
\int\limits_{V_3}\! \Phi_H(r_3) d\tau_3 = 1 \text{ and } \int\limits_{V_{12}}\! \Phi_{Ps}(r_{12}) d\tau_{12} = 1,
\label{eq:PsHNormalization}
\eeq
we now have
\beq
\label{eq:IlPsiVariation8}
\delta I_\ell^\prime = (L_\ell^t - L_\ell) \Det{\textbf{u}} \int\limits_{S_\rho} \SphericalHarmonicY{\ell}{0}{\theta_\rho}{\phi_\rho}^2 \sin\theta_\rho d\theta_\rho d\phi_\rho.
\eeq
The spherical harmonics are normalized so that \cite[p.788]{Arfken2005}
\beq
\label{eq:SphHarmNorm}
\int\limits_{S_\rho} \SphericalHarmonicY{\ell}{0}{\theta_\rho}{\phi_\rho}^2 d\Omega = 1.
\eeq
This gives that
\beq
\label{eq:IlPsiVariation9}
\delta I_\ell^\prime = (L_\ell^t - L_\ell) \Det{\textbf{u}}.
\eeq

From \cref{eq:IlPrimeDef,eq:IlPsiVariation9},
\beq
\label{eq:KatoIdent}
\delta I_\ell^\prime = (L_\ell^t - L_\ell) \Det{\textbf{u}} + (\delta\Psi_\ell, \mathcal{L} \,\delta\Psi_\ell).
\eeq
This is the Kato identity \cite{Kato1951a}. For the Kohn variational methods, the last term is neglected, since it is quadratic in $\delta\Psi_\ell$. Using $\delta I_\ell^\prime \approx \delta I_\ell$, we have
\beq
\delta I_\ell = I_\ell[\Psi_\ell^t] - I_\ell[\Psi_\ell] = (L_\ell^t - L_\ell) \Det{\textbf{u}}.
\eeq
Replacing the exact $L_\ell$ by the variational $L_\ell^v$ and rearranging, we finally get the general Kohn variational method of
\beq
\label{eq:GenKohn}
L_\ell^v = L_\ell^t - I_\ell[\Psi_\ell^t] / \! \Det{\textbf{u}}.
\eeq


\section{Application of the Kohn Methods}
\label{sec:KohnApplied}

We use the general Kohn variational method (\cref{eq:GenKohn}) with our full trial wavefunction to get
\beq
\label{eq:GenKohnApplied}
L_\ell^v = L_\ell^t - \tfrac{1}{\Det{\textbf{u}}} \Big((\widetilde{S} + L_\ell^t \, \widetilde{C} + \sum_i c_i \bar{\phi}_i^t), \mathcal{L} (\widetilde{S} + L_\ell^t \, \widetilde{C} + \sum_j c_j \bar{\phi}_j^t )\Big).
\eeq
The property of the Kohn functional that it is stationary with respect to variations in the linear parameters \cite{Joachain1979} can be written in our case as
\beq
\frac{\partial L_\ell^v}{\partial L_\ell^t} = 0  \text{ and } \frac{\partial L_\ell^v}{\partial c_i} = 0 \text{, where $i = 1,\ldots,N$}.
\label{eq:KohnStationary}
\eeq

Performing the first variation gives
\beq
0 = \pderiv{L_\ell^v}{L_\ell^t} = 1 - \left[(\widetilde{S},\mathcal{L}\widetilde{C}) + (\widetilde{C},\mathcal{L}\widetilde{S}) + \frac{\partial}{\partial L_\ell^t}(L_\ell^t \widetilde{C},\mathcal{L} L_\ell^t \widetilde{C}) + (\widetilde{C}, L \sum_i c_i \bar{\phi}_i) + (\sum_i c_i \bar{\phi}_i, \mathcal{L} \widetilde{C}) \right].
\label{eq:PdLambda1}
\eeq

\noindent The third term in brackets becomes
\beq
\pderiv{}{L_\ell^t} (L_\ell^t \widetilde{C},\mathcal{L} L_\ell^t \widetilde{C}) = (\widetilde{C},\mathcal{L} \widetilde{C}) \frac{\partial}{\partial L_\ell^t} {L_\ell^t}^2 = 2(\widetilde{C},\mathcal{L}\widetilde{C}) L_\ell^t.
\eeq

\noindent The last two terms of \cref{eq:PdLambda1} are equal to each other, and we can use \cref{eq:SLCandCLSBar} to rewrite this.
\beq
0 = -(\widetilde{C},\mathcal{L}\widetilde{S}) - (\widetilde{C},\mathcal{L}\widetilde{S}) - 2 L_\ell^t (\widetilde{C},\mathcal{L}\widetilde{C}) - 2 \sum_i c_i (\widetilde{C},\mathcal{L}\bar{\phi}_i)
\eeq

\noindent Rearranging gives
\beq
-(\widetilde{C},\mathcal{L}\widetilde{S}) = L_\ell^t (\widetilde{C},\mathcal{L}\widetilde{C}) + \sum_i c_i (\widetilde{C},\mathcal{L}\bar{\phi}_i)
\label{eq:PdLambda}
\eeq

Now we perform the variation with respect to a general $c_k$ as in \cref{eq:KohnStationary}.
\beq
0 = \frac{\partial \mathcal{L}_v}{\partial c_k} = -\left[ (\widetilde{S},\mathcal{L} \bar{\phi}_k) + L_\ell^t (\widetilde{C},\mathcal{L} \bar{\phi}_k) + (\bar{\phi}_k,\mathcal{L} \widetilde{S}) + L_\ell^t (\bar{\phi}_k,\mathcal{L} \widetilde{C}) + \frac{\partial}{\partial c_k} (\sum_i c_i \bar{\phi}_i, \mathcal{L} \sum_j c_j \bar{\phi}_j) \right]
\label{eq:PdCk1}
\eeq

If $c_i \ne c_j$,
\begin{subequations}
\begin{align}
\frac{\partial}{\partial c_i} (c_i \bar{\phi}_i, \mathcal{L} \sum_{j \ne i} c_j \bar{\phi}_j) &= \sum_{j \ne i} c_j (\bar{\phi}_i, \mathcal{L} \bar{\phi}_j) \text{ and} \\
\frac{\partial}{\partial c_j} (\sum_{i \ne j} c_i \bar{\phi}_i, \mathcal{L} c_j \bar{\phi}_j) &= \sum_{i \ne j} c_i (\bar{\phi}_i, \mathcal{L} \bar{\phi}_j).
\end{align}
\end{subequations}

\noindent These two equations are equivalent, since $\left( \bar{\phi}_i, \mathcal{L} \bar{\phi}_j \right) = \left( \bar{\phi}_j, \mathcal{L}L \bar{\phi}_i \right)$ by \cref{PhiLPhiPerm}.

If $c_i = c_j$,
\beq
\frac{\partial}{\partial c_i} \left( c_i \bar{\phi}_i, L c_j \bar{\phi}_j \right) = \frac{\partial}{\partial c_i} \left( c_i \bar{\phi}_i, \mathcal{L} c_i \bar{\phi}_i \right) = \frac{\partial}{\partial c_i} c_i^2 \left( \bar{\phi}_i, \mathcal{L} \bar{\phi}_j \right) = 2 \, c_i \left( \bar{\phi}_i, \mathcal{L} \bar{\phi}_j \right).
\eeq

\noindent We can also use \cref{eq:PhiLSPerm,eq:PhiLCPerm} to reduce \cref{eq:PdCk1} to
\beq
0 = -\Big[ 2 (\bar{\phi}_k, \mathcal{L} \widetilde{S}) + 2 L_\ell^t (\bar{\phi}_k, \mathcal{L} \widetilde{C}) + 2 \sum_i (\bar{\phi}_k, \mathcal{L} c_i \bar{\phi}_i) \Big].
\eeq

\noindent
Rearranging gives
\beq
-\left( \bar{\phi}_k, \mathcal{L} \widetilde{S} \right) = L_\ell^t \left( \bar{\phi}_k, \mathcal{L} \widetilde{C} \right) + \sum_i \left( \bar{\phi}_k, \mathcal{L} c_i \bar{\phi}_i \right).
\label{eq:PdCk}
\eeq

The set of linear equations in \cref{eq:PdLambda,eq:PdCk} can be written in matrix form as
\begin{equation}
\label{eq:GeneralKohnMatrix}
\begin{bmatrix} 
 (\widetilde{C},\mathcal{L}\widetilde{C}) & (\widetilde{C},\mathcal{L}\bar{\phi}_1) & \cdots & (\widetilde{C},\mathcal{L}\bar{\phi}_j) & \cdots\\
 (\bar{\phi}_1,\mathcal{L}\widetilde{C}) & (\bar{\phi}_1,\mathcal{L}\bar{\phi}_1) & \cdots & (\bar{\phi}_1,\mathcal{L}\bar{\phi}_j) & \cdots\\
 \vdots & \vdots & \ddots & \vdots \\
 (\bar{\phi}_i,\mathcal{L}\widetilde{C}) & (\bar{\phi}_i,\mathcal{L}\bar{\phi}_1) & \cdots & (\bar{\phi}_i,\mathcal{L}\bar{\phi}_j) & \cdots\\
 \vdots & \vdots & & \vdots & \\
\end{bmatrix}
\begin{bmatrix}
L_\ell^t\\
c_1\\
\vdots\\
c_i\\
\vdots
\end{bmatrix}
= -
\begin{bmatrix}
(\widetilde{C},\mathcal{L}\widetilde{S}) \\
(\bar{\phi}_1,\mathcal{L}\widetilde{S}) \\
\vdots \\
(\bar{\phi}_i,\mathcal{L}\widetilde{S}) \\
\vdots
\end{bmatrix}.
\end{equation}

\noindent This matrix equation can be rewritten as
\beq
\label{eq:GenKohnMatrixAXB}
\textbf{\emph{AX = -B}}.
\eeq

\noindent Solving this for $\textbf{\emph{X}}$ gives
\beq
\textbf{\emph{X = $-A^{-1}$B}}.
\eeq

To obtain $\mathcal{L}_v$ from this matrix equation, we must next expand \cref{eq:GenKohnApplied}.

\begin{align}
\nonumber \mathcal{L}_v = L_\ell^t - &\left[ (\widetilde{S},\mathcal{L}\widetilde{S}) + L_\ell^t (\widetilde{S},\mathcal{L}\widetilde{C}) + \sum_i c_i (\widetilde{S},\bar{\phi}_i) + L_\ell^t (\widetilde{C},\mathcal{L}\widetilde{S}) + {L_\ell^t}^2 (\widetilde{C},\mathcal{L}\widetilde{C}) + L_\ell^t \sum_i c_i (\widetilde{C},\mathcal{L} \bar{\phi}_i) \right. \\
& + \left. \sum_i c_i (\bar{\phi}_i, \mathcal{L} \widetilde{S}) + L_\ell^t \sum_i c_i (\bar{\phi}_i, \mathcal{L} \widetilde{C}) + \sum_i \sum_j c_i c_j (\bar{\phi}_i, \mathcal{L} \bar{\phi}_j) \right]
\end{align}

\noindent By substituting \cref{eq:GenSLCandCLS} in for $(\widetilde{S},\mathcal{L}\widetilde{C})$, the first $L_\ell^t$ above is canceled, leaving

\begin{align}
\label{eq:GenKohnApplied2}
\nonumber \mathcal{L}_v = - & \left[ (\widetilde{S},\mathcal{L}\widetilde{S}) + L_\ell^t (\widetilde{C},\mathcal{L}\widetilde{S}) + \sum_i c_i (\widetilde{S},\bar{\phi}_i) + L_\ell^t (\widetilde{C},\mathcal{L}\widetilde{S}) + {L_\ell^t}^2 (\widetilde{C},\mathcal{L}\widetilde{C}) + L_\ell^t \sum_i c_i (\widetilde{C},\mathcal{L} \bar{\phi}_i) \right. \\
& + \left. \sum_i c_i (\bar{\phi}_i, \mathcal{L} \widetilde{S}) + L_\ell^t \sum_i c_i (\bar{\phi}_i, \mathcal{L} \widetilde{C}) + \sum_i \sum_j c_i c_j (\bar{\phi}_i, \mathcal{L} \bar{\phi}_j) \right].
\end{align}

Using the following definitions of
\beq
D = 
\begin{bmatrix}
L_\ell^t & c_1 & \cdots & c_N
\end{bmatrix}
\text{ and}
\eeq
\beq
\label{eq:GenFandD}
F =
\begin{bmatrix}
(\boldsymbol{\widetilde{C},\mathcal{L}\widetilde{C}}) & (\boldsymbol{\widetilde{C},\mathcal{L}\bar{\phi}}) & (\boldsymbol{\widetilde{C},\mathcal{L}\widetilde{S}}) \\
(\boldsymbol{\bar{\phi},\mathcal{L}\widetilde{C}}) & (\boldsymbol{\bar{\phi},\mathcal{L}\bar{\phi}}) & (\boldsymbol{\bar{\phi},\mathcal{L}\widetilde{S}}) \\
(\boldsymbol{\widetilde{C},\mathcal{L}\widetilde{S}}) & (\boldsymbol{\widetilde{S},\mathcal{L}\bar{\phi}}) & (\boldsymbol{\widetilde{S},\mathcal{L}\widetilde{S}})
\end{bmatrix},
\eeq
\cref{eq:GenKohnApplied2} can be rewritten as the following matrix equation:

\beq
\label{eq:GenDFDT}
L_\ell^v = - D F D^T.
\eeq

\noindent Using \cref{eq:GenKohnMatrix,eq:GenKohnMatrix} in \cref{eq:GenDFDT} and expanding gives
\begin{align}
\label{eq:GenDFDT2}
\nonumber L_\ell^v &= - 
\begin{bmatrix}
\boldsymbol{X^T} & 1 
\end{bmatrix}
\begin{bmatrix}
\boldsymbol{A} & \boldsymbol{B} \\
\boldsymbol{B^T} & \boldsymbol{(\widetilde{S},\mathcal{L}\widetilde{S})}
\end{bmatrix}
\begin{bmatrix}
\boldsymbol{X} \\
1
\end{bmatrix}
= -
\begin{bmatrix}
\boldsymbol{X^T} & 1 
\end{bmatrix}
\begin{bmatrix}
0 \\
\boldsymbol{B^T X} + (\widetilde{S},\mathcal{L}\widetilde{S})
\end{bmatrix} \\
&= -\boldsymbol{B^T X} - (\widetilde{S},\mathcal{L}\widetilde{S}),
\end{align}
where
\beq
\boldsymbol{B^T X} = L_\ell^t (\widetilde{C},\mathcal{L}\widetilde{S}) + \sum_i c_i (\bar{\phi}_i, \mathcal{L} \widetilde{S}).
\eeq

\noindent A more compact way of writing \cref{eq:GenDFDT2} is by
\beq
L_\ell^v = -\left( \Psi^{t,0},\mathcal{L} \widetilde{S} \right).
\eeq
$\Psi^{t,0}$ is the full general wavefunction in \cref{eq:GeneralWaveTrial} with its nonlinear parameters optimized.






















The $\textbf{u}$ and $\mathcal{L}_l$ for the various Kohn methods are described now. Note that for each of these, $\Det{\textbf{u}} = 1$.

\subsubsection*{Kohn}
\beq
\textbf{u} =
\begin{bmatrix}
1 & 0 \\
0 & 1 
\end{bmatrix}
\label{eq:uKohn}
\eeq

\beq
\mathcal{L}_l = \lambda_t = K_t
\label{eq:LKohn}
\eeq


\subsubsection*{Inverse Kohn}
\beq
\textbf{u} =
\begin{bmatrix}
0 & 1 \\
-1 & 0 
\end{bmatrix}
\label{eq:uInvKohn}
\eeq

\beq
\mathcal{L}_l = -\mu_t = -K^{-1}_t = -\bar{K}_t
\label{eq:LInvKohn}
\eeq


\subsubsection*{Generalized Kohn}
\beq
\textbf{u} =
\begin{bmatrix}
\cos\tau & \sin\tau \\
-\sin\tau & \cos\tau 
\end{bmatrix}
\label{eq:uGenKohn}
\eeq

\beq
\mathcal{L}_l = a_t
\label{eq:LGenKohn}
\eeq

\noindent The generalized Kohn method is described by Cooper et al.\ \cite{Cooper2009, Cooper2010}.  When $\tau = 0$ is substituted in \cref{eq:uGenKohn}, the $\textbf{u}$-matrix for the Kohn method is generated (\cref{eq:uKohn}). Similarly, when $\tau = \frac{\pi}{2}$, the $\textbf{u}$-matrix for the inverse Kohn method is generated (\cref{eq:uInvKohn}). As to be expected, the Kohn and inverse Kohn methods are special cases of the generalized Kohn method.


\subsubsection*{Complex Kohn $T$-matrix}
\beq
\textbf{u} =
\begin{bmatrix}
1 & 0 \\
\ii & 1
\end{bmatrix}
\label{eq:uCompTKohn}
\eeq

\beq
\mathcal{L}_l = T
\label{eq:LCompTKohn}
\eeq

\noindent Lucchese denotes this as $\mathcal{L}_l = -\pi T$ \cite{Lucchese1989}, but to be consistent with our definition of the $T$-matrix, we must use equation \cref{eq:LCompTKohn}.


\subsubsection*{Complex Kohn $S$-matrix}
\beq
\textbf{u} =
\begin{bmatrix}
-\ii & 1 \\
\ii & 1
\end{bmatrix}
\label{eq:uCompSKohn}
\eeq

\beq
\mathcal{L}_l = 2 \ii S_\ell
\label{eq:LCompSKohn}
\eeq
\todoi{This is probably not correct now.}

The Lucchese version of $\textbf{u}$ differs from ours, since he uses a different definition for the $S$-matrix  \cite{Lucchese1989}. The form of the $S$-matrix we are using with our wavefunction is related to the $K$-matrix by
\beq
K_\ell = \frac{\ii(1-S_\ell)}{1+S_\ell},
\eeq
which is satisfied by the above $\textbf{u}$-matrix.

\subsubsection*{Generalized $T$ matrix Kohn}
\beq
\textbf{u} =
\begin{bmatrix}
\cos\tau & \sin\tau \\
-\sin\tau + \ii \cos\tau & \cos\tau + \ii \sin\tau
\end{bmatrix}
\label{eq:uGenKohn}
\eeq


\subsubsection*{Generalized $S$ matrix Kohn}
\beq
\textbf{u} =
\begin{bmatrix}
\ii \cos\tau - \sin\tau & -\ii \sin\tau + \cos\tau \\
\ii \cos\tau - \sin\tau & \ii \sin\tau + \cos\tau
\end{bmatrix}
\label{eq:uGenKohn}
\eeq


\section{Schwartz Singularities}
\label{eq:SchwartzSing}





\end{document}