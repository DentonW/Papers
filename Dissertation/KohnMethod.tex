\documentclass[Dissertation.tex]{subfiles} 
\begin{document}


\chapter{Wavefunction and Kohn Variational Methods}


\section{General Wavefunction}

The applications of the Kohn, inverse Kohn, complex Kohns, generalized Kohn, and generalized complex Kohns to our trial wavefunctions for each of the partial waves through the H-wave are all very similar in form. The trial wavefunctions for the partial waves (\cref{eq:SWaveTrialSimple,eq:PWaveSimple,eq:DWaveSimple}) can be written in a general form as

\beq
\Psi_\ell^{\pm,t} = \widetilde{S}_\ell + L_\ell^{\pm,t} \, \widetilde{C}_\ell + \sum_{i=1}^N c_i \bar{\Phi}_i.
\label{eq:GeneralWaveTrial}
\eeq

The short-range $\bar{\Phi}_i$ terms can represent terms of different symmetries, such as the $\bar{\phi}_{1i}$ and $\bar{\phi}_{2j}$ of the P-wave in \cref{eq:PWaveSimple}. The only requirement in this derivation is that these are Hylleraas-type short-range terms. In addition to letting the $\widetilde{S}$ and $\widetilde{C}$ represent the $\bar{S}$ and $\bar{C}$ for the different partial waves, we can define them in such a way as to use multiple Kohn methods (Kohn, inverse Kohn, etc.). We begin by defining a $2\times 2$ matrix $\textbf{u}$ which satisfies

\beq
\label{eq:GenSCMatrix}
\begin{bmatrix}
\widetilde{S} \\
\widetilde{C}
\end{bmatrix}
=
\textbf{u}
\begin{bmatrix}
\bar{S} \\
\bar{C}
\end{bmatrix}
=
\begin{bmatrix}
u_{00} & u_{01} \\
u_{10} & u_{11}
\end{bmatrix}
\begin{bmatrix}
\bar{S} \\
\bar{C}
\end{bmatrix}.
\eeq

\noindent This notation is similar to that of Lucchese \cite{Lucchese1989} and Cooper, Plummer, and Armour \cite{Cooper2010}. From this, it can easily be seen that
\begin{subequations}
\label{eq:TildeSCDef}
\begin{align}
\widetilde{S}_\ell &= u_{00} \bar{S}_\ell + u_{01} \bar{C}_\ell  \label{eq:TildeSDef} \\
\widetilde{C}_\ell &= u_{10} \bar{S}_\ell + u_{11} \bar{C}_\ell. \label{eq:TildeCDef}
\end{align}
\end{subequations}

Generally,
\beq
\bar{S}_\ell = \frac{1}{\sqrt{2}}(S_\ell \pm S_\ell^\prime) \text{ and } \bar{C}_\ell = \frac{1}{\sqrt{2}}(C_\ell \pm C_\ell^\prime),
\eeq
where $S_\ell^\prime = P_{23}S_\ell$ and $C_\ell^\prime = P_{23}C_\ell$.
The general form for the long-range terms $S_\ell$ and $C_\ell$ is
\begin{subequations}
\label{eq:GenSandC}
\begin{align}
S_\ell = \,&\SphericalHarmonicY{\ell}{0}{\theta_\rho}{\phi_\rho} \Phi_{Ps}\left(r_{12}\right) \Phi_H\left(r_3\right) \sqrt{2\kappa} \,j_\ell\!\left(\kappa\rho\right) \label{eq:GenSDef} \\
C_\ell = -&\SphericalHarmonicY{\ell}{0}{\theta_\rho}{\phi_\rho} \Phi_{Ps}\left(r_{12}\right) \Phi_H\left(r_3\right) \sqrt{2\kappa} \,n_\ell\!\left(\kappa\rho\right) f_\ell(\rho) \label{eq:GenCDef}.
\end{align}
\end{subequations}
The shielding function, $f_\ell(\rho)$, removes the singularity at the origin due to the Neumann function, $n_\ell$. The form that we have chosen for this is
\begin{equation}
  \label{eq:PartialWaveShielding}
  f_\ell(\rho) = \left[1 - \ee^{-\mu \rho} \left(1+\frac{\mu}{2}\rho\right)
  \right]^{m_\ell}.
\end{equation}
At a minimum, $m_\ell$ is chosen so that $C_\ell$ behaves like $S_\ell$ as $\rho \to 0$. For more discussion of this, see \Cref{sec:ShieldingFunc}. The values used for the different partial waves are given in \cref{tab:Shielding}.

\todoi{Put in plots from ``Shielding Function Plots.nb''}

The Hylleraas-type short-range terms are similar to that used in the bound state in \cref{eq:BoundWavefn}. These are chosen to have two symmetries, one with a prefactor of $r_1^\ell$ and the other with a prefactor of $r_2^\ell$. From Refs.~\cite{Schwartz1961a,VanReethThesis}, the D-wave and higher can have additional symmetries where the angular momentum is shared between the Ps and H. We do not consider these mixed terms here. \todo{Justification}

\begin{subequations}
\begin{align}
  \label{eq:PhiDef}
  \bar{\phi}_{1i} &= \left(1 \pm P_{23}\right) \SphericalHarmonicY{\ell}{0}{\theta_1}{\phi_1}
  e^{-(\alpha r_1 + \beta r_2 + \gamma r_3)}
  r_1^{\ell} r_1^{k_i} r_2^{l_i} r_{12}^{m_i} r_3^{n_i} r_{13}^{p_i} r_{23}^{q_i} \\
  \bar{\phi}_{2j} &= \left(1 \pm P_{23}\right) \SphericalHarmonicY{\ell}{0}{\theta_2}{\phi_2}
  e^{-(\alpha r_1 + \beta r_2 + \gamma r_3)}
  r_2^{\ell} r_1^{k_j} r_2^{l_j} r_{12}^{m_j} r_3^{n_j} r_{13}^{p_j} r_{23}^{q_j}.
\end{align}
\end{subequations}

The S-wave only has a single symmetry, so $\bar{\Phi}_i$ is a single set of terms.
Similar to \cref{sec:BoundWavefn}, the $\frac{1}{\sqrt{2}}$ is absorbed into $c_i$. The full S-wave trial wavefunction can be written as
\begin{equation}
  \label{eq:TrialWave}
  \Psi_0^{\pm,t} = \widetilde{S}_0 + L_0^{\pm,t} \, \widetilde{C}_0 + \sum_{i=1}^{N(\omega)} c_{i,0} \bar{\phi}_{i1}.
\end{equation}

For the P-wave and higher ($\ell > 0$), 

\begin{equation}
  \label{eq:TrialWaveHigher}
  \Psi_\ell^{\pm,t} = \widetilde{S}_\ell + L^{\pm,t}_\ell \, \widetilde{C}_\ell
  + \sum_{i=1}^{N(\omega)} c_{i,\ell} \bar{\phi}_{i1}
  + \!\!\!\sum_{i=N(\omega)+1}^{2N(\omega)} \!\! d_{i,\ell} \bar{\phi}_{i2}.
\end{equation}

$\rho$ and $\rho'$ are defined as (refer to \cref{fig:PsHCoords} and \Cref{chp:RhoDef})
\begin{subequations}
\begin{align}
\vec{\rho} &= \frac{1}{2}\left(\vec{r_1} + \vec{r_2}\right) \label{eq:RhoDef}\\
\vec{\rho}^\prime &= \frac{1}{2}\left(\vec{r_1} + \vec{r_3}\right) \label{eq:RhopDef}.
\end{align}
\end{subequations}


\section{General Kohn Principle Derivation}
\label{sec:KohnDerivation}

Much of this derivation is similar to that in Peter Van Reeth's thesis \cite{
VanReethThesis} but is for single channel scattering and also generalized to 
a variety of Kohn variational methods. His thesis covers the Kohn and inverse 
Kohn methods for two channel e$^+$-He scattering. For this derivation, I will 
use \cref{eq:GeneralWaveTrial} but drop the short-range $\bar{\phi}_i^t$ 
terms. The derivation follows through the same with these terms, but it is 
clearer to ignore them here. Likewise, we only consider the direct terms 
here, unless otherwise specified. The final result of this section applies 
equally well to both the direct and exchanged terms.

The functional $I_\ell$ is defined as \cite{Adhikari1998}
\begin{equation}
I[\Psi_\ell^t]\equiv \left<{\Psi_\ell^t}^\star | \mathcal{L} | \Psi_\ell^t \right> = \left(\Psi_\ell^t, \mathcal{L} \Psi_\ell^t \right) = \int \Psi_\ell^t \mathcal{L}
  \Psi_\ell^t \,d\tau,
\label{eq:IlDefPsi}
\end{equation}
where the operator $\mathcal{L}$ is given by
\beq
\label{eq:LDef}
\mathcal{L} = 2(H-E).
\eeq
Note that the exact wavefunction $\Psi_\ell$ solves the Schr\"{o}dinger equation, giving
\beq
\label{eq:Il0}
I[\Psi_\ell] = 0.
\eeq
Normally, the bra in bra-ket notation is conjugated, but as noted by Cooper et al. \cite{Cooper2010}, the bra is not conjugated for the Kohn variational methods.

The trial wavefunction is related to the exact solution by
\beq
\label{eq:PsilTrialRelation}
\Psi_\ell^t = \Psi_\ell + \delta \Psi_\ell.
\eeq
The variation of $I_\ell$ is
\begin{align}
\label{eq:IlPsiVariation1}
\nonumber \delta I_\ell &= I_\ell[\Psi_\ell^t] - I_\ell[\Psi_\ell] \\
\nonumber &= I_\ell[\Psi_\ell + \delta \Psi_\ell] - I_\ell[\Psi_\ell] \\
&= (\Psi_\ell, \mathcal{L} \Psi_\ell) + (\Psi_\ell, \mathcal{L} \,\delta\Psi_\ell) + (\delta\Psi_\ell, \mathcal{L} \Psi_\ell) + (\delta\Psi_\ell, \mathcal{L} \,\delta\Psi_\ell) - (\Psi_\ell, \mathcal{L} \Psi_\ell).
\end{align}
The first and last terms are equal to 0, by virtue of \cref{eq:Il0}. %We drop the $\ell$ subscript from this point.

The $(\delta\Psi_\ell, \mathcal{L} \,\delta\Psi_\ell)$ term is of second order in $\delta\Psi_\ell$, so this can be neglected. Denoting this approximation as $\delta I_\ell^\prime$ gives
\beq
\label{eq:IlPrimeDef}
\delta I_\ell^\prime = \delta I_\ell - (\delta\Psi_\ell, \mathcal{L} \,\delta\Psi_\ell).
\eeq

Since $\mathcal{L}\Psi_\ell = 0$,
\beq
(\delta\Psi_\ell, \mathcal{L} \Psi_\ell) = -(\delta\Psi_\ell, \mathcal{L} \Psi_\ell),
\eeq

\noindent which combined with the definition of $\mathcal{L}$ from \cref{eq:LDef}, allows us to write the above equation as
\beq
\delta I_\ell^\prime = 2 \Braket{\Psi_\ell^\star | H\!-\!E | \delta\Psi_\ell} - 2 \Braket{\delta\Psi_\ell^\star | H\!-\!E | \Psi_\ell }.
\label{eq:IlPsiVariation2}
\eeq

The wavenumber is defined as
\beq
\kappa = \frac{\sqrt{2 m E}}{\hbar}
\label{eq:Wavenumber}
\eeq

From \cref{eq:Wavenumber}, since Ps has twice the mass of a positron, $m \to 2m$, the kinetic energy of the positronium atom is $T = \frac{1}{4} \kappa^2$, where $\kappa$ is the wavenumber of Ps. Including the dissociation energy of H and Ps to get the total energy gives
\beq
E = E_H + E_{Ps} + \frac{1}{4} \kappa^2.
\eeq
For the ground states of H and Ps,
\beq
E = -\frac{1}{2} - \frac{1}{4} + \frac{1}{4} \kappa^2 = -\frac{3}{4} + \frac{1}{4} \kappa^2 \:\: \text{(in a.u.)}.
\label{eq:EnergyTotal}
\eeq

The Hamiltonian for the fundamental Coulombic system is
\begin{align}
\label{eq:Hamiltonian1}
H = -\frac{1}{2} \Laplacian_{r_1} - \frac{1}{2} \Laplacian_{r_2} - \frac{1}{2} \Laplacian_{r_3} + \frac{1}{r_1} - \frac{1}{r_2} - \frac{1}{r_3} - \frac{1}{r_{12}} -\frac {1}{r_{13}} + \frac{1}{r_{23}}.
\end{align}

\noindent The Hamiltonian can also be expressed in terms of other variables as
\begin{align}
H = -\frac{1}{4} \Laplacian_\rho - \frac{1}{2} \Laplacian_{r_3} - \Laplacian_{r_{12}} + \frac{1}{r_1} - \frac{1}{r_2} - \frac{1}{r_3} - \frac{1}{r_{12}} - \frac{1}{r_{13}} + \frac{1}{r_{23}}
\label{eq:Hamiltonian2}
\end{align}
and
\begin{align}
H = -\frac{1}{4} \Laplacian_{\rho^\prime} - \frac{1}{2} \Laplacian_{r_2} - \Laplacian_{r_{13}} + \frac{1}{r_1} - \frac{1}{r_2} - \frac{1}{r_3} - \frac{1}{r_{12}} - \frac{1}{r_{13}} + \frac{1}{r_{23}}.
\label{eq:Hamiltonian3}
\end{align}

\noindent Substituting the second form of $H$ in \cref{eq:IlPsiVariation2} and using the total energy from \cref{eq:TotalEnergy} gives
\begin{align}
\label{eq:IlPsiVariation3}
\delta I_\ell^\prime = 2 \int\limits_{V_{12}} \int\limits_{V_3} \int\limits_{V_\rho} \Psi_\ell &\left[-\frac{1}{4} \Laplacian_\rho - \frac{1}{2} \Laplacian_{r_3} - \Laplacian_{r_{12}} + \frac{1}{r_1} - \frac{1}{r_2} - \frac{1}{r_3} \right. \nonumber \\
  &- \left. \frac{1}{r_{12}} - \frac{1}{r_{13}} + \frac{1}{r_{23}} - E_H - E_{Ps} - \frac{1}{4}\kappa^2 \right] \delta \Psi_\ell \,d\tau_\rho d\tau_{r_3} d\tau_{r_{12}} \nonumber \\
 - 2 \int\limits_{V_{12}} \int\limits_{V_3} \int\limits_{V_\rho} \delta \Psi_\ell &\left[-\frac{1}{4} \Laplacian_\rho - \frac{1}{2} \Laplacian_{r_3} - \Laplacian_{r_{12}} + \frac{1}{r_1} - \frac{1}{r_2} - \frac{1}{r_3} \right. \nonumber \\
  &- \left. \frac{1}{r_{12}} - \frac{1}{r_{13}} + \frac{1}{r_{23}} - E_H - E_{Ps} - \frac{1}{4}\kappa^2 \right] \Psi_\ell \,d\tau_\rho d\tau_{r_3} d\tau_{r_{12}}.
\end{align}

Some important cancellations are made in the matrix element equations in \cref{sec:??} by using the H and Ps ground state wave equations. Separately, the H and Ps equations are respectively (for large values of $\rho$)
\begin{subequations}
\label{eq:HPsEqn}
\begin{align}
\left(-\frac{1}{2} \Laplacian_{r_3} - \frac{1}{r_3}\right) \Phi_H(r_3) &= E_H \Phi_H(r_3) \label{eq:HEqn} \\
\left(-\Laplacian_{r_{12}} - \frac{1}{r_{12}}\right) \Phi_{Ps}(r_{12}) &= E_{Ps} \Phi_{Ps}(r_{12}) \label{eq:PsEqn}
\end{align}
\end{subequations}
Realizing then that the Hamiltonians for H and Ps are given by
\begin{subequations}
\label{eq:HPsHamil}
\begin{align}
H_H =& -\frac{1}{2} \Laplacian_{r_3} - \frac{1}{r_3} \label{eq:HHamil} \\
H_{Ps} =& -\Laplacian_{r_{12}} - \frac{1}{r_{12}} \label{eq:PsHamil}
\end{align}
\end{subequations}
and rearranging terms, \cref{eq:IlPsiVariation3} becomes
\begin{align}
\label{eq:IlPsiVariation4}
\delta I_\ell^\prime = -&\frac{1}{2} \int\limits_{V_{12}} \int\limits_{V_3} \int\limits_{V_\rho} \left[\Psi_\ell \Laplacian_\rho \,\delta\Psi_\ell - \delta\Psi_\ell \Laplacian_\rho \Psi_\ell \right] \,d\tau_\rho d\tau_{r_3} d\tau_{r_{12}} \nonumber \\
+ &2 \int\limits_{V_{12}} \int\limits_{V_3} \int\limits_{V_\rho} \Psi_\ell \left[H_H + H_{Ps} + \frac{1}{r_1} - \frac{1}{r_2} - \frac{1}{r_{13}} + \frac{1}{r_{23}} - E_H - E_{Ps} - \frac{1}{4}\kappa^2 \right] \delta\Psi_\ell \,d\tau_\rho d\tau_{r_3} d\tau_{r_{12}} \nonumber \\
- &2 \int\limits_{V_{12}} \int\limits_{V_3} \int\limits_{V_\rho} \delta\Psi_\ell \left[H_H + H_{Ps} + \frac{1}{r_1} - \frac{1}{r_2} - \frac{1}{r_{13}} + \frac{1}{r_{23}} - E_H - E_{Ps} - \frac{1}{4}\kappa^2 \right] \Psi_\ell \,d\tau_\rho d\tau_{r_3} d\tau_{r_{12}}.
\end{align}
Due to the exponential form of $\Phi_{Ps}(r_{12})$ and $\Phi_H(r_3)$ given in \cref{}, the last two lines cancel each other.

From Green's theorem,
\begin{align}
\label{eq:GreensThm}
\nonumber \Braket{\Psi_\ell^\star | \Laplacian_\rho | \delta\Psi_\ell } - \Braket{\delta\Psi_\ell^\star | \laplacian_\rho | \Psi_\ell}
&= \int\limits_{V_3} \int\limits_{V_{12}} \int\limits_{V_\rho} \left[ \Psi_\ell \laplacian_\rho \,\delta\Psi_\ell - \delta\Psi_\ell \laplacian_\rho \Psi_\ell \right] d\tau_\rho \, d\tau_{12} d\tau_3 \\
&= \int\limits_{V_3} \int\limits_{V_{12}} \int\limits_{S_\rho} \left[ \Psi_\ell \grad_\rho \delta\Psi - \delta\Psi_\ell \grad_\rho \Psi_\ell \right] \cdot d\bm{\sigma}_\rho d\tau_{12} d\tau_3.
\end{align}
$S_\rho$ is the surface at $\rho \rightarrow \infty$.

Since we are only considering the direct terms in \cref{eq:GeneralWaveTrial} so far, let us define a direct term only version of \cref{eq:TildeSCDef} with
\begin{subequations}
\label{eq:TildeSCDefDir}
\begin{align}
\widetilde{S}_d &= u_{00} S + u_{01} C \\
\widetilde{C}_d &= u_{10} S + u_{11} C.
\end{align}
\end{subequations}

From \cref{eq:PsilTrialRelation} and \cref{eq:GeneralWaveTrial},
\beq
\label{eq:DeltaPsi}
\delta \Psi = \Psi_\ell^t - \Psi_\ell = (\widetilde{S}_d + L_\ell^t \, \widetilde{C}_d) - (\widetilde{S}_d + L_\ell \, \widetilde{C}_d) = (L_\ell^t - L_\ell) \widetilde{C}_d \,.
\eeq
Substituting this into \cref{eq:IlPsiVariation4} with \cref{eq:GreensThm},
\beq
\label{eq:IlPsiVariation5}
\delta I_\ell^\prime = -\frac{1}{2} (L_\ell^t - L_\ell) \int\limits_{V_{12}} \int\limits_{V_3} \int\limits_{S_\rho} \left[(\widetilde{S}_d + L_\ell \, \widetilde{C}_d) \grad_\rho \widetilde{C}_d - \widetilde{C}_d \grad_\rho (\widetilde{S}_d + L_\ell \, \widetilde{C}_d) \right] \cdot d\bm{\sigma}_\rho d\tau_{12} d\tau_3.
\eeq

From \cref{eq:SphBesDerRel,eq:GenSandC}, to first order, the gradient acting on $\widetilde{S}_d$ and $\widetilde{C}_d$ in \cref{eq:TildeSCDef} gives
\begin{subequations}
\label{eq:GradSC}
\begin{align}
\grad_\rho \widetilde{S}_d &= \kappa \left[u_{00} C - u_{01} S \right] \hat{\bm\rho} \\
\grad_\rho \widetilde{C}_d &= \kappa \left[u_{10} C - u_{11} S \right] \hat{\bm\rho} \,.
\end{align}
\end{subequations}
Substituting this into \cref{eq:IlPsiVariation5} and dropping the dot product, since the surface elements are in the same direction as $\hat{\bm\rho}$, this becomes
\begin{align}
\label{eq:IlPsiVariation6a}
\delta I_\ell^\prime = -\frac{1}{2} (L_\ell^t - L_\ell) & \int\limits_{V_{12}} \int\limits_{V_3} \int\limits_{S_\rho}  \left\{(\widetilde{S}_d + L_\ell \, \widetilde{C}_d) \kappa (u_{10} C - u_{11} S) \right. \nonumber \\
 &- \left. \widetilde{C}_d \kappa \left[(u_{00} C - u_{01} S) + L_\ell (u_{10} C - u_{11} S) \right] \right\} d\sigma_\rho d\tau_{12} d\tau_3.
\end{align}
Omitting terms quadratic in $L_\ell$ or $L_\ell^t$, including $L_\ell^t L_\ell$,
\begin{align}
\label{eq:IlPsiVariation6b}
\delta I_\ell^\prime = -\frac{1}{2} \kappa (L_\ell^t - L_\ell) & \int\limits_{V_{12}} \int\limits_{V_3} \int\limits_{S_\rho} \left[\widetilde{S}_d (u_{10} C - u_{11} S) - \widetilde{C}_d (u_{00} C - u_{01} S) \right] d\sigma_\rho d\tau_{12} d\tau_3 \nonumber \\
= -\frac{1}{2} \kappa (L_\ell^t - L_\ell) & \int\limits_{V_{12}} \int\limits_{V_3} \int\limits_{S_\rho} \left[\widetilde{S}_d u_{10} C - \widetilde{S}_d u_{11} S - \widetilde{C}_d u_{00} C + \widetilde{C}_d u_{01} S \right] d\sigma_\rho d\tau_{12} d\tau_3 \nonumber \\
= -\frac{1}{2} \kappa (L_\ell^t - L_\ell) & \int\limits_{V_{12}} \int\limits_{V_3} \int\limits_{S_\rho} \left[u_{00} u_{10} S C + u_{01} u_{10} C^2 - u_{00} u_{11} S^2 - u_{01} u_{11} C S\right. \nonumber \\
& \left. - u_{10} u_{00} S C - u_{11} u_{00} C^2 + u_{10} u_{01} S^2 + u_{11} u_{01} C S   \right] d\sigma_\rho d\tau_{12} d\tau_3 \nonumber \\
= -\frac{1}{2} \kappa (L_\ell^t - L_\ell) & \int\limits_{V_{12}} \int\limits_{V_3} \int\limits_{S_\rho} (u_{10} u_{01} - u_{00} u_{11}) \left( S^2 + C^2 \right) d\sigma_\rho d\tau_{12} d\tau_3 \nonumber \\
= \frac{1}{2} \kappa (L_\ell^t - L_\ell) & \Det{\textbf{u}} \int\limits_{V_{12}} \int\limits_{V_3} \int\limits_{S_\rho} \left( S^2 + C^2 \right) d\sigma_\rho d\tau_{12} d\tau_3.
\end{align}

The rest of this derivation considers only the direct terms. The final result applies as well when the exchanged terms are included. Since we are considering the surface as $\rho \rightarrow \infty$, $f_\ell(\rho)$ in \cref{eq:GenCDef} becomes 1. Then from \cref{eq:GenSandC},
\beq
S^2 + C^2 = \SphericalHarmonicY{\ell}{0}{\theta_\rho}{\phi_\rho}^2 \Phi_{Ps}\left(r_{12}\right)^2 \Phi_H\left(r_3\right)^2 (2 \kappa) \left[j_\ell\!\left(\kappa\rho\right)^2 + n_\ell\!\left(\kappa\rho\right)^2\right].
\eeq
The asymptotic forms of $j_\ell$ and $n_\ell$ are given by \cite[p.729]{Arfken2005}
\begin{subequations}
\label{eq:AsymSphBes}
\begin{align}
\lim_{\rho \to \infty} j_\ell\!\left(\kappa\rho\right) &= \frac{1}{\kappa\rho} \sin\left(\kappa\rho - \frac{n \pi}{2}\right) \label{eq:AsymJl} \\
\lim_{\rho \to \infty} n_\ell\!\left(\kappa\rho\right) &= \frac{1}{\kappa\rho} \cos\left(\kappa\rho - \frac{n \pi}{2}\right). \label{eq:AsymNl}
\end{align}
\end{subequations}
As $\rho \to \infty$,
\beq
S^2 + C^2 = \SphericalHarmonicY{\ell}{0}{\theta_\rho}{\phi_\rho}^2 \Phi_{Ps}\left(r_{12}\right)^2 \Phi_H\left(r_3\right)^2 (2 \kappa) \frac{1}{\kappa^2 \rho^2}.
\eeq

Substituting this in \cref{eq:IlPsiVariation6b} and expanding the $d\sigma_\rho$ differential,
\begin{align}
\label{eq:IlPsiVariation7}
\delta I_\ell^\prime =& \kappa (L_\ell^t - L_\ell) \Det{\textbf{u}} \int\limits_{V_{12}} \int\limits_{V_3} \int\limits_{S_\rho} \SphericalHarmonicY{\ell}{0}{\theta_\rho}{\phi_\rho}^2 \Phi_{Ps}\left(r_{12}\right)^2 \Phi_H\left(r_3\right)^2 \frac{1}{\kappa \rho^2} \rho^2 \sin\theta_\rho d\theta_\rho d\phi_\rho d\tau_{12} d\tau_3 \nonumber \\
=& (L_\ell^t - L_\ell) \Det{\textbf{u}} \int\limits_{V_{12}} \int\limits_{V_3} \int\limits_{S_\rho} \SphericalHarmonicY{\ell}{0}{\theta_\rho}{\phi_\rho}^2 \Phi_{Ps}\left(r_{12}\right)^2 \Phi_H\left(r_3\right)^2 \sin\theta_\rho d\theta_\rho d\phi_\rho d\tau_{12} d\tau_3.
\end{align}
Since the Ps and H eigenfunctions are normalized, i.e.
\beq
\int\limits_{V_3}\! \Phi_H(r_3) d\tau_3 = 1 \text{ and } \int\limits_{V_{12}}\! \Phi_{Ps}(r_{12}) d\tau_{12} = 1,
\label{eq:PsHNormalization}
\eeq
we now have
\beq
\label{eq:IlPsiVariation8}
\delta I_\ell^\prime = (L_\ell^t - L_\ell) \Det{\textbf{u}} \int\limits_{S_\rho} \SphericalHarmonicY{\ell}{0}{\theta_\rho}{\phi_\rho}^2 \sin\theta_\rho d\theta_\rho d\phi_\rho.
\eeq
The spherical harmonics are normalized so that \cite[p.788]{Arfken2005}
\beq
\label{eq:SphHarmNorm}
\int\limits_{S_\rho} \SphericalHarmonicY{\ell}{0}{\theta_\rho}{\phi_\rho}^2 d\Omega = 1.
\eeq
This gives that
\beq
\label{eq:IlPsiVariation9}
\delta I_\ell^\prime = (L_\ell^t - L_\ell) \Det{\textbf{u}}.
\eeq

From \cref{eq:IlPrimeDef,eq:IlPsiVariation9},
\beq
\label{eq:KatoIdent}
\delta I_\ell^\prime = (L_\ell^t - L_\ell) \Det{\textbf{u}} + (\delta\Psi_\ell, \mathcal{L} \,\delta\Psi_\ell).
\eeq
This is the Kato identity \cite{Kato1951a}. For the Kohn variational methods, the last term is neglected, since it is quadratic in $\delta\Psi_\ell$. Using $\delta I_\ell^\prime \approx \delta I_\ell$, we have
\beq
\delta I_\ell = I_\ell[\Psi_\ell^t] - I_\ell[\Psi_\ell] = (L_\ell^t - L_\ell) \Det{\textbf{u}}.
\eeq
Replacing the exact $L_\ell$ by the variational $L_\ell^v$ and rearranging, we finally get the general Kohn variational method of
\beq
\label{eq:GenKohn}
L_\ell^v = L_\ell^t - I_\ell[\Psi_\ell^t] / \! \Det{\textbf{u}}.
\eeq
This was only derived using the direct terms, but the exchange terms follow the same steps with $\rho^\prime$ instead of $\rho$. This was also only shown for the long-range terms, but it applies equally as well to the short-range terms.


\section{Application of the Kohn Methods}
\label{sec:KohnApplied}

We use the general Kohn variational method (\cref{eq:GenKohn}) with our full trial wavefunction to get
\beq
\label{eq:GenKohnApplied}
L_\ell^v = L_\ell^t - \tfrac{1}{\Det{\textbf{u}}} \Big((\widetilde{S} + L_\ell^t \, \widetilde{C} + \sum_i c_i \bar{\phi}_i^t), \mathcal{L} (\widetilde{S} + L_\ell^t \, \widetilde{C} + \sum_j c_j \bar{\phi}_j^t )\Big).
\eeq
The property of the Kohn functional that it is stationary with respect to variations in the linear parameters \cite{Joachain1979} can be written in our case as
\beq
\frac{\partial L_\ell^v}{\partial L_\ell^t} = 0  \text{ and } \frac{\partial L_\ell^v}{\partial c_i} = 0 \text{, where $i = 1,\ldots,N$}.
\label{eq:KohnStationary}
\eeq

Performing the first variation gives
\beq
0 = \pderiv{L_\ell^v}{L_\ell^t} = 1 - \left[(\widetilde{S},\mathcal{L}\widetilde{C}) + (\widetilde{C},\mathcal{L}\widetilde{S}) + \frac{\partial}{\partial L_\ell^t}(L_\ell^t \widetilde{C},\mathcal{L} L_\ell^t \widetilde{C}) + (\widetilde{C}, L \sum_i c_i \bar{\phi}_i) + (\sum_i c_i \bar{\phi}_i, \mathcal{L} \widetilde{C}) \right].
\label{eq:PdLambda1}
\eeq

\noindent The third term in brackets becomes
\beq
\pderiv{}{L_\ell^t} (L_\ell^t \widetilde{C},\mathcal{L} L_\ell^t \widetilde{C}) = (\widetilde{C},\mathcal{L} \widetilde{C}) \frac{\partial}{\partial L_\ell^t} {L_\ell^t}^2 = 2(\widetilde{C},\mathcal{L}\widetilde{C}) L_\ell^t.
\eeq

\noindent The last two terms of \cref{eq:PdLambda1} are equal to each other, and we can use \cref{eq:SLCandCLSBar} to rewrite this.
\beq
0 = -(\widetilde{C},\mathcal{L}\widetilde{S}) - (\widetilde{C},\mathcal{L}\widetilde{S}) - 2 L_\ell^t (\widetilde{C},\mathcal{L}\widetilde{C}) - 2 \sum_i c_i (\widetilde{C},\mathcal{L}\bar{\phi}_i)
\eeq

\noindent Rearranging gives
\beq
-(\widetilde{C},\mathcal{L}\widetilde{S}) = L_\ell^t (\widetilde{C},\mathcal{L}\widetilde{C}) + \sum_i c_i (\widetilde{C},\mathcal{L}\bar{\phi}_i)
\label{eq:PdLambda}
\eeq

Now we perform the variation with respect to a general $c_k$ as in \cref{eq:KohnStationary}.
\beq
0 = \frac{\partial \mathcal{L}_v}{\partial c_k} = -\left[ (\widetilde{S},\mathcal{L} \bar{\phi}_k) + L_\ell^t (\widetilde{C},\mathcal{L} \bar{\phi}_k) + (\bar{\phi}_k,\mathcal{L} \widetilde{S}) + L_\ell^t (\bar{\phi}_k,\mathcal{L} \widetilde{C}) + \frac{\partial}{\partial c_k} (\sum_i c_i \bar{\phi}_i, \mathcal{L} \sum_j c_j \bar{\phi}_j) \right]
\label{eq:PdCk1}
\eeq

If $c_i \ne c_j$,
\begin{subequations}
\begin{align}
\frac{\partial}{\partial c_i} (c_i \bar{\phi}_i, \mathcal{L} \sum_{j \ne i} c_j \bar{\phi}_j) &= \sum_{j \ne i} c_j (\bar{\phi}_i, \mathcal{L} \bar{\phi}_j) \text{ and} \\
\frac{\partial}{\partial c_j} (\sum_{i \ne j} c_i \bar{\phi}_i, \mathcal{L} c_j \bar{\phi}_j) &= \sum_{i \ne j} c_i (\bar{\phi}_i, \mathcal{L} \bar{\phi}_j).
\end{align}
\end{subequations}

\noindent These two equations are equivalent, since $\left( \bar{\phi}_i, \mathcal{L} \bar{\phi}_j \right) = \left( \bar{\phi}_j, \mathcal{L}L \bar{\phi}_i \right)$ by \cref{PhiLPhiPerm}.

If $c_i = c_j$,
\beq
\frac{\partial}{\partial c_i} \left( c_i \bar{\phi}_i, L c_j \bar{\phi}_j \right) = \frac{\partial}{\partial c_i} \left( c_i \bar{\phi}_i, \mathcal{L} c_i \bar{\phi}_i \right) = \frac{\partial}{\partial c_i} c_i^2 \left( \bar{\phi}_i, \mathcal{L} \bar{\phi}_j \right) = 2 \, c_i \left( \bar{\phi}_i, \mathcal{L} \bar{\phi}_j \right).
\eeq

\noindent We can also use \cref{eq:PhiLSPerm,eq:PhiLCPerm} to reduce \cref{eq:PdCk1} to
\beq
0 = -\Big[ 2 (\bar{\phi}_k, \mathcal{L} \widetilde{S}) + 2 L_\ell^t (\bar{\phi}_k, \mathcal{L} \widetilde{C}) + 2 \sum_i (\bar{\phi}_k, \mathcal{L} c_i \bar{\phi}_i) \Big].
\eeq

\noindent
Rearranging gives
\beq
-\left( \bar{\phi}_k, \mathcal{L} \widetilde{S} \right) = L_\ell^t \left( \bar{\phi}_k, \mathcal{L} \widetilde{C} \right) + \sum_i \left( \bar{\phi}_k, \mathcal{L} c_i \bar{\phi}_i \right).
\label{eq:PdCk}
\eeq

The set of linear equations in \cref{eq:PdLambda,eq:PdCk} can be written in matrix form as
\begin{equation}
\label{eq:GeneralKohnMatrix}
\begin{bmatrix} 
 (\widetilde{C},\mathcal{L}\widetilde{C}) & (\widetilde{C},\mathcal{L}\bar{\phi}_1) & \cdots & (\widetilde{C},\mathcal{L}\bar{\phi}_j) & \cdots\\
 (\bar{\phi}_1,\mathcal{L}\widetilde{C}) & (\bar{\phi}_1,\mathcal{L}\bar{\phi}_1) & \cdots & (\bar{\phi}_1,\mathcal{L}\bar{\phi}_j) & \cdots\\
 \vdots & \vdots & \ddots & \vdots \\
 (\bar{\phi}_i,\mathcal{L}\widetilde{C}) & (\bar{\phi}_i,\mathcal{L}\bar{\phi}_1) & \cdots & (\bar{\phi}_i,\mathcal{L}\bar{\phi}_j) & \cdots\\
 \vdots & \vdots & & \vdots & \\
\end{bmatrix}
\begin{bmatrix}
L_\ell^t\\
c_1\\
\vdots\\
c_i\\
\vdots
\end{bmatrix}
= -
\begin{bmatrix}
(\widetilde{C},\mathcal{L}\widetilde{S}) \\
(\bar{\phi}_1,\mathcal{L}\widetilde{S}) \\
\vdots \\
(\bar{\phi}_i,\mathcal{L}\widetilde{S}) \\
\vdots
\end{bmatrix}.
\end{equation}
\todoi{Be clear that in the long-range code, we only calculate the Kohn elements, and then we rearrange them to create the others in the phase shift code.}

\noindent This matrix equation can be rewritten as
\beq
\label{eq:GenKohnMatrixAXB}
\textbf{\emph{AX}} = -\textbf{\emph{B}}.
\eeq

\noindent Solving this for $\textbf{\emph{X}}$ gives
\beq
\textbf{\emph{X}} = -\textbf{\emph{A}}^{-1}\textbf{\emph{B}}.
\eeq

To obtain $\mathcal{L}_v$ from this matrix equation, we must next expand \cref{eq:GenKohnApplied}.

\begin{align}
\nonumber \mathcal{L}_v = L_\ell^t - &\left[ (\widetilde{S},\mathcal{L}\widetilde{S}) + L_\ell^t (\widetilde{S},\mathcal{L}\widetilde{C}) + \sum_i c_i (\widetilde{S},\bar{\phi}_i) + L_\ell^t (\widetilde{C},\mathcal{L}\widetilde{S}) + {L_\ell^t}^2 (\widetilde{C},\mathcal{L}\widetilde{C}) + L_\ell^t \sum_i c_i (\widetilde{C},\mathcal{L} \bar{\phi}_i) \right. \\
& + \left. \sum_i c_i (\bar{\phi}_i, \mathcal{L} \widetilde{S}) + L_\ell^t \sum_i c_i (\bar{\phi}_i, \mathcal{L} \widetilde{C}) + \sum_i \sum_j c_i c_j (\bar{\phi}_i, \mathcal{L} \bar{\phi}_j) \right]
\end{align}

\noindent By substituting \cref{eq:GenSLCandCLS} in for $(\widetilde{S},\mathcal{L}\widetilde{C})$, the first $L_\ell^t$ above is canceled, leaving

\begin{align}
\label{eq:GenKohnApplied2}
\nonumber \mathcal{L}_v = - & \left[ (\widetilde{S},\mathcal{L}\widetilde{S}) + L_\ell^t (\widetilde{C},\mathcal{L}\widetilde{S}) + \sum_i c_i (\widetilde{S},\bar{\phi}_i) + L_\ell^t (\widetilde{C},\mathcal{L}\widetilde{S}) + {L_\ell^t}^2 (\widetilde{C},\mathcal{L}\widetilde{C}) + L_\ell^t \sum_i c_i (\widetilde{C},\mathcal{L} \bar{\phi}_i) \right. \\
& + \left. \sum_i c_i (\bar{\phi}_i, \mathcal{L} \widetilde{S}) + L_\ell^t \sum_i c_i (\bar{\phi}_i, \mathcal{L} \widetilde{C}) + \sum_i \sum_j c_i c_j (\bar{\phi}_i, \mathcal{L} \bar{\phi}_j) \right].
\end{align}

Using the following definitions of
\beq
D = 
\begin{bmatrix}
L_\ell^t & c_1 & \cdots & c_N
\end{bmatrix}
\text{ and}
\eeq
\beq
\label{eq:GenFandD}
F =
\begin{bmatrix}
(\boldsymbol{\widetilde{C},\mathcal{L}\widetilde{C}}) & (\boldsymbol{\widetilde{C},\mathcal{L}\bar{\phi}}) & (\boldsymbol{\widetilde{C},\mathcal{L}\widetilde{S}}) \\
(\boldsymbol{\bar{\phi},\mathcal{L}\widetilde{C}}) & (\boldsymbol{\bar{\phi},\mathcal{L}\bar{\phi}}) & (\boldsymbol{\bar{\phi},\mathcal{L}\widetilde{S}}) \\
(\boldsymbol{\widetilde{C},\mathcal{L}\widetilde{S}}) & (\boldsymbol{\widetilde{S},\mathcal{L}\bar{\phi}}) & (\boldsymbol{\widetilde{S},\mathcal{L}\widetilde{S}})
\end{bmatrix},
\eeq
\cref{eq:GenKohnApplied2} can be rewritten as the following matrix equation:

\beq
\label{eq:GenDFDT}
L_\ell^v = - D F D^T.
\eeq

\noindent Using \cref{eq:GenKohnMatrix,eq:GenKohnMatrix} in \cref{eq:GenDFDT} and expanding gives
\begin{align}
\label{eq:GenDFDT2}
\nonumber L_\ell^v &= - 
\begin{bmatrix}
\boldsymbol{X^T} & 1 
\end{bmatrix}
\begin{bmatrix}
\boldsymbol{A} & \boldsymbol{B} \\
\boldsymbol{B^T} & \boldsymbol{(\widetilde{S},\mathcal{L}\widetilde{S})}
\end{bmatrix}
\begin{bmatrix}
\boldsymbol{X} \\
1
\end{bmatrix}
= -
\begin{bmatrix}
\boldsymbol{X^T} & 1 
\end{bmatrix}
\begin{bmatrix}
0 \\
\boldsymbol{B^T X} + (\widetilde{S},\mathcal{L}\widetilde{S})
\end{bmatrix} \\
&= -\boldsymbol{B^T X} - (\widetilde{S},\mathcal{L}\widetilde{S}),
\end{align}
where
\beq
\boldsymbol{B^T X} = L_\ell^t (\widetilde{C},\mathcal{L}\widetilde{S}) + \sum_i c_i (\bar{\phi}_i, \mathcal{L} \widetilde{S}).
\eeq

\noindent A more compact way of writing \cref{eq:GenDFDT2} is by
\beq
L_\ell^v = -\left( \Psi^{t,0},\mathcal{L} \widetilde{S} \right).
\eeq
$\Psi^{t,0}$ is the full general wavefunction in \cref{eq:GeneralWaveTrial} with its nonlinear parameters optimized.



Finally, to obtain the phase shifts, we use the relation given by Ref.~\cite{Lucchese1989} as
\begin{equation}
\label{eq:GenKohnL}
K_\ell = \tan \delta_\ell = (u_{01} + u_{11} L_\ell)(u_{00} + u_{10} L_\ell)^{-1}.
\end{equation}

The $\textbf{u}$ and $\mathcal{L}_l$ for the various Kohn methods are described now. Note that for each of these, $\Det{\textbf{u}} = 1$.

\subsubsection*{Kohn}
\label{sec:Kohn}
\beq
\textbf{u} =
\begin{bmatrix}
1 & 0 \\
0 & 1 
\end{bmatrix}
\label{eq:uKohn}
\eeq

\beq
\mathcal{L}_l = \lambda_t = K_t
\label{eq:LKohn}
\eeq


\subsubsection*{Inverse Kohn}
\label{sec:InvKohn}
\beq
\textbf{u} =
\begin{bmatrix}
0 & 1 \\
-1 & 0 
\end{bmatrix}
\label{eq:uInvKohn}
\eeq

\beq
\mathcal{L}_l = -\mu_t = -K^{-1}_t = -\bar{K}_t
\label{eq:LInvKohn}
\eeq


\subsubsection*{Generalized Kohn}
\label{sec:GenKohn}
\beq
\textbf{u} =
\begin{bmatrix}
\cos\tau & \sin\tau \\
-\sin\tau & \cos\tau 
\end{bmatrix}
\label{eq:uGenKohn}
\eeq

\beq
\mathcal{L}_l = a_t
\label{eq:LGenKohn}
\eeq

\noindent The generalized Kohn method is described by Cooper et al.\ \cite{Cooper2009, Cooper2010}.  When $\tau = 0$ is substituted in \cref{eq:uGenKohn}, the $\textbf{u}$-matrix for the Kohn method is generated (\cref{eq:uKohn}). Similarly, when $\tau = \frac{\pi}{2}$, the $\textbf{u}$-matrix for the inverse Kohn method is generated (\cref{eq:uInvKohn}). As to be expected, the Kohn and inverse Kohn methods are special cases of the generalized Kohn method.


\subsubsection*{Complex Kohn $T$-matrix}
\label{sec:ComplexTKohn}
\beq
\textbf{u} =
\begin{bmatrix}
1 & 0 \\
\ii & 1
\end{bmatrix}
\label{eq:uCompTKohn}
\eeq

\beq
\mathcal{L}_l = T
\label{eq:LCompTKohn}
\eeq

\noindent Lucchese denotes this as $\mathcal{L}_l = -\pi T$ \cite{Lucchese1989}, but to be consistent with our definition of the $T$-matrix, we must use equation \cref{eq:LCompTKohn}.


\subsubsection*{Complex Kohn $S$-matrix}
\label{sec:ComplexSKohn}
\beq
\textbf{u} =
\begin{bmatrix}
-\ii & 1 \\
\ii & 1
\end{bmatrix}
\label{eq:uCompSKohn}
\eeq

\beq
\mathcal{L}_l = 2 \ii S_\ell
\label{eq:LCompSKohn}
\eeq
\todoi{This is probably not correct now.}

The Lucchese version of $\textbf{u}$ differs from ours, since he uses a different definition for the $S$-matrix  \cite{Lucchese1989}. The form of the $S$-matrix we are using with our wavefunction is related to the $K$-matrix by
\beq
K_\ell = \frac{\ii(1-S_\ell)}{1+S_\ell},
\eeq
which is satisfied by the above $\textbf{u}$-matrix.

\subsubsection*{Generalized $T$ matrix Kohn}
\label{sec:GenComplexTKohn}
\beq
\textbf{u} =
\begin{bmatrix}
\cos\tau & \sin\tau \\
-\sin\tau + \ii \cos\tau & \cos\tau + \ii \sin\tau
\end{bmatrix}
\label{eq:uGenTKohn}
\eeq
This is a generalized form of the $T$ matrix Kohn, similar to how the generalized Kohn works. When $\tau = 0$, this reduces to the $T$ matrix Kohn.


\subsubsection*{Generalized $S$ matrix Kohn}
\label{sec:GenComplexSKohn}
\beq
\textbf{u} =
\begin{bmatrix}
\ii \cos\tau - \sin\tau & -\ii \sin\tau + \cos\tau \\
\ii \cos\tau - \sin\tau & \ii \sin\tau + \cos\tau
\end{bmatrix}
\label{eq:uGenSKohn}
\eeq
This is a generalized form of the $S$ matrix Kohn. When $\tau = 0$, this reduces to the $S$ matrix Kohn.


\section{Schwartz Singularities}
\label{eq:SchwartzSing}




\section{Matrix Elements Involving Long-Range Terms}
\label{sec:MatrixLong}
The short-range--long-range and long-range--long-range matrix elements have a similar analysis. For all of these, the effect of the $\mathcal{L} = 2(H-E)$ operator on the long-range terms must be considered, and then integrations over the external angles (see \Cref{chp:AngularInt}) are performed. The remaining 6-dimensional integral is then numerically integrated as described in sections \ref{sec:LongLongInt} and \ref{sec:ShortLongInt}.

\subsection{\texorpdfstring{$\mathcal{L}S$}{LS} Terms}
\label{sec:LSTerms}
The $(\widetilde{S},\mathcal{L}\widetilde{S})$, $(\widetilde{C},\mathcal{L}\widetilde{S})$, and $(\bar{\phi}_i,\mathcal{L}\widetilde{S})$ matrix elements in \cref{eq:GeneralKohnMatrix} require us to first determine $\mathcal{L}\widetilde{S}$. We start with examining $\mathcal{L}S$ first.

Using \cref{eq:GenSDef,eq:LDef}, 
\begin{align}
\label{eq:LS1}
\mathcal{L}S_\ell = &\left(-\frac{1}{2} \Laplacian_\rho - \Laplacian_{r_3} - 2 \Laplacian_{r_{12}} + \frac{2}{r_1} - \frac{2}{r_2} - \frac{2}{r_3} - \frac{2}{r_{12}} - \frac{2}{r_{13}} + \frac{2}{r_{23}} - 2 E_H - 2 E_{Ps} - \frac{1}{2} \kappa^2\right) \nonumber \\
& \times \SphericalHarmonicY{\ell}{0}{\theta_\rho}{\phi_\rho} \Phi_{Ps}\left(r_{12}\right) \Phi_H\left(r_3\right) \sqrt{2\kappa} \,j_\ell\!\left(\kappa\rho\right).
\end{align}
Since $S_\ell$ is independent of $r_3$ and $r_{12}$ except for the $\Phi_H$ and $\Phi_{Ps}$ functions, respectively, using \cref{eq:HPsEqn,eq:HPsHamil} simplifies this to
\begin{align}
\label{eq:LS2}
\mathcal{L}S_\ell = \left(-\frac{1}{2} \Laplacian_\rho + \frac{2}{r_1} - \frac{2}{r_2} - \frac{2}{r_{13}} + \frac{2}{r_{23}} - \frac{1}{2} \kappa^2\right) \SphericalHarmonicY{\ell}{0}{\theta_\rho}{\phi_\rho} \Phi_{Ps}\left(r_{12}\right) \Phi_H\left(r_3\right) \sqrt{2\kappa} \,j_\ell\!\left(\kappa\rho\right).
\end{align}
From \cref{sec:SphBess2}, we find that $\SphericalHarmonicY{\ell}{0}{\theta_\rho}{\phi_\rho} j_\ell(\kappa\rho)$ is an eigenfunction of $\Laplacian_\rho$ with eigenvalue $-\kappa^2$:
\begin{equation}
\Laplacian_\rho \left[\SphericalHarmonicY{\ell}{0}{\theta_\rho}{\phi_\rho} j_\ell(\kappa\rho) \right] = \frac{(-\kappa^2 \rho^2) \LegendreP{\ell, \cos\theta} } {\rho^2 \LegendreP{\ell, \cos\theta}}
= -\kappa^2 \, \SphericalHarmonicY{\ell}{0}{\theta_\rho}{\phi_\rho} j_\ell(\kappa\rho).
\end{equation}
Then \cref{eq:LS2} reduces down to
\begin{equation}
\label{eq:LS3}
\mathcal{L}S_\ell = \left(\frac{2}{r_1} - \frac{2}{r_2} - \frac{2}{r_{13}} + \frac{2}{r_{23}}\right) \SphericalHarmonicY{\ell}{0}{\theta_\rho}{\phi_\rho} \Phi_{Ps}\left(r_{12}\right) \Phi_H\left(r_3\right) \sqrt{2\kappa} \,j_\ell\!\left(\kappa\rho\right)
\end{equation}
or
\begin{equation}
\label{eq:LSFinal}
\mathcal{L}S_\ell = \left(\frac{2}{r_1} - \frac{2}{r_2} - \frac{2}{r_{13}} + \frac{2}{r_{23}}\right) S_\ell.
\end{equation}

$\mathcal{L}S_\ell^\prime$ is simply the same as \cref{eq:LSFinal} but with $2 \leftrightarrow 3$ due to the permutation operator, or
\begin{equation}
\label{eq:LSPrimeFinal}
\mathcal{L}S_\ell^\prime = \left(\frac{2}{r_1} - \frac{2}{r_3} - \frac{2}{r_{12}} + \frac{2}{r_{23}}\right) S_\ell^\prime.
\end{equation}


\subsection{\texorpdfstring{$\mathcal{L}C$}{LC} Terms}
\label{sec:LCTerms}
To calculate the matrix elements in \cref{eq:GeneralKohnMatrix} that include $\widetilde{C}$ in the ket, we start by writing a general form of $\mathcal{L}C$ using \cref{eq:LDef,eq:GenCDef}:
\begin{align}
\label{eq:LC1}
\mathcal{L}C = -&\left(-\frac{1}{2} \Laplacian_\rho - \Laplacian_{r_3} - 2 \Laplacian_{r_{12}} + \frac{2}{r_1} - \frac{2}{r_2} - \frac{2}{r_3} - \frac{2}{r_{12}} - \frac{2}{r_{13}} + \frac{2}{r_{23}} - 2 E_H - 2 E_{Ps} - \frac{1}{2} \kappa^2\right) \nonumber \\
& \times \SphericalHarmonicY{\ell}{0}{\theta_\rho}{\phi_\rho} \Phi_{Ps}\left(r_{12}\right) \Phi_H\left(r_3\right) \sqrt{2\kappa} \,n_\ell\!\left(\kappa\rho\right) f_\ell(\rho) 
\end{align}
Similar to \cref{}, using \cref{eq:HPsEqn,eq:HPsHamil} reduces this to
\begin{align}
\label{eq:LC2}
\mathcal{L}C = -&\left(-\frac{1}{2} \Laplacian_\rho + \frac{2}{r_1} - \frac{2}{r_2} - \frac{2}{r_{13}} + \frac{2}{r_{23}} - 2 E_H - 2 E_{Ps} - \frac{1}{2} \kappa^2\right) \nonumber \\
& \times \SphericalHarmonicY{\ell}{0}{\theta_\rho}{\phi_\rho} \Phi_{Ps}\left(r_{12}\right) \Phi_H\left(r_3\right) \sqrt{2\kappa} \,n_\ell\!\left(\kappa\rho\right) f_\ell(\rho).
\end{align}
Unlike with $\mathcal{L}S$ in \cref{sec:LSTerms}, there is not a cancellation with the $\Laplacian_\rho$ and $\kappa^2$ terms. The combination of these terms was calculated in the ``First Partial Waves LC.nb'' \emph{Mathematica} notebook using the code given in \cref{fig:LCMath}. This is for the F-wave, and replacing the the $\ell$-value 3 in \texttt{SphericalBesselY} allows this to be used for any partial wave. The results of these derivations is given in \cref{tab:LCList}. The $\frac{1}{2} \left(\Laplacian_\rho + \kappa^2\right) \SphericalHarmonicY{\ell}{0}{\theta_\rho}{\phi_\rho} n_\ell(\kappa\rho) f_\ell(\rho)$ given in this table is substituted in \cref{eq:LC2} to find the full $\mathcal{L}C$.

\begin{figure}[H]
	\centering
	\includegraphics[width=6.5in]{LC}
	\caption{Listing of \emph{Mathematica} code in ``First Partial Waves LC.nb'' to calculate part of LC for the F-wave}
	\label{fig:LCMath}
\end{figure}

{
\renewcommand{\arraystretch}{3}  % To space out rows - could also look at the array environment.
\begin{table}[H]
\centering
\begin{tabular}{c l}
\toprule \\[-2.7cm]
Partial & \\[-1.3cm]
Wave & $\frac{1}{2} \left(\Laplacian_\rho + \kappa^2\right) \SphericalHarmonicY{\ell}{0}{\theta_\rho}{\phi_\rho} n_\ell(\kappa\rho) f_\ell(\rho)$ \\
\midrule
S-Wave & {$\!\begin{aligned} % http://tex.stackexchange.com/q/98482/16595 
               \frac{2 \kappa  f^\prime(\rho ) \sin (\kappa  \rho )-f^{\prime\prime}(\rho ) \cos (\kappa  \rho )}{2 \kappa  \rho } \\    % http://tex.stackexchange.com/q/78788/16595
           \end{aligned}$} \\
P-Wave & {$\!\begin{aligned}
				-\frac{\rho  f^{\prime\prime}(\rho ) \left[\kappa  \rho  \sin (\kappa  \rho )+\cos (\kappa  \rho )\right]+2 f^\prime(\rho ) \left[\left(\kappa ^2 \rho ^2-1\right) \cos (\kappa  \rho )-\kappa  \rho  \sin (\kappa \rho )\right]}{2 \kappa ^2 \rho ^3}
			\end{aligned}$} \\
D-Wave & {$\!\begin{aligned}
		-\frac{1}{2 \kappa ^3 \rho ^4} &\left\{\rho  f^{\prime\prime}(\rho ) \left[\left(3-\kappa ^2 \rho ^2\right) \cos (\kappa  \rho )+3 \kappa  \rho  \sin (\kappa  \rho )\right] \right. \\
		& \left.+2 f^\prime(\rho ) \left[\kappa  \rho  \left(\kappa ^2 \rho ^2-6\right) \sin (\kappa  \rho )+3 \left(\kappa ^2 \rho ^2-2\right) \cos (\kappa  \rho )\right] \right\}
		\end{aligned}$} \\[0.6cm]
F-Wave & {$\!\begin{aligned}
	\frac{1}{2 \kappa ^4 \rho ^5} &\left\{\rho  f^{\prime\prime}(\rho ) \left[\kappa  \rho  \left(\kappa ^2 \rho ^2-15\right) \sin (\kappa  \rho )+3 \left(2 \kappa ^2 \rho ^2-5\right) \cos (\kappa  \rho )\right] \right. \\
	& \left. +2 f^\prime(\rho ) \left[3 \kappa  \rho  \left(15-2 \kappa ^2 \rho ^2\right) \sin (\kappa  \rho )+\left(\kappa ^4 \rho ^4-21 \kappa ^2 \rho ^2+45\right) \cos (\kappa  \rho )\right]\right\}
	\end{aligned}$} \\[1cm] 
G-Wave & {$\!\begin{aligned}
	\frac{1}{2 \kappa ^5 \rho ^6} & \left\{2 f^\prime(\rho ) \left[\kappa  \rho  \left(\kappa ^4 \rho ^4-55 \kappa ^2 \rho ^2+420\right) \sin (\kappa  \rho ) \right. \right. \\
	& \left. +5 \left(2 \kappa ^4 \rho ^4-39 \kappa ^2 \rho ^2+84\right) \cos (\kappa  \rho )\right]-\rho  f^{\prime\prime}(\rho ) \left[5 \kappa  \rho  \left(21-2 \kappa ^2 \rho ^2\right) \sin (\kappa  \rho ) \right. \\
	& \left. \left. +\left(\kappa ^4 \rho ^4-45 \kappa ^2 \rho ^2+105\right) \cos (\kappa  \rho )\right] \right\}
	\end{aligned}$} \\[1.5cm]
H-Wave & {$\!\begin{aligned}
	-\frac{1}{2 \kappa ^6 \rho ^7} & \left\{\rho  f^{\prime\prime}(\rho ) \left[\kappa  \rho  \left(\kappa ^4 \rho ^4-105 \kappa ^2 \rho ^2+945\right) \sin (\kappa  \rho ) \right. \right. \\
	& \left. +15 \left(\kappa ^4 \rho ^4-28 \kappa ^2 \rho ^2+63\right) \cos (\kappa  \rho )\right] \\
	& + 2 f^\prime(\rho ) \left[\left(\kappa ^6 \rho ^6-120 \kappa ^4 \rho ^4+2205 \kappa ^2 \rho ^2-4725\right) \cos (\kappa  \rho ) \right. \\
	& \left. \left. -15 \kappa  \rho  \left(\kappa ^4 \rho ^4-42 \kappa ^2 \rho ^2+315\right) \sin (\kappa  \rho )\right]\right\}
	\end{aligned}$} \\
\bottomrule
\end{tabular}
\caption{Part of LC derived for each partial wave}
\label{tab:LCList}
\end{table}
}

Matrix elements involving $\mathcal{L}C^\prime$ look similar but have the 2 and 3 coordinates swapped. In other words, $\rho \leftrightarrow \rho^\prime$.

From \cref{eq:PartialWaveShielding}, the general shielding function for $\widetilde{C}$ to keep it regular at the origin is given by
\begin{equation}
  %\label{eq:PartialWaveShielding}
  f_\ell(\rho) = \left[1 - \ee^{-\mu \rho} \left(1+\frac{\mu}{2}\rho\right)
  \right]^{m_\ell}.
\end{equation}
From \cref{tab:LCList}, the derivatives $f_\ell^\prime(\rho)$ and $f_\ell^{\prime\prime}(\rho)$ are needed. In the \emph{Mathematica} notebook ``General Shielding Function.nb'', I found out that the derivatives can be written generally as
\beq
\label{eq:Shielding1Der}
f_\ell^\prime(\rho) = -\frac{\mu m_\ell (\mu  \rho +1) \left[1-\frac{1}{2} e^{-\mu  \rho } (\mu  \rho +2)\right]^{m_\ell}}{\mu  \rho -2 e^{\mu  \rho }+2}
\eeq
and
\beq
\label{eq:Shielding1Der}
f_\ell^{\prime\prime}(\rho) = \frac{\mu^2 m_\ell \left[-2 \mu  \rho  e^{\mu  \rho }+ m_\ell (\mu  \rho +1)^2-1\right] \left[1-\frac{1}{2} e^{-\mu  \rho } (\mu  \rho +2)\right]^{m_\ell}} {\left(\mu  \rho -2 e^{\mu  \rho }+2\right)^2}.
\eeq


\subsection{\texorpdfstring{$(\bar{S}_\ell,\mathcal{L}\bar{S}_\ell)$}{SLS} and \texorpdfstring{$(\bar{C}_\ell,\mathcal{L}\bar{S}_\ell)$}{CLS} Matrix Elements}
\label{sec:SLSandCLS}

From \cref{eq:TildeSCDef}, we see that, in general, any matrix element in \cref{eq:GeneralKohnMatrix} containing only long-range terms will contain both $(\bar{S}_\ell,\mathcal{L}\bar{S}_\ell)$ and $(\bar{C}_\ell,\mathcal{L}\bar{S}_\ell)$, along with the other two combinations given in \cref{sec:}. When $(\bar{S}_\ell,\mathcal{L}\bar{S}_\ell)$ is expanded,
\beq
(\bar{S}_\ell,\mathcal{L}\bar{S}_\ell) = \frac{1}{2} \left[ (S_\ell,\mathcal{L}S_\ell) \pm (S_\ell^\prime,\mathcal{L}S_\ell) \pm (S_\ell,\mathcal{L}S_\ell^\prime) \pm (S_\ell^\prime,\mathcal{L}S_\ell^\prime) \right]
\eeq
\todoi{Need to give reason why!}
Due to \cref{},
\beq
(S_\ell,\mathcal{L}S_\ell) = (S_\ell^\prime,\mathcal{L}S_\ell^\prime) \text{ and } (S_\ell^\prime,\mathcal{L}S_\ell) = (S_\ell,\mathcal{L}S_\ell^\prime),
\eeq
so this becomes
\beq
(\bar{S}_\ell,\mathcal{L}\bar{S}_\ell) = (S_\ell,\mathcal{L}S_\ell) \pm (S_\ell^\prime,\mathcal{L}S_\ell).
\eeq

From \cref{eq:LSFinal,eq:LSPrimeFinal}, the Laplacian on $S_\ell$ and $S_\ell^\prime$ leaves only some of the potential terms. The potential terms in \cref{eq:LSFinal} are antisymmetric upon the $1 \leftrightarrow 2$ swap, and the terms in \cref{eq:LSPrimeFinal} are antisymmetric upon the $1 \leftrightarrow 3$ swap. $S_\ell$ and $S_\ell^\prime$ are symmetric with the $1 \leftrightarrow 2$ and $1 \leftrightarrow 3$ swaps, respectively. When these are integrated over these coordinates, the combination of symmetric with antisymmetric functions causes the integral to be 0:
\beq
\label{eq:SLS0Test}
(S_\ell,\mathcal{L}S_\ell) = (S_\ell^\prime,\mathcal{L}S_\ell^\prime) = 0.
\eeq
\todoi{Reference this in computation section}
Therefore,
\beq
(\bar{S}_\ell,\mathcal{L}\bar{S}_\ell) = \pm (S_\ell^\prime,\mathcal{L}S_\ell) = \pm (S_\ell,\mathcal{L}S_\ell^\prime).
\eeq

Since $C_\ell$ and $C_\ell^\prime$ are also symmetric with their respective swaps, the $(\bar{C}_\ell,\mathcal{L}\bar{S}_\ell)$ matrix element is a similar form given by
\beq
(\bar{C}_\ell,\mathcal{L}\bar{S}_\ell) = \pm (C_\ell^\prime,\mathcal{L}S_\ell) = \pm (C_\ell,\mathcal{L}S_\ell^\prime).
\eeq

%{
%\renewcommand{\arraystretch}{2}
%\begin{table}[H]
%\centering
%\begin{tabular}{l l l l}
%\toprule\\[-1.5cm]
%Partial Wave & $\SphericalHarmonicY{\ell}{0}{\theta_\rho}{\phi_\rho}$ & $j_\ell(x)$ & $n_\ell(x)$ \\
%\midrule
%S-Wave & $\frac{1}{\sqrt{4\pi}}$ & & \\
%P-Wave & $\sqrt{\frac{3}{4\pi}} \cos \theta$ & & \\
%D-Wave & $\sqrt{\frac{5}{16\pi}} \left(3 \cos^2\theta - 1 \right)$ &  &  \\
%F-Wave & $\sqrt{\frac{7}{16\pi}} \left(5 \cos^3\theta - 3 \cos\theta \right)$ &  &  \\
%G-Wave & $\sqrt{\frac{9}{256\pi}} \left(35 \cos^4\theta - 30 \cos^2\theta + 3 \right)$ &  &  \\
%H-Wave & $\sqrt{\frac{11}{256\pi}} \left(63 \cos^5\theta - 70 \cos^3\theta + 15 \cos\theta \right)$ &  &  \\
%\bottomrule
%\end{tabular}
%\caption{Spherical harmonics, spherical Bessel, and spherical Neumann functions for partial waves through $\ell = 5$}
%\label{tab:SphHarmBessel}
%\end{table}
%}


\biblio
\end{document}