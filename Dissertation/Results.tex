\documentclass[Dissertation.tex]{subfiles} 
\begin{document}


\chapter{Results}

\section{Our Results}
\subsection{Bound State: Singlet}

\begin{table}[H]
\begin{center}
\begin{tabular}{|c|c|c|c|c|c|}
\hline
Description & $\alpha$, $\beta$ and $\gamma$ & $\omega$ & N($\omega$) & Used terms & Energy\\
\hline
Double, $10^{-7}$?, with asymptotic expansion & 0.6, 0.6, 1.0 & 6 & 924 & 916 & -0.7891695098 \\
Double, $10^{-7}$?, with asymptotic expansion & 0.6, 0.6, 1.0 & 7 & 1716 & 1585 & -0.7891895683 \\
Double, $10^{-7}$?, with asymptotic expansion & 0.6, 0.6, 1.0 & 8 & 3003 & 1925 & -0.7891945593 \\
Quadruple, $10^{-6}$, with asymptotic expansion & 0.6, 0.6, 1.0 & 8 & 3003 & 2067 & -0.7891945847\\
Double, $10^{-7}$?, with asymptotic expansion & 0.6, 0.6, 1.0 & 9 & 5005 & 2155 & -0.7891958600 \\
\rowcolor{LightCyan}
Double, $10^{-7}$?, with asymptotic expansion & 0.6, 0.6, 1.0 & 10 & 8008 & 2205 & -0.7891963235 \\

Double, $10^{-7}$?, with asymptotic expansion & 0.58, 0.6, 1.0 & 9 & 5005 & 2166 & -0.7891958308\\
Double, $10^{-7}$?, with asymptotic expansion & 0.58, 0.6, 1.0 & 11 & 12376 & 1674 & -0.7891961505\\

Double, $10^{-7}$, with asymptotic expansion & 0.5, 0.6, 1.1 & 6 & 924 & 924 & -0.7891619764\\
Double, $10^{-7}$, without asymptotic expansion & 0.5, 0.6, 1.1 & 6 & 924 & 924 & -0.7891619769\\
\rowcolor{LightCyan}
Double, $10^{-6}$, with asymptotic expansion & 0.5, 0.6, 1.1 & 7 & 1716 & 1608 & -0.7891870914\\
Double, $10^{-6}$, without asymptotic expansion & 0.5, 0.6, 1.1 & 7 & 1716 & 1658 & -0.7891870935\\
Quadruple, $10^{-6}$, with asymptotic expansion & 0.5, 0.6, 1.1 & 7 & 1716 & 1661 & -0.7891870935\\
Double, $10^{-6}$, with asymptotic expansion & 0.5, 0.6, 1.1 & 8 & 3003 & 1653 & -0.7891935019\\
Double, $10^{-6}$, without asymptotic expansion & 0.5, 0.6, 1.1 & 8 & 3003 & 1403 & -0.7891932577\\
Quadruple, $10^{-6}$, with asymptotic expansion & 0.5, 0.6, 1.1 & 8 & 3003 & 1619 & -0.7891934661\\
\hline
\end{tabular}
\caption{Bound State Energy for Singlet PsH}
\label{tab:BoundSingletResults}
\end{center}
\end{table}

The PsH bound state energy code was run for several sets of nonlinear parameters and increasing values of $\omega$.  Some runs were also performed using quadruple precision, but the results for that are preliminary at this point.  Table \ref{tab:BoundSingletResults} contains the results of these runs.  One criteria of Todd's method (\ref{sec:ToddBound}) is that the eigenvalues from the upper and lower triangular matrices must differ by no more than a threshold value.  This threshold is given in the first column.

The best run is given by the first highlighted row, with $\omega = 10$.  We performed a run for $\omega = 11$, but linear dependence issues caused it to returned only 1674 terms, less than the 2205 for $\omega = 10$.  The energy is also higher, so $\omega = 10$ is the practical limit for our calculations without improvements in the code to solve the generalized eigenvalue problem (\ref{eq:BoundGenEig}).  The scattering problem is much more demanding with regards to computation time and linear dependence, so the current set of short-range terms that we use is denoted by the second highlighted row with $\omega = 7$.  

\textbf{@TODO:} Further discussion on the linear dependence in another section


\setlength{\abovecaptionskip}{6pt}   % 0.5cm as an example
\setlength{\belowcaptionskip}{6pt}   % 0.5cm as an example
\begin{table}[H]
\centering
\begin{tabular}{c c c c c c c}
\toprule
$\omega$ & Terms & $\alpha$ & $\beta$ & $\gamma$ & Total Energy (au) & Binding Energy (eV) \\ [0.5ex]
\midrule
0 & 1 & 0.60 & 0.60 & 1.00 & -0.541 492 378 889 & --- \\
1 & 7 & 0.60 & 0.60 & 1.00 & -0.744 334 244 165 & --- \\
2 & 28 & 0.60 & 0.60 & 1.00 & -0.778 357 106 972 & 0.771 636 156 726 \\
3 & 84 & 0.60 & 0.60 & 1.00 & -0.786 807 448 395 & 1.001 581 651 009 \\
4 & 210 & 0.60 & 0.60 & 1.00 & -0.788 685 563 109 & 1.052 687 753 648 \\
5 & 462 & 0.60 & 0.60 & 1.00 & -0.789 082 645 582 & 1.063 492 917 716 \\
6 & 916 & 0.60 & 0.60 & 1.00 & -0.789 169 509 836 & 1.065 856 614 384 \\
7 & 1585 & 0.60 & 0.60 & 1.00 & -0.789 189 568 390 & 1.066 402 435 425 \\
8 & 1925 & 0.60 & 0.60 & 1.00 & -0.789 194 559 324 & 1.066 538 245 640 \\
9 & 2166 & 0.60 & 0.60 & 1.00 & -0.789 195 830 870 & 1.066 572 846 182 \\
10 & 2205 & 0.60 & 0.60 & 1.00 & -0.789 196 323 586 & 1.066 586 253 647 \\
11 & 1674 & 0.58 & 0.60 & 1.00 & -0.789 196 284 600 & 1.066 585 192 793 \\
\bottomrule
\end{tabular}
\caption{Ground-state energy of positronium hydride} % title of Table
\label{tab:BoundEnergy}
\end{table}


\setlength{\abovecaptionskip}{6pt}   % 0.5cm as an example
\setlength{\belowcaptionskip}{6pt}   % 0.5cm as an example
\begin{table}[H]
\centering
\begin{tabular}{c c c c c c c}
\toprule
$\omega$ & Terms & $\alpha$ & $\beta$ & $\gamma$ & Total Energy (au) & Binding Energy (eV) \\ [0.5ex]
\midrule
0 & 1 & 0.50 & 0.60 & 1.10 & -0.609 615 339 155 & --- \\
1 & 7 & 0.50 & 0.60 & 1.10 & -0.744 479 903 766 & --- \\
2 & 28 & 0.50 & 0.60 & 1.10 &    -0.777 557 969 817 & 0.749 890 527 892 \\
3 & 84 & 0.50 & 0.60 & 1.10 &    -0.786 424 552 475 & 0.991 162 522 688 \\
4 & 210 & 0.50 & 0.60 & 1.10 &   -0.788 573 334 213 & 1.049 633 849 950 \\
5 & 462 & 0.50 & 0.60 & 1.10 &   -0.789 054 818 106 & 1.062 735 693 547 \\
6 & 924 & 0.50 & 0.60 & 1.10 &   -0.789 161 976 450 & 1.065 651 620 503 \\
7 & 1608 & 0.50 & 0.60 & 1.10 &  -0.789 187 091 408 & 1.066 335 033 300 \\
8 & 1653 & 0.50 & 0.60 & 1.10 &  -0.789 193 426 168 & 1.066 507 410 889 \\
9 & 1629 & 0.50 & 0.60 & 1.10 &  -0.789 195 191 874 & 1.066 555 458 198 \\
10 & 1797 & 0.50 & 0.60 & 1.10 & -0.789 195 743 623 & 1.066 570 472 057 \\
\bottomrule
\end{tabular}
\caption{Ground-state energy of positronium hydride} % title of Table
\label{tab:BoundEnergy}
\end{table}




\subsection{``Bound State'': Triplet}

\begin{table}[H]
\centering
\begin{tabular}{|c|c|c|c|c|c|}
\hline
Description & $\alpha$, $\beta$ and $\gamma$ & $\omega$ & N($\omega$) & Used terms & Energy\\
\hline
Double, $10^{-7}$, without asymptotic expansion & 0.5, 0.6, 1.1 & 6 & 924 & 924 & -0.7337271181 \\
Double, $10^{-7}$, without asymptotic expansion & 0.5, 0.6, 1.1 & 7 & 1716 & 1050 & -0.7377948557 \\
Double, $10^{-7}$, without asymptotic expansion & 0.5, 0.6, 1.1 & 8 & 3003 & 1035 & -0.7396804167 \\

Double, $10^{-7}$, without asymptotic expansion & 0.33, 0.36, 0.9 & 6 & 924 & 924 & -0.7403629263 \\
Double, $10^{-7}$, with asymptotic expansion & 0.323, 0.334, 0.975 & 7 & 1716 & 450 & -0.7433297421 \\
\rowcolor{LightCyan}
Double, $10^{-6}$, with asymptotic expansion & 0.323, 0.334, 0.975 & 7 & 1716 & 1327 & -0.7433869876 \\
Double, $10^{-7}$, without asymptotic expansion & 0.323, 0.334, 0.975 & 7 & 1716 & 450 & -0.7433303800 \\
Double, $10^{-6}$, without asymptotic expansion & 0.323, 0.334, 0.975 & 7 & 1716 & 900 & -0.7433728119 \\
Quadruple, $10^{-6}$, with asymptotic expansion & 0.323, 0.334, 0.975 & 7 & 1716 & 993 & -0.7433812904 \\
Quadruple, $10^{-7}$, with asymptotic expansion & 0.323, 0.334, 0.975 & 7 & 1716 & 700 & -0.7433679278 \\
Double, $10^{-6}$, with asymptotic expansion & 0.323, 0.334, 0.975 & 8 & 3003 & 500 & -0.7450315235 \\
Double, $10^{-6}$, without asymptotic expansion & 0.323, 0.334, 0.975 & 8 & 3003 & 532 & -0.7450273221 \\
\hline
\end{tabular}
\caption{``Bound State'' Energy for Triplet PsH}
\label{tab:BoundTripletResults}
\end{table}

\indent Our code does not predict a true triplet bound state, making the singlet state the only possible configuration for PsH.  Mitroy and Bromley have published a paper claiming a stable triplet bound state \cite{Mitroy2007}, but our code may not have the appropriate type of wavefunction to see this.  Despite not predicting a bound state, we run the bound state code for the triplet so that we can use this for the short-range terms for the scattering program.

This output is also used in the program for Todd's method, which is then fed into the scattering program.  The triplet case is much more sensitive to the nonlinear parameters, as evidenced by the results in Table \ref{tab:BoundTripletResults}.  The lowest energy and larger number of terms for $\omega = 7$ is given by the highlighted row.  This set of terms is used in our triplet scattering calculations.  We have been unable to use more terms for $\omega = 8$ than $\omega = 7$, though the energy is lower than any of those given for $\omega = 7$.


\subsection{S-Wave Scattering Results: Singlet}
All runs here were performed using Peter's set, with an extra 10 points in all coordinates, except for the 2nd, 3rd and 4th (which gained an extra 5 points instead).  The phaseshift is given at the last term.  The value in brackets after the extrapolated phase shift is the starting value of $\omega$ for the extrapolation.

The following table shows some sample runs for one value of $\kappa$, $0.1$.  The extrapolation becomes less stable for higher $\kappa$, as is discussed in section (\ref{sec:Extrapolation}).  The highlighted line shows the choice of parameters that are used, since this leads to a more well-converged set of results.

\begin{center}
\begin{tabular}{|c|c|c|c|c|c|c|}
\hline
Description & $\alpha$, $\beta$ and $\gamma$ & $\omega$ & $\kappa$ & Terms & Phase Shift & Extrapolated\\
\hline
\rowcolor{LightCyan} Double, $10^{-6}$, with asymptotic expansion & 0.5, 0.6, 1.1 & 7 & 0.1 & 1450 & -0.426666 & -0.4254 [3] \\
& & & & & & -0.4257 [4] \\
Quadruple, $10^{-6}$, with asymptotic expansion & 0.5, 0.6, 1.1 & 7 & 0.1 & 1595 & -0.42647 & -0.42517 [3] \\
Double, $10^{-6}$, with asymptotic expansion & 0.5, 0.6, 1.1 & 8 & 0.1 & 1614 & -0.426646 & -0.4254 [3] \\
Double, $10^{-6}$, with asymptotic expansion & 0.6, 0.6, 1.0 & 8 & 0.1 & 1925 & -0.42662 & -0.4258 [3] \\
& & & & & & -0.4259 [4] \\
& & & & & & -0.4259 [5] \\
\hline
\end{tabular}
\end{center}

\subsection{S-Wave Scattering Results: Phase Shifts}

\begin{table}[H]
\begin{center}
\begin{tabular}{|c !{\vrule width 1pt} c|c !{\vrule width 1pt} c|c !{\vrule width 1pt} c | c !{\vrule width 1pt} c | c|}
\hline
$\kappa$ & $\delta^+ (\omega = 6)$ & $\delta^- (\omega = 6)$ & $\delta^+ (\omega = 7)$ & $\delta^- (\omega = 7)$ & $\delta^+ (\omega \rightarrow \infty)$ & $\delta^- (\omega \rightarrow \infty)$ & \% Diff$^+$ & \% Diff$^-$ \\
\hline
0.1 & -0.427 & -0.215 & -0.427 & -0.215 & -0.425 & -0.214 & 0.35\% & 0.62\% \\
0.2 & -0.820 & -0.432 & -0.820 & -0.432 & -0.818 & -0.431 & 0.26\% & 0.36\% \\
0.3 & -1.162 & -0.646 & -1.161 & -0.646 & -1.160 & -0.644 & 0.17\% & 0.32\% \\
0.4 & -1.447 & -0.851 & -1.446 & -0.850 & -1.445 & -0.849 & 0.084\% & 0.49\% \\
0.5 & -1.678 & -1.043 & -1.678 & -1.041 & -1.676 & -1.039 & 0.18\% & 0.72\% \\
0.6 & -1.858 & -1.219 & -1.858 & -1.217 & -1.856 & -1.213 & 0.22\% & 0.59\% \\
0.7 & -1.965 & -1.378 & -1.964 & -1.376 & -1.962 & -1.370 & 0.21\% & 0.58\% \\
\hline
\end{tabular}
\caption{S-Wave Phase Shifts}
\label{tab:SWavePhase}
\end{center}
\end{table}

Table \ref{tab:SWavePhase} contains the phase shifts for regular intervals of $\kappa$, which we can compare to the results from other groups in Tables \ref{tab:SWaveSingletOther} and \ref{tab:SWaveTripletOther} starting on page \pageref{tab:SWaveSingletOther}.  Our phase shifts at $\omega = 7$ from this table are exactly the same as Van Reeth and Humberston's results for $\omega = 6$ \cite{VanReeth2003}, with some exceptions in the last digit.  Figure \ref{fig:SWavePhaseOmega=7} has the fuller set of phase shifts plotted with respect to the positronium momentum, $\kappa$.

\textbf{@TODO:} What are our results for $\omega = 6$?

As mentioned in Section \ref{sec:Extrapolation}, extrapolation to $\omega = \infty$ has been problematic with larger numbers of terms.

\begin{figure}[H]
	\centering
	\includegraphics[width=7in]{Omega=7}
	\caption{S-Wave Singlet and Triplet Phase Shifts}
	\label{fig:SWavePhaseOmega=7}
\end{figure}


\subsection{S-Wave Scattering Results: Resonance Parameters}
Section \ref{sec:SWaveResonances} showed our calculations of the resonance parameters for the S-wave singlet.  For $\omega = 6$, the position of the first resonance ranges from 4.0054 to 4.0075, depending on the weighting function used for the nonlinear fitting.  The weighting functions that give the smallest residuals (Bisquare, Andrews and Welsh) give an average $^1$E$_R$ $=4.0067$.  For $\omega = 7$, $^1$E$_R$ ranges from 4.0051 to 4.0062, with an average of 4.0061.  The rest of the fitting parameters from the weighting fits with the smallest residuals are given in table \ref{tab:SWaveResonancesCopy}.  This table is identical to table \ref{tab:SWaveResonances}, which has more discussion in section \ref{sec:ResonanceErrors}.

\begin{table}[H]
\begin{center}
\begin{tabular}{c c c c c}
\toprule
$\omega$ & $^1E_R$ & $^1\Gamma$ & $^2E_R$ & $^2\Gamma$ \\
\midrule
6 & $4.0067 \pm 0.0004$ & $0.0955 \pm 0.0002$ & $5.0265 \pm 0.0015$ & $0.0596 \pm 0.0008$ \\
7 & $4.0061 \pm 0.0001$ & $0.0955 \pm 0.0001$ & $4.9754 \pm 0.0008$ & $0.0493 \pm 0.0002$ \\
\bottomrule
\end{tabular}
\caption{S-Wave Singlet Resonance Parameters}
\label{tab:SWaveResonancesCopy}
\end{center}
\end{table}

\textbf{@TODO:} Discussion of Peter's previous calculations and omitted terms.
The errors are greater for the higher $\omega$, though they are still small.  Our $\omega = 6$ results are nearly exactly the same as Peter Van Reeth's earlier calculation using almost the same set of terms \cite{VanReeth2004}.  When $\omega$ is raised to 7, the values shift slightly and match closely with Yan and Ho's results \cite{Yan1999}.  Particularly interesting is the fact that the second resonance shifts to a lower energy, matching better with Yan and Ho's second resonance position.  Both Van Reeth's and Yan's resonance parameters can be seen in Table \ref{tab:SWaveResonOther}.


\section{Other Groups' Results}
\subsection{Bound State Results}
\begin{table}[H]
\begin{center}
%\begin{tabular}{|l|l|c|l|l|}
\begin{tabular}{l l c l l}
\toprule
Group & Method & Terms & Total Energy (au) & Binding Energy (eV)\\
\midrule
Ho (1986) \cite{Ho1986} & Variational with Hylleraas (?) & 396 & -0.788 945$^\star$ & 1.059 75 \\
Frolov (1997) \cite{Frolov1997a} & James-Coolidge variational & 924 & -0.789 136 9$^\star$ & 1.064 969 \\
Frolov (1997) \cite{Frolov1997a} & James-Coolidge variational & --- & -0.789 181 8$^\star$ & 1.066 191 \\
Frolov (1997) \cite{Frolov1997b} & Kolesnikov-Tarasov variational & --- & -0.789 179 4$^\star$ & 1.066 126 \\
Adhikari (1999) \cite{Adhikari1999} & Five-state & --- & -0.7886 & 1.05 \\
Yan (1999) \cite{Yan1999} & Variational with Hylleraas $(\omega = 10)$ & 3501 & -0.789 196 607 6$^\star$ & 1.066 593 98 \\
Yan (1999) \cite{Yan1999} & Variational with Hylleraas $(\omega = 12)$ & 5741 & -0.789 196 705 1$^\star$ & 1.066 596 635 \\
Yan (1999) \cite{Yan1999} & Variational with Hylleraas $(\omega \rightarrow \infty)$ & - & -0.789 196 714 7$^\star$ & 1.066 596 896 \\
Blackwood (2002) \cite{Blackwood2002} & ?Close-coupling 14Ps14H? & --- & -0.786 5 & 0.994$^\star$ \\
Van Reeth (2003) \cite{VanReeth2003} & Variational with Hylleraas $(\omega = 6)$ & 721 & -0.789 156 & 1.0655$^\star$ \\
Walters (2004) \cite{Walters2004} & ?Close-coupling 14Ps14H + $\text{H}^-$? & --- & -0.787 9 & 1.03$^\star$\\
Mitroy (2006) \cite{Mitroy2006} & ECGs with SVM & 1800 & -0.789 196 740$^\star$ & 1.066 597 58 \\
Bubin (2006) \cite{Bubin2006} & ECGs ?variational? & 5000 & -0.789 196 765 251$^\star$ & 1.066 598 271 959 \\
Frolov (2010) \cite{Frolov2010} & Semi-exponential basis & 84 & -0.788 516 419$^\star$ & 1.048 085 11 \\
\bottomrule
\end{tabular}
\caption{Bound State Energies from Other Groups - Starred values are the reported values.  Unstarred values are obtained by using the conversion factor given in section \ref{sec:Units}.}
\label{tab:BoundEnergyOther}
\end{center}
\end{table}


\subsection{S-Wave Scattering Results: Phase Shifts}

\begin{table}[H]
\begin{center}
\begin{tabular}{|c|c|c|c|c|c|c|}
\hline
 & $(\omega = 6)$ & $(\omega \rightarrow \infty)$ &  &  &  & \\
$\kappa$ & $\delta^+$ \cite{VanReeth2003} & $\delta^+$ \cite{VanReeth2003} & $\delta^+$ \cite{Blackwood2002} & $\delta^+$ \cite{Walters2004} & $\delta^+$ \cite{Ray1997} & $\delta^+$ \cite{Adhikari1999} \\
\hline
0.1 & -0.427 & -0.425 & -0.434 & -0.428 & -0.692 & -0.362 \\
0.2 & -0.820 & -0.817 & -0.834 & -0.825 & -1.212 & -0.702 \\
0.3 & -1.161 & -1.158 & -1.178 & -1.167 & -1.592 & -1.002 \\
0.4 & -1.446 & -1.443 & -1.467 & -1.453 & -1.902 & -1.252 \\
0.5 & -1.677 & -1.674 & -1.704 & -1.685 & -2.142 & -1.462 \\
0.6 & -1.857 & -1.852 & -1.890 & -1.867 & -2.362 & -1.622 \\
0.7 & -1.964 & -1.959 & -2.018 & -1.992 & -2.512 & -1.712 \\
0.8 &    --- &    --- &    --- &    --- & -2.652 & -2.163 \\
\hline
\end{tabular}
\caption{S-Wave Singlet Results from Other Groups}
\label{tab:SWaveSingletOther}
\end{center}
\end{table}


\begin{table}[H]
\begin{center}
\begin{tabular}{|c|c|c|c|c|c|}
\hline
 & $(\omega = 6)$ & $(\omega \rightarrow \infty)$ &  &  &   \\
$\kappa$ & $\delta^-$ \cite{VanReeth2003} & $\delta^-$ \cite{VanReeth2003} & $\delta^-$ \cite{Blackwood2002} & $\delta^-$ \cite{Ray1997} & $\delta^-$ \cite{Adhikari1999} \\
\hline
0.1 & -0.215 & -0.214 & -0.206 & -0.252 & -0.167 \\
0.2 & -0.432 & -0.431 & -0.414 & -0.502 & -0.327 \\
0.3 & -0.645 & -0.645 & -0.624 & -0.722 & -0.474 \\
0.4 & -0.850 & -0.849 & -0.838 & -0.942 & -0.602 \\
0.5 & -1.040 & -1.038 & -1.037 & -1.142 & -0.706 \\
0.6 & -1.215 & -1.211 & -1.213 & -1.332 & -0.784 \\
0.7 & -1.373 & -1.366 & -1.367 & -1.502 & -0.833 \\
0.8 &    --- &    --- &    --- & -1.652 & -0.851 \\
\hline
\end{tabular}
\caption{S-Wave Triplet Results from Other Groups}
\label{tab:SWaveTripletOther}
\end{center}
\end{table}


\begin{table}[H]
\begin{center}
\begin{tabular}{c c c c c c}
\toprule
 & $(\omega = 6)$ & $(\omega \rightarrow \infty)$ &  &  &   \\
$\kappa$ & $\delta^-$ \cite{VanReeth2003} & $\delta^-$ \cite{VanReeth2003} & $\delta^-$ \cite{Blackwood2002} & $\delta^-$ \cite{Ray1997} & $\delta^-$ \cite{Adhikari1999} \\
\midrule
0.1 & -0.215 & -0.214 & -0.206 & -0.252 & -0.167 \\
0.2 & -0.432 & -0.431 & -0.414 & -0.502 & -0.327 \\
0.3 & -0.645 & -0.645 & -0.624 & -0.722 & -0.474 \\
0.4 & -0.850 & -0.849 & -0.838 & -0.942 & -0.602 \\
0.5 & -1.040 & -1.038 & -1.037 & -1.142 & -0.706 \\
0.6 & -1.215 & -1.211 & -1.213 & -1.332 & -0.784 \\
0.7 & -1.373 & -1.366 & -1.367 & -1.502 & -0.833 \\
0.8 &    --- &    --- &    --- & -1.652 & -0.851 \\
\bottomrule
\end{tabular}
\caption{S-Wave Triplet Results from Other Groups}
\label{tab:SWaveTripletOther}
\end{center}
\end{table}


\begin{figure}[H]
	\centering
	\includegraphics[width=7in]{OtherSingletResults}
	\caption{Comparison of S-Wave Singlet Results from Other Groups}
	\label{fig:OtherSingletResults}
\end{figure}

Figure \ref{fig:OtherSingletResults} shows the results from various groups for calculations of the singlet S-wave.  Our results are extremely close to Van Reeth's \cite{VanReeth2003}, so they would follow along the solid line as well.  Several groups have results that cluster very closely to his and ours, namely Blackwood \cite{Blackwood2002}, Walters \cite{Walters2004}, Chiesa \cite{Chiesa2002} and Ivanov \cite{Ivanov2002}.

\begin{figure}[H]
	\centering
	\includegraphics[width=7in]{OtherTripletResults}
	\caption{Comparison of S-Wave Triplet Results from Other Groups}
	\label{fig:OtherTripletResults}
\end{figure}

\textbf{@TODO:} Discussion of the techniques used by each of the groups


\subsection{S-Wave Scattering Results: Scattering Length and Effective Range}
\begin{table}[H]
\begin{center}
\begin{tabular}{l l l l l}
\toprule
Method & \multicolumn{1}{c}{$a^+$} & \multicolumn{1}{c}{$r_0^+$} & \multicolumn{1}{c}{$a^-$} & \multicolumn{1}{c}{$r_0^-$}\\
\midrule
Static-exchange (Hara \emph{et al} 1975) \cite{Hara1975} & 7.275 & \,\,--- & 2.476 & \,\,--- \\
Kohn 35 terms (Page 1976) \cite{Page1976} & 5.844 & 2.90 & 2.319 & \,\,--- \\
Stabilization (Drachman \emph{et al} 1975) \cite{Drachman1975} & 5.33 & 2.54 & \,\,--- & \,\,--- \\
Stabilization (Drachman \emph{et al} 1976) \cite{Drachman1976} & \,\,--- & \,\,--- & 2.36 & 1.31 \\
9-state R-matrix (Campbell \emph{et al} 1998) \cite{Campbell1998} & 5.51 & 2.63 & 2.45 & 1.33 \\
22-state R-matrix (Campbell \emph{et al} 1998) \cite{Campbell1998} & 5.20 & 2.52 & 2.45 & 1.32 \\
5-state (Adhikari \emph{et al} 1999) \cite{Adhikari1999} & 3.72 & 1.67 & --- & --- \\
6-state close coupling (Sinha \emph{et al} 2000) \cite{Sinha2000} & 5.90 & 2.73 & 2.32 & 1.29 \\
Variational basis-set (Adhikari \emph{et al} 2001) \cite{Adhikari2001b} & 3.49 & \,\,--- & 2.46 & \,\,--- \\
Diffusion Monte Carlo (Chiesa \emph{et al} 2002) \cite{Chiesa2002} & 4.375 & 2.228 & 2.246 & 1.425 \\
Stochastic variational (Ivanov \emph{et al} 2002) \cite{Ivanov2002} & 4.34 & 2.39 & 2.22 & 1.29 \\
R-matrix 14Ps14H (Blackwood \emph{et al} 2002) \cite{Blackwood2002} & 4.41 & 2.19 & 2.06 & 1.47 \\
Kohn 721 terms (Van Reeth \emph{et al} 2003) \cite{VanReeth2003} & 4.334 & \,\,--- & 2.143 & \,\,--- \\
Kohn extrapolated (Van Reeth \emph{et al} 2003) \cite{VanReeth2003} & 4.311 & 2.27 & 2.126 & 1.39 \\
R-matrix 14Ps14H+H$^-$ (Walters \emph{et al} 2004) \cite{Blackwood2002} & 4.327 & \,\,--- & \,\,--- & \,\,--- \\
\bottomrule
\end{tabular}
\caption{S-Wave Scattering Length and Effective Range}
\label{tab:SWaveScatLenOther}
\end{center}
\end{table}


\subsection{S-Wave Scattering Results: Resonance Parameters}
\setlength{\abovecaptionskip}{6pt}   % 0.5cm as an example
\setlength{\belowcaptionskip}{6pt}   % 0.5cm as an example
\begin{table}[H]
\centering
\begin{tabular}{c c c c c c}
\toprule
Method & $^1E_R \text{ (eV)}$ & $^1\Gamma \text{ (eV)}$ & $^2E_R \text{ (eV)}$ & $^2\Gamma \text{ (eV)}$ \\
\midrule
Complex rotation (Drachman \emph{et al} 1975) \cite{Drachman1975} & $4.455 \pm 0.010$ & $0.062 \pm 0.015$ & --- & --- \\
Complex rotation (Ho 1978) \cite{Ho1978} & $4.013 \pm 0.014$ & $0.075 \pm 0.027$ & --- & --- \\
Complex rotation (Yan \emph{et al} 1999) \cite{Yan1999} & $4.0058 \pm 0.0005$ & $0.0952 \pm 0.0011$ & $4.9479 \pm 0.0014$ & $0.0585 \pm 0.0027$ \\
Close coupling (Blackwood \emph{et al} 2002) \cite{Blackwood2002} & $4.37$ & $0.10$ & --- & --- \\
Kohn variational (Van Reeth \emph{et al} 2004) \cite{VanReeth2004} & $4.0072 \pm 0.0020$ & $0.0956 \pm 0.010$ & $5.0267 \pm 0.0020$ & $0.0597 \pm 0.0010$ \\
Close coupling (Walters \emph{et al} 2004) \cite{Walters2004} & $4.149$ & $0.103$ & $4.877$ & $0.0164$ \\
\bottomrule
\end{tabular}
\caption{S-Wave Resonance Parameters} % title of Table
\label{tab:SWaveResonancesOther}
\end{table}


\end{document}

